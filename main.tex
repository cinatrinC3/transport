\RequirePackage{etex}
\documentclass[12pt,a4paper]{article}
\usepackage[a4paper,left=1cm,right=1cm,top=2cm,bottom=2cm]{geometry}
\usepackage{CJKutf8}
\usepackage[colorlinks, linkcolor=blue]{hyperref}
\usepackage{pgfplots}
\usepackage{amsmath}
\usepackage{amssymb}
\usepackage{cancel}
\usepackage{fancyhdr}
\usepackage{soul}
\usepackage{chemarr}
\usepackage{circuitikz}
\usepackage{ifthen}
\usepackage{esint}
\usepackage{textcomp}
\usepackage{catchfilebetweentags}
\usepackage{subfiles}
\usepackage{tikz}
\usetikzlibrary{patterns,patterns.meta,arrows.meta,intersections,calc,decorations.pathmorphing}
\usepackage{tikz-3dplot}
\usepackage{float}
\usepackage{chngcntr}
\usepackage{makecell}
\usepackage{tocloft}
\usepackage{dutchcal}
\usepackage{bm}
\counterwithin{figure}{section}
\graphicspath{images}
\pgfplotsset{compat=1.18} 
\setlength{\headheight}{26pt}
\numberwithin{equation}{subsection}

\renewcommand\cftsecafterpnum{\par\addvspace{1pt}}
\newcommand{\mcirc}[1]{\text{\textcircled{\footnotesize #1}}}
\newcommand \dbar {{d\mkern-6mu\mathchar'26\mkern-3mu}}
\newcommand*{\rttensor}[1]{\bar{\bar{#1}}}
\makeatletter
\newcommand{\rightbotmark}{\expandafter\@rightmark\botmark}
\DeclareFontFamily{U}{tipa}{}
\DeclareFontShape{U}{tipa}{m}{n}{<->tipa10}{}
\newcommand{\arc@char}{{\usefont{U}{tipa}{m}{n}\symbol{62}}}%
\newcommand{\arc}[1]{\mathpalette\arc@arc{#1}}
\newcommand{\arc@arc}[2]{%
  \sbox0{$\m@th#1#2$}%
  \vbox{
    \hbox{\resizebox{\wd0}{\height}{\arc@char}}
    \nointerlineskip
    \box0
  }%
}
\def\cantox@vector#1#2#3#4#5#6#7#8{%
  \dimen@.5\p@
  \setbox\z@\vbox{\boxmaxdepth.5\p@
   \hbox{\kern-1.2\p@\kern#1\dimen@$#7{#8}\m@th$}}%
  \ifx\canto@fil\hidewidth  \wd\z@\z@ \else \kern-#6\unitlength \fi
  \ooalign{%
    \canto@fil$\m@th \CancelColor
    \vcenter{\hbox{\dimen@#6\unitlength \kern\dimen@
      \multiply\dimen@#4\divide\dimen@#3 \vrule\@depth\dimen@\@width\z@
      \vector(#3,-#4){#5}%
    }}_{\raise-#2\dimen@\copy\z@\kern-\scriptspace}$%
    \canto@fil \cr
    \hfil \box\@tempboxa \kern\wd\z@ \hfil \cr}}
\def\bcancelto#1#2{\let\canto@vector\cantox@vector\cancelto{#1}{#2}}
\makeatother
\title{單元操作筆記}
\author{B10504001 化工五 唐麒鈞}
\fancyhf{}
\cfoot{\thepage}
\renewcommand{\sectionmark}[1]{\markboth{#1}{}}
\renewcommand{\subsectionmark}[1]{\markright{\thesubsection\ \ #1}}
\begin{document}
\begin{CJK*}{UTF8}{bkai}
\pagenumbering{roman}
\pagestyle{plain}
\maketitle
\thispagestyle{empty}
\newpage
\tableofcontents
\newpage
\listoffigures
\newpage
\listoftables
\newpage
\pagenumbering{arabic}
\renewcommand{\headrule}{%
  \hrule width\headwidth height 0.4pt % The standard horizontal line
  \vspace{2pt}                        % Small gap between line and text
  {\footnotesize \hfill \rightbotmark}   % \hfill pushes text to the right
}
\pagestyle{fancy}
\rhead{單元操作-\leftmark}
\lhead{B10504001$\textrm{  }$化工五$\textrm{  }$唐麒鈞}
\section{各種數學工具}
\subfile{sections/ch0_1_math_tools}
\subfile{sections/ch0_2_coordinates}
\newpage
\section{流體力學導論}
\subfile{sections/ch1_fluid_intro}
\subfile{sections/ch1_2_dimensionless}
\newpage
\section{動量輸送}
\subfile{sections/ch2_0_momentum_transport}
\subfile{sections/ch2_1_laminar_flow}
\subfile{sections/ch2_2_laminar_differential}
\subfile{sections/ch2_3_multi_variable_steady}
\subfile{sections/ch2_4_unsteady_state}
\subfile{sections/ch2_5_boundary_layer_dev}
\newpage
\section{熱量輸送}
\subfile{sections/ch3_1_heat_transport_intro}
\subfile{sections/ch3_2_heat_transfer_solve}
\subfile{sections/ch3_3_heat_transfer_bessel}
\subfile{sections/ch3_4_heat_transfer_unsteady}
\subfile{sections/ch3_5_heat_transfer_graetz}
\subfile{sections/ch3_6_heat_transfer_macro}
\newpage
\section{質量傳輸}
\subfile{sections/ch4_1_mass_transport_intro}
\subfile{sections/ch4_2_mass_transport_basic}
\subfile{sections/ch4_3_mass_transport_stefan}
\subfile{sections/ch4_4_mass_transport_boundary_layer}
\subfile{sections/ch4_5_mass_momentum_transport}
\subfile{sections/ch4_6_mass_thermal_transport}
\subfile{sections/ch4_7_mass_transport_reaction}
\newpage
\section{單元操作}
\subfile{sections/ch5_0_unit_operation_intro}
\subfile{sections/ch5_1_unit_operation_exchanger}
\subfile{sections/ch5_2_unit_operation_evaporator}
\subsection{蒸餾 Distillation}
藉由加熱使混合物之各成分因蒸氣壓的不同而分離
\begin{itemize}
  \item 分類:
  \begin{itemize}
    \item 簡單蒸餾 Simple Distillation :\\
    液體受熱所產生的蒸汽直接排出,不再冷凝,回流,或與其他蒸氣進行液汽接觸
    \begin{enumerate}
      \item 閃蒸、驟沸蒸餾 Flash distillation\\
      進料經加熱後,進入閃蒸槽,將槽內壓力降低,則液體部分企劃蒸發而達氣液平衡
      \item 微分蒸餾 Differential distillation\\
      液體(L)經加熱會產生蒸氣(V),經冷凝而成為塔頂產物,不再回流至器內。\\
      殘留液的量(L)及組成($x_A$),與塔頂產物的量(V)及組成($y_A$)皆會隨時間改變(unsteadt state)\\
      長時間來看,L,V為未平衡關係,不可視為平衡操作\\
      但短時間來看,L,V仍為平衡關係,故$y_A,x_A$還是可以看平衡線
    \end{enumerate}
    \item 精餾 Rectification:\\
    具有冷凝、回流裝置
    \begin{enumerate}
      \item 批式蒸餾 Batch Distillation
      \item 連續蒸餾 Continuous Distillation\\
      可以拆分為三個物流、三個操作線、兩個影響因素
      \begin{itemize}
        \item 物流:
        \begin{itemize}
          \item 進料($F$),Feed
          \item 塔頂產物($D$),Distillate
          \item 塔底產物($W$),Bottoms
        \end{itemize}
        \item 操作線 Operating Line:
        \begin{itemize}
          \item q-line,Feed Section Operating Line
          \item 增濃段,Enriching Section Operating Line
          \item 氣提段(回收段),Stripping(Recovery) Section Operating Line
        \end{itemize}
        \item 影響因素:
        \begin{itemize}
          \item 溫度\\
          若塔溫提高,則分離效果較差,$x_D$下降,$x_W$上升\\
          若回流比提高或再沸器功率降低,塔溫會下降,分離效果較好
          \item 壓力\\
          若壓力上升,則分離效果較差,$x_D$下降,$x_W$上升
        \end{itemize}
      \end{itemize}
    \end{enumerate}
    \item 共沸蒸餾 Azeotrope Distillation
    \item 萃取蒸餾 Extraction Distillation
    \item 水蒸氣蒸餾 Steam Distillation
  \end{itemize}
  \item 重要名詞
  \begin{itemize}
    \item 相對揮發度 Relative volatility,$\alpha_{AB}$\\
    定義:
    \begin{equation}
      \alpha_{AB}=\frac{\frac{y_A}{x_A}}{\frac{y_B}{x_B}} = \frac{y_A\left(1-x_A\right)}{x_A\left(1-y_A\right)}
    \end{equation}
    \item 平衡線 Equilibrium line\\
    將上式移項後,若作圖$y_A,x_A$,則會得到平衡線
    \begin{equation}
       y_A = \frac{\alpha_{AB}x_A}{1+\left(\alpha_{AB}-1\right)x_A}
    \end{equation}
    平衡線可判斷分離$A$,$B$的難易程度\\
    若$\alpha_{AB}>1$,$A$易揮發\\
    若$\alpha_{AB}<1$,$B$易揮發\\
    若$\alpha_{AB}=1$,$A$,$B$達共沸點(Azeotrope point)\\
    而$\alpha_{AB}$越偏離1,則$A$,$B$越易分離
  \end{itemize}
  \item Flash Distillation 的計算範例:\\
  假設進料$F$,含輕成分$x_F$,進入閃蒸槽後,產生蒸氣$V$,含$y_A$,液體$L$,含$x_A$
  \begin{itemize}
    \item 整體質量平衡:
    \begin{equation}
      F = L+V
    \end{equation}
    \item 輕成分的質量平衡:
    \begin{equation}
      F\cdot x_F = L\cdot x_A +V\cdot y_A
    \end{equation}
    解得操作式,必過($x_F,x_F$)點
    \begin{equation}
      y_A = \frac{f-1}{f}x_A + \frac{x_f}{f}
    \end{equation}
    並定義$f$,代表進料中被汽化的比例:
    \begin{equation}
      f = \frac{V}{F}
    \end{equation}
    \item 自點($x_F,x_F$),做斜率$\left(\frac{f-1}{f}\right)$的直線,會交平衡線於$P$點\\
    $P$點及為出口端氣體與液體的組成\\
    另外也可以做通過點($x_F,x_F$)和點($0,\frac{x_F}{f}$)的直線,同樣是$P$點
  \end{itemize}
  \item Differential Distillation 微分蒸餾的計算範例\\
  (同樣適用於批次蒸餾(Batch Distillation))
  假設有一具有共$L$ mole,內含$x_A$,的液體,在定功率的加熱下,蒸餾出$V$ mole 的東西,內含$y_A$\\
  $t=0$時,只有$L$以及在$L$裡的$x_A$,令為$L_1$,$x_{A_1}$\\
  $t=t$時,液體剩下$L-dL$,$X_A-dX_A$,令為$L_2$,$x_{A_1}$
  \begin{itemize}
    \item Overall Mass Balance:
    \begin{equation}
      L_1 = L_2 + V,\quad V = dL
    \end{equation}
    蒸出量$V$等於液體減少量$dL$
    \item material Balance of $A$:
    \begin{equation}
      L\cdot x_A = \left(L-dL\right)\cdot\left(x_A-dx_A\right)+V\cdot y_A
    \end{equation}
    \item 化簡會得到,Rayleigh Equation
    \begin{equation}
      \int_{L_1}^{L_2}\frac{dL}{L} = \ln\frac{L_2}{L_1} = \int_{X_{A_1}}^{X_{A_2}}\frac{dX_A}{y_A-x_A}
    \end{equation}
    \item 定義Average Composition of total material distillated, $y_{av}$
    \begin{equation}
      y_{av} = \frac{L_1x_{A_1}-L_2x_{A_2}}{L_1-L_2}
    \end{equation}
    使得:
    \begin{equation}
      L_1 x_{A_1} = L_2 x_{A_2} = \left(L_2-L_1\right)\cdot y_{av}
    \end{equation}
    \item 解法1:\\
    由Rayleigh Equation,對平衡時的$y_A,x_A$實驗結果做($\frac{1}{y_A-x_A}-x_A$)的圖\\
    利用Simpson's rule或其他積分方法得到:
    \begin{equation}
      \int_{X_{A_1}}^{X_{A_2}}\frac{dX_A}{y_A-x_A} = \ln \frac{L_2}{L_1}
    \end{equation}
    \item 解法2:\\
    利用$\alpha_{AB}$,平衡線會是:
    \begin{equation}
      y_A = \frac{\alpha_{AB}x_A}{1+\left(\alpha_{AB}-1\right)x_A}\nonumber
    \end{equation}
    代入 Rayleigh Equation,得到:
    \begin{equation}
      \ln \frac{L_2}{L_1} = \frac{1}{\alpha_{AB}-1}\left[
        \ln\frac{x_{A_2}}{x_{A_1}} - \alpha_{AB}\cdot\ln\left(
          \frac{1-x_{A_2}}{1-x_{A_1}}
        \right)
      \right]
    \end{equation}
  \end{itemize}
  \item Continuous Distillation 連續蒸餾的計算範例\\
  假設一連續蒸餾裝置,進料$F$,含$x_F$,進入塔後,產生塔頂產物$V_1$,含$y_1$,塔底產物$W$,含$x_W$\\
  上方Enriching Section有n個板,下方Stripping Section有m個板\\
  也就是具有共計n個理論板在高於進料端,由最頂端的板開始計數\\
  也就是具有共計m個理論板在低於進料端,由靠近進料端的板開始計數\\
  上方Enriching Section的冷凝後物流會有一部分回流,為$L$,含$x_D$,另一部分為$D$,同樣含$x_D$,有著回流比$R$\\
  下方Stripping Section的經過再沸器將$W$,含$x_W$送出,並回收$y_W$
  \begin{itemize}
    \item 整體質量平衡:
    \begin{equation}
      F = D+W
    \end{equation}
    \item 輕成分的質量平衡:
    \begin{equation}
      F\cdot x_F = D\cdot y_D + W\cdot x_W
    \end{equation}
    \item 上方的Enriching Section的整體的質量平衡:
    \begin{equation}
      V_{n+1} = L_n + D
    \end{equation}
    P.S. 從進料端上面的板子是$n$,而在$1$板出去的物流為$V_1$故為$V_{n+1}$\\
    $L_n$則為從$n$板掉回來的物流,因為在$1$板掉回來的是$L_1$\\
    回流比的定義,$R$:
    \begin{equation}
      R = \frac{L_n}{D}
    \end{equation}
    \item 代入後得到Enriching Section的操作線:
    \begin{equation}
      y_{n+1} = \frac{R}{R+1}x_n + \frac{1}{R+1}x_D
    \end{equation}
    \item 當$n=0$時,$x_n=x_D$,代入可得到$y_1=x_D$,此為起點,開始作圖
    \begin{enumerate}
      \item 做($y-x$)圖,並繪製平衡線
      \item 在圖上再繪製一$45^\circ$的直線,在此直線上點出$(x_D,x_D)$也就是起點
      \item 過起點做一斜率為$\frac{R}{R+1}$的直線$M$
      \item 從起點開始,做一水平直線,交平衡線於$1$點,此點即為塔頂產物的組成($x_1,y_1$)
      \item 從$1$點開始,向下做一垂直線,交$M$後向左做一水平線,交平衡線於$2$點,此點即為第二層組成($x_2,y_2$)
      \item 重複上述步驟,直到最後一層$n$,即為入料端向上第一板的組成($x_n,y_n$)
    \end{enumerate}
    \item 下方的Stripping Section的整體的質量平衡:
    \begin{equation}
      V_{m+1} = L_m -W
    \end{equation}
    \item 代入輕成分的質量平衡,得到Stripping Section的操作線:
    \begin{equation}
      y_{m+1}=\frac{L_m}{V_{m+1}}x_m - \frac{W}{V_{m+1}}x_W
    \end{equation}
    \item 當$m=m$時(第二個$m$是總Stripping段板數),$x_m=x_W$,代入可得到$y_{m+1}=x_W$,此為起點,開始作圖
    \begin{enumerate}
      \item 做($y-x$)圖,並繪製平衡線
      \item 在圖上再繪製一$45^\circ$的直線,在此直線上點出$(x_W,x_W)$也就是起點
      \item 過起點做一斜率為$\frac{L_m}{V_{m+1}}$的直線$N$
      \item 從起點開始,向上做一垂直線,交平衡線於$(M)$點,此點即為塔底產物的組成($x_M,y_M$)
      \item 從$M$點開始,向右做一水平直線,交$N$後向上做一垂直線,交平衡線於$(M-1)$點,此點即為第二層組成($x_{M-1},y_{M-1}$)
      \item 重複上述步驟,直到第一層上方,$0$,即為入料端向下第一板的組成($x_1,y_1$)
    \end{enumerate}
    \item 進料端的操作線:\\
    定義$q$,蒸發1莫爾的進料組成需要花費的熱,比上進料組成的莫爾潛熱
    \begin{equation}
      q = \frac{H_V-H_F}{H_V-H_L}
    \end{equation}
    可以把這個比值想成是,想要掉下去的程度,也就是$F$進料後,會有$Fq$的液體掉下去,也就是$L_m$,$F(1-q)$的液體會被蒸發,也就是$V_n$
    \item 整體進料端的質量平衡:
    \begin{align}
      L_m &= L_n + qF\\
      V_n &= V_m + (1-q)F
    \end{align}
    \item 整體進料端的輕成分質量平衡,代回剛剛那兩個操作線:
    \begin{align}
      V_n y &= L_n x + Dx_D \\
      V_m y &= L_m x + Wx_W
    \end{align}
    \item 合併以上四條會得到Q-line的方程式
    \begin{equation}
      y = \frac{1}{q-1}x -\frac{x_F}{q-1}
    \end{equation}
    \item 當$x=x_F$時,$y=x_F$,此為起點,開始作圖
    \begin{enumerate}
      \item 做($y-x$)圖,並繪製平衡線
      \item 在圖上再繪製一$45^\circ$的直線,在此直線上點出$(x_F,x_F)$也就是起點
      \item 過起點做一斜率為$\frac{q}{q-1}$的直線$Q$
    \end{enumerate}
    \item Q-line的用途,不知道板子數時,找到到底該在哪裡放進料,一共要多少個板子
    \begin{enumerate}
      \item 做($y-x$)圖,並由實驗數據或$y=\frac{\alpha_{AB}x}{1+(\alpha_{AB}-1)x}$繪製平衡線
      \item 點出($x_D,x_D$),叫做$D$點
      \item 點出($x_W,x_W$),叫做$W$點
      \item 點出($x_F,x_F$),叫做$F$點
      \item 計算$q$值(往上與往下的理論分配量),並由$F$點做一斜率為$\frac{q}{q-1}$的直線,叫做$M$線
      \item 由$D$點做一斜率為$\frac{R}{R+1}$的直線,交$M$線於$O$點
      \item 連接$O$點和$W$點
      \item OWD三點形成一個三角形,包於平衡線內
      \item 由$D$點向右出發,每碰到一次平行線,就是一個板子,碰到後向下碰到三角形OWD後,向左碰到平行線,就是另一個板子
      \item 畫板子值到板子交點小於$x_W$
      \item 進料板的位置就是$M$線與平衡線的交點在哪兩個板子間
      \item 而向上需要多少板子,以及向下需要多少板子也會得出
    \end{enumerate}
    \item $q$的斜率特質:
    \begin{itemize}
      \item 若進料是過冷液體,$q>1$,$\frac{q}{q-1}$斜率$>1$,是朝右上方的斜直線
      \item 若進料是飽和液體,$q=1$,$\frac{q}{q-1}=\infty$,是向上的垂直線
      \item 若進料是氣液共存,$0<q<1$,$-\infty<\frac{q}{q-1}<0$,是向左上方的斜直線
      \item 若進料是飽和蒸汽,$q=0$,$\frac{q}{q-1}=0$,是向左的水平線
      \item 若進料是過熱蒸汽,$q<0$,$0<\frac{q}{q-1}<1$,是向左下方的斜直線
    \end{itemize}
  \end{itemize}
  \item Stripping-column Distillation 的設計:\\
  其實就是只有下半段的連續蒸餾,只有Stripping Section,沒有Enriching Section\\
  整體質量平衡
  \begin{equation}
    F = W + V_D
  \end{equation}
  輕成分的質量平衡
  \begin{equation}
    F\cdot x_F = W\cdot x_W + V_D\cdot y_D
  \end{equation}
  操作線
  \begin{equation}
    y_{m+1}= \frac{L_m}{V_{m+1}}x_m - \frac{W}{V_{m+1}}x_W
  \end{equation}
  Q-line:
  \begin{equation}
    y = \frac{1}{q-1}x - \frac{x_F}{q-1}
  \end{equation}
  \item Distillation with direct stream injection:\\
  連續蒸餾,但不使用Reboiler改為直接通入蒸氣$S$的純水($y_S=0$,蒸氣不含輕成分)\\
  質量平衡,及輕成分質量平衡
  \begin{equation}
    F + S = D + W,\quad  F\cdot x_F = D\cdot x_D + W\cdot x_W
  \end{equation}
\end{itemize}
\end{CJK*}
\end{document}