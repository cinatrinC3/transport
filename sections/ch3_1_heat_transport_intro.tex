\documentclass[../main.tex]{subfiles}
\begin{document}
\begin{CJK*}{UTF8}{bkai}
\subsection{熱量傳輸概論}
Same Analogy as Momentum Transport
\begin{figure}[H]
  \centering
  \begin{tikzpicture}[>=Latex, line cap=round, line join=round, thick]
    \draw[->] (-2,0) -- (-1,0) node[anchor=west] {$x$};
    \draw[->] (-2,0) -- (-2,1) node[anchor=south] {$y$};
    \draw (0,0) rectangle (5, -0.5);
    \draw[->] (5,-0.25) -- (6,-0.25) node[anchor=west] {$U$};
    \draw[->] (5,1) -- (5,2) node[anchor=south] {$\tau_{yx}$};
    \draw[dashed] (1,0) -- (1,3);
    \draw[blue] (4,0) .. controls (1,0) .. (1,0.5);
    \draw[blue] (4,0) .. controls (1.5, 0) and (1, 0.5) .. (1,1);
    \draw[blue] (4,0) .. controls (2, 0) and (1, 1) .. (1,2);
    \draw[blue] (4,0) .. controls (2.5, 0) and (1, 2) .. (1,3);
    \draw[->, red] (1,0) -- +(2,2) node[anchor=south west] {$t$};
    \node[anchor=north west, blue] at (1,3) {$u_x(y,t)$};
    \draw (0,3) rectangle (5, 3.5);
    \node[anchor=north] at (2.5, -0.5) {Momentum Flux};
    \node[anchor=north] at (2.5, -1) {$\tau_{yx}=-\mu\frac{du_x}{dy}\quad[(kg\cdot m/s)/m^2 s]$};
    \def\xShift{7.5}
    \draw (0+\xShift,0) -- (5+\xShift, -0);
    \fill[pattern=north east lines] (0+\xShift,0) rectangle (5+\xShift, -0.2);
    \draw (0+\xShift,3) -- (5+\xShift, 3);
    \fill[pattern=north east lines] (0+\xShift,3) rectangle (5+\xShift, 3.2);
    \draw[blue] (4+\xShift,0) -- (4+\xShift, 0.3) node [above] {$T_H$};
    \node[anchor=north east, blue] at(1+\xShift,3) {$T_C$};
    \draw[->] (0.5+\xShift,1) -- (0.5+\xShift,2) node[anchor=south] {$y$};
    \draw[dashed] (1+\xShift,0) -- (1+\xShift,3);
    \draw[blue] (4+\xShift,0) .. controls (1+\xShift,0) .. (1+\xShift,0.5);
    \draw[blue] (4+\xShift,0) .. controls (1.5+\xShift, 0) and (1+\xShift, 0.5) .. (1+\xShift,1);
    \draw[blue] (4+\xShift,0) .. controls (2+\xShift, 0) and (1+\xShift, 1) .. (1+\xShift,2);
    \draw[blue] (4+\xShift,0) .. controls (2.5+\xShift, 0) and (1+\xShift, 2) .. (1+\xShift,3);
    \draw[->, red] (1+\xShift,0) -- (2.5+\xShift,1.5);
    \draw[red] (4+\xShift,0) -- (1+\xShift,3) node[midway, anchor=south west] {$T(y)\text{ at S.S.}$};
    \node[anchor=south east, blue] at (1+\xShift,0) {$T(y,t)$};
    \node[anchor=north] at (2.5+\xShift, -0.5) {Energy Flux};
    \node[anchor=north] at (2.5+\xShift, -1) {$q_y=-k\frac{dT}{dy}\quad[J/m^2 s]$};
  \end{tikzpicture}
  \caption{Momentum Transport vs. Heat Transport Analogy}
\end{figure}
\begin{itemize}
  \item Govering Equation:
    \begin{itemize}
      \item Fourier 1st law of heat conduction (熱傳導)\\
        \fbox{傳導Flux},\fbox{Anaolgy to $\overline{\overline{\bm \tau}}$}\\
        不過這邊是\fbox{向量而不是張量}
        \begin{equation}
          \vec{\bm q} = -k\nabla T
        \end{equation}
        不過對於一些固體,各個方向的傳導效率不一樣\\
        甚至來自某些方向可能更好傳去某些方向等等\\
        會用$\overline{\overline{\bm \kappa}}$來表示熱導係數的張量形式
        \begin{equation}
          \vec{\bm q} = -\overline{\overline{\bm \kappa}}\cdot \nabla T
        \end{equation}
        對$q$的Order不變\\
        不過通常運算上只會在某$x$方向上的傳導(下標$k$表示傳導)
        \begin{equation}
          q_k=-kA\frac{dT}{dx}
        \end{equation}
        $q_k$是因為傳導,所產生的熱傳導速率,單位是$W$\\
        P.S. 會是下標$k$是來自德文的konduktion\\
        $k$是熱導係數,單位是$W/m\cdot K$\\
        $A$是傳導的面積,單位是$m^2$\\
        $\frac{dT}{dx}$是溫度梯度,單位是$K/m$\\
        $k$有些時候不是常數,而會跟溫度與壓力有關,但是主要影響還是物質特性\\
        若要對影響排序,則是: 物質特性$>$溫度$>$壓力(只有氣體會受壓力影響)\\
        另外在物質當中,又以分子間距離遠近為主要影響\\
        故\fbox{固體$>$液體$>$氣體}\\
        液體中:$k\propto T$\\
        而扯到氣體,因為氣體的熱傳主要靠分子間的碰撞\\
        所以會跟溫度(動能上升)和壓力(分子距離靠近)有比較大的關係
        \begin{equation}
          k\propto v = \frac{8RT}{\pi M} \propto \sqrt{\frac{T}{M}} \quad k\propto P
        \end{equation}
        所以如果是氣體的時候Fourier's law通常只能算穩態(溫度、壓力不變)下的熱傳\\
        另外在固體中,金屬$>$非金屬,因為金屬有\fbox{自由電子},所以熱傳導效率會比較好\\
        排序是: 銀$>$銅$>$金$>$鋁\\
        不過在固體中,非金屬的$k\propto T$\\
        而金屬則\fbox{有些是反比於溫度的},主要跟自由電子受溫度影響有關
        \begin{align}
          k\propto T & \quad \text{鋁、金、鉑、不鏽鋼} \nonumber\\
          k\propto \frac{1}{T} & \quad \text{銅、銀、鐵、鉛、鎳、鋅、錫} \nonumber
        \end{align}
      \item Convective transfer (熱對流)\\
        \fbox{對流Flux},\fbox{Anaolgy to $\rho \vec{\bm u}\vec{\bm u}$}\\
        代表由流體整體移動所帶來的熱量傳遞
        \begin{equation}
          \text{一坨流體}\cdot\text{用流速}\cdot \text{攜帶單位能量} = \rho \vec{\bm u} \hat E
        \end{equation}
        而整陀流體的能量($\rho \hat E \vec{\bm u}$)又可以分成動能和內能
        \begin{equation}
          \left(\frac{1}{2}\rho \left|\vec{\bm u}\right|^2 + \rho \hat U\right)\vec{\bm u} = \rho \hat E \vec{\bm u} \label{eq:convective_energy_flux}
        \end{equation}
        \begin{itemize}
        \item 發生條件:
          \begin{enumerate}
            \item 溫度差
            \item \fbox{流動的}連續介質,所以固體就沒有了
          \end{enumerate}
        \item 以流動方向做分類:
        \begin{figure}[H]
          \centering
          \begin{tikzpicture}[>=Latex, line cap=round, line join=round, thick]
            \draw (0,0) rectangle (4,3);
            \fill[pattern=north east lines, draw=black] (0,0) rectangle (4,0.3);
            \fill[pattern=north east lines, draw=black] (0,2.7) rectangle (4,3);
            \draw[->] (-2,1.5) -- (0,1.5) node[midway, above] {$v_\infty$};
            \draw[decorate, decoration={snake,amplitude=3pt,segment length=5pt}] (2,0.3) -- (2,1.05);
            \draw[-> ](2,1.05) -- (2,1.5) node[above] {$q_c$};
            \node[anchor=north west, red] at (2.5, 2.67) {$A$};
            \draw[red, dashed] (1.5,0.33) rectangle (2.5,2.67);
            \draw[->] (0.5,1.5) -- (1.5,1.5) node[midway, above] {$\dot H$};
            \node[anchor=north] at (1,1.5) {$\dot m$};
          \end{tikzpicture}
          \caption{熱對流方向示意圖}
        \end{figure}
        \begin{enumerate}
          \item 徑向熱對流 (Radial convection): 熱傳方向與流體流動方向垂直\\
          不過因為這些$\hat E$之類的在一團東西流動時很難取得\\
          不過因為這些$\hat E$之類的在一團東西流動時很難取得\\
          所以有牛頓冷卻定律 (Newton's Law of Cooling)
          \begin{equation}
            q_c = hA\left(T_s - T_\infty\right)
          \end{equation}
          $h$是Heat Transfer Coefficient,或是Film Coefficient\\
          $A$是熱傳方向的垂直面積\\
          只要知道這個$h$,就可以算出來熱對流的速率\\
          但實務上$h$很難獲得,而且\fbox{這個$h$不是物性},而是經驗值\\
          所以當用Governing Equation算熱對流時,\fbox{不會一開始就出現$h$}\\
          而是要先算出熱對流的速率,再反推$h$\\
          計算方式分成兩大類:
          \begin{itemize}
            \item 無因次數(Nusselt Number, $\text{Nu}$)法
            \begin{equation}
              \text{Nu} = \text{Nu}_N + \text{Nu}_F = 
              f(\text{Pr}, \text{Gr}, \frac{L}{D}) + f(\text{Pr}, \text{Re}, \frac{L}{D})
            \end{equation}
            \fbox{拆成強制對流與自然對流兩種情況}\\
            \fbox{適用於各種流動狀況}\\
            P.S. Re 和 Gr 不會在同一條式子裡同時出現\\
            但由於在流體各處這些無因次數會不同\\
            所以須訂一些\fbox{整體平均}的描述方式
            \renewcommand{\arraystretch}{2}
            \begin{table}[H]
              \centering
              \begin{tabular}{c|c|c}
                變數 & local & average \\
                \hline
                Heat Transfer Coefficient, $h$  & $h_x \propto x^{-\frac{1}{2}}=\frac{a}{\sqrt{x}}$ 
                & $\overline{h} = \frac{1}{L}\int_0^L h_x dx = 2\frac{a}{\sqrt{L}}$ \\
                Fanning Friction Factor, $C_f$ & 
                $C_{f,x} = \frac{\tau_{w,x}}{\frac{1}{2}\rho u_\infty^2}$ & 
                $\overline{C_{f}} = \frac{1}{L}\int_0^L C_{f,x} dx$ \\
                \hline
                Reynolds Number & $\text{Re}_x = \frac{\rho u_\infty x}{\mu}$ & 
                  $\text{Re}_L = \frac{\rho u_\infty L}{\mu}$ \\
                \hline
                Nusselt Number & $\text{Nu}_x = \frac{h_x x}{k_f}=\frac{a\sqrt{x}}{k_f}$ & 
                  $\overline{\text{Nu}_L} = \frac{\overline{h} L}{k_f} = 2\frac{a\sqrt{L}}{k_f}$ \\
                $\frac{\overline{\text{Nu}_L}}{\text{Nu}_L} = 2$ & & \\
              \end{tabular}
              \caption{熱傳經驗式的局部與整體平均}
            \end{table}
            \item 無因次數(Stanton Number, $\text{St}_H$)法
            \begin{equation}
              \text{St}_H = f(\text{Pr}, C_f)  = \frac{\text{Nu}}{\text{Re}\cdot \text{Pr}}
            \end{equation}
            \fbox{只能用於強制對流、直角坐標、圓柱座標$z$向流動}
          \end{itemize}
          又根據動力不同可分成
          \begin{itemize}
            \item 自然對流(Nature convection):\\
              又稱為Free convection\\
              流體原本不流動,但因流體內部溫度不同\\
              因而產生溫度差,溫度不同造成密度的不同,進而產生對流\\
              所以主要會跟浮力(Buoyant force)有關
            \item 強制對流(Forced convection):\\
              又稱為機械對流(Mechanical convection)\\
              由機械裝置,如泵浦、風扇、壓縮機,所造成的流動,進而產生對流
          \end{itemize}
          而強制對流的$h$會比自然對流的大很多\\
          以Nusselt Number進行計算時\\
          會根據Grashof Number(Gr) (\ref{eq:grashof_number})大小來選擇忽略自然對流與否
          \begin{equation}
            \text{Gr} = \frac{L^3\rho^2g\beta\left(T_s-T_\infty\right)}{\mu^2} =
            \frac{\text{Buoyancy Force}}{\text{Viscous Force}}
          \end{equation}
          若只考慮\fbox{自然對流,則Re可忽略},$\text{Nu}_N = f(\text{Pr},\text{Gr},\frac{L}{D})$\\
          若只考慮\fbox{強制對流,則Gr可忽略},$\text{Nu}_F = f(\text{Pr},\text{Re},\frac{L}{D})$\\
          因此,會將Nusselt Number分成自然對流和強制對流兩項評估:
          \begin{equation}
            \text{Nu} = \text{Nu}_N \pm \text{Nu}_F
          \end{equation}
          Richardson Number (Ri),則量化了自然對流和強制對流的相對重要性
          \begin{equation}
            \text{Ri} = \frac{\text{Gr}}{\text{Re}^2} = \frac{\text{Buoyancy Force}}{\text{Shear Force}}
          \end{equation}
          之所以會是平方,是因為浮力大小正比於密度差,但慣性力跟剪力都是正比於速度平方\\
          或單純,只是這樣訂數量級差距能在1附近
          \begin{equation}
            \begin{cases}
              \text{Ri} \gg 1 & \text{Nu} = \text{Nu}_N \\
              \text{Ri} \ll 1 & \text{Nu} = \text{Nu}_F \\
              \text{Ri} \approx 1 & \text{Nu} = \text{Nu}_N \pm \text{Nu}_F
            \end{cases}
          \end{equation}
          \item 當流體沒有流動時,熱對流會只剩下自然對流\\
          假想一顆熱球在空氣中降溫,只計算\fbox{空氣的熱傳導}是不夠精確的\\
          因為靠近球體的空氣會被加熱,密度變小而上升\\
          進而帶走熱量,所以會有自然對流的現象發生\\
          或者說,當流體沒流流動時,會額外假想它外圍一圈的氣體在流動\\
          因此定義Rayleigh Number (Ra)\\
          來量化自然對流相較熱傳導的影響(類似Peclet Number)
          \begin{equation}
            \text{Ra} = \text{Gr}\cdot \text{Pr} = \frac{L^3\rho^2g\beta\left(T_s-T_\infty\right)}{\mu^2}
            \cdot \frac{C_p \mu}{k} = \frac{\text{Buoyancy Force}}{\text{Thermal Diffusion}}
          \end{equation}\
          若$\text{Ra} > 10^6$,要考慮自然對流
          \item 軸向熱對流 (Axial convection): 熱傳方向與流體流動方向平行\\
          流體流走會帶走熱量,所以熱量會隨著流體流動而改變
          \begin{equation}
            \boxed{\dot H = \dot m C_p \left(T_\infty - T_\infty\right) 
            = \rho u A C_p \left(T_\infty - T_\infty\right)
            = \rho C_p \vec{\bm u}\cdot\nabla T}
          \end{equation}
          P.S. 可以看出,對流當中的軸向改變,是會併入Governing Equation中的\\
          因此$\vec{\bm q}_c$並不會出現在軸向熱對流的計算上\\
          所以其實有一個廣義的熱對流,包含軸向和徑向兩種熱對流\\
          以及一個狹義的熱對流,只有徑向熱對流,$h$和$q_c$都是指狹義的熱對流
        \end{enumerate}
      \end{itemize}
      \item 熱輻射(Heat radiation): 由電磁波的方式傳遞熱量\\
        只要有溫差就會發生\\
        而固體因為會擋住光,通常會忽略熱輻射\\
        使用Stefan-Boltzmann Law計算:
        \begin{equation}
          q_r(1\to 2) = A_1 F_{1\to 2} \sigma \left(T_1^4 - T_2^4\right) \label{eq:stefan_boltzmann_law}
        \end{equation}
        只會考定義,但不會考計算,另外溫度要代絕對溫度($K$)
        \begin{figure}[H]
          \centering
          \begin{tikzpicture}[>=Latex, line cap=round, line join=round, thick]
            \draw (0,0) -- (4,0);
            \fill[pattern=north east lines] (0,0) rectangle (4,-0.3);
            \node[anchor=south] at (3,0) {$T_2$(Low)};
            \draw (0,3) -- (3.5, 4) node[midway, below right] {$T_1$(High)};
            \fill [pattern=north east lines] (0,3) -- (3.5,4) -- (3.3,4.2) -- (-0.2,3.2) -- cycle;
            \node[anchor=north] at (2,-0.3) {Surface 2};
            \node[anchor=south east] at (1.45, 3.7) {Surface 1 ($A_1$)};
            \draw[red, ->] (1,3.1) -- (2, 0.5) node[midway, above right] {$q_r$};
            \draw[red, dashed,domain=225:270, smooth] plot ({3*cos(\x)}, {3+3*sin(\x)});
            \draw[red, dashed,domain=276:315, smooth] plot ({3.5+4.0311*cos(\x)}, {4+4.0311*sin(\x)});
          \end{tikzpicture}
          \caption{熱輻射示意圖}
          \label{fig:heat_radiation}
        \end{figure}
        其中$F_{1\to 2}$是View Factor或稱為Shape Factor\\
        指的是1號面積所發出的光,到達2表面積的比例\\
        所以如果是球殼內部給球殼外部,那就是1了\\
        跟接觸者的面積有約略正比的關係\\
        有Reciprocity Relation:
        \begin{equation}
          A_1 F_{1\to 2} = A_2 F_{2\to 1}
        \end{equation}
        而$\sigma$是Stefan-Boltzmann constatnt=$5.67\times 10^{-8}\quad [W/m^2K^4]$\\
        同熱對流遇到的問題$q_r$也是只能用在巨觀的統計\\
        但不同的事,確實是可以把這項塞入Governing Equation中的熱平衡\\
        \fbox{對流之所以不行,是因為他混了物質運動和能量傳遞兩種機制}\\
        可是輻射純粹是能量傳遞,所以可以直接塞入Governing Equation裡。\\
        但是是沒有意義的,因為對每一個微小控制體積\\
        都要去計算$\nabla \cdot \vec{\bm q}_r$,實務上不會這樣做\\
        而且物體內部的輻射,大概會變成實驗測量的熱傳係數上的誤差\\
        所以通常還是以巨觀的熱平衡來處理
        \begin{equation}
          -k\frac{\partial T}{\partial n}\big|_{\text{surface}} = h\left(T_s - T_\infty\right) + \varepsilon \sigma \left(T_s^4 - T_\infty^4\right)
        \end{equation}
        $n$是法向量方向\\
        $\varepsilon$是Emissivity,代表表面積對黑體的輻射效率\\
        黑體的$\varepsilon=1$,而一般物體$0<\varepsilon<1$\\
        主要跟表面積的材質、顏色、粗糙度有關\\
        黑色、粗糙的表面積,會比較接近黑體\\
        而光滑、銀色的表面積,會比較接近鏡面反射\\
        但當然你還是能把他們全部併入$h$,所以才說$h$很雜\\
        P.S. Rosseland Model假設在流體非常稠(光學無法透視)的情況下\\
        熱輻射可以近似成熱傳導的一種形式
        \begin{equation}
          \vec{\bm q}_r \approx -\frac{16\sigma T^3}{3k_{\text{abs}}}\nabla T
        \end{equation}
        $k_{\text{abs}}$是Absorption Coefficient,代表流體對輻射的吸收能力\\
        這樣就可以把熱輻射也塞入Governing Equation中了
      \item 將上面三種$q_k,q_c,q_r$加在一起,針對這三個$q$的\fbox{巨觀能量守恆}\\
      會將\fbox{熱傳導放在$E_{\text{in}},~E_{\text{out}}$中}\\
      而\fbox{熱對流和熱輻射放在$E_{\text{gen}}$中},因為這兩個是\fbox{單向}的能量傳遞\\
      只會從控制體積中流出或流入,但這也是方便而已,不用糾結為什麼對流是$E_{\text{gen}}$\\
      另外,如果有電熱、化學反應等熱源產生,合理也會放在$E_{\text{gen}}$中
      \item Work associated with molecular motion,\fbox{分子運動作功Flux}\\
        定義一個控制體積對各方向的作用力是:
        \begin{equation}
          \overline{\overline{\bm \pi}} = P{\bm \delta\bm \delta} + \overline{\overline{\bm \tau}}
        \end{equation}
        其實就是一個單位向量$I$乘上壓力,變成對角矩陣,再加上黏性應力張量\\
        來源於做功的定義,一坨流體移動了一個距離\\
        分子運動作功的Flux:
        \begin{equation}
          \left[\overline{\overline{\bm \pi}} \cdot \vec{\bm u}\right]
        \end{equation}
      \item 定義Combined energy flux,$\vec{\bm e}$:
        \begin{equation}
          \vec{\bm e} = \rho \hat E \vec{\bm u} + \overline{\overline{\bm \pi}} \cdot \vec{\bm u}  + \vec{\bm q}
          \label{eq:energy_flux_definition}
        \end{equation}
        注意我們使用$\vec{\bm q}$,因為並不是所有熱傳導都符合Fourier's Law
      \item Equation of Energy\\
        由上述總能量和,如同動量傳輸的推導,對任一個控制體積:
        \begin{equation}
          \underbrace{-\iint_S(\vec {\bm e}\cdot \vec{\bm n}) dS}_{E_{\text{in}}-E_{\text{out}}} 
          + \underbrace{\iiint_V \rho \left(\vec{\bm v} \cdot \vec{\bm g}\right)dV}_{\text{整陀東西被作功}=E_{\text{gen}}} 
          = \underbrace{\frac{d}{dt}\iiint_V \rho \hat E dV}_{E_\text{acc}}
        \end{equation}
        P.S. 當使用Shell balance思考時,同介紹$q$時所說\\
        熱對流和熱輻射是單向的能量傳遞,會放在$E_{\text{gen}}$中\\
        而流入與流出也會再加上$\dot H$,也就是來自周遭流體給予的能量\\
        因為根據題目不同很可能會把流場的傳遞、來自熱源的輸入等等簡併為焓\\
        根據高斯定理,將表面積分轉成體積積分:
        \begin{equation}
          \underbrace{- \nabla \cdot \vec{\bm e}}_{\text{能量進出}} + 
          \underbrace{\rho \left(\vec{\bm v} \cdot \vec{\bm g}\right)}_{\text{整陀東西被作功}} = 
          \underbrace{\frac{\partial}{\partial t} (\rho \hat E)}_{\text{能量變化}} 
        \end{equation}
        將$\vec{\bm e}$用(\ref{eq:energy_flux_definition})式代入:
        \begin{equation}
          \underbrace{\frac{\partial}{\partial t} (\rho \hat E)}_{\text{能量變化}} = \underbrace{- 
          \overbrace{\nabla \cdot \left(\rho \hat E \vec{\bm u}\right)}^{\text{整陀流走}}  - 
          \overbrace{\nabla \cdot \left(\overline{\overline{\bm \pi}} \cdot \vec{\bm u}\right)}^{\text{分子運動作功}} -
          \overbrace{\nabla \cdot \vec{\bm q}}^{\text{熱傳導}}
          }_{\text{能量進出}} + \underbrace{\rho \left(\vec{\bm v} \cdot \vec{\bm g}\right)}_{\text{整陀東西被作功}}
        \end{equation}
        同樣的,因為在參考座標不動的偏微分觀察那個\fbox{整陀流走}\\
        包含了\fbox{他的慣性和外加的變化},所以改為實時導數\\
        利用以下性質:
        \begin{equation}
          \nabla \cdot s \vec{\bm u} = \nabla s \cdot \vec{\bm u} + s \left(\nabla \cdot \vec{\bm u}\right)
        \end{equation}
        展開$\nabla \cdot \left(\rho \hat E \vec{\bm u}\right)$:
        \begin{align}
          & {\color{blue}\frac{\partial}{\partial t} (\rho \hat E)} = - {\color{red}\nabla \cdot \left(
            \rho \hat E \vec{\bm u}\right)}  - \nabla \cdot \left(\overline{\overline{\bm \pi}} \cdot \vec{\bm u}\right)
             - \nabla \cdot \vec{\bm q} + \rho \left(\vec{\bm v} \cdot \vec{\bm g}\right) \nonumber\\
          \implies & 
          {\color{blue}\rho \frac{\partial \hat E}{\partial t} + \hat E \frac{\partial \rho}{\partial t} } +
          {\color{red}\nabla\hat E \cdot \rho \vec{\bm u} + \hat E \left(\nabla \cdot \rho \vec{\bm u}\right)}  = 
           - \nabla \cdot \vec{\bm q} 
          - \nabla \cdot \left[\left(\cancel{
            P\bm \delta\bm \delta} + \overline{\overline{\bm \tau}}
          \right) \cdot \vec{\bm u}\right]
          + \rho \left(\vec{\bm v} \cdot \vec{\bm g}\right) \nonumber\\
          &
          \underbrace{\rho \frac{\partial \hat E}{\partial t}}_{\text{開始那陀能量變化}}
          + \underbrace{\nabla\hat E \cdot \rho \vec{\bm u}}_{\text{開始那陀慣性流走}}
          + \underbrace{\hat E \frac{\partial \rho}{\partial t}}_{\text{額外質量變化}}
          + \underbrace{\hat E \left(\nabla \cdot \rho \vec{\bm u}\right)}_{\text{額外質量流走}} \nonumber\\
          &\quad\quad = 
           - \underbrace{\nabla \cdot \vec{\bm q}}_{\text{熱傳導}}
           - \underbrace{\nabla \cdot\left(\overline{\overline{\bm \tau}} \cdot \vec{\bm u}\right)}_{\text{分子運動作功}} 
           + \underbrace{\rho \left(\vec{\bm v} \cdot \vec{\bm g}\right)}_{\text{整陀東西被作功}}
        \end{align}
        P.S. $\bm \delta\bm \delta$,會變成單位矩陣$I$哦\\
        所以$\nabla \cdot I = 0$,沒有散度\\
        所以壓力才沒有梯度\\
        把$\hat E$丟到左邊,並利用Equation of Continuity:\\
        質量變化等於質量流入流出差:
        \begin{equation}
          \frac{\partial \rho}{\partial t} + \nabla \cdot \rho \vec{\bm u} = 0
        \end{equation}
        消除掉兩項後,前兩項以實時導數合併:
        \begin{align}
          &\rho \left(
            \frac{\partial \hat E}{\partial t} + \vec{\bm u} \cdot \nabla \hat E
          \right) + \hat E \left(
            \cancel{\frac{\partial \rho}{\partial t} + \nabla \cdot \rho \vec{\bm u}}
          \right) = 
           - \nabla \cdot \vec{\bm q}
           - \nabla \cdot\left(\overline{\overline{\bm \tau}} \cdot \vec{\bm u}
           \right) + \rho \left(\vec{\bm v} \cdot \vec{\bm g}\right) \nonumber\\
           &
           \boxed{\underbrace{\rho \frac{D\hat E}{Dt}}_{\text{能量變化}} = 
           - \underbrace{\nabla \cdot \vec{\bm q}}_{\text{熱傳導}}
           - \underbrace{\nabla \cdot\left(\overline{\overline{\bm \tau}} \cdot \vec{\bm u} \right)}_{\text{分子運動作功}}
          + \underbrace{\rho \left(\vec{\bm v} \cdot \vec{\bm g}\right)}_{\text{整陀東西被作功}}} \label{eq:total_energy_equation}
        \end{align}
        將$\hat E$展開成動能和內能項(\ref{eq:convective_energy_flux})式:
        \begin{equation}
          \hat E = \underbrace{\left(\frac{1}{2}\rho\left|\vec{\bm u}\right|^2\right)\vec{\bm u}}_{\text{動能}} 
          + \underbrace{\left(\rho \hat U\right)\vec{\bm u}}_{\text{內能}} \label{eq:total_energy_decomposition}
        \end{equation}
        而動能又能透過(\ref{eq:mechanical_energy_equation})式,展開
        \begin{equation}
          \boxed{\underbrace{\rho \frac{D}{Dt}\left(\frac{1}{2}\left|\vec{\bm u}\right|^2\right)}_{\text{動能變化}} = 
          - \underbrace{\vec{\bm u} \cdot \nabla P}_{\text{可逆的動能轉內能}} 
          - \underbrace{\nabla \cdot \left(\overline{\overline{\bm \tau}} \cdot \vec{\bm u}\right)}_{\text{黏力做功}}
          - \underbrace{\left[- \overline{\overline{\bm \tau}} : \left(\nabla \vec{\bm u}\right)\right]}_{\text{不可逆的動能轉內能}}
          \underbrace{+ \rho \left(\vec{\bm u} \cdot \vec{\bm g}\right)}_{\text{外力做功}}} \label{eq:kinetic_energy_equation}
        \end{equation}
        P.S. 不可逆的動能轉內能,可以看做\fbox{力$\times$形變速率}\\
        代表為了抵抗控制體積的形變("耗散"),所必須耗損的能量\\
        但因為這樣$\hat E$,包含太多項了\\
        我們將包含所有能量的(\ref{eq:total_energy_equation}),減去代表動能的(\ref{eq:kinetic_energy_equation})式\\
        就能得到只包含內能的能量方程式:
        \begin{equation}
          \boxed{\underbrace{\rho \frac{D\hat U}{Dt}}_{\text{內能變化}} = 
          - \underbrace{\nabla \cdot \vec{\bm q}}_{\text{熱傳導}} 
          - \underbrace{P  \nabla \cdot \vec{\bm u}}_{\text{可逆的動能轉內能}} 
          - \underbrace{\overline{\overline{\bm \tau}} : \left(\nabla \vec{\bm u}\right)}_{\text{不可逆的動能轉內能}}}
        \end{equation}
        P.S. 這邊$\overline{\overline{\bm \tau}} : \left(\nabla \vec{\bm u}\right)$的正負跟動能的不一樣哦\\
        畢竟是內能動能互轉的關係\\
        (沒有耗散能+內能,因為耗散會以熱或形變轉為內能,只是無法利用而已)\\
        而耗散能,也就會是物化中的Entropy Production
        \begin{equation}
          \frac{\partial}{\partial t}\rho\hat S= -\left(
            \nabla \cdot (\rho \hat S \vec{\bm u})
          \right) - \left(
            \nabla \cdot \frac{\vec{\bm q}}{T}
          \right) - \frac{1}{T^2}\left(
            \vec{\bm q} \cdot \nabla T
          \right) - \frac{1}{T}\left(
            \overline{\overline{\bm \tau}} : \nabla \vec{\bm u}
          \right)
        \end{equation}
        P.S. 右邊第一項跟流體慣性有關\\
        第二、三項是熱傳導造成的可以用可逆方式轉回的能量$Q=T\Delta S$\\
        第四項則是黏性應力造成的不可逆耗散能\\
        不過大部分時候,我們不會想以等體積的描述,$U, C_V$來計算\\
        而可逆的動能轉內能,與內能變化相加,正是\fbox{焓的變化}\\
        $\hat H = \hat U + P\hat V = \hat U + \frac{P}{\rho}$(P.S. 單位質量體積就是密度)\\
        但因為左邊已經是實時導數,所以透過將左邊展開,來跟右邊的壓力項消去:
        \begin{equation}
          \hat U = \hat H - \frac{P}{\rho} \implies \rho \frac{D\hat U}{Dt} 
          = \rho \frac{D\hat H}{Dt} - \rho \frac{D}{Dt}\left(\frac{P}{\rho}\right)
        \end{equation}
        代入後得到:
        \begin{align}
          \rho \frac{D\hat H}{Dt} -\rho \frac{D}{Dt}\left(\frac{P}{\rho}\right) &= 
          -\nabla \cdot \vec{\bm q} - P \nabla \cdot \vec{\bm u}
           - \overline{\overline{\bm \tau}} : \left(\nabla \vec{\bm u}\right) \nonumber\\
           \rho \frac{D\hat H}{Dt} -\frac{D P}{Dt} + \frac{P}{\rho}\frac{D \rho}{Dt} &=
           -\nabla \cdot \vec{\bm q} - P \nabla \cdot \vec{\bm u}
           - \overline{\overline{\bm \tau}} : \left(\nabla \vec{\bm u}\right)
        \end{align}
        利用Equation of Continuity: 
        \begin{align}
          \frac{\partial \rho}{\partial t} + \nabla \cdot( \rho \vec{\bm u}) &= 0\nonumber\\
          \underbrace{\frac{\partial \rho}{\partial t} + \vec{\bm u} \cdot \nabla \rho}_{\frac{D \rho}{Dt}} 
          + \rho \left(\nabla \cdot \vec{\bm u}\right) &= 0
        \end{align}
        所以:
        \begin{equation}
          \frac{D \rho}{Dt} = - \rho \left(\nabla \cdot \vec{\bm u}\right)
        \end{equation}
        代入後得到:
        \begin{align}
          & \rho\frac{D\hat H}{Dt} - \frac{D P}{Dt} 
          + \frac{P}{\cancel{\rho}}\left(-\cancel{\rho} \nabla \cdot \vec{\bm u}\right) = 
          -\nabla \cdot \vec{\bm q} 
          - P \nabla \cdot \vec{\bm u}
          - \overline{\overline{\bm \tau}} : \left(\nabla \vec{\bm u}\right) \nonumber\\
          & \rho\frac{D\hat H}{Dt} - \frac{D P}{Dt} 
          -\cancel{ P \nabla \cdot \vec{\bm u}} = 
          -\nabla \cdot \vec{\bm q} 
          -\cancel{ P \nabla \cdot \vec{\bm u}}
          - \overline{\overline{\bm \tau}} : \left(\nabla \vec{\bm u}\right) \nonumber\\
          \implies \quad &\boxed{\rho \frac{D\hat H}{Dt} = -\nabla \cdot \vec{\bm q} + \frac{D P}{Dt} 
          - \overline{\overline{\bm \tau}} : \left(\nabla \vec{\bm u}\right)} \label{eq:energy_equation_intermediate}
        \end{align}
        但大部分時候,考慮熱傳問題時\\
        我們會希望能以溫度來表示能量方程式\\
        因為能假設固定熱容,固定熱傳導率等等\\
        為了要把式子變成溫度的形式,需利用熱力學關係式\\
        利用物化的$\hat H(T,P)$,換掉$\rho \frac{D\hat H}{Dt}$:
        \begin{align}
          \hat H(T,P) \implies d\hat H &= \left(
            \frac{\partial \hat H}{\partial T}
          \right)_P dT + \left(
            \frac{\partial \hat H}{\partial P}
          \right)_T dP \nonumber\\
          &= \hat C_p dT + \left[
            \hat V - T\left(
              \frac{\partial \hat V}{\partial T}
            \right)_P
          \right] dP \nonumber\\
          &= \hat C_p dT + \left[
            \frac{1}{\rho} - T\left(
              \frac{\partial }{\partial T}\left(\frac{1}{\rho}\right)
            \right)_P
          \right] dP
        \end{align}
        取實時導數:
        \begin{align}
          \rho \frac{D\hat H}{Dt} &= \rho \hat C_p \frac{DT}{Dt} + \rho \left[
            \frac{1}{\rho} - T\left(
              \frac{\partial }{\partial T}\left(\frac{1}{\rho}\right)
            \right)_P
          \right] \frac{D P}{Dt} \nonumber\\
          &= \rho \hat C_p \frac{DT}{Dt} + \left[
            1 - \rho T\left(
              \frac{\partial }{\partial T}\left(\frac{1}{\rho}\right)
            \right)_P
          \right] \frac{D P}{Dt} \nonumber\\
          &= \rho \hat C_p \frac{DT}{Dt} + \left[
            1 - \left(\frac{T}{\rho}\right)\left(
              \frac{\partial \rho}{\partial T}
            \right)_P
          \right] \frac{D P}{Dt} \nonumber\\
          &= \rho \hat C_p \frac{DT}{Dt} + \left[
            1 - \left(
              \frac{\partial \ln \rho}{\partial \ln T}
            \right)_P
          \right] \frac{D P}{Dt}
        \end{align}
        代入(\ref{eq:energy_equation_intermediate})式:
        \begin{equation}
          \rho\hat C_p \frac{DT}{Dt} + \left[
            \cancel{1} - \left(
              \frac{\partial \ln \rho}{\partial \ln T}
            \right)_P
          \right] \frac{D P}{Dt} = -\nabla \cdot \vec{\bm q} + \cancel{\frac{D P}{Dt} }
          - \overline{\overline{\bm \tau}} : \left(\nabla \vec{\bm u}\right)  
        \end{equation}
        會得到最終移除力學能項,以溫度表達的能量方程式:
        \begin{equation}
          \boxed{\rho \hat C_p \frac{DT}{Dt} = -\nabla \cdot \vec{\bm q} 
          - \overline{\overline{\bm \tau}} : \left(\nabla \vec{\bm u}\right)
          + \left(
            \frac{\partial \ln \rho}{\partial \ln T}
          \right)_P\frac{DP}{Dt}}
        \end{equation}
        P.S. 如果是理想氣體,$\left(\frac{\partial \ln \rho}{\partial \ln T}\right)_P=-1$\\
        至此,我們透過將能量(位能、動能、耗散)間加減\\
        以及將內能換為以溫度表示的方式,列出了數條能量方程式\\
        將其餘能量平衡式表列如下:
        \begin{table}[H]
          \centering
          \begin{tabular}{c|c|c}
            \hline
            描述能量 & 能量方程式  & 備註\\
            \hline
            $\hat E=\hat K + \hat U + \hat \Phi$ &
            $\underbrace{\rho \frac{D\hat E}{Dt}}_{\text{能量變化}} = 
           - \underbrace{\nabla \cdot \vec{\bm q}}_{\text{熱傳導}}
           - \underbrace{\nabla \cdot\left(\overline{\overline{\bm \tau}} \cdot \vec{\bm u} \right)}_{\text{分子運動作功}}
          + \underbrace{\rho \left(\vec{\bm v} \cdot \vec{\bm g}\right)}_{\text{整陀東西被作功}}$ & 通用\\
          \hline 
            $\hat U$ & $\underbrace{\rho \frac{D\hat U}{Dt}}_{\text{內能變化}} = 
          - \underbrace{\nabla \cdot \vec{\bm q}}_{\text{熱傳導}} 
          - \underbrace{P  \nabla \cdot \vec{\bm u}}_{\text{可逆動能耗損}} 
          - \underbrace{\overline{\overline{\bm \tau}} : 
            \left(\nabla \vec{\bm u}\right)}_{\text{不可逆動能耗損}}$ & 密度固定\\
          \hline
            $\hat H$ & $\underbrace{\rho \frac{D\hat H}{Dt}}_{\text{焓變化}} = 
          - \underbrace{\nabla \cdot \vec{\bm q}}_{\text{熱傳導}} + \underbrace{\frac{D P}{Dt}}_{\text{壓力變化}} 
          - \underbrace{\overline{\overline{\bm \tau}} : \left(\nabla \vec{\bm u}\right)}_{\text{不可逆動能耗損}}$ & 密度可變\\
          \hline
            $\hat K$ & $\underbrace{\rho \frac{D}{Dt}\left(\frac{1}{2}\left|\vec{\bm u}\right|^2\right)}_{\text{動能變化}} = 
          - \underbrace{\vec{\bm u} \cdot \nabla P}_{\text{可逆動能轉內能}} 
          - \underbrace{\vec{\bm u} \cdot \left[
            \nabla \cdot \overline{\overline{\bm \tau}}
          \right]}_{\text{黏力做功}}
          \underbrace{+ \rho \left(\vec{\bm u} \cdot \vec{\bm g}\right)}_{\text{外力做功}}$ & 動量守恆\\
          \hline
            $\hat K+\hat U$ &
            $\underbrace{\rho \frac{D}{Dt}\left(\frac{1}{2}\left|\vec{\bm u}\right|^2 + \hat U\right)}_{\text{機械能變化}} = 
          - \underbrace{\nabla \cdot \vec{\bm q}}_{\text{熱傳導}}
          - \underbrace{\nabla \cdot (P \vec{\bm u})}_{\text{可逆動能轉內能}}
          - \underbrace{\overline{\overline{\bm \tau}} : \left(\nabla \vec{\bm u}\right)}_{\text{不可逆動能轉內能}}
          + \underbrace{\rho \left(\vec{\bm u} \cdot \vec{\bm g}\right)}_{\text{外力做功}}$ & 無耗散能\\
          \hline
          $\hat C_P, T$ &
          $\rho \hat C_p \frac{DT}{Dt} = -\nabla \cdot \vec{\bm q} 
          - \overline{\overline{\bm \tau}} : \left(\nabla \vec{\bm u}\right)
          + \left(
            \frac{\partial \ln \rho}{\partial \ln T}
          \right)_P\frac{DP}{Dt}$ & 熱力學\\
          \hline
          $\hat C_V, T$ &
          $\rho \hat C_v \frac{DT}{Dt} = -\nabla \cdot \vec{\bm q} 
          - \overline{\overline{\bm \tau}} : \left(\nabla \vec{\bm u}\right)
          - T\left(
            \frac{\partial P}{\partial T}
          \right)_\rho\left(\nabla \cdot \vec{\bm u}\right)$ &
          熱力學\\
          \hline
          $\hat S$ &
          $\frac{\partial}{\partial t}\rho\hat S= -\left(
            \nabla \cdot (\rho \hat S \vec{\bm u})
          \right) - \left(
            \nabla \cdot \frac{\vec{\bm q}}{T}
          \right) - \frac{1}{T^2}\left(
            \vec{\bm q} \cdot \nabla T
          \right) - \frac{1}{T}\left(
            \overline{\overline{\bm \tau}} : \nabla \vec{\bm u}
          \right)$ & 熵產生\\
          \hline
          \end{tabular}
          \caption{各種能量方程式總整理}
          \label{tab:various_energy_equations}
        \end{table}
      \item Equation of Energy的簡化形式\\
        根據不同的假設條件,可以將上式進一步簡化,常見的假設條件有:
        \begin{enumerate}
          \item Fourier Law:
          \begin{equation}
            - \nabla \cdot \vec{\bm q} = k\nabla^2 T 
          \end{equation}
          \item Newtonian Fluid:
          \begin{equation}
            \overline{\overline{\bm \tau}} : \left(\nabla \vec{\bm u}\right) = \mu \bm \Phi_u + \kappa \bm \Phi_b
          \end{equation}
          進一步Newtonian + Incompressible:
          \begin{equation}
            \overline{\overline{\bm \tau}} : \left(\nabla \vec{\bm u}\right) = \mu \bm \Phi_u
          \end{equation}
          再進一步,計算熱傳時,通常以溫度為主要影響,會忽略它
          \item 定壓:
          \begin{equation}
            \frac{D P}{D t} = 0
          \end{equation}
          \item 定密度:
          \begin{equation}
            \left(
              \frac{\partial \ln \rho}{\partial \ln T}
            \right)_P\frac{D P}{D t} = 0
          \end{equation}
          \item 一塊靜止的固體或液體
          \begin{equation}
            \vec{\bm u} = 0 \implies \overline{\overline{\bm \tau}} : \left(\nabla \vec{\bm u}\right) = 0,\frac{DT}{Dt} 
            = \frac{\partial T}{\partial t}
          \end{equation}
        \end{enumerate}
        各種狀況下的衍生型式:
        \begin{itemize}
          \item 對固體或液體,$k$是常數
            \begin{equation}
              \rho C_p\frac{DT}{Dt} = \rho C_p \left(\frac{\partial T}{\partial t}+\vec {\bm u} \cdot \nabla T\right) = k\nabla^2 T + \dot q
            \end{equation}
            \item 對固體或液體,但$k$不是常數
            \begin{equation}
              \rho C_p \left(\frac{\partial T}{\partial t}+\vec {\bm u} \cdot \nabla T\right) = \nabla\cdot(k\nabla T) + \dot q
            \end{equation}
            \item 對固體或液體,沒有熱源提供($\dot q=0$),但$k$不是常數:
            \begin{equation}
              \rho C_p \left(\frac{\partial T}{\partial t}+\vec {\bm u} \cdot \nabla T\right) = k\nabla^2 T
            \end{equation}
            \item 對固體或液體,$k$是常數,系統內流體沒有流動$\vec {\bm u}=0$
            \begin{equation}
              \rho C_p \frac{\partial T}{\partial t} = k\nabla^2 T + \dot q
            \end{equation}
            \item 對固體或液體,$k$是常數,系統內流體沒有流動$\vec {\bm u}=0$,沒有熱源提供($\dot q=0$)
              \begin{equation}
                \rho C_p \frac{\partial T}{\partial t} = k\nabla^2 T
              \end{equation}
              此為傅立葉第二定律,及其滿足條件\\
              移項可得到:
              \begin{equation}
                \frac{\partial T}{\partial t} = \frac{k}{\rho C_p}\nabla^2 T
              \end{equation}
              P.S. 從此式也定義了熱擴散係數$\alpha = \frac{k}{\rho C_p}$
            \item 沒有流體流動,穩態:
              \begin{equation}
                \boxed{\nabla^2T + \frac{\dot q}{k} = 0}
              \end{equation}
              此為: Poisson Equation
            \item 沒有流體流動,穩態,沒有熱源提供($\dot q=0$):
              \begin{equation}
                \boxed{\nabla^2T = 0}
              \end{equation}
              此為: Laplace Equation
        \end{itemize}
    \end{itemize}
  \item 常用的邊界條件:
    \begin{itemize}
      \item Dirichlet Boundary Condition: 指定溫度
      \begin{equation}
        T = T_{\text{wall}}
      \end{equation}
      \item Neumann Boundary Condition: 指定熱通量\\
      例如在牆壁有穩定熱源提供
      \begin{equation}
        q = q_{\text{wall}}
      \end{equation}
      \item 在物體表面時,熱通量連續
      \begin{equation}
        T_1 = T_2, \quad q_1 = q_2
      \end{equation}
    \end{itemize}
  \item 名詞定義:
  \begin{itemize}
    \item Thermal Resistance 熱阻:
    \begin{equation}
      R = \frac{L}{kA}
    \end{equation}
    $L$是傳導距離,單位是$m$\\
    $k$是熱導係數,單位是$W/m\cdot K$\\
    $A$是傳導的面積,單位是$m^2$\\
    熱阻的單位是$K/W$
    \item 熱導效率是,溫差的驅動力除以熱阻
    \begin{equation}
      q = \frac{\Delta T}{R} = \frac{\text{Driving Force}}{\text{Resistance}}
    \end{equation}
    \item 熱擴散係數($\alpha$):
    \begin{equation}
      \alpha = \frac{k}{\rho \hat{C}_p}
    \end{equation}
  \end{itemize}
  \item 熱量不同傳輸方式下的熱阻
  \begin{itemize}
    \item 傳導(Conduction): 物質內部的熱量傳遞\\
    只要有溫差,連續的介質,非真空,就會有熱傳導
    \begin{itemize}
      \item 熱傳導的串並聯:\\
      將熱阻想像成電阻,電壓想成溫度差,$q$想成電流\\
      使用通用型分子傳遞方程式:
      \begin{equation}
        \text{Transfer Rate} = \frac{\text{Driving Force}}{\text{Resistance}}
      \end{equation}
      或者就說是歐姆定律就行了\\
      P.S. 所以巨觀的流體力學也有,只是比較少用到:
      \begin{equation}
        Q = \frac{\Delta P}{R_f}
      \end{equation}
      $R_f$是流體阻力\\
      換算到熱傳導:
      \begin{equation}
        q_k = \frac{\Delta T}{R_k}
      \end{equation}
      P.S. \fbox{阻抗一定要是正的}\\
      熱阻的串並聯關係:
      \begin{itemize}
        \item 串聯:
        \begin{equation}
          \begin{cases}
            q = q_1 = q_2 = q_3 = \cdots = q_n\\
            R = R_1 + R_2 + R_3 + \cdots + R_n
          \end{cases}
        \end{equation}
        \item 並聯:
        \begin{equation}
          \begin{cases}
            q = q_1 + q_2 + q_3 + \cdots + q_n\\
            \frac{1}{R} = \frac{1}{R_1} + \frac{1}{R_2} + \frac{1}{R_3} + \cdots + \frac{1}{R_n}
          \end{cases}
        \end{equation}
      \end{itemize}
      \item 平板的熱傳導:\\
      假設一個厚度$L$的板子,面積$A$,$x$方向左端溫度為$T_1$,右端溫度為$T_2$
      \begin{figure}[H]
        \centering
        \begin{tikzpicture}[>=Latex, line cap=round, line join=round, thick]
          \draw (0,0) rectangle (3,5);
          \draw[<->] (0, 0.2) -- (3, 0.2) node[midway, above] {$L$};
          \draw (0,0) -- (0,-0.5);
          \draw [->] (0,-0.5) -- (1,-0.5) node[right] {$x$};
          \node[anchor=north] at (0, -0.5) {$x=0$};
          \node[anchor=north] at (3, -0.5) {$x=L$};
          \node[blue, anchor=north] at (2,0) {$dx$};
          \fill[pattern=north east lines, pattern color=blue, draw=black] (1.8,0) rectangle (2.2,5);
          \fill[red, draw=black] (0,2.2) -- (2.5,2.2) -- (2.5,1.8) -- (3, 2.5) -- (2.5,3.3) -- (2.5,2.8) -- (0,2.8) -- cycle;
          \node[red, anchor=north] at(1.25, 2.2) {$+q_k$};
          \node[red, anchor=east] at (0, 4) {$T_1$};
          \node[red, anchor=west] at (3, 1) {$T_2$};
          \draw[dashed, red] (0,4) -- (3,1);
          \node[anchor=south] at (1.5, 5) {$T_1>T_2$};
          \draw[dashed] (0,5) -- +(1.5,1.5);
          \draw[dashed] (3,5) -- +(1.5,1.5);
          \draw[dashed] (3,0) -- +(1.5,1.5);
          \node at (3,2.5,-2) {$A$};
        \end{tikzpicture}
        \caption{平板熱傳導示意圖}
      \end{figure}
      \begin{itemize}
        \item 傅立葉第一定律:
        \begin{equation}
          q_k = -kA\frac{\Delta T}{\Delta x} 
        \end{equation}
        \item 假設是Steady State,則$q_k=\text{constant}$,積分:
        \begin{equation}
          \int_0^L q_k dx = -\int_{T_1}^{T_2} kA dT
        \end{equation}
        \item 板子,$kA$是常數
        \begin{equation} 
          q_k = \frac{kA}{L}\left(T_1-T_2\right)
        \end{equation}
        \item 故平板的熱導熱阻$R_k$:
        \begin{equation}
          R_k = \frac{L}{kA} \label{eq:thermal_resistance_flat_plate}
        \end{equation}
      \end{itemize}
      \item 圓柱的熱傳導:\\
      假設一個管子(空心圓柱),內徑$r_i$,外徑$r_o$,長度$L$,管內外溫度分別為$T_i$和$T_o$
      \begin{figure}
        \centering
        \begin{tikzpicture}[>=Latex, line cap=round, line join=round, thick]
          \draw (0,0) circle (1.5);
          \draw (0,0) circle (3);
          \fill[pattern=crosshatch dots, pattern color=blue] (2.7,0) arc (0:180:2.7) -- (-2.4,0) arc (180:0:2.4) -- cycle;
          \fill[pattern=crosshatch dots, pattern color=blue] (2.7,0) arc (0:-180:2.7) -- (-2.4,0) arc (-180:0:2.4) -- cycle;
          \draw [dashed, blue] (0,0) circle (2.7);
          \draw [dashed, blue] (0,0) circle (2.4);
          \draw[<->] (0,0) -- (-1.5,0) node[midway, above] {$r_i$};
          \draw[<->] (0,0) -- (45:3);
          \node[anchor=south east] at (45:2.1) {$r_o$};
          \node[anchor=south west] at (45:3) {$T_o$};
          \node[anchor=east] at (-1.5, 0) {$T_i$};
          \fill[red, draw=black] (1.5,-0.2) -- (2.5, -0.2) -- (2.5,-0.4) -- (3,0) -- (2.5,0.4) -- (2.5,0.2) -- (1.5,0.2) -- cycle;
          \node[red, anchor=south] at (1.75, 0.2) {$+q_k$};
          \node[anchor=north] at (0,-3) {$T_i>T_o$};
          \draw[->] (225:3.2) -- (225:2.7);
          \draw[->] (225:1.9) -- (225:2.4);
          \node[anchor=north east] at (225:3.2) {$dr$};
        \end{tikzpicture}
        \caption{圓柱熱傳導示意圖}
      \end{figure}
      \begin{itemize}
        \item 傅立葉第一定律,設控制體積為厚度$dr$的薄圓柱,側表面積為$2\pi rL$:
        \begin{equation}
          q_k = -k(2\pi rL)\frac{dT}{dr}
        \end{equation}
        \item 假設是Steady State,則$q_k=\text{constant}$、圓管,$k(2\pi rL)$是常數,積分:
        \begin{equation}
          q_k\int_{r_i}^{r_o} dr = -k(2\pi L)\int_{T_i}^{T_o} dT
        \end{equation}
        \item 得到$q_k$:
        \begin{equation}
          q_k = \frac{2\pi kL}{\ln\left(\frac{r_o}{r_i}\right)}\left(T_i-T_o\right)
        \end{equation}
        \item 故圓管的熱導熱阻$R_k$為
        \begin{equation}
          R_k = \frac{\ln\left(\frac{r_o}{r_i}\right)}{2\pi kL}
        \end{equation}
      \end{itemize}
      \item 球殼的熱傳導:\\
        假設一個球體,內半徑$r_i$,外半徑$r_o$,球體的內外溫度分別為$T_i$和$T_o$\\
        圖片同上,只是把圓柱改成球體
        \begin{itemize}
          \item 傅立葉第一定律,設控制體積為厚度$dr$的薄球殼,側表面積為$4\pi r^2$:
            \begin{equation}
              q_k = -k(4\pi r^2)\frac{dT}{dr}
            \end{equation}
          \item 假設是Steady State,則$q_k=\text{constant}$、球體,$k(4\pi r^2)$是常數,積分:
            \begin{equation}
              q_k\int_{r_i}^{r_o} dr = -k(4\pi)\int_{T_i}^{T_o} dT
            \end{equation}
          \item 得到$q_k$:
            \begin{equation}
              q_k = \frac{4\pi k r_or_i}{r_o-r_i}\left(T_i-T_o\right)
            \end{equation}
          \item 故球體的熱導熱阻$R_k$為
            \begin{equation}
              R_k = \frac{r_o-r_i}{4\pi k r_or_i}
            \end{equation}
        \end{itemize}
    \end{itemize}
    \item 對流(Convection):對流熱阻$R_c$:
      \begin{equation}
        R_c = \frac{1}{hA} \label{eq:thermal_resistance_convection}
      \end{equation}
    \item 輻射(Heat Radiation): 由電磁波的方式傳遞熱量\\
      由(\ref{eq:stefan_boltzmann_law})
      \begin{equation}
        q_r(1\to 2) = A_1 F_{1\to 2} \sigma \left(T_1^4 - T_2^4\right)
      \end{equation}
      試圖以牛頓冷卻定律的形式來表示輻射熱流:
      \begin{equation}
        q_r = h_r A_1 \Delta T
      \end{equation}
      透過將兩式相等,來求出$h_r$\\
      而熱阻$R_r$就會是$\frac{1}{h_r A_1}$\\
      故:
      \begin{equation}
        q_r = h_r A_1 \Delta T = A_1 F_{1\to 2} \sigma \left(T_1^4 - T_2^4\right)
      \end{equation}
      整理後:
      \begin{equation}
        h_r = \frac{F_{1\to 2} \sigma \left(T_1^4 - T_2^4\right)}{T_1 - T_2}
      \end{equation}
      故輻射熱阻$R_r$:
      \begin{equation}
        \boxed{R_r = \frac{1}{A_1 F_{1\to 2} \sigma \frac{T_1^4 - T_2^4}{T_1 - T_2}}} \label{eq:thermal_resistance_radiation}
      \end{equation}
      P.S. 這個任何座標都可以用
      用\fbox{並聯}的方式結合起來:
      \begin{equation}
        q_{\text{total}} = q_c + q_r = \frac{\Delta T}{R_c} + \frac{\Delta T}{R_r} = \frac{\Delta T}{R_{\text{total}}}
      \end{equation}
      所以總熱阻$R_{\text{total}}$:
      \begin{equation}
        \frac{1}{R_{\text{total}}} = \frac{1}{R_c} + \frac{1}{R_r}
      \end{equation}
      代入後:
      \begin{equation}
        \boxed{q_{\text{total}} = \Delta T hA + A_1 F_{1\to 2} \sigma\left( T_1^4 - T_2^4\right)}
      \end{equation}
    \item  相變化熱可以用\fbox{串聯}的方式與熱傳結合起來\\
      因為熱傳需要$\Delta T$,相變化熱需$\Delta T=0$\\
      所以熱傳與相變化不可能在同一個空間下發生\\
      而在串聯時,要注意單位必須相同\\
      $q_k$的單位會是$J/s$(積分面積後),而相變化熱的單位會是$J$\\
      故除上時間後,即可相加,使用$\dot H$表示相變化熱流率\\
      例如熔化速率、汽化速率等\\
      假設汽化熱為$\Delta H_{vap}$,質量流率為$\dot m$:
      \begin{equation}
        q_{\text{total}} = \frac{\Delta T}{R_{\text{total}}} \pm \dot H_{vap} = \frac{\Delta T}{R_{\text{total}}} 
        \pm \dot m \Delta {\hat H}_{vap}
      \end{equation}
      $\hat H$是單位質量的相變化熱\\
      不過這裡要注意正負號的關係,看相變化在熱傳上是吸熱或放熱而有所不同\\
      例如,如果是固體熔化,這件事會吸熱,所以是要減去熔化熱的
    \item 電熱與熱傳也是\fbox{串聯}的方式結合起來\\
      因為電熱是熱源,會提供額外的熱量給系統\\
      假設電熱功率為$q_e$:
      \begin{equation}
        q_{\text{total}} = \frac{\Delta T}{R_{\text{total}}} + q_e
      \end{equation}
      至於$q_e$單位也會是$J/s$,也就是功率\\
      至於功率要怎麼算,就高中的物理:
      \begin{equation}
        q_e = IV = \frac{V^2}{R} = I^2 R
      \end{equation}
    \item 絕熱材料的臨界半徑(Critical thickness)\\
      絕熱材料是用來減少熱傳的材料\\
      但是在圓柱或球體上包覆絕熱材料時,會有一個臨界半徑的概念\\
      當小於臨界半徑時,包覆絕熱材料反而會增加熱傳\\
      直覺上來說,一個物體原本很細,熱對流就不快了\\
      但你去增加它的半徑,增加了表面積,反而就增加熱傳了\\
      臨界半徑公式:
      \begin{itemize}
        \item 圓柱:
          \begin{equation}
            r_{\text{crit}} = \frac{k}{h}
          \end{equation}
        \item 球體:
          \begin{equation}
            r_{\text{crit}} = \frac{2k}{h}
          \end{equation}
      \end{itemize}
      圓柱的推導:
      \begin{itemize}
        \item 在包覆絕緣材料前
        \begin{figure}[H]
          \centering
          \begin{circuitikz}[american]
            \draw (0,0) to [R, l=$R_c$, o-o] (2,0);
            \node[anchor=east] at (0,0) {$T_i$};
            \node[anchor=west] at (2,0) {$T_o$};
            \draw[->] (0,-1) -- (2,-1) node[midway, above] {$q_c$};
          \end{circuitikz}
          \caption{包覆前熱阻等效電路}
        \end{figure}
        等效熱阻:
        \begin{equation}
          R_{\text{total}} = R_c = \frac{1}{h(2\pi r_i L)}
        \end{equation}
        熱傳功率:
        \begin{equation}
          q_c = \frac{T_i - T_o}{R_{\text{total}}} = h(2\pi r_i L)(T_i - T_o)
        \end{equation}
        \item 包覆絕緣材料後\\
        假設包完之後的外半徑為$r_o$
        \begin{figure}[H]
          \centering
          \begin{circuitikz}[american]
            \draw (0,0) to [R, l=$R_k$, o-o] (2,0) to [R, l=$R_c$, o-o] (4,0);
            \node[anchor=east] at (0,0) {$T_i$};
            \node[anchor=south] at (2,0) {$T_m$};
            \node[anchor=west] at (4,0) {$T_o$};
            \draw[->] (0,-1) -- (4,-1) node[midway, above] {$q_c$};
          \end{circuitikz}
          \caption{包覆後熱阻等效電路}
        \end{figure}
        等效熱阻:
        \begin{equation}
          R_{\text{total}} = R_k + R_c = \frac{\ln\left(\frac{r_o}{r_i}\right)}{2\pi kL} + \frac{1}{h(2\pi r_o L)}
        \end{equation}
        熱傳功率:
        \begin{equation}
          q_c = \frac{T_i - T_o}{R_{\text{total}}} = \frac{2\pi kL (T_i - T_o)}{\ln\left(\frac{r_o}{r_i}\right) + \frac{k}{h r_o}}
        \end{equation}
      \item 找臨界半徑\\
        欲求臨界半徑,及將包覆後減去包覆前的熱傳功率,並對$r_o$微分,尋找極值
        \begin{equation}
          \frac{d}{dr_o}\left[q_{c,\text{after}} - q_{c,\text{before}}\right] = 0
        \end{equation}
        但因為包覆前的熱傳功率與$r_o$無關,所以只需對包覆後的微分
        \begin{align}
          \frac{d}{dr_o}\left[\frac{2\pi kL (T_i - T_o)}{\ln\left(\frac{r_o}{r_i}\right) + \frac{k}{h r_o}}\right] &=
          2\pi kL (T_i - T_o) \frac{d}{dr_o}\left[\frac{1}{\ln\left(\frac{r_o}{r_i}\right) + \frac{k}{h r_o}}\right] \nonumber\\
          &= -2\pi kL (T_i - T_o) \frac{\frac{1}{r_o} - \frac{k}{h r_o^2}}{\left[\ln\left(\frac{r_o}{r_i}\right) + \frac{k}{h r_o}\right]^2}
          = 0
        \end{align}
        令分子為0,解出:
        \begin{align}
          \frac{1}{r_o} - \frac{k}{h r_o^2} &= 0 \nonumber\\
          r_o &= \boxed{\frac{k}{h} = r_{\text{crit}}}
        \end{align}
        \item P.S. 其實還須證明,極值發生時,$q_{c,\text{after}} - q_{c,\text{before}}$是最大值\\
        故需對$q$做二次微分:
        \begin{align}
          \frac{d^2}{dr_o^2}\left[q_{c,\text{after}} - q_{c,\text{before}}\right]&=
          -2\pi kL (T_i - T_o) \frac{d}{dr_o}\left[\frac{\frac{1}{r_o} 
          - \frac{k}{h r_o^2}}{\left[\ln\left(\frac{r_o}{r_i}\right) + \frac{k}{h r_o}\right]^2}\right] \nonumber\\
          &=-2\pi kL (T_i - T_o) \frac{-\frac{1}{r_o^2} + \frac{2k}{h r_o^3}}{\left[\ln\left(\frac{r_o}{r_i}\right) + \frac{k}{h r_o}\right]^2} \nonumber\\
          &\quad + 2\pi kL (T_i - T_o) \frac{\left(\frac{1}{r_o} - \frac{k}{h r_o^2}\right) 
          \left(\frac{1}{r_o} - \frac{k}{h r_o^2}\right)}{\left[\ln\left(\frac{r_o}{r_i}\right) + \frac{k}{h r_o}\right]^3}
        \end{align}
        化簡後得到:
        \begin{equation}
          \frac{d^2}{dr_o^2}\left[q_{c,\text{after}} - q_{c,\text{before}}\right] =
          -2\pi kL (T_i - T_o) \frac{-\frac{1}{r_o^2} + \frac{2k}{h r_o^3}}{\left[\ln\left(\frac{r_o}{r_i}\right) + \frac{k}{h r_o}\right]^2}
        \end{equation}
        代入$r_o = \frac{k}{h}$:
        \begin{align}
          \frac{d^2}{dr_o^2}\left[q_{c,\text{after}} - q_{c,\text{before}}\right] &=
          -2\pi kL (T_i - T_o) \frac{-\frac{1}{\left(\frac{k}{h}\right)^2} + 
          \frac{2k}{h \left(\frac{k}{h}\right)^3}}{\left[\ln\left(\frac{\frac{k}{h}}{r_i}\right) + 1\right]^2} \nonumber\\
          &=-2\pi kL (T_i - T_o) \frac{\frac{1}{\left(\frac{k}{h}\right)^2}}{\left[\ln\left(\frac{\frac{k}{h}}{r_i}\right) + 1\right]^2} < 0
        \end{align}
        因此確定為最大值
      \end{itemize}
  \end{itemize}
  \item 一些熱量傳輸的經驗式:\\
    $f(\text{Re},\text{Pr},\frac{L}{D})$函數,題目會給\\
    $\text{St}_H$是$f(\text{Pr},C_f)$的函數,$C_f$是范寧摩擦因子
    \begin{itemize}
      \item 流體流過平板: (見\ref{eq:graetz_nusselt_number_final})
        \begin{equation}
          \overline{\text{Nu}}_L = 0.664\text{Re}_L^{0.5}\text{Pr}^{1/3}
        \end{equation}
      \item 層流流過長直圓管,其壁面溫度均一(泡熱水)
        \begin{equation}
          \overline{\text{Nu}}_L = 3.66 + f(\text{Re},\text{Pr},\frac{L}{D})
        \end{equation}
        見(\ref{eq:leveque_entry_length}),當$L\gg D,\quad \frac{L}{D}>40$\\
        可忽略$f(\text{Re},\text{Pr},\frac{L}{D})$項
        \begin{equation}
          \overline{\text{Nu}}_L = 3.66
        \end{equation}
      \item 層流流過長直圓管,給予均勻熱通量(\fbox{火焰加熱煮水})
        \begin{equation}
          \overline{\text{Nu}}_L = \frac{48}{11} + f(\text{Re},\text{Pr},\frac{L}{D})
        \end{equation}
        同樣,當$L\gg D,\quad \frac{L}{D}>40$,可忽略$f(\text{Re},\text{Pr},\frac{L}{D})$項
        \begin{equation}
          \boxed{\overline{\text{Nu}}_L = \frac{48}{11} \approx 4.36}
        \end{equation}
      \item Plug Flow,流過長直圓管,給予均勻熱通量(\fbox{火焰加熱煮空氣})
        \begin{equation}
          \overline{\text{Nu}}_L = 8 + f(\text{Re},\text{Pr},\frac{L}{D})
        \end{equation}
        同樣,當$L\gg D,\quad \frac{L}{D}>40$,可忽略$f(\text{Re},\text{Pr},\frac{L}{D})$項
        \begin{equation}
          \boxed{\overline{\text{Nu}}_L = 8}
        \end{equation}
      \item 流過球面的流:
        \begin{equation}
          \overline{\text{Nu}}_L = 2 + f(\text{Re},\text{Pr})
        \end{equation}
        同樣,當流體是\fbox{Creeping flow},$\text{Re}\ll 1$,可忽略$f(\text{Re},\text{Pr})$項
        \begin{equation}
          \boxed{\overline{\text{Nu}}_L = 2}
        \end{equation}
      \item Reynolds Analogy:(適用於$Pr=1$,No Form drag)\\
        P.S. 也也是因為要No form drag所所以只能是直角坐標系
        \begin{equation}
          \frac{h}{\rho C_p u_\infty} = \boxed{\text{St}_H = \frac{C_f}{2}}
        \end{equation}
      \item Colburn Analogy:(適用於$0.5<\text{Pr}<50$,No Form drag)
        \begin{equation}
          j_H = \left(\frac{\text{Nu}}{\text{Re}\text{Pr}}\right)\cdot \text{Pr}^{\frac{2}{3}} = \frac{C_f}{2}
        \end{equation}
        $j_H$是Colburn $j$ factor
        而且
        \begin{equation}
          \boxed{\text{St}_H = \frac{h}{\rho C_p u_\infty} = j_H \text{Pr}^{-\frac{2}{3}}}
        \end{equation}
      \item Prandtl Analogy:(適用於忽略Drag)
        \begin{equation}
          \text{St}_H = \frac{\frac{C_f}{2}}{1+5\sqrt{\frac{C_f}{2}}\left(\text{Pr}-1\right)} = \frac{h}{\rho C_p u}
        \end{equation}
      \item Von Karman Analogy:(適用於忽略Drag)
        \begin{equation}
          \text{St}_H = \frac{\frac{C_f}{2}}{1+5\sqrt{\frac{C_f}{2}\left\{
            \text{Pr}-1+\ln\left[1+\frac{5}{6}\left(\text{Pr}-1\right)\right]
          \right\}}} = \frac{h}{\rho C_p u}
        \end{equation}
  \end{itemize}
\end{itemize}
\end{CJK*}
\end{document}