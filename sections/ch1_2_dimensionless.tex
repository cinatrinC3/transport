\documentclass[../main.tex]{subfiles}
\begin{document}
\begin{CJK*}{UTF8}{bkai}
\subsection{輸送現象,定義與無因次群}
\begin{itemize}
\item 擴散係數:
\begin{itemize}
  \item 流力擴散係數,(Kinematic viscosity, momentum diffusivity):
  \begin{equation}
    \nu = \frac{\mu}{\rho} = \frac{\text{動量擴散}}{\text{質量}},quad [=] \mathrm{m^2/s}
  \end{equation}
  \item 熱擴散係數 (Thermal diffusivity):
  \begin{equation}
    \alpha = \frac{k}{\rho C_P} = \frac{\text{熱傳擴散}}{\text{質量}},\quad [=] \mathrm{m^2/s}
  \end{equation}
  \item 質量擴散係數 (Mass diffusivity):
  \begin{equation}
    D_{AB} = \frac{\text{質量擴散}}{\text{質量}} ,\quad [=] \mathrm{m^2/s}
  \end{equation}
  \item 流力/熱傳 $\implies$ Prandtl Number (Pr)
  \begin{equation}
    \text{Pr} = \frac{\nu}{\alpha} = \frac{C_P \mu}{k} = \frac{\text{Momentum Diffusivity}}{\text{Thermal Diffusivity}}
  \end{equation}
  \item 流力/質傳 $\implies$ Schmidt Number (Sc)
  \begin{equation}
    \text{Sc} = \frac{\nu}{D_{AB}} = \frac{\mu}{\rho D_{AB}} = \frac{\text{Momentum Diffusivity}}{\text{Mass Diffusivity}}
  \end{equation}
  \item 熱傳/質傳 $\implies$ Lewis Number (Le)
  \begin{equation}
    \text{Le} = \frac{\alpha}{D_{AB}} = \frac{k}{\rho C_P D_{AB}} = \frac{\text{Thermal Diffusivity}}{\text{Mass Diffusivity}}
  \end{equation}
\end{itemize}
\item 無因次群:\\
其實解題還是可以用無因次分析去得到一些無因次群\\
只是這樣很麻煩,所以要記一些有物理意義、有名字的無因次群\\
質傳還會有,見表以快速從表目錄查詢:\\
故意寫成不約分的形式,以方便與物理意義對應\\
很多地方的特徵長度可能會要求是水力直徑$D_h$、特徵流速是平均流速$\left<u\right>$
\renewcommand{\arraystretch}{2}
\begin{table}[H]
  \centering
  \begin{tabular}{c|c|c|c|c}
    名稱 & 符號 & 定義 & 意義 & 判定/用途\\
    \hline
    Reynolds Number & Re & $\frac{\rho u^2/L}{\mu u/L^2}$
      & $\frac{\text{慣性力}}{\text{黏滯力}}$& $\uparrow$慣性力主導\\
    Euler Number & Eu & $\frac{P}{\rho u^2}$ 
      & $\frac{\text{壓力力}}{\text{慣性力}}$& \\
    Prandtl Number & Pr & $\frac{\nu}{\alpha},\left(\frac{\delta}{\delta_t}\right)^3$ 
      & $\frac{\text{動量擴散}}{\text{熱傳擴散}}$ & $\uparrow$無熱邊界層\\
    Nusselt Number & Nu & $\frac{hL}{k}$
      & $\frac{\text{徑向熱對流}}{\text{徑向熱傳導}}$& 由$k$導$h$\\
    Peclet Number (Heat) & $\text{Pe}_H$ & $\frac{\rho \hat C_P u\Delta T/L}{k\Delta T/L^2}$
      & $\frac{\text{軸向熱對流}}{\text{軸向熱傳導}}$& $\uparrow$忽略軸向熱傳導\\
    Grashof Number & Gr & $\frac{L^3\rho^2g\beta\left(T_s-T_\infty\right)}{\mu^2}$
      & $\frac{\text{浮力}}{\text{黏滯力}}$ & $\downarrow$強制對流 \\
    Richardson Number & Ri & $\frac{\text{Gr}}{\text{Re}^2}$
      & {\tiny 讓Gr與Re比較} & $\downarrow$強制-1-自然$\uparrow$ \\
    Rayleigh Number & Ra & $\text{Gr}\cdot \text{Pr}$
      & $\frac{\text{自然對流}}{\text{熱傳導}}$ & $\downarrow$忽略-$10^6$-自然\\
    Froude Number & Fr & $\frac{\rho u^2/L}{\rho g}$ 
      & $\frac{\text{慣性力}}{\text{重力}}$&\\
    Brinkman Number & Br & $\frac{\mu (u/L)^2}{k\Delta T/L^2}$ 
      & $\frac{\text{黏滯耗散}}{\text{熱傳導}}$&\\
    Schmidt Number & Sc & $\frac{\nu}{D_{AB}}$
      & $\frac{\text{動量擴散}}{\text{質量擴散}}$&\\
    Lewis Number & Le & $\frac{\alpha}{D_{AB}}$
     & $\frac{\text{熱傳擴散}}{\text{質量擴散}}$&\\
    Sherwood Number & Sh & $\frac{h_\infty L}{D_{AB}}$
      &&\\
    Biot Number & Bi & $\frac{1/k_S}{1/h\left(\frac{V}{A}\right)}$
      & $\frac{\text{內部阻力}}{\text{外部阻力}}$ & $\downarrow$均-0.1-異$\uparrow$ \\
    Fourier Number & Fo & $\frac{kA\Delta T/L}{\rho V C_P \Delta T/t}$
      & $\frac{\text{熱傳導}}{\text{熱儲存速率}\dot H}$&\\
    Stanton Number (Heat) & $\text{St}_H$ & $\frac{\text{Nu}}{\text{Re}\cdot\text{Pr}}$
      &&
  \end{tabular}
  \caption{Heat and Mass Transfer Dimensionless Groups}
\end{table}
\begin{itemize}
  \item Prandtl Number (Pr): \fbox{流傳和熱傳}的比例
    \begin{equation}
      \text{Pr} = \frac{\nu}{\alpha} = \frac{C_p\mu}{k} = \frac{\text{Momentum Diffusivity}}{\text{Thermal Diffusivity}}
    \end{equation}
    或是\fbox{速度邊界層厚度和熱傳邊界層厚度的比例}
    \begin{equation}
      \text{Pr} = \left(\frac{\delta}{\delta_t}\right)^3 = \left(\frac{\text{Momentum Boundary Layer}}{\text{Thermal Boundary Layer}}\right)^3
    \end{equation}
    用於熱對流經驗式:
    \begin{equation}
      \begin{cases}
        \text{Pr} > 1 & \delta > \delta_t, \text{液體, 水2.8} \\
        \text{Pr} < 1 & \delta < \delta_t, \text{氣體, 空氣0.686} \\
        \text{Pr} = 1 & \delta = \delta_t, \text{超臨界流體,水蒸氣}\\
        \text{Pr} \approx 0 & \delta \approx 0, \text{液態金屬,汞}
      \end{cases}
    \end{equation}
  \item Schmidt Number:
    \begin{equation}
      \text{Sc} = \frac{\nu}{D_{AB}} = \frac{\mu}{\rho D_{AB}} = \frac{\text{Momentum Diffusivity}}{\text{Mass Diffusivity}}
    \end{equation}
  \item Lewis Number:
    \begin{equation}
      \text{Le} = \frac{\alpha}{D_{AB}} = \frac{k}{\rho C_p D_{AB}} = \frac{\text{Thermal Diffusivity}}{\text{Mass Diffusivity}}
    \end{equation}
  \item Nusselt Number: \fbox{徑向熱傳和徑向熱傳}的比例\\
    或者說\fbox{對流和傳導}的比例\\
    在物理意義上來自在邊界上,徑向的熱傳導會轉換為熱對流繼續傳遞\\
    而目的是透過傳導來間接求出,影響因素複雜的$h$的經驗式
    \begin{equation}
      \boxed{q_k\big|_{y=0} = q_c} \implies 
      \boxed{-k_fA\frac{\partial T}{\partial y}\bigg|_{y=0} = hA\left(T_s - T_\infty\right)}
    \end{equation}
    因此Nusselt Number可寫成:
    \begin{equation}
      \text{Nu} =
      \frac{\left.\frac{\partial \left(T_s -T\right)}{\partial y}\right|_{y=0}}{\frac{\left(T_s-T_\infty\right)}{L}} 
      = \frac{hL}{k} = \frac{\text{Radius Heat Convection Rate}}{\text{Radius Heat Conduction Rate}}
    \end{equation}
    P.S. 不要和Biot Number搞混了,特徵長度是橫向的\\
    若系統為圓柱或球:
    \begin{equation}
      \text{Nu}_L = \frac{hD}{k}
    \end{equation}
    P.S. 因為這裡可以用直徑當特徵長度,而直徑和卡氏用的$L$是不同的意義\\
    為了避免錯誤比較與混淆,故加上下標$L$表示用直徑當特徵長度
  \item Sherwood Number:
    \begin{equation}
      \text{Sh} = \frac{h_\infty L}{D_{AB}}
    \end{equation}
  \item Biot Number:\\
    應用於固體或靜止流體內的熱傳導,系統內是不流動的物質,而外部有流動流體
    \begin{equation}
      \text{Bi} = \frac{h\left(\frac{V}{A}\right)}{k_S} = \frac{\text{Internal Resistance}}{\text{External Resistance}}
    \end{equation}
    $k_S$是\fbox{固體}的熱傳導係數\\
    可由Bi值判斷\fbox{系統內部是否有熱傳}存在(固體內部溫度是否均一)\\
    \fbox{不要和Nusselt Number搞混了},這裡根本沒有用特徵長度,而是特徵體積/表面積\\
    如果小於0.1,代表內部熱傳可以忽略,溫度可視為均一
  \item Peclet Number (Heat):\fbox{軸向熱對流和軸向熱傳導}的比例
    \begin{equation}
      \boxed{\text{Pe}_H =\text{Re}\cdot\text{Pr} = \frac{\rho C_p\nu D}{k} 
      = \frac{\rho C_p \vec{\bm u}\cdot \nabla T}{k\nabla^2 T}
      = \frac{\text{Heat Convection Rate}}{\text{Heat Conduction Rate}}}
    \end{equation}
    若$\text{Pe}_H$很大,代表軸向上的熱傳導可以被忽略
  \item Grashof Number:\\
    重力對熱對流的影響,\fbox{自然對流比較大}時用
    \begin{equation}
      \text{Gr} = \frac{L^3\rho^2g\beta\left(T_s-T_\infty\right)}{\mu^2} 
      = \frac{\text{Buoyancy Force}}{\text{Viscous Force}} \label{eq:grashof_number}
    \end{equation}
    $\beta$是熱膨脹係數
    \begin{equation}
      \boxed{\beta = \frac{1}{V}\left(
        \frac{\partial V}{\partial T}
      \right)
      = -\frac{1}{\rho}\left(
        \frac{\partial \rho}{\partial T}
      \right)}, \quad [=] \mathrm{K^{-1}}
    \end{equation}
    代表溫度改變單位體積的體積變化率\\
    P.S. 理想氣體$\beta = \frac{1}{T}$\\
    分子會有$\rho^2$,其實來自於$L^3 \rho g \times \Delta \rho$\\
    也就是物體所受淨浮力,會是\fbox{排開液體重$\times$密度差}\\
    當物體也是同樣液體時(對流),驅動力就從定義密度差($\Delta \rho$)\\
    改為溫度差$\times$體積變化靈敏度($\rho \beta \Delta T$)\\
    另外,若只考慮\fbox{自然對流,則Re可忽略},$\text{Nu}_N = f(\text{Pr},\text{Gr},\frac{L}{D})$\\
    若只考慮\fbox{強制對流,則Gr可忽略},$\text{Nu}_F = f(\text{Pr},\text{Re},\frac{L}{D})$
  \item Stanton Number:
    \begin{equation}
      \text{St}_H = \frac{\text{Nu}}{\text{Re}\cdot\text{Pr}} = \frac{h}{\rho C_p u}
    \end{equation}
    是Dimensionless的整體熱傳係數(包含熱對流和熱傳導)
  \end{itemize}
\end{itemize}
\end{CJK*}
\end{document}