\documentclass[../main.tex]{subfiles}
\begin{document}
\begin{CJK*}{UTF8}{bkai}
\subsection{動量輸送的介紹}
\begin{itemize}
  \item 影響流速的四個力:
  \begin{itemize}
    \item Inertial Force (慣性力): 使流體繼續保持原本的流動狀態\\
    實際上為Inertial Stress
    \begin{equation}
      \text{Inertial Force(慣性力)} = \rho \vec {\bm u} \vec {\bm u}
    \end{equation}
    $\vec {\bm u} \vec {\bm u}$為二階張量,有9個分量:
    \begin{equation}
      \rho \vec {\bm u} \vec {\bm u} = \begin{bmatrix}
        \rho u_x u_x & \rho u_x u_y & \rho u_x u_z \\
        \rho u_y u_x & \rho u_y u_y & \rho u_y u_z \\
        \rho u_z u_x & \rho u_z u_y & \rho u_z u_z
      \end{bmatrix} \label{eq:inertial_stress_tensor}
    \end{equation}
    或者寫為單位向量展開形式:
    \begin{align}
      \rho \vec {\bm u} \vec {\bm u} =& \phantom{+}\rho u_x u_x \bm \delta_x \bm \delta_x + \rho u_x u_y \bm \delta_x \bm \delta_y + \rho u_x u_z \bm \delta_x \bm \delta_z \nonumber \\
      &+ \rho u_y u_x \bm \delta_y \bm \delta_x + \rho u_y u_y \bm \delta_y \bm \delta_y + \rho u_y u_z \bm \delta_y \bm \delta_z \nonumber \\
      &+ \rho u_z u_x \bm \delta_z \bm \delta_x + \rho u_z u_y \bm \delta_z \bm \delta_y + \rho u_z u_z \bm \delta_z \bm \delta_z
    \end{align}
    慣性「力」的定義是使流體保持$u$之力,是動量/時間\\
    但Inertial Force實際上是Inertial Stress,是力/面積
    \begin{align}
      \text{Inertial Stress} &= \frac{\text{真慣性力}}{A} = \frac{\frac{m\vec {\bm u}}{t}}{A} = \frac{\dot m\vec {\bm u}}{A} \nonumber \\
      &= \frac{\rho Q \vec {\bm u}}{A} = \frac{\rho \vec {\bm u} A \vec {\bm u}}{A} = \rho \vec {\bm u} \vec {\bm u}
    \end{align}
    \item Pressure Force (壓力),純量,$P$
    \item Viscous Force (黏滯力),$\overline {\overline \tau}$: 抵抗流體流動的力\\
    實際上為Shear Stress,張量,有9個分量
    \begin{equation}
      \overline{\overline{\bm \tau}} = \begin{bmatrix}
        \tau_{xx} & \tau_{xy} & \tau_{xz} \\
        \tau_{yx} & \tau_{yy} & \tau_{yz} \\
        \tau_{zx} & \tau_{zy} & \tau_{zz}
      \end{bmatrix} \label{eq:shear_stress_tensor}
    \end{equation}
    也可以寫成單位向量的展開形式:
    \begin{align}
      \overline{\overline{\bm  \tau}} =& \phantom{+}\tau_{xx}\bm \delta_x\bm \delta_x + \tau_{xy}\bm \delta_x\bm \delta_y + \tau_{xz}\bm \delta_x\bm \delta_z \nonumber \\
      &+ \tau_{yx}\bm \delta_y\bm \delta_x + \tau_{yy}\bm \delta_y\bm \delta_y + \tau_{yz}\bm \delta_y\bm \delta_z \nonumber \\
      &+ \tau_{zx}\bm \delta_z\bm \delta_x + \tau_{zy}\bm \delta_z\bm \delta_y + \tau_{zz}\bm \delta_z\bm \delta_z
    \end{align}
    其中對角線上的分量為Normal Stress,其他為Shear Stress
    \begin{equation}
      \overline{\overline{\bm \tau}} = -\mu\left(
        \nabla \vec {\bm u} + (\nabla \vec {\bm u})^T
      \right) + \left(
        \frac{2}{3}\mu - \kappa
      \right)\left(\nabla \cdot \vec {\bm u}\right) \bm\delta
    \end{equation}
    而因為\fbox{數學上黏度會被乘進去壓力項}\\
    所以加上$\kappa$,可以想成各向抗壓性(Dilatational Viscosity)\\
    來消掉斜對角多算的部分,另外因為黏度這個純量是來自9個項,所以在定義上有其權重分配:\\
    實際上可以由角動量守恆證明以下性質,但大學就不講這麼多了
    \begin{equation}
      \tau_{xy} = \tau_{yx} = -\mu \left[
        \left(\frac{\partial u}{\partial y}\right) + \left(\frac{\partial v}{\partial x}\right)
      \right]
    \end{equation}
    以及:
    \begin{equation}
      \tau_{xx} = -\mu \left[
        2\left(\frac{\partial u}{\partial x}\right)
      \right] + \left(\frac{2}{3}\mu - \kappa\right)\left(\nabla \cdot \vec {\bm u}\right)
    \end{equation}
    \item Gravity Force (重力),純量,$\rho \vec {\bm g}$\\
    實際上也不是力,而是力/體積
    \begin{equation}
      \text{Gravity Force} = \frac{\text{真重力}}{V} = \frac{m\vec {\bm g}}{V} = \frac{\rho V \vec {\bm g}}{V} = \rho \vec {\bm g}
    \end{equation}
  \end{itemize}
  \item 動量輸送的Governing Equation
  \begin{itemize}
    \item Equation of continuity:
      \begin{equation}
        \frac{\partial \rho}{\partial t} + \nabla \cdot (\rho \vec {\bm u}) = 0
      \end{equation}
    \item Equation of motion:\\
      最初的形式其實就是力平衡\\
      牛頓第二定律可以解釋為物體所受淨利等於該物體度量隨時間的變化率
      \begin{equation}
        F = \frac{d(m\vec {\bm u})}{dt}
      \end{equation}
      而一個流體的力平衡則由四個力組成: 慣性力、壓力、黏滯力、重力
      \begin{equation}
        \underbrace{\frac{\partial}{\partial t}(\rho \vec {\bm u})}_{\text{動量累積}} = 
        - \underbrace{\nabla \cdot (\rho \vec {\bm u} \vec {\bm u})}_{\text{慣性力}} -
        \underbrace{\nabla P}_{\text{壓力}} +
        \underbrace{\nabla \cdot \overline{\overline{\bm \tau}}}_{\text{黏滯力}} +
        \underbrace{\rho \vec {\bm g}}_{\text{重力}}
      \end{equation}
      而將$\nabla \cdot (\rho \vec {\bm u} \vec {\bm u})$移到等號左邊\\
      與$\frac{\partial}{\partial t}(\rho \vec {\bm u})$合併\\
      這是將固定座標的微分,轉換為移動座標的微分
      \begin{equation}
        \frac{\partial}{\partial t}(\rho \vec {\bm u}) + \nabla \cdot (\rho \vec {\bm u} \vec {\bm u}) = \frac{D(\rho \vec {\bm u})}{Dt}
      \end{equation}
      此步驟的推導可以看(\ref{eq:material_derivative_relation})\\
      甚至還要用到Equation of continuity\\
      因此此步驟其實假設了恆溫恆密度流體
      \begin{equation}
        \rho \frac{D\vec {\bm u}}{Dt} = -\nabla P + \nabla \cdot \overline{\overline{\bm \tau}} + \rho \vec {\bm g}
      \end{equation}
      若為牛頓流體,且黏度、密度不變,則可化簡為Navier-Stokes Equation
      \begin{equation}
        \rho\left(
          \frac{\partial \vec {\bm u}}{\partial t} + \vec {\bm u} \cdot \nabla \vec {\bm u}
        \right) = \rho \frac{D\vec {\bm u}}{Dt} = -\nabla P + \mu \nabla^2 \vec {\bm u} + \rho \vec {\bm g}
      \end{equation}
      當流體流速很慢,加速度項可以忽略時,則會變成Stokes Flow Equation,或是Creeping Flow Equation
      \begin{equation}
        0 = -\nabla P + \mu \nabla^2 \vec {\bm u} + \rho \vec {\bm g}
      \end{equation}
      而當$\mu =0$時,會再進一步變成Euler Equation(或者說雷諾數無限大)
      \begin{equation}
        \rho\left(
          \frac{\partial \vec {\bm u}}{\partial t} + \vec {\bm u} \cdot \nabla \vec {\bm u}
        \right)= \rho \frac{D\vec {\bm u}}{Dt}  = -\nabla P + \rho \vec {\bm g}
      \end{equation}
    \item  Equation of Mechanical Energy\\
      有了質量守恆、力平衡,再往下還有能量守恆:
      \begin{align}
        \underbrace{\frac{\partial}{\partial t}\left(\frac{1}{2}\rho u^2\right)}_{\text{總動能變化}}
        =&-\underbrace{\left(\nabla \cdot \frac{1}{2}\rho u^2 \vec{\bm u}\right)}_{\text{傳送的動能變化}}
        -\underbrace{\nabla \cdot P \vec{\bm u}}_{\text{周遭壓力做功}} \nonumber \\
        &-\underbrace{\left[P\left(-\nabla \cdot \vec{\bm u}\right)\right]}_{\text{可逆的動能轉內能}}
        -\underbrace{\left[-\overline{\overline{\bm \tau}}  : \nabla \vec{\bm u}\right]}_{\text{不可逆的動能轉內能}} \nonumber \\
        &-\underbrace{\nabla\cdot [\overline{\overline{\bm \tau}} \cdot \vec{\bm u}]}_{\text{黏力做功}} 
        + \underbrace{\rho \vec{\bm u}\cdot \vec{\bm g} }_{\text{外力做功}} \label{eq:mechanical_energy_equation}
      \end{align}
      P.S.\fbox{前面的正負號,只是流出減流入做高斯定理,出和入的方向,都和法向量相反!}\\
      所以以下探討的正負,都不管前面的符號,只管括弧中的正負\\
      注意中間那兩項\fbox{動能轉內能},可逆的結果可正可負,視體積收縮或膨脹而定\\
      \fbox{而因為散度為正,代表體積傾向膨脹,對外做功,所以括弧內有負號}\\
      \fbox{不可逆是一種耗散,括弧內恆正,所以括弧內有負號}\\
      大概因為其他人都是能量的微分形式,唯有重力不是,所以定義單位質量的重力位能$\hat{\bm \Phi}$
      \begin{equation}
        \vec{\bm g} = - \nabla \hat{\bm \Phi}
      \end{equation}
      如此一來便可以將最後一項改寫成:
      \begin{equation}
        \rho  \vec{\bm u}\cdot \vec{\bm g} =
          -\rho ( \vec{\bm u}\cdot \nabla \hat{\bm \Phi})
          = -\left(
          \nabla \cdot (\rho \vec{\bm u} \hat{\bm \Phi})
          \right) + \hat{\bm \Phi} \left(\nabla \cdot \rho \vec{\bm u}\right)
      \end{equation}
      後面那項可以利用Equation of continuity換成$-\frac{\partial \rho}{\partial t}$\\
      變成$\hat{\bm \Phi} \left(-\frac{\partial \rho}{\partial t}\right)$後就可以丟到左邊跟時間導數合併\\
      最後得到Equation of change for mechanical energy(因為合併了動能+位能)
      \begin{align}
        \frac{\partial}{\partial t}\left(\frac{1}{2}\rho u^2 + \rho \hat{\bm \Phi}\right) = 
        &-\left[
          \nabla \cdot \left(
            \frac{1}{2}\rho u^2 + \rho \hat{\bm \Phi}
          \right) \vec{\bm u}
        \right] - \nabla \cdot P \vec{\bm u} \nonumber \\
        &- P\left(-\nabla \cdot \vec{\bm u}\right) - (-\overline{\overline{\bm \tau}} : \nabla \vec{\bm u}) \nonumber \\
        &- \nabla \cdot [\overline{\overline{\bm \tau}} \cdot \vec{\bm u}] 
        \label{eq:mechanical_energy_equation_modified}
      \end{align}
      而其中不可逆的動能轉內能項,$-\overline{\overline{\bm \tau}} : \nabla \vec{\bm u}$\\
      實際上就是流體的耗散函數(Dissipation function)\\
      如同後面例題會提到的假設,流體在管中造成的壓力損失\\
      是來自為了抵抗形變所耗損的能量,所以才會是\fbox{力($\tau$)乘以形變速率($u$)的形式}\\
      如果$\overline{\overline{\bm \tau}}$是牛頓流體,則可以代入其定義
      \begin{equation}
        -\overline{\overline{\bm \tau}}: \nabla \vec{\bm u} =\frac{2}{3}\mu \sum_{i=1}^{3}\sum_{j=1}^{3}\left[
          \left(
          \frac{\partial u_i}{\partial x_i} + \frac{\partial u_j}{\partial x_j}
        \right) - \frac{2}{3}\delta_{ij} \nabla \cdot \vec{\bm u}
        \right]^2 + \kappa \left(
          \nabla \cdot \vec{\bm u}
        \right)^2 = \mu \bm \Phi_u + \kappa \bm \Phi_b \label{eq:viscous_dissipation}
      \end{equation}
      $\bm \Phi_u$為剪切耗散函數(Shear Dissipation Function)\\
      $\bm \Phi_b$為體積耗散函數(Bulk Dissipation Function)\\
      兩者皆恆正,代表不可逆的能量耗散,並分別代表流體的抗剪和抗壓能力
    \item 巨觀的能量平衡式:\\
      積分式中對於流入或流出控制體積的前三項
      \begin{equation}
        \int_{\text{CV}} \left[
          -\nabla \cdot \left(
            \frac{1}{2}\rho u^2 + \rho \hat{\bm \Phi}
          \right) \vec{\bm u}
          - \nabla \cdot P \vec{\bm u}
          - \nabla \cdot [\overline{\overline{\bm \tau}} \cdot \vec{\bm u}]
        \right] dV
      \end{equation}
      可以利用Gauss定理轉換成表面積分
      \begin{equation}
        \int_{\text{CS}} \left[
          \left(
            \frac{1}{2}\rho u^2 + \rho \hat{\bm \Phi}
          \right) \vec{\bm u}
          + P \vec{\bm u}
          + \overline{\overline{\bm \tau}} \cdot \vec{\bm u}
        \right] \cdot \vec{\bm n} dA
      \end{equation}
      而控制體積內的後三項
      \begin{equation}
        \int_{\text{CV}} \left[
          - P\left(-\nabla \cdot \vec{\bm u}\right)
          - (-\overline{\overline{\bm \tau}} : \nabla \vec{\bm u})
        \right] dV
      \end{equation}
      則代表流體內部因為可逆和不可逆的動能轉內能
      \begin{equation}
        E_c = -\int_{\text{CV}} P\left(-\nabla \cdot \vec{\bm u}\right) dV
      \end{equation}
      以及
      \begin{equation}
        E_v = -\int_{\text{CV}} \left[-\overline{\overline{\bm \tau}} : \nabla \vec{\bm u} \right] dV
      \end{equation}
      因此整個控制體積的機械能平衡式為:
      \begin{equation}
        \frac{d}{dt}\left(
          K_{\text{tot}} + \bm \Phi_{\text{tot}}
        \right) = \int_{\text{CS}} \left[
          \left(
            \frac{1}{2}\rho u^2 + \rho \hat{\bm \Phi}
          \right) \vec{\bm u}
          + P \vec{\bm u}
          + \overline{\overline{\bm \tau}} \cdot \vec{\bm u}
        \right] \cdot \vec{\bm n} dA  + E_c + E_v
      \end{equation}
      其中$K_{\text{tot}}$為控制體積內的總動能:
      \begin{equation}
        K_{\text{tot}} = \int_{\text{CV}} \frac{1}{2}\rho u^2 dV
      \end{equation}
      $\bm \Phi_{\text{tot}}$為控制體積內的總位能:
      \begin{equation}
        \bm \Phi_{\text{tot}} = \int_{\text{CV}} \rho \hat{\bm \Phi} dV
      \end{equation}
    \item 應用於單一進口和單一出口的系統:\\
      假設控制體積只有一個進口和一個出口,且流體在截面上的速度分布均勻\\
      則表面積分可以寫成:
      \begin{align}
        &\int_{\text{CS}} \left[
          \left(
            \frac{1}{2}\rho u^2 + \rho \hat{\bm \Phi}
          \right) \vec{\bm u}
          + P \vec{\bm u}
          + \overline{\overline{\bm \tau}} \cdot \vec{\bm u}
        \right] \cdot \vec{\bm n} dA \nonumber \\
        =& \left[ \left(
          \frac{1}{2}\rho_1 \left<u_1^3\right> + \rho_1 \hat{\bm \Phi}_1 \left<u_1\right> + P_1 \left<u_1\right>
        \right) S_1 \right] \nonumber \\
        &- \left[ \left(
          \frac{1}{2}\rho_2 \left<u_2^3\right> + \rho_2 \hat{\bm \Phi}_2 \left<u_2\right> + P_2 \left<u_2\right>
        \right) S_2 \right]
      \end{align}
      其中$S_1$和$S_2$分別為進口和出口的截面積\\
      $\left<u_1\right>$和$\left<u_2\right>$分別為進口和出口的平均速度\\
      $\left<u_1^3\right>$和$\left<u_2^3\right>$分別為進口和出口的速度三次方的平均值\\
      另外假設控制體積內有外力對流體做功$W_m$\\
      則整個機械能平衡式為:
      \begin{align}
        \frac{d}{dt}\left(
          K_{\text{tot}} + \bm \Phi_{\text{tot}}
        \right) =&\left[ \left(
          \frac{1}{2}\rho_1\left<u_1^3\right> + \rho_1\hat{\bm \Phi}_1 \left<u_1\right> + P_1 \left<u_1\right>
        \right) S_1 \right]\nonumber\\
        & - \left[\left(
          \frac{1}{2}\rho_2\left<u_2^3\right> + \rho_2\hat{\bm \Phi}_2 \left<u_2\right> + P_2 \left<u_2\right>
        \right) S_2 \right] \nonumber \\
        & + W_m + E_c + E_v
      \end{align}
      若假設Steady State,且控制體積內的位能不變,則左邊為0\\
      並且將位能改寫成高度$h$的形式:
      \begin{equation}
        \hat{\bm \Phi} = gh
      \end{equation}
      則得到:
      \begin{align}
        0 =& \left[ \left(
          \frac{1}{2}\rho_1\left<u_1^3\right> + \rho_1 gh_1 \left<u_1\right> + P_1 \left<u_1\right>
        \right) S_1 \right] \nonumber \\
        & - \left[\left(
          \frac{1}{2}\rho_2\left<u_2^3\right> + \rho_2 gh_2 \left<u_2\right> + P_2 \left<u_2\right>
        \right) S_2 \right] \nonumber \\
        & + W_m + E_c + E_v
      \end{align}
      將上式除以$w$,並整理,得到機械能守恆方程式:
      \begin{equation}
        \frac{1}{2}\left(\frac{\left<u_2^3\right>}{\left<u_2\right>} - 
        \frac{\left<u_1^3\right>}{\left<u_1\right>}\right) + g(h_2 - h_1) 
        + \left(\frac{P_2}{\rho_2} - \frac{P_1}{\rho_1}\right) 
        = \frac{-W_m + E_c + E_v}{w}
      \end{equation}
    \item 機械能守恆方程式的衍生,以及動能修正因子:\\
      若流體不可壓縮,則密度不變,$E_c =0$\\
      若流體可壓,但是穩態,則可以將$E_c$改寫成:
      \begin{equation}
        E_c = \int_{\text{CV}} \frac{p}{\rho} (\nabla \rho \cdot \vec{\bm u}) dV
      \end{equation}
      若假設任一截面下的密度和壓力相同,則可以近似為:
      \begin{equation}
        E_c = -w\Delta \left(\frac{p}{\rho}\right) + w\int_1^2\frac{1}{\rho} dP 
      \end{equation}
      往前併入機械能守恆方程式,則變成:
      \begin{equation}
        \boxed{\Delta \left(\frac{1}{2}\left<u^2\right>\right)  + g\Delta h + \int_1^2 \frac{1}{\rho} dP =
        \frac{-W_m + E_v}{w}}
      \end{equation}
      另外若流速為紊流,速度並不均勻,若以平均速度來寫時應改為:
      \begin{equation}
        \frac{\left<u^3\right>}{\left<u\right>} = \alpha \left<u\right>^2
      \end{equation}
      其中$\alpha$為動能修正因子,通常大於1,這個數值大概是1.02到1.1之間
  \end{itemize}
\end{itemize}
\end{CJK*}
\end{document}