\documentclass[../main.tex]{subfiles}
\begin{document}
\begin{CJK*}{UTF8}{bkai}
\subsection{Shell-Balance解題}
\begin{itemize}
  \item Shell-balance Method: 利用控制體積做力平衡\\
    先求出$\tau$,再代入流體模型求出$u_x(y)$\\
    優點: 任何種類的流體都可以計算\\
    缺點,只能解\fbox{直角座標}或\fbox{圓柱座標},流體流動方向在$z$方向\\
    否則會很困難
  \item 若以Shell-Balance Method求解,會依序算出的物理量:
  \begin{enumerate}
    \item Shear stress, Momentum Flux: $\tau$,各處的剪應力大小
    \item 速度分布:\\
      $u_x(y)$: 垂直流向上的速度分布
    \item 最大速度 $u_{\text{max}}$: \\
      尋找$u_x(y)$的最大值
    \item 平均速度 $\left<u_x\right>$: 
    \begin{equation}
      \left<u_x\right> = \frac{1}{A}\iint u_x dA
    \end{equation}
    \item  體積流率 $Q$:
    \begin{equation}
      Q = \iint u_x dA = \left<u_x\right> A
    \end{equation}
    \item 雷諾數 $\text{Re}$: 
    \begin{equation}
      \text{Re} = \frac{\rho \left<u_x\right> D}{\mu}
    \end{equation}
    \item 濕潤表面所給予的阻力 $F_x$:\\
    Drag force of plates
    \begin{equation}
      F_x = 2\iint \tau_{yx}|_{y=B}dA_S
    \end{equation}
    $A_S$是\fbox{濕潤的表面積},不是截面積
    \item 范寧摩擦因子(Fanning friction factor) $f$:
    \begin{equation}
      f = \frac{\tau_{yx}|_{y=B}}{\frac{1}{2}\rho \left<u_x\right>^2}
    \end{equation}
  \end{enumerate}
  \item 解題步驟:
  \begin{enumerate}
    \item 寫出The Equation of continuity
    \begin{equation}
      \frac{\partial \rho}{\partial t}+\nabla \cdot (\rho \vec {\bm u}) = 0
    \end{equation}
    物理意義為C.V.內的質量守恆\\
    從質量平衡式而來:
    \begin{equation}
      \text{in} - \text{out} + \cancel{\text{gen}} = \text{acc}
    \end{equation}
    假設一個邊長為$dx,dy,dz$的控制體積
    \begin{figure}[H]
      \centering
      \begin{tikzpicture}[>=Latex, line cap=round, line join=round, thick]
        \draw (0,0,0) -- (4,0,0) -- (4,3,0) -- (0,3,0) -- cycle;
        \draw [dashed] (0,0,0) -- (0,0,-3);
        \draw (4,0,0) -- (4,0,-3);
        \draw (4,3,0) -- (4,3,-3);
        \draw (0,3,0) -- (0,3,-3);
        \draw [dashed] (0,3,-3) -- (0,0,-3) -- (4,0,-3);
        \draw (4,0,-3) -- (4,3,-3) -- (0,3,-3);
        \node[anchor=north] at (2,0,0) {$dx$};
        \node[anchor=east] at (0,1.5,0) {$dy$};
        \node[anchor=north west] at (3,0, -1.5) {$dz$};
        \draw[->] (-2,1.5,-1.5) -- (0,1.5,-1.5);
        \node[anchor=east] at (-2,1.5,-1.5) {$\dot m_x\big|_x$};
        \draw[->] (4,1.5,-1.5) -- (6,1.5,-1.5);
        \node[anchor=west] at (6,1.5,-1.5) {$\dot m_x\big|_{x+dx}$};
        \draw[->] (2,-2,-1.5) -- (2,0,-1.5);
        \node[anchor=north] at (2,-2,-1.5) {$\dot m_y\big|_y$};
        \draw[->] (2,3,-1.5) -- (2,5,-1.5);
        \node[anchor=south] at (2,5,-1.5) {$\dot m_y\big|_{y+dy}$};
        \draw[->] (2,1.5,4) -- (2,1.5,0);
        \node[anchor=north east] at (2,1.5,4) {$\dot m_z\big|_z$};
        \draw[->] (2,1.5,-3) -- (2,1.5,-7);
        \node[anchor=south west] at (2,1.5,-7) {$\dot m_z\big|_{z+dz}$};
        \fill[pattern=dots, pattern color=blue] (0,0,0) -- (4,0,0) -- (4,3,0) -- (0,3,0) -- cycle;
        \fill[pattern=dots, pattern color=blue] (0,0,-3) -- (4,0,-3) -- (4,3,-3) -- (0,3,-3) -- cycle;
        \fill[pattern=dots, pattern color=blue] (0,0,0) -- (0,0,-3) -- (0,3,-3) -- (0,3,0) -- cycle;
        \fill[pattern=dots, pattern color=blue] (4,0,0) -- (4,0,-3) -- (4,3,-3) -- (4,3,0) -- cycle;
        \fill[pattern=dots, pattern color=blue] (0,0,0) -- (4,0,0) -- (4,0,-3) -- (0,0,-3) -- cycle;
        \fill[pattern=dots, pattern color=blue] (0,3,0) -- (4,3,0) -- (4,3,-3) -- (0,3,-3) -- cycle; 
      \end{tikzpicture}
      \caption{Control Volume for Equation of continuity}
    \end{figure}
    所以質量平衡式為:
    \begin{align}
      &\rho u_x dy dz\big|_{x} - \rho u_x dy dz\big|_{x+dx} +
      \rho u_y dx dz\big|_{y} - \rho u_y dx dz\big|_{y+dy} \nonumber\\
      &\phantom{\rho u_y dx dz\big|_{y+dy}}+ \rho u_z dx dy\big|_{z} - \rho u_z dx dy\big|_{z+dz} = \frac{\partial}{\partial t}(\rho dx dy dz) \nonumber \\
      \text{同除}dxdydz\Rightarrow&
      \frac{\rho u_x|_{x} - \rho u_x|_{x+dx}}{dx} + 
      \frac{\rho u_y|_{y} - \rho u_y|_{y+dy}}{dy} +
      \frac{\rho u_z|_{z} - \rho u_z|_{z+dz}}{dz} = \frac{\partial \rho}{\partial t} \nonumber \\
      \text{同乘}-1\Rightarrow&
      -\frac{\rho u_x\big|_{x+dx} - \rho u_x\big|_{x}}{dx} -
      \frac{\rho u_y\big|_{y+dy} - \rho u_y\big|_{y}}{dy} -
      \frac{\rho u_z\big|_{z+dz} - \rho u_z\big|_{z}}{dz} = \frac{\partial \rho}{\partial t} \nonumber \\
      \Rightarrow&
      -\frac{\partial}{\partial x}(\rho u_x) -
      \frac{\partial}{\partial y}(\rho u_y) -
      \frac{\partial}{\partial z}(\rho u_z) = \frac{\partial \rho}{\partial t} \nonumber \\
      \Rightarrow&
      -\nabla \cdot (\rho \vec {\bm u}) = \frac{\partial \rho}{\partial t} \nonumber \\
      \Rightarrow&
      \frac{\partial \rho}{\partial t} + \nabla \cdot (\rho \vec {\bm u}) = 0 \label{eq:continuity_equation}
    \end{align}
    \item 利用題目假設得到解的形式為$v_x = f(t,x,y,z)$中那些項:\\
    有以下假設可以使用:
    \begin{itemize}
      \item Steady-state:
      \begin{equation}
        \dot m_{\text{acc}}=0 \implies \frac{\partial}{\partial t} = 0 \implies \boxed{u = f(x,y,z)}
      \end{equation}
      \item Fully developed flow (發展完全流動)\\
        流體各處沒有在並非整體流動方向外的流動(假設流向$x$)
        \begin{equation}
          \boxed{u_y = u_z = 0}
        \end{equation}
        速度分布曲線不再隨流動方向的深度而改變(假設流向$x$)
        \begin{equation}
          \boxed{u_x \neq f(x)} \implies u_x = f(t,y,z) \implies \frac{\partial u_x}{\partial x} = 0
        \end{equation}
        以管子來說,會是\fbox{邊界層重疊}之後的流動狀態
      \item End-effects can be neglect: 管子長度足夠長,使得兩端效應可以忽略\\
        兩端的意思就是\fbox{邊界層尚未重疊之處}
      \item 假設因摩擦損耗導致單位管長的壓力損失$\frac{dP}{dx}$為常數\\
        可以由微分形式的白努力定律說明
        \begin{equation}
          \frac{d u^2}{2g_c} +\frac{g}{g_c}dz + \frac{1}{\rho}dP + \underbrace{dh_f}_{\text{摩擦}} = 0 \label{eq:diff_bernoulli}
        \end{equation}
        假設是水平管子,且流體速度相同:
        \begin{equation}
          \frac{1}{\rho}dP + \underbrace{dh_f}_{\text{摩擦}} = 0
        \end{equation}
        後續單操設計會說,摩擦係數與管線設計有關
        \begin{equation}
          \frac{1}{\rho} dP + 4f \frac{dx}{D_H}\frac{u^2}{2g_c} = 0 
        \end{equation}
        故$\frac{dP}{dx}$為常數
        \begin{equation}
          -\frac{dP}{dx} = \frac{4f}{D_H}\frac{\rho u^2}{2g_c} = \text{常數}
        \end{equation}
        而如果是常數時
        \begin{equation}
          -\left(\frac{\partial P}{\partial x}\right) = \frac{P_0-P_L}{L}
        \end{equation}
      \item Incompressible: $\rho = \text{constant}$
      \item 是否受重力影響: $\rho \vec {\bm g} = 0$\\
        若題目受重力影響要使用Modified Pressure,$\frac{d\mathbb{P}}{dt}$
        \begin{equation}
          \boxed{\mathbb{P} = P + \rho g z}
        \end{equation}
      \item Laminar flow: $\text{Re} < 2100$ 或 4000 取決於平板或圓管
      \item No-slip: 管壁與流體速度相同
      \item Newtonian fluid: $\tau_{yx} = -\mu \frac{du_x}{dy}$,當然也可以換成任一流體模型
      \item Continuum hypothesis: 流體是連續的
      \item Isothermal: 流體溫度不變
    \end{itemize}
    \item 找一Control Volume,寫出這個Shell的Force Balance($\sum F=0$)
    \item 求出$\tau_{yx}$
    \item 代入流體的模型,求出$u_x$
  \end{enumerate}
  \item 例1:直角坐標、層流、穩態、不可壓縮、牛頓流體:\\
  假設有兩個平行板,兩平行板間距離$2B$,長度$L$,寬$W$,流體在$x$方向上流動\\
  並定義原點在中心,$y$方向向上,$x$方向向右\\
  水平放置,忽略重力影響
  \begin{figure}[H]
    \centering
    \begin{tikzpicture}[>=Latex, line cap=round, line join=round, thick]
      \draw (2,0) -- (10,0);
      \draw[dashed] (2,0) -- +(0.5,1);
      \draw[dashed] (10,0) -- +(0.5,1);
      \draw (2,4) -- (10,4);
      \draw[dashed] (2,4) -- +(0.5,1);
      \draw[dashed] (10,4) -- +(0.5,1);
      \draw[<->, dashed] (10.5,4) -- +(0.5, 1) node[midway, below right] {$W$}; 
      \draw[<->] (2,-0.5) -- (10, -0.5) node[midway, below] {$L$};
      \draw[<->] (8,0) -- (8,4) node[midway, right] {$2B$};
      \node[anchor=south west] at (10,0) {Fixed Plate};
      \node[anchor=north west] at (10,4) {Fixed Plate};
      \fill[pattern=north east lines, pattern color=red, draw=red] (0,0) rectangle (0.3,4);
      \node[anchor=east, red] at (0,2) {Pump};
      \draw[blue] (0.5,0) -- (0.5,4);
      \draw[blue, dashed] (1.5,0) -- (1.5,4);
      \draw[->, blue, dashed] (0.5,0) -- (1.5,0);
      \draw[->, blue, dashed] (0.5,0.5) -- (1.5,0.5);
      \draw[->, blue, dashed] (0.5,1) -- (1.5,1);
      \draw[->, blue, dashed] (0.5,1.5) -- (1.5,1.5);
      \draw[->, blue, dashed] (0.5,2) -- (1.5,2);
      \draw[->, blue, dashed] (0.5,2.5) -- (1.5,2.5);
      \draw[->, blue, dashed] (0.5,3) -- (1.5,3);
      \draw[->, blue, dashed] (0.5,3.5) -- (1.5,3.5);
      \draw[->, blue, dashed] (0.5,4) -- (1.5,4);
      \node[anchor=south, blue] at (1,4) {Poiseuille Flow};
      \draw[->] (2,2) -- (2,3) node[above] {$y$};
      \draw[->] (2,2) -- (3,2) node[right] {$x$};
      \draw[blue] (4,0) -- (4,4);
      \draw[blue,dashed,domain=0:4,smooth,variable=\y] plot ({5.2 - 0.3*(\y-2)*(\y-2)},\y);
      \draw[->,blue,dashed] (4, 0.5) -- ({5.2 - 0.3*(0.5-2)*(0.5-2)}, 0.5);
      \draw[->,blue,dashed] (4, 1) -- ({5.2 - 0.3*(1-2)*(1-2)}, 1);
      \draw[->,blue,dashed] (4, 1.5) -- ({5.2 - 0.3*(1.5-2)*(1.5-2)}, 1.5);
      \draw[->,blue,dashed] (4, 2) -- ({5.2 - 0.3*(2-2)*(2-2)}, 2);
      \draw[->,blue,dashed] (4, 2.5) -- ({5.2 - 0.3*(2.5-2)*(2.5-2)}, 2.5);
      \draw[->,blue,dashed] (4, 3) -- ({5.2 - 0.3*(3-2)*(3-2)}, 3);
      \draw[->,blue,dashed] (4, 3.5) -- ({5.2 - 0.3*(3.5-2)*(3.5-2)}, 3.5);
      \fill[pattern=north east lines, draw=black] (6,1.8) rectangle (6.4,2.2);
      \node[anchor=west] at (6.4,2) {C.V.};
    \end{tikzpicture}
    \caption{兩固定平行板,Poiseuille Flow}
  \end{figure}
  \begin{enumerate}
    \item 根據Equation of continuity:
    \begin{equation}
      \frac{\partial \rho}{\partial t} + \nabla \cdot (\rho\vec {\bm u})  = 0
    \end{equation}
    \item 因為不可壓縮$\rho = \text{constant}$,且Steady-state,所以$\frac{\partial P}{\partial t} = 0$
    \begin{equation}
      \nabla \cdot \vec{u} =0
    \end{equation}
    \item 直角坐標:
    \begin{equation}
      \frac{\partial u_x}{\partial x} + \frac{\partial u_y}{\partial y} + \frac{\partial u_z}{\partial z} = 0
    \end{equation}
    \item Fully developed
    \begin{equation}
      \frac{\partial u_x}{\partial x} = 0, \quad u_y = 0, \quad u_z = 0
    \end{equation}
    說明$u_x$與$x$無關
    \item 根據Steady state, Fully Develop, 2D
    \begin{equation}
      u_x=f(\cancelto{\text{S.S.}}{t},\cancelto{\text{上式}}{x},y,\cancelto{2D}{z}) \implies \boxed{u_x = u_x(y)}
    \end{equation}
    \item 給定一個控制殼體積,邊長$dx,~dy,~dz$,做Shell-balance(單位時間的力平衡)
    \begin{itemize}
      \item yz平面上進出動量
      \begin{align}
        P dy dz\big|_{x} & -P dy dz\big|_{x+dx} \nonumber\\
        \rho u_x u_x dy dz\big|_{x} & -\rho u_x u_x dy dz\big|_{x+dx}
      \end{align}
      \item xz平面上進出動量
      \begin{equation}
        \tau_{yx} dy dz\big|_{y} - \tau_{yx} dy dz\big|_{y+dy}
      \end{equation}
      \item xy平面上進出動量
      \begin{equation}
        \tau_{zx} dx dz\big|_{z} - \tau_{zx} dx dz\big|_{z+dz}
      \end{equation}
      \item 忽略重力影響,$\rho \vec {\bm g} = 0$
      \item 加總:$\text{in} -\text{out}+ \cancel{\text{gen}} =\cancelto{S.S.}{\text{acc}}$
      \begin{align}
        &\rho u_x u_x dy dz|_{x} - \rho u_x u_x dy dz|_{x+dx} + 
        \tau_{yx} dy dz|_{y} - \tau_{yx} dy dz|_{y+dy} \nonumber \\
        &\phantom{\tau_{yx} dy dz|_{y}}+ \tau_{zx} dx dz|_{z} - \tau_{zx} dx dz|_{z+dz}  
        + P dy dz|_{x} - P dy dz|_{x+dx} = 0
      \end{align}
      \item 同除$dx\cdot dy\cdot dz$
      \begin{equation}
        \frac{\rho u_x u_x|_{x} - \rho u_x u_x|_{x+dx}}{dx} +
        \frac{\tau_{yx}|_{y} - \tau_{yx}|_{y+dy}}{dy} +
        \frac{\tau_{zx}|_{z} - \tau_{zx}|_{z+dz}}{dz} +
        \frac{P|_{x} - P|_{x+dx}}{dx} = 0 
      \end{equation}
      \item 根據微分定義:
      \begin{align}
        -\cancelto{u_x\neq f(x)}{\frac{\partial}{\partial x}\left(\rho u_x u_x\right)} -
        \frac{\partial \tau_{yx}}{\partial y} - 
        \cancelto{2D}{\frac{\partial \tau_{zx}}{\partial z}} - 
        \frac{\partial P}{\partial x} &= 0 \nonumber \\
         \frac{\partial \tau_{yx}}{\partial y} &= -\left(\frac{\partial P}{\partial x}\right)
      \end{align}
      \item 積分
      \begin{equation}
        \tau_{yx} = -\int \frac{\partial P}{\partial x} dy +  C_1
      \end{equation}
      \item 代入此題邊界條件$y=0$時,$\tau_{yx}=0$,因為兩平板中心對稱,解得$C_1$
      \begin{equation}
        C_1 = 0
      \end{equation}
      \item 得到Shear stress distribution
      \begin{equation}
        \boxed{\tau_{yx} = -\frac{\partial P}{\partial x}}
      \end{equation}
      若忽略重力,$ -\frac{\partial P}{\partial x} = \frac{P_L-P_0}{L}$
      \begin{equation}
        \boxed{\tau_{yx} = \frac{P_0-P_L}{L} y}
      \end{equation}
      \begin{figure}[H]
        \centering
        \begin{tikzpicture}[>=Latex, line cap=round, line join=round, thick]
          \draw (2,0) -- (10,0);
          \draw[dashed] (2,0) -- +(0.5,1);
          \draw[dashed] (10,0) -- +(0.5,1);
          \draw (2,4) -- (10,4);
          \draw[dashed] (2,4) -- +(0.5,1);
          \draw[dashed] (10,4) -- +(0.5,1);
          \draw[<->, dashed] (10.5,4) -- +(0.5, 1) node[midway, below right] {$W$}; 
          \draw[<->] (2,-0.5) -- (10, -0.5) node[midway, below] {$L$};
          \draw[<->] (9,0) -- (9,4) node[midway, right] {$2B$};
          \node[anchor=south west] at (10,0) {Fixed Plate};
          \node[anchor=north west] at (10,4) {Fixed Plate};
          \fill[pattern=north east lines, pattern color=red, draw=red] (0,0) rectangle (0.3,4);
          \node[anchor=east, red] at (0,2) {Pump};
          \draw[blue] (0.5,0) -- (0.5,4);
          \draw[blue, dashed] (1.5,0) -- (1.5,4);
          \draw[->, blue, dashed] (0.5,0) -- (1.5,0);
          \draw[->, blue, dashed] (0.5,0.5) -- (1.5,0.5);
          \draw[->, blue, dashed] (0.5,1) -- (1.5,1);
          \draw[->, blue, dashed] (0.5,1.5) -- (1.5,1.5);
          \draw[->, blue, dashed] (0.5,2) -- (1.5,2);
          \draw[->, blue, dashed] (0.5,2.5) -- (1.5,2.5);
          \draw[->, blue, dashed] (0.5,3) -- (1.5,3);
          \draw[->, blue, dashed] (0.5,3.5) -- (1.5,3.5);
          \draw[->, blue, dashed] (0.5,4) -- (1.5,4);
          \node[anchor=south, blue] at (1,4) {Poiseuille Flow};
          \draw[->] (2,2) -- (2,3) node[above] {$y$};
          \draw[->] (2,2) -- (3,2) node[right] {$x$};
          \draw[blue] (4,0) -- (4,4);
          \draw[blue,dashed,domain=0:4,smooth,variable=\y] plot ({5.2 - 0.3*(\y-2)*(\y-2)},\y);
          \draw[->,blue,dashed] (4, 0.5) -- ({5.2 - 0.3*(0.5-2)*(0.5-2)}, 0.5);
          \draw[->,blue,dashed] (4, 1) -- ({5.2 - 0.3*(1-2)*(1-2)}, 1);
          \draw[->,blue,dashed] (4, 1.5) -- ({5.2 - 0.3*(1.5-2)*(1.5-2)}, 1.5);
          \draw[->,blue,dashed] (4, 2) -- ({5.2 - 0.3*(2-2)*(2-2)}, 2);
          \draw[->,blue,dashed] (4, 2.5) -- ({5.2 - 0.3*(2.5-2)*(2.5-2)}, 2.5);
          \draw[->,blue,dashed] (4, 3) -- ({5.2 - 0.3*(3-2)*(3-2)}, 3);
          \draw[->,blue,dashed] (4, 3.5) -- ({5.2 - 0.3*(3.5-2)*(3.5-2)}, 3.5);
          \draw[teal] (7,0) -- (7,4);
          \draw[teal,domain=0:4,smooth,variable=\y, dashed] plot ({7+0.6*(\y-2)}, {\y});
          \draw[->,teal,dashed] (7, 0.5) -- ({7+0.6*(0.5-2)}, 0.5);
          \draw[->,teal,dashed] (7, 1) -- ({7+0.6*(1-2)}, 1);
          \draw[->,teal,dashed] (7, 1.5) -- ({7+0.6*(1.5-2)}, 1.5);
          \draw[->,teal,dashed] (7, 2.5) -- ({7+0.6*(2.5-2)}, 2.5);
          \draw[->,teal,dashed] (7, 3) -- ({7+0.6*(3-2)}, 3);
          \draw[->,teal,dashed] (7, 3.5) -- ({7+0.6*(3.5-2)}, 3.5);
          \node[anchor=south, teal] at (7,4) {$\tau_{yx}$};
        \end{tikzpicture}
        \caption{兩固定平行板,Poiseuille flow,Shear Stress Distribution}
      \end{figure}
    \end{itemize}
    \item 代入流體模型(牛頓流體):$\tau_{yx} = -\mu \frac{du_x}{dy}$
      \begin{equation}
        -\mu \frac{du_x}{dy} = \left(-\frac{\partial P}{\partial x}\right)y
      \end{equation}
      移項後,再對$y$積分
      \begin{equation}
        u_x = -\frac{1}{2\mu} \left(-\frac{\partial P}{\partial x}\right)y^2 + C_2
      \end{equation}
      代入此題邊界條件$y=B$時$u_x=0$,(no-slip),解得$C_2$
      \begin{equation}
        C_2 = \frac{B^2}{2\mu}\left(-\frac{\partial P}{\partial x}\right)
      \end{equation}
      代回得到Velocity Profile $u_x(y)$
      \begin{equation}
        \boxed{u_x = \frac{B^2}{2\mu}\left(-\frac{\partial P}{\partial x}\right)\left[
          1-\left(\frac{y}{B}\right)^2
        \right]}
      \end{equation}
      若忽略重力 $ -\frac{\partial P}{\partial x} = \frac{P_L-P_0}{L}$
      \begin{equation}
        \boxed{u_x = \frac{B^2}{2\mu}\left(\frac{P_0-P_L}{L}\right)\left[
          1-\left(\frac{y}{B}\right)^2
        \right]}
      \end{equation}
    \item 求最大速度$u_{\text{max}}$\\
      此題也就是在$y=0$時(兩板中心),不需要用微分為零來算,直接將$y=0$代入
      \begin{equation}
        \boxed{u_{\text{max}} = \frac{B^2}{2\mu}\left(-\frac{\partial P}{\partial x}\right)}
      \end{equation}
      若忽略重力 $ -\frac{\partial P}{\partial x} = \frac{P_L-P_0}{L}$
      \begin{equation}
        \boxed{u_{\text{max}} = \frac{B^2}{2\mu}\left(\frac{P_0-P_L}{L}\right)}
      \end{equation}
    \item 求平均速度$\left<u_x\right>$
      由均值定理:
      \begin{equation}
        \left<u_x\right> =\frac{1}{A}\iint u_x dA
      \end{equation}
      此題$A=2BW$,且$u_x$與$z$無關
      \begin{equation}
        \left<u_x\right> = \frac{
          \int_0^W\int_{-B}^{B} u_x dy dz
        }{\int_0^W\int_{-B}^{B} dy dz} = \frac{\frac{B^2}{2\mu}\left(-\frac{\partial P}{\partial x}\right)\cdot W \cdot \left( \left.
          y- \frac{y^3}{3B^2}
        \right|_{-B}^{B}\right)}{2BW}\nonumber
      \end{equation}
      化簡為
      \begin{equation}
        \boxed{\left<u_x\right> = \frac{B^2}{3\mu}\left(-\frac{\partial P}{\partial x}\right) = \frac{2}{3}u_{\text{max}}}
      \end{equation}
      若忽略重力 $ -\frac{\partial P}{\partial x} = \frac{P_L-P_0}{L}$
      \begin{equation}
        \boxed{\left<u_x\right> = \frac{B^2}{3\mu}\left(\frac{P_0-P_L}{L}\right)}
      \end{equation}
    \item 求體積流率,$Q$
      體積流率求法:
      \begin{equation}
        Q =\iint u_xdA \left(u_x\right)A
      \end{equation}
      此題$A=2BW$,且$u_x$與$z$無關
      \begin{equation}
        \boxed{Q = \frac{2B^3W}{3\mu}\left(-\frac{\partial P}{\partial x}\right)}
      \end{equation}
      若忽略重力 $ -\frac{\partial P}{\partial x} = \frac{P_L-P_0}{L}$
      \begin{equation}
        \boxed{Q = \frac{2B^3W}{3\mu}\left(\frac{P_0-P_L}{L}\right)}
      \end{equation}
    \item 求雷諾數Re:\\
      雷諾數定義:
      \begin{equation}
        \text{Re} = \frac{\rho \left<u_x\right> D_H}{\mu}
      \end{equation}
      此題不是圓管,需求等效直徑:
      \begin{equation}
        D_H = 4\frac{A}{P} = 4\frac{2BW}{2W} = 4B
      \end{equation}
      故雷諾數為:
      \begin{equation}
        \boxed{\text{Re} = \frac{4\rho B^2\left(-\frac{\partial P}{\partial x}\right)}{3\mu^2}}
      \end{equation}
      若忽略重力 $ -\frac{\partial P}{\partial x} = \frac{P_L-P_0}{L}$
      \begin{equation}
        \boxed{\text{Re} = \frac{4\rho B^2\left(\frac{P_0-P_L}{L}\right)}{3\mu^2}}
      \end{equation}
    \item 求Drag Force,拖曳力,$F_x$\\
      Drag Force 定義:板子被流體沖走的力量,是Shear Stress乘上面積
      \begin{equation}
        F_x = \iint \tau_{yx}|_{y=B} dA_S
      \end{equation}
      積分後
      \begin{equation}
        \boxed{F_x = \left(-\frac{\partial P}{\partial x}\right)2BLW}
      \end{equation}
      若忽略重力 $ -\frac{\partial P}{\partial x} = \frac{P_L-P_0}{L}$
      \begin{equation}
        \boxed{F_x = (P_0-P_L)2BW}
      \end{equation}
      這也可以看出,$P_0-P_L$為損失的壓力\\
      而這代表損失的壓力都被拿去讓板子克服拖曳力
    \item 求Fanning friction factor,$f_F$\\
      Fanning friction factor 定義:
      \begin{equation}
        f_F = \frac{\tau_{w}}{\frac{1}{2}\rho \left<u\right>^2} = \frac{\text{viscous force}}{\text{inertial force}}
      \end{equation}
      也可以看出,Fanning Friction Factor和雷諾數成反比關係\\
      其中$\tau_w$為壁面剪應力,此題為$\tau_{yx}|_{y=B}$
      \begin{equation}
        f_F = \frac{\tau_{yx}|_{y=B}}{\frac{1}{2}\rho \left<u_x\right>^2}
      \end{equation}
      由於Fanning Friction Factor 通常是用來看出與$\text{Re}$的關係\\
      最後要換成以雷諾數表示,不需要把速度換回去
      \begin{align}
        f_F &= \frac{\left(
          -\frac{\partial P}{\partial x}
        \right)B}{\frac{1}{2}\rho\left<u_x\right>\cdot\frac{B^2}{3\mu}\left(-\frac{\partial P}{\partial x}\right)} \nonumber\\
        &= \frac{6\mu}{\rho\left<u_x\right>B} \nonumber\\
        &= \frac{24\mu}{\rho\left<u_x\right>4B}
      \end{align}
      得到Fanning Friction Factor 與雷諾數的關係
      \begin{equation}
        \boxed{f_F = \frac{24}{\text{Re}}}
      \end{equation}
  \end{enumerate}
  \item 例2,同例1,但將\fbox{兩平板立起來},流體改為由上往下流入,$x$方向向下,$y$方向向右,2D\\
    可以注意到前面改變的地方只有,在Shell-balance時,要多加一項重力影響,也就是整個控制體積受到的力\\
    另外注意Slit是指兩板之間,而Slide是單板之上,邊界是不一樣的
    \begin{equation}
      \rho g \cdot dx\cdot dy\cdot dz
    \end{equation}
    而同除$dx\cdot dy\cdot dz$後,下一步的Balance Equation中,就會變成
    \begin{equation}
      \frac{\tau_{yx}}{\partial y} = -\frac{\partial P}{\partial x} + \rho g \label{eq:ch2_laminar_flow_gravity_balance}
    \end{equation}
    定義Modified Pressure Drop,$\mathbb{P}$:
    \begin{equation}
      \boxed{\mathbb{P} = P \pm \rho g x}
    \end{equation}
    這樣就能在微分之後,變成(\ref{eq:ch2_laminar_flow_gravity_balance})的右邊兩項併成一項
    \begin{equation}
      -\frac{\partial \mathbb{P}}{\partial x} = -\frac{\partial P}{\partial x} \mp \rho g 
    \end{equation}
    注意正負號的選擇,若流體是由上往下流入,重力會幫忙推動流體前進,故選加號\\
    反之,若流體是由下往上流入,重力會阻擋流體前進,故選減號\\
    取代後就會變成
    \begin{equation}
      \frac{\tau_{yx}}{\partial y} = -\frac{\partial \mathbb{P}}{\partial x}
    \end{equation}
    而之後的所有步驟,可以發現原先的$-\frac{\partial P}{\partial x}$都沒有動到\\
    故最後的結果,只是把上一節中的$\frac{\partial P}{\partial x}$換成$\frac{\partial \mathbb{P}}{\partial x}$即可\\
    另外,若壓力梯度是常數,則
    \begin{equation}
      - \frac{\mathbb{P}}{\partial x} = \frac{P_L-P_0}{L} + \rho g
    \end{equation}
    另外也可以注意到Fanning Friction Factor也是一樣的,只是$\text{Re}$的定義會有所不同
  \item 例3: 同例2,但改為圓管,$R$為半徑,長度$L$,流體由上往下流入,$z$方向向下,$r$方向向外,3D
    \begin{equation}
      \nabla \cdot \vec {\bm u} = 0 \Rightarrow \frac{1}{r}\frac{\partial}{\partial r}(r u_r) + \frac{1}{r}\frac{\partial u_{\theta}}{\partial \theta} +
      \frac{\partial u_z}{\partial z} = 0
    \end{equation}
    同樣利用Fully-developed,$u_{\theta}=0$,且$u_r = 0$,代入
    \begin{equation}
      \frac{\partial u_z}{\partial z} = 0
    \end{equation}
    得出
    \begin{equation}
      u_z = f(\cancelto{S.S.}{t},r,\cancelto{2D}{\theta},\cancelto{\text{上式}}{z}) = u_z(r)~\text{only}
    \end{equation}
    Shell 會是一個邊長分別為$r dr,~d\theta,~dz$的立方體\\
    Shell balance的重力項:
    \begin{equation}
      \rho g \cdot r \cdot dr \cdot d\theta \cdot dz
    \end{equation}
    平衡式:
    \begin{equation}
      - \frac{\partial \left(\tau_{rz}\cdot r\right)}{\partial r} -
      \cancelto{\text{F.D.}}{\frac{\partial \tau_{\theta z}}{\partial \theta}} -
      \frac{\partial \left(P\cdot r\right)}{\partial z} -
      \frac{\partial}{\partial z} \cancelto{u_z\neq f(z)}{\left(\rho r u_z u_z\right)} = 0
    \end{equation}
    化簡後得到:
    \begin{equation}
      \frac{\partial\left(\tau_{rz}\cdot r\right)}{\partial r} = 
      \left(-\frac{\partial P}{\partial z}\right)r + \rho g r = \left[
        -\left(\frac{\partial P}{\partial z}\right) + \rho g
      \right]r
    \end{equation}
    一樣定義Modified Pressure $\mathbb{P}$
    \begin{equation}
      \mathbb{P} = P + \rho g z
    \end{equation}
    使得$\frac{\partial \mathbb{P}}{\partial z} = \frac{\partial P}{\partial z} + \rho g$,代入得到
    \begin{equation}
      \frac{\partial\left(\tau_{rz}\cdot r\right)}{\partial r} = \left(-\frac{\partial \mathbb{P}}{\partial z}\right)r
    \end{equation}
    同樣若壓力梯度是常數,則
    \begin{equation}
      -\frac{\partial \mathbb{P}}{\partial z} = \frac{P_L-P_0}{L} + \rho g
    \end{equation}
    後面代入牛頓流體後,經過相同的步驟會得到以下結果:
    \begin{align}
      \tau_{rz} &= \boxed{\frac{1}{2}\left(-\frac{\mathbb P}{\partial z}\right)r} \label{eq:ch2_laminar_flow_circular_pipe_shear_stress}\\
      u_z & = \boxed{\frac{R}{4\mu}\left(-\frac{\partial \mathbb P}{\partial z}\right)\left[
        1-\left(\frac{r}{R}\right)^2
      \right]} \\
      u_{\text{max}} &= \boxed{\frac{R^2}{4\mu}\left(-\frac{\partial \mathbb P}{\partial z}\right)} \\
      \left<u_z\right> &= \frac{R^2}{8\mu}\left(-\frac{\partial \mathbb P}{\partial z}\right)  = \boxed{\frac{1}{2}u_{\text{max}}}\\
      Q &= \boxed{\frac{\pi R^4}{8\mu}\left(-\frac{\partial \mathbb P}{\partial z}\right) ~\text{(Hagen-Poiseuille equation)}} \\
      \text{Re} &= \frac{\rho R^3\left(-\frac{\partial \mathbb P}{\partial z}\right)}{4\mu^2}\\
      F_z &= \boxed{\left(-\frac{\partial \mathbb P}{\partial z}\right)\pi R^2L} \\
      f &= \boxed{\frac{16}{\text{Re}}}
    \end{align}
    其中,Hagen-Poiseuille equation可以用作測量黏度的方法,在已知尺寸的圓管中,只要知道$Q$,和兩端的壓力差,即可計算出黏度$\mu$\\
    P.S. Hagen-Poiseuille equation的適用條件:
    \begin{itemize}
      \item Laminar flow
      \item Steady-state
      \item Incompressible flow
      \item Newtonian fluid
      \item Flow in a circular pipe
    \end{itemize}
  \item Hagen Poiseuille equation的廣義應用:
    \begin{equation}
      \boxed{Q = \frac{\pi R^4}{8\mu}\left(\frac{\mathbb P}{\partial z}\right)}
    \end{equation}
    Q - $\mu$ - $(P_0-P)$ 三者之間的關係式
    \begin{itemize}
      \item 已知$Q$,和$(P_0-P)$,求$\mu$
      \item 已知$\mu$,和$(P_0-P)$,求$Q$
      \item 已知$Q$,和$\mu$,求$(P_0-P)$
    \end{itemize}
    當看到題目給予一些圓管的測試結果時,就可以想到用Hagen-Poiseuille equation來解題
  \item Psuedo Steady-state或Quasi Steady-state:\\
    Steady-state是指不論長時間或短時間,系統與時間無關\\
    Psuedo Steady-state是指系統在\fbox{短時間}內可以視為Steady-state
  \item Efflux time: 容器內流體流完所需時間
  \begin{figure}[H]
    \centering
    \begin{tikzpicture}[>=Latex, line cap=round, line join=round, thick]
      \draw (3, 0) arc(0:-180:0.5 and 0.2);
      \draw [dashed] (3,0) arc(0:180:0.5 and 0.2);
      \draw (2,0) -- (2,5);
      \draw (3,0) -- (3,5);
      \fill[white, draw=black] (2.5,5) ellipse (2.5 and 1);
      \fill[white] (0,5) rectangle (5,6.5);
      \draw [dashed] (2,0) -- (2,5);
      \draw [dashed] (3,0) -- (3,5);
      \draw (5,5) arc(0:-180:2.5 and 1);
      \draw [dashed] (5,5) arc(0:180:2.5 and 1);
      \draw (0,5) -- (0,8);
      \draw (2.5, 8) ellipse (2.5 and 1);
      \draw (5,5) -- (5,8);
      \draw [dashed] (2.5,5) ellipse (0.5 and 0.2);
      \draw [blue, densely dashed] (5,5.75) arc(0:-180:2.5 and 1);
      \draw [blue, densely dashed] (5,5.75) arc(0:180:2.5 and 1);
      \fill [pattern = crosshatch dots, pattern color=blue] (2.5,5) ellipse (2.5 and 1);
      \fill [pattern = crosshatch dots, pattern color=blue] (2.5,5.75) ellipse (2.5 and 1);
      \fill [pattern = crosshatch dots, pattern color=blue] (0,5) rectangle (5,5.75);
      \fill [pattern = crosshatch dots, pattern color=blue] (3,5) -- (3,0) arc(0:-180:0.5 and 0.2) -- (2,5) -- cycle;
      \draw [->] (1.5,3) -- (2,3);
      \draw [->] (3.5,3) -- (3,3);
      \node [anchor=west] at (3,3.5) {$2R_0$};
      \draw [<->] (5.3,5.75) -- (5.3,5) node[midway, right] {$h(t)$};
      \draw [<->] (2.5,8) -- (5, 8) node[midway, above] {$R$};
      \draw [<->] (-0.3, 0) -- (-0.3, 5) node[midway, left] {$L$};
      \draw [<->] (-0.3,5) -- (-0.3,8) node[midway, left] {$H$};
      \draw [dashed] (-0.3,0) -- (2,0);
      \draw [dashed] (-0.3,5) -- (2,5);
      \draw [dashed] (-0.3,8) -- (0,8);
      \fill [red, draw=black] (2.4, 5) -- (2.4, 4.5) -- (2.25, 4.5) -- (2.5, 4.2) -- (2.75, 4.5) -- (2.6, 4.5) -- (2.6, 5) -- cycle;
    \end{tikzpicture}
    \caption{Efflux time示意圖}
  \end{figure}
  \begin{itemize}
    \item 將流出分成兩個階段,第一段為清空槽
      \begin{equation}
        0 \leq h \leq H
      \end{equation}
      由Mass Balance可知:
      \begin{align}
        \cancel{\text{in}} -\text{out} +\cancel{\text{generation}} &= \text{accumulation} \nonumber\\
        0- \dot m_{z} +0 &= \frac{dM}{dt} \nonumber\\
        -\rho \left<u_z\right>\pi R_0^2 &= \frac{d}{dt} \left(\rho \pi R^2 h\right) \nonumber\\
        -\left<u_z\right> R_0^2 &= R^2 \frac{dh}{dt} \label{eq:ch2_laminar_flow_efflux_mass_balance}
      \end{align}
    \item 由於是圓管,故可以使用Hagen-Poiseuille equation求出$\left<u_z\right>$\\
      P.S. 可以假設為\fbox{Pseudo Steady-state},因為Mass Balance中,右側仍有$dt$\\
      所以可以視為一個極短時間內下,會符合Hagen-Poiseuille equation,然後積分
      \begin{align}
        Q &= \frac{\pi R_0^4}{8\mu}\left(-\frac{\partial \mathbb P}{\partial z}\right) \nonumber\\
        &= \frac{\pi R_0^4}{8\mu}\left[
          -\frac{\partial P}{\partial z} + \rho g
        \right] \nonumber\\
        &= \frac{\pi R_0^4}{8\mu}\left[
          -\frac{\int_{P_1}^{P_2}dP}{\int_{0}^{L}dz} + \rho g
        \right] \nonumber\\
        \left<u_z\right>\pi R_0^2 &= \frac{\pi R_0^4}{8\mu}\left[
          \frac{P_1-P_2}{L} + \rho g
        \right] \nonumber\\
        \left<u_z\right> &= \frac{R_0^2}{8\mu}\left[
          \frac{P_{\text{atm}}+\rho g h - P_{\text{atm}}}{-L} + \rho g
        \right] \nonumber\\
        &= \frac{R_0^2\rho g(h+L)}{8\mu L}
      \end{align}
      P.S. 由$\left<u_z\right>$可以看出,殘餘高度與平均速度成正比\\
      代入(\ref{eq:ch2_laminar_flow_efflux_mass_balance})
      \begin{equation}
        -\frac{R_0^2\rho g(h+L)}{8\mu L} R_0^2 = R^2 \frac{dh}{dt}
      \end{equation}
      分離變數
      \begin{equation}
        dt = -\frac{8\mu L R^2}{\rho g R_0^4} \frac{1}{h+L} dh
      \end{equation}
      積分
      \begin{equation}
        \int_0^{t} dt = -\frac{8\mu L R^2}{\rho g R_0^4} \int_H^{h} \frac{1}{h+L} dh
      \end{equation}
      得到
      \begin{equation}
        \boxed{t = \frac{8\mu L R^2}{\rho g R_0^4} \ln\left(\frac{H+L}{h+L}\right)}
      \end{equation}
      此式有三個用途
      \begin{itemize}
        \item 已知$H$,求流完所需時間$t_{\text{eff}}$,當$h=0$
        \item 已知$t$,求剩餘高度$h$
        \item 已知$h$,求達到此高度所需時間$t$
      \end{itemize}
      當$h=0$時,得到Efflux time(槽體流完所需時間)
      \begin{equation}
        \boxed{t_{\text{eff}} = \frac{8\mu L R^2}{\rho g R_0^4} \ln\left(\frac{H+L}{L}\right)}
      \end{equation}
    \end{itemize}
  \item 同上個舉例,但流體改為紊流時
    \begin{itemize}
      \item 改以紊流的經驗假設:\\
        紊流的整體Shear stress,為$r$為水力半徑時的Shear stress\\
        P.S. 水力半徑為
        \begin{equation}
          r_H = \frac{A}{P} = \frac{\pi R_0^2}{2\pi R_0} = \frac{R_0}{2}
        \end{equation}
      \item 使用Fanning Friction Factor來計算
        \begin{equation}
          f = \frac{\tau_s}{\frac{1}{2}\rho \left<u_z\right>^2}
        \end{equation}
      \item 由上式移項,得到平均速度$\left<u_z\right>$
        \begin{equation}
          \left<u_z\right> = \sqrt{\frac{2\tau_s}{\rho f}} = \sqrt{\frac{2\tau_{rz}\big|_{r=r_H=\frac{R_0}{2}}}{\rho f}} \label{eq:ch2_laminar_flow_efflux_turbulent_velocity}
        \end{equation}
      \item 求出殘留高度與時間的關係$h(t)$\\
        由Hagen-Poiseuille推導過程中的$\tau_{rz}$(\ref{eq:ch2_laminar_flow_circular_pipe_shear_stress}),求得$\tau_{rz}\big|_{r=r_H=\frac{R_0}{2}}$:
        \begin{align}
          \tau_{rz}\big|_{r=\frac{R_0}{2}} &= \frac{1}{2}\left(-\frac{\partial \mathbb P}{\partial z}\right)\frac{R_0}{2} \nonumber\\
          &= \frac{R_0}{4}\left[
            -\frac{P_{\text{atm}}+\rho g h - P_{\text{atm}}}{L} + \rho g
          \right] \nonumber\\
          &= \frac{R_0 \rho g (h+L)}{4L}
        \end{align}
        代入(\ref{eq:ch2_laminar_flow_efflux_turbulent_velocity})
        \begin{equation}
          \left<u_z\right> = \sqrt{\frac{2\cdot \frac{R_0 \rho g (h+L)}{4L}}{\rho f}} = \boxed{\sqrt{\frac{R_0 g (h+L)}{2L f}}}
        \end{equation}
        P.S. 由$\left<u_z\right>$可以看出,\fbox{平均速度與殘餘高度的平方根成正比},與層流時不同\\
        再代入Mass Balance(\ref{eq:ch2_laminar_flow_efflux_mass_balance})
        \begin{equation}
          -\sqrt{\frac{R_0 g (h+L)}{2L f}} R_0^2 = R^2 \frac{dh}{dt}
        \end{equation}
        分離變數
        \begin{equation}
          dt = -\frac{R^2}{R_0^2}\sqrt{\frac{2L f}{R_0 g}} \frac{1}{\sqrt{h+L}} dh
        \end{equation}
        積分
        \begin{equation}
          \int_0^{t} dt = -\frac{R^2}{R_0^2}\sqrt{\frac{2L f}{R_0 g}} \int_H^{h} \frac{1}{\sqrt{h+L}} dh
        \end{equation}
        得到
        \begin{equation}
          \boxed{t = \frac{2 R^2}{R_0}\sqrt{\frac{2 f}{g}} \left(\sqrt{H+L} - \sqrt{h+L}\right)}
        \end{equation}
      \item 最後,當$h=0$時,得到紊流下的Efflux time
        \begin{equation}
          \boxed{t_{\text{eff}} = \frac{2 R^2}{R_0}\sqrt{\frac{2 f}{g}} \left(\sqrt{H+L} - \sqrt{L}\right)}
        \end{equation}
    \end{itemize}
  \item 同上個舉例,但孔很小致使流出液體不連續時
    \begin{itemize}
      \item 無法使用層流或紊流的假設,而須以能量平衡來求解$\left<u_z\right>$
      \item 使用Bernoulli Equation(能量平衡式)來求解
      \begin{equation}
        0 = \underbrace{\frac{1}{2g_c}\left(u_1^2-u_2^2\right)}_{\text{損失動能}} 
        + \underbrace{\frac{g}{g_c}\left(z_1-z_2\right)}_{\text{損失位能}}
        + \underbrace{\frac{1}{\rho}\left(P_1-P_2\right)}_{\text{損失壓力能}}
        - \underbrace{h_f}_{\text{摩擦}}
      \end{equation}
      也就是將摩擦過程所損失的能量,表現於動能、位能、壓力能的損失上\\
      同(\ref{eq:diff_bernoulli}),後續章節會詳細介紹:
      \begin{equation}
        h_f = 4 f_f \frac{L}{D_H} \frac{\left<u\right>^2}{2g_c}
      \end{equation}
      注意這裡又變成水力直徑$D_H$了
      \item 求出平均速度$\left<u_2\right>$:\\
      代入Bernoulli Equation\\
      令槽體上方為點1,槽的出孔口為點2
      \begin{equation}
        0 = \frac{1}{2g_c}\left(\left<u_1\right>^2-\left<u_2\right>^2\right) 
        + \frac{g}{g_c}\left(z_1-z_2\right) 
        + \frac{1}{\rho}\left(P_1-P_2\right) 
        - 4 f_f \frac{L}{D_H} \frac{\left<u_2\right>^2}{2g_c}
      \end{equation}
      而因為孔非常小,$A_1 \gg A_2$,故可以忽略$u_1$($u_1^2\ll u_2^2$)\\
      如果要寫清楚這件事,可以由Mass Balance
      \begin{equation}
        \dot m_1 = \dot m_2 \Rightarrow \rho A_1 u_1 = \rho A_2 u_2 \Rightarrow u_1 = \frac{A_2}{A_1}u_2
      \end{equation}
      若$A_1 \gg A_2$,則$u_1 \ll u_2$\\
      另外因為孔非常小,可以視為\fbox{Nozzle},水用滴的,可以假想下方的管子$L\to 0$\\
      而$z_1 -z_2 = h$,$P_1$和$P_2$皆為大氣壓\\
      故Bernoulli Equation化簡為:
      \begin{equation}
        0 = - \frac{1}{2g_c}\left<u_2\right>^2 
        + \frac{g}{g_c}h 
      \end{equation}
      移項後得到
      \begin{equation}
        \boxed{\left<u_z\right> = \sqrt{2gh}}
      \end{equation}
      \item 代回Mass Balance(\ref{eq:ch2_laminar_flow_efflux_mass_balance})求出殘留高度與時間的關係$h(t)$
      \begin{equation}
        -\sqrt{2gh} R_0^2 = R^2 \frac{dh}{dt}
      \end{equation}
      分離變數
      \begin{equation}
        dt = -\frac{R^2}{R_0^2} \frac{1}{\sqrt{2gh}} dh
      \end{equation}
      積分
      \begin{equation}
        \int_0^{t} dt = -\frac{R^2}{R_0^2} \frac{1}{\sqrt{2g}} \int_H^{h} \frac{1}{\sqrt{h}} dh
      \end{equation}
      得到
      \begin{equation}
        \boxed{t = \frac{R^2}{R_0^2} \sqrt{\frac{2}{g}} \left(\sqrt{H} - \sqrt{h}\right)}
      \end{equation}
      \item 最後,當$h=0$時,得到孔很小致使流出液體不連續時的Efflux time
      \begin{equation}
        \boxed{t_{\text{eff}} = \frac{R^2}{R_0^2} \sqrt{\frac{2}{g}} \sqrt{H}}
      \end{equation}
    \end{itemize}
  \item 同上個舉例,但是是圓錐,一樣假設下方孔很小
  \begin{figure}[H]
    \centering
    \begin{tikzpicture}[>=Latex, line cap=round, line join=round, thick]
      \draw[dashed] (3,0) -- (2.5, 1);
      \draw[dashed] (3,0) -- (3.5, 1);
      \draw[dashed] (3.5, 1) arc(0:180:0.5 and 0.2);
      \draw (3.5,1) arc(0:-180:0.5 and 0.2);
      \draw [->] (2, 1) -- (2.5,1);
      \draw [->] (4, 1) -- (3.5,1);
      \node [anchor=north west] at (3.5,1) {$2R_0$};
      \draw (2.5,1) -- (0,6);
      \draw (3.5,1) -- (6,6);
      \draw (3,6) ellipse (3 and 1);
      \draw[dashed, blue] (3, 4) ellipse (2 and 0.6);
      \fill [pattern = crosshatch dots, pattern color=blue] (3,4) ellipse (2 and 0.6);
      \fill [pattern = crosshatch dots, pattern color=blue] (2.5,1) -- (3.5,1) arc(0:-180:0.5 and 0.2) -- cycle;
      \fill [pattern = crosshatch dots, pattern color=blue] (2.5,1) -- (3.5,1) -- (5,4) arc(0:-180:2 and 0.6) -- cycle;
      \draw[dashed] (3,-0.5) -- (3,7.5);
      \draw [<->] (3,6) -- (6,6) node[midway, above] {$R$};
      \draw [<->] (-0.3,1) -- (-0.3,6) node[midway, left] {$H$};
      \draw [<->] (-0.3,1) -- (-0.3,0) node[midway, left] {$h_2$};
      \draw [<->] (5.3,4) -- (5.3,1) node[midway, right] {$h(t)$};
      \draw [<->] (3,4) -- (5,4) node[midway, above] {$r(h)$};
      \draw [loosely dotted] (-0.5,1) -- (5.5,1);
      \draw [loosely dotted] (-0.5,6) -- (3,6);
      \draw [loosely dotted] (5.3,4) -- (3,4);
      \draw [loosely dotted] (-0.5,0) -- (3,0);
    \end{tikzpicture}
    \caption{圓錐體流出示意圖}
  \end{figure}
  \begin{itemize}
    \item 在質量平衡式的Accumulation項,就不能是$\rho \pi R^2 h$,而是會跟高度有關的體積$V(h)$
    \begin{equation}
      V(h) = \frac{1}{3}\pi r^2 h - \frac{1}{3}\pi R_0^2*h_2
    \end{equation}
    所以質量平衡式會變成:
    \begin{align}
      \cancel{\text{in}} -\text{out} +\cancel{\text{generation}} &= \text{accumulation}\nonumber\\
      0 - \rho \left<u_z\right> \pi R_0^2 &= \frac{d}{dt}\left(\rho V(h)\right) \nonumber\\
      - \left<u_z\right> R_0^2 &= \frac{1}{3} \frac{d}{dt} \left[
        r^2 h - R_0^2 h_2
      \right] \label{eq:ch2_laminar_flow_conical_efflux_mass_balance}
    \end{align}
    利用\fbox{相似形},將$r$表示成$h$的函數:
    \begin{equation}
      \frac{R_0}{r} = \frac{h_2}{h+h_2} \implies  r = R_0 \left(\frac{h+h_2}{h_2}\right) \label{eq:ch2_laminar_flow_conical_efflux_similar_triangle}
    \end{equation}
    代入質量平衡式(\ref{eq:ch2_laminar_flow_conical_efflux_mass_balance}):
    \begin{align}
      - \left<u_z\right> R_0^2 &= \frac{1}{3} \frac{d}{dt} \left[
        R_0^2 \left(\frac{h+h_2}{h_2}\right)^2 h - R_0^2 h_2
      \right] \nonumber\\
      &= \frac{R_0^2}{3} \frac{d}{dt} \left[
        \frac{(h+h_2)^2}{h_2^2} h - h_2
      \right] \nonumber\\
      &= \frac{R_0^2}{3} \frac{d}{dt} \left[
        \frac{h^3 + 2h_2 h^2 + h_2^2 h - h_2^3}{h_2^2}
      \right] \nonumber\\
      &= \frac{R_0^2}{3 h_2^2} \frac{d}{dt} \left[
        h^3 + 2h_2 h^2 + h_2^2 h - h_2^3
      \right] \nonumber\\
      &= \frac{R_0^2}{3 h_2^2} \left(
        3h^2 + 4h_2 h + h_2^2
      \right) \frac{dh}{dt} \label{eq:ch2_laminar_flow_conical_efflux_mass_balance_final}
    \end{align}
    \item 因為孔很小,故使用Bernoulli Equation,能量平衡求出$\left<u_z\right>$
    \begin{equation}
      \frac{1}{2g_c}\left(\left<u_1\right>^2-\left<u_2\right>^2\right) 
      + \frac{g}{g_c}\left(z_1-z_2\right) 
      + \frac{1}{\rho}\left(P_1-P_2\right) 
      - h_f =0
    \end{equation}
    由於孔非常小,$A_1 \gg A_2$,故可以忽略$u_1$($u_1^2\ll u_2^2$)\\
    另外因為孔非常小,可以視為\fbox{Nozzle},水用滴的,可以假想下方的管子$L\to 0$,忽略$h_f$\\
    而$z_1 -z_2 = h$,$P_1$和$P_2$皆為大氣壓\\
    故Bernoulli Equation化簡為:
    \begin{equation}
      0 = - \frac{1}{2g_c}\left<u_2\right>^2 
      + \frac{g}{g_c}h 
    \end{equation}
    移項後得到
    \begin{equation}
      \boxed{\left<u_z\right> = \sqrt{2gh}} \label{eq:ch2_laminar_flow_conical_efflux_velocity}
    \end{equation}
    \item 將平均速度代入質量平衡式(\ref{eq:ch2_laminar_flow_conical_efflux_mass_balance_final})
      \begin{equation}
        - \sqrt{2gh} R_0^2 = \frac{R_0^2}{3 h_2^2} \left(
          3h^2 + 4h_2 h + h_2^2
        \right) \frac{dh}{dt}
      \end{equation}
      分離變數
      \begin{align}
        -\sqrt{2gh}\cancel{R_0^2} dt &= \frac{\cancel{R_0^2}}{3 h_2^2} \left(
          3h^2 + 4h_2 h + h_2^2
        \right) dh \nonumber\\
        dt &= -\frac{1}{3 h_2^2 \sqrt{2g}} \left(
          3h^2 + 4h_2 h + h_2^2
        \right) \cdot h^{-\frac{1}{2}} dh \nonumber\\
        &= - \frac{1}{3 h_2^2 \sqrt{2g}} \left(
          3h^{\frac{3}{2}} + 4h_2 h^{\frac{1}{2}} + h_2^2 h^{-\frac{1}{2}}
        \right) dh
      \end{align}
      積分
      \begin{equation}
        \int_0^{t} dt = - \frac{1}{3 h_2^2 \sqrt{2g}} \int_H^{h} \left(
          3h^{\frac{3}{2}} + 4h_2 h^{\frac{1}{2}} + h_2^2 h^{-\frac{1}{2}}
        \right) dh \label{eq:ch2_laminar_flow_conical_efflux_integral}
      \end{equation}
      計算右側積分
      \begin{align}
        &\int_H^{h} \left(
          3h^{\frac{3}{2}} + 4h_2 h^{\frac{1}{2}} + h_2^2 h^{-\frac{1}{2}}
        \right) dh \nonumber \\
        =& \left[
          \frac{6}{5} h^{\frac{5}{2}} + \frac{8}{3} h_2 h^{\frac{3}{2}} + 2 h_2^2 h^{\frac{1}{2}}
        \right]_{H}^{h} \nonumber\\
        =& \left(
          \frac{6}{5} h^{\frac{5}{2}} + \frac{8}{3} h_2 h^{\frac{3}{2}} + 2 h_2^2 h^{\frac{1}{2}}
        \right) - \left(
          \frac{6}{5} H^{\frac{5}{2}} + \frac{8}{3} h_2 H^{\frac{3}{2}} + 2 h_2^2 H^{\frac{1}{2}}
        \right)
      \end{align}
      代入式(\ref{eq:ch2_laminar_flow_conical_efflux_integral})
      \begin{align}
        t &= - \frac{1}{3 h_2^2 \sqrt{2g}} \left[
          \left(
            \frac{6}{5} h^{\frac{5}{2}} + \frac{8}{3} h_2 h^{\frac{3}{2}} + 2 h_2^2 h^{\frac{1}{2}}
          \right) - \left(
            \frac{6}{5} H^{\frac{5}{2}} + \frac{8}{3} h_2 H^{\frac{3}{2}} + 2 h_2^2 H^{\frac{1}{2}}
          \right)
        \right] \nonumber\\
        &= \frac{1}{3 h_2^2 \sqrt{2g}} \left[
          \left(
            \frac{6}{5} H^{\frac{5}{2}} + \frac{8}{3} h_2 H^{\frac{3}{2}} + 2 h_2^2 H^{\frac{1}{2}}
          \right) - \left(
            \frac{6}{5} h^{\frac{5}{2}} + \frac{8}{3} h_2 h^{\frac{3}{2}} + 2 h_2^2 h^{\frac{1}{2}}
          \right)
        \right]
      \end{align}
    \item 將上式整理後得到殘留高度與時間的關係$t(h)$,以根號表示
      \begin{equation}
        \boxed{t = \frac{1}{3 h_2^2 \sqrt{2g}} \left[
          \left(
            \frac{6}{5} \sqrt{H^5} + \frac{8}{3} h_2 \sqrt{H^3} + 2 h_2^2 \sqrt{H}
          \right) - \left(
            \frac{6}{5} \sqrt{h^5} + \frac{8}{3} h_2 h^{\frac{3}{2}} + 2 h_2^2 \sqrt{h}
          \right)
        \right]}
      \end{equation}
      \item 最後,當$h=0$時,得到圓錐體流出時的Efflux time
      \begin{equation}
        \boxed{t_{\text{eff}} = \frac{1}{3 h_2^2 \sqrt{2g}} \left(
          \frac{6}{5} \sqrt{H^5} + \frac{8}{3} h_2 \sqrt{H^3} + 2 h_2^2 \sqrt{H}
        \right)}
      \end{equation}
      \item 若$h_2$很小,括號內第二、三項可以忽略,得到近似式
      \begin{equation}
        \boxed{t_{\text{eff}} \approx \frac{2}{5 h_2^2 \sqrt{2g}} \sqrt{H^5}}
      \end{equation}
      P.S. 於步驟(\ref{eq:ch2_laminar_flow_conical_efflux_similar_triangle})中\\
      若$h_2$很小,可以直接令$r(h) = R_0 \frac{h}{h_2}$,後續推導相同
      \begin{equation}
      \frac{R_0}{r} = \frac{h_2}{h+h_2} \implies  r = R_0 \left(\frac{h+h_2}{h_2}\right) \approx R_0 \frac{h}{h_2}
    \end{equation}
    \end{itemize}
\end{itemize}
\end{CJK*}
\end{document}