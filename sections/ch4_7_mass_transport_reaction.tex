\documentclass[../main.tex]{subfiles}
\begin{document}
\begin{CJK*}{UTF8}{bkai}
\subsection{質量輸送,有反應發生}
\begin{itemize}
  \item Steady-State, 1-D molecular diffusion, heterogeneous reaction
  \begin{figure}[H]
    \centering
    \begin{tikzpicture}[>=Latex, line cap=round, line join=round, thick]
      \draw (-2,-0.3) -- (0,-0.3) -- (0, -2.665) -- (10, -2.665)
      -- (10, -0.3) -- (12, -0.3);
      \draw (-2, 0.3) -- (0, 0.3) -- (0, 2.665) -- (10, 2.665)
      -- (10, 0.3) -- (12, 0.3);
      \draw[->] (-3,0) -- (-2,0) node[midway, above] {$A$};
      \draw[->] (12,0) -- (13,0) node[midway, above] {$A+B$};
      \node[anchor=south] at (5, 2.665) {Reaction: $2A_{(g)}\rightarrow B_{(g)}$};
      \def\r{0.5}           % Radius
      \def\diam{1.0}        % Diameter
      \def\dx{1.0}          % Horizontal spacing (1 * diameter)
      \def\dy{0.866025}     % Vertical spacing (sin(60) * diameter)
      \def\yBottom{-2.665}  % The bottom y-coordinate of your box
      \foreach \j in {0,...,5} {
        \pgfmathsetmacro{\y}{\yBottom + \r + \j*\dy}
        \ifodd\j 
            \foreach \i in {0,...,8} {
                \pgfmathsetmacro{\x}{\diam + \i*\dx}
                \draw[blue] (\x,\y) circle (\r);
            }
        \else 
            \foreach \i in {0,...,9} {
                \pgfmathsetmacro{\x}{\r + \i*\dx}
                \draw[blue] (\x,\y) circle (\r);
            }
        \fi
    }
    \end{tikzpicture}
    \caption{heterogeneous reaction example}
  \end{figure}
  \begin{itemize}
    \item 假設$A$在氣相中擴散到固體表面,並在表面發生反應生成$B$\\
    並假設這個\fbox{反應瞬間完成},也就是它不是個反應工程問題,而是個質傳問題\\
    並且假設\fbox{觸媒很大顆可以視為平面}\\
    所以在控制體積中,實際上並沒有反應發生,稱為\fbox{非均相反應}\\
    這時可以將問題簡化成
    \begin{figure}[H]
      \centering
      \begin{tikzpicture}[>=Latex, line cap=round, line join=round, thick]
        \fill[pattern=north east lines, pattern color=red, draw=red] (0,0) rectangle (10,-0.3);
        \node[anchor=east] at (0,0) {$z=\delta$};
        \draw (0,3) -- (10, 3);
        \node[anchor=east] at (0,3) {$z=0$};
        \draw[->] (2.5, 3) -- (2.5, 0) node[midway, left] {$N_A$};
        \draw[->] (7.5, 0) -- (7.5, 3) node[midway, right] {$N_B$};
        \node[anchor=south east] at (2.5,0) {$y_{As}$};
        \node[anchor=north east] at (2.5,3) {$y_{A0}$};
        \node[anchor=south] at (5,0) {2A $\rightarrow$ B};
        \node[anchor=south] at (5,3) {A,B 濃度不再變化};
        \draw[dashed] (0,1.75) rectangle (10,2.25);
        \node at (5,2) {\tiny C.V.};
        \node[anchor=south] at (5, 2.25) {A,B Counterdiffusion};
      \end{tikzpicture}
      \caption{non-homogeneous reaction simplification}
    \end{figure}
    因反應在觸媒表面發生,A會被消耗,於氣相中產生濃度差,產生質傳\\
    而反應生成了B,B與氣相也產生濃度差,產生質傳
    \item 寫出Fick's 1st Law:
      \begin{equation}
        N_A = -D_{AB}C\frac{d y_A}{dz}+ y_A(N_A + N_B)
      \end{equation}
      由於反應是$2A \rightarrow B$,所以$N_B = -\frac{1}{2}N_A$,代入上式得
      \begin{equation}
        N_A = -D_{AB}C\frac{d y_A}{dz}+ \frac{1}{2}y_A N_A \label{eq:ch4_7_ficks_law_reaction_start}
      \end{equation}
      此為1階微分方程式,整理得
      \begin{align}
        \int_0^\delta N_A dz &= -D_{AB}C \int_{y_{A0}}^{y_{As}} \frac{dy_A}{1-\frac{1}{2}y_A} \nonumber \\
        (u=1-\frac{1}{2}y_A) & (du = -\frac{1}{2}dy_A) \nonumber\\
        N_A\delta &= 2D_{AB}C \int_{u_0}^{u_s} \frac{du}{u} \nonumber \\
        N_A\delta &= 2D_{AB}C [\ln|u|]_{u_0}^{u_s} \nonumber \\
        N_A\delta &= -2D_{AB}C \ln\left(\frac{1-\frac{1}{2}y_{As}}{1-\frac{1}{2}y_{A0}}\right) \nonumber \\
        \therefore \quad N_A &= \frac{-2D_{AB}C}{\delta}
         \ln\left(\frac{1-\frac{1}{2}y_{As}}{1-\frac{1}{2}y_{A0}}\right) \label{eq:ch4_7_ficks_law_reaction}
      \end{align}
      因為\fbox{瞬間反應},所以在表面$y_{As}=0$,代入上式得
      \begin{equation}
        \boxed{N_A = \frac{2D_{AB}C}{\delta} \ln\left(\frac{1}{1-\frac{1}{2}y_{A0}}\right)}
      \end{equation}
      P.S. 如果為1級反應
      \begin{equation}
        N_A\big|_{z=\delta} = k C_{As} = k C y_{As} \implies y_{As} = \frac{N_A(\delta)}{kC}
      \end{equation}
      代入(\ref{eq:ch4_7_ficks_law_reaction})得
      \begin{equation}
        N_A = \frac{-2D_{AB}C}{\delta}
         \ln\left(\frac{1-\frac{1}{2}\frac{N_A(\delta)}{kC}}{1-\frac{1}{2}y_{A0}}\right)
      \end{equation}
      這時$N_A$無法直接求出,需要用Try-and-Error的方法求解
    \item 求濃度分布:\\
    透過Equation of Continuity可知,在steady-state下,質傳通量不變
    \begin{equation}
      \cancel{\frac{\partial C_A}{\partial t}} +\nabla \cdot N_A = \cancel{R_A}
      \implies \nabla \cdot N_A = 0
    \end{equation}
    在1-D下,將Control Volume的$N_A$代入Fick's 1st Law\\
    P.S. 不能用剛剛(\ref{eq:ch4_7_ficks_law_reaction})式,因為那是整個區域的通量,不是局部的通量\\
    要用在他積分之前的式子(\ref{eq:ch4_7_ficks_law_reaction_start})
    \begin{align}
      N_A &= -D_{AB}C\frac{d y_A}{dz}+ \frac{1}{2}y_A N_A \nonumber\\
      N_A &= \frac{-D_{AB}C}{1-\frac{1}{2}y_A}\frac{d y_A}{dz} \label{eq:ch4_7_ficks_law_reaction_local}
    \end{align}
    再代入$\frac{d N_A}{dz} = 0$中
    \begin{align}
      \frac{d}{dz}\left(
        \frac{-D_{AB}C}{1-\frac{1}{2}y_A}\frac{d y_A}{dz}
      \right) &= 0 \nonumber \\
      \implies \quad \frac{d}{dz}\left(
        \frac{1}{1-\frac{1}{2}y_A}\frac{d y_A}{dz}
      \right) &= 0
      \implies \quad 
      \frac{1}{1-\frac{1}{2}y_A}\frac{d y_A}{dz} = C_1 \nonumber\\
      \implies \quad 
      2 \ln \frac{1}{1-\frac{1}{2}y_A} &= C_1 z + C_2 \label{eq:ch4_7_concentration_distribution_reaction}
    \end{align}
    而邊界條件
    \begin{align}
      y_A(0) &= y_{A0} \\
      y_A(\delta) &= y_{As}
    \end{align}
    代入(\ref{eq:ch4_7_concentration_distribution_reaction})式中可求得$C_1, C_2$
    \begin{align}
      y_A(0) = y_{A0} &\implies 2 \ln \frac{1}{1-\frac{1}{2}y_{A0}} = C_2 \\
      y_A(\delta) = y_{As} &\implies 2 \ln \frac{1}{1-\frac{1}{2}y_{As}} = 
      C_1 \delta + 2 \ln \frac{1}{1-\frac{1}{2}y_{A0}} \nonumber \\
      &\implies C_1 = \frac{2}{\delta} \ln \frac{1-\frac{1}{2}y_{A0}}{1-\frac{1}{2}y_{As}}
    \end{align}
    代入(\ref{eq:ch4_7_concentration_distribution_reaction})式中,得
    \begin{align}
      2 \ln \frac{1}{1-\frac{1}{2}y_A} &= 
      \frac{2z}{\delta} \ln \frac{1-\frac{1}{2}y_{A0}}{1-\frac{1}{2}y_{As}} +
      2 \ln \frac{1}{1-\frac{1}{2}y_{A0}} \nonumber \\
      \implies \quad 
      \ln \left(
        \frac{1}{1-\frac{1}{2}y_A}
      \right) &= \frac{z}{\delta} \ln \left(
        \frac{1-\frac{1}{2}y_{A0}}{1-\frac{1}{2}y_{As}}
      \right) + \ln \left(
        \frac{1}{1-\frac{1}{2}y_{A0}}
      \right) 
    \end{align}
    整理得
    \begin{equation}
      \ln \left(
        \frac{1-\frac{1}{2}y_{As}}{1-\frac{1}{2}y_{A}}
      \right) = \frac{z}{\delta} \ln \left(
        \frac{1-\frac{1}{2}y_{As}}{1-\frac{1}{2}y_{A0}}
      \right)
    \end{equation}
  如果快速反應,$y_{As}=0$,則
  \begin{equation}
    \ln \left(
      \frac{1}{1-\frac{1}{2}y_{A}}
    \right) = \frac{z}{\delta} \ln \left(
      \frac{1}{1-\frac{1}{2}y_{A0}}
    \right)
  \end{equation}
  如果一級反應,$y_{As} = \frac{N_A(\delta)}{kC}$,則
  \begin{equation}
    \ln \left(
      \frac{1-\frac{1}{2}\frac{N_A(\delta)}{kC}}{1-\frac{1}{2}y_{A}}
    \right) = \frac{z}{\delta} \ln \left(
      \frac{1-\frac{1}{2}\frac{N_A(\delta)}{kC}}{1-\frac{1}{2}y_{A0}}
    \right)
  \end{equation}
  \end{itemize}
  \item Steady-State, 1-D molecular diffusion, homogeneous reaction
  \begin{figure}[H]
    \centering
    \begin{tikzpicture}[>=Latex, line cap=round, line join=round, thick]
      \draw (0,0) rectangle (3,4);
      \draw (0,4) -- (0,5);
      \draw (3,4) -- (3,5);
      \fill[pattern=crosshatch dots, pattern color=blue] (0,0) rectangle (3,4);
      \node[anchor=east] at (0,0) {$z=\delta$};
      \node[anchor=east] at (0,4) {$z=0$};
      \draw[->] (1.5, 4.5) -- (1.5, 3.5);
      \node[anchor=south] at (1.5,4.5) {Gas $A$};
      \node[anchor=east] at (0,2) {Liquid $B$};
      \node[anchor=west] at (3,2) {$aA_{(l)}+bB_{(l)}\rightarrow P_{(l)}$};
      \draw[dashed] (0,0.8) rectangle (3,1.2);
      \node[anchor=east] at (0,1) {C.V.};
    \end{tikzpicture}
    \caption{homogeneous reaction example}
  \end{figure}
  \begin{itemize}
    \item 以質傳視角來看,有1級勻相反應發生\\
    雖然看起來是非均相反應\\
    但實際上是$A$先變成溶液中的$A_{(l)}$,再與$B_{(l)}$反應生成$P_{(l)}$\\
    而同時液體中,有濃度差,故Control Volume中有反應發生
    \item 記得要\fbox{先解出濃度分布},再解通量
    畢竟Fick's 1st Law根本就沒考慮反應項\\
    P.S. 這題目跟反應係數無關\\
    然後這種題目只會出現一級或零級反應,因為超過解不出來
    \item 寫出Equation of Continuity:
    \begin{equation}
      \frac{\partial C_A}{\partial t} +\nabla \cdot N_A = R_A
    \end{equation}
    在steady-state下,$\frac{\partial C_A}{\partial t} = 0$,且在1-D下
    \begin{equation}
      \frac{d N_A}{dz} = R_A
    \end{equation}
    假設為1級反應,$R_A = -k C_A$,代入上式得
    \begin{equation}
      \frac{d N_A}{dz} = -k C_A
    \end{equation}
    再代入Fick's 1st Law,並假設氣體在\fbox{液體中濃度極低},$y_A \approx 0$
    \begin{align}
      N_A &= -D_{AB}\frac{d C_A}{dz} + \cancel{y_A}(N_A + N_B) \nonumber \\
      N_A &= -D_{AB}\frac{d C_A}{dz} \nonumber \\
      \implies \quad 
      \frac{d}{dz}\left(
        -D_{AB}\frac{d C_A}{dz}
      \right) = -k C_A \nonumber \\
      \implies \quad 
      D_{AB}\frac{d^2 C_A}{dz^2} - k C_A &= 0 \label{eq:ch4_7_homogeneous_reaction_diff_eq}
    \end{align}
    此為2階微分方程式,令:
    \begin{equation}
      C_A = e^{\lambda z} \implies D_{AB}\lambda^2 e^{\lambda z} - k e^{\lambda z} = 0
    \end{equation}
    整理得特徵方程式:
    \begin{equation}
      D_{AB}\lambda^2 - k = 0 \implies \lambda = \pm \sqrt{\frac{k}{D_{AB}}}
    \end{equation}
    假設為相異實數根,且邊界條件是無限的(可以消去一個常數)
    \begin{equation}
      C_A = C_1 e^{\sqrt{\frac{k}{D_{AB}}} z} + C_2 e^{-\sqrt{\frac{k}{D_{AB}}} z}
    \end{equation}
    如果相異實數,但邊界條件有限,則通解為
    \begin{equation}
      C_A = C_1 \cosh \left(
        \sqrt{\frac{k}{D_{AB}}} z
      \right) + C_2 \sinh \left(
        \sqrt{\frac{k}{D_{AB}}} z
      \right) \label{eq:ch4_7_homogeneous_reaction_concentration_distribution}
    \end{equation}
    若為重根,則通解為
    \begin{equation}
      C_A = C_1 + C_2 z
    \end{equation}
    若為複根,則通解為
    \begin{equation}
      C_A = e^{\alpha z} (C_1 \cos \beta z + C_2 \sin \beta z)
    \end{equation}
    \begin{enumerate}
      \item 相異實根,並假設$z=\delta$時,沒有殘留A:
      \begin{align}
        C_A(0) &= C_{A0} = \frac{CPy_A}{k_A}, \quad k_A(\text{亨利常數})\\
        C_A(\delta) &= 0
      \end{align}
      用(\ref{eq:ch4_7_homogeneous_reaction_concentration_distribution})式
      \begin{equation}
        C_A = C_1 \cosh \left(
        \sqrt{\frac{k}{D_{AB}}} z
      \right) + C_2 \sinh \left(
        \sqrt{\frac{k}{D_{AB}}} z
      \right)
      \end{equation}
      代入邊界條件得
      \begin{align}
        C_{A0} &= C_1\\
        0 &= C_{A0} \cosh \left(
          \sqrt{\frac{k}{D_{AB}}} \delta
        \right) + C_2 \sinh \left(
          \sqrt{\frac{k}{D_{AB}}} \delta
        \right) \nonumber \\
        \implies \quad C_2 &= -C_{A0} \coth \left(
          \sqrt{\frac{k}{D_{AB}}} \delta
        \right)
      \end{align}
      代入通解得濃度分布
      \begin{equation}
        C_A = C_{A0} \left[
          \cosh \left(
            \sqrt{\frac{k}{D_{AB}}} z
          \right) - \coth \left(
            \sqrt{\frac{k}{D_{AB}}} \delta
          \right) \sinh \left(
            \sqrt{\frac{k}{D_{AB}}} z
          \right)
        \right]
      \end{equation}
      再代入Fick's 1st Law得通量
      \begin{align}
        N_A &= -D_{AB}\frac{d C_A}{dz} \nonumber\\
        &= 
        -D_{AB} C_{A0} \sqrt{\frac{k}{D_{AB}} \left[
          \sinh \left(
            \sqrt{\frac{k}{D_{AB}}} z
          \right) - \coth \left(
            \sqrt{\frac{k}{D_{AB}}} \delta
          \right) \cosh \left(
            \sqrt{\frac{k}{D_{AB}}} z
          \right)
        \right]} \nonumber \\
        \therefore \quad N_A\big|_{z=0} &= 
         C_{A0} \sqrt{k\cdot D_{AB}} \coth \left(
          \sqrt{\frac{k}{D_{AB}}} \delta
        \right) \label{eq:ch4_7_homogeneous_reaction_flux}
      \end{align}
      P.S. 若反應速率無限快,$k \rightarrow \infty$,則
      $\coth(\infty) \rightarrow 1$,代入(\ref{eq:ch4_7_homogeneous_reaction_flux})式得
      \begin{equation}
        N_A\big|_{z=0} = C_{A0} \sqrt{k\cdot D_{AB}} \quad (k \rightarrow \infty)
      \end{equation}
      \item 假設杯子有底部,底部會殘留A,但$A$卡在那$N_A=0$:
      \begin{align}
        C_A(0) &= C_{A0} = \frac{CPy_A}{k_A}, \quad k_A(\text{亨利常數})\\
        N_A(\delta) &= 0 \implies \boxed{\frac{d C_A}{dz}\big|_{z=\delta}} = 0
      \end{align}
      第二個邊界條件跟絕熱的邊界條件很像\\
      用(\ref{eq:ch4_7_homogeneous_reaction_concentration_distribution})式
      \begin{equation}
        C_A = C_1 \cosh \left(
        \sqrt{\frac{k}{D_{AB}}} z
      \right) + C_2 \sinh \left(
        \sqrt{\frac{k}{D_{AB}}} z
      \right)
      \end{equation}
      代入邊界條件得
      \begin{align}
        C_{A0} &= C_1\\
        0 &= -C_1 \sqrt{\frac{k}{D_{AB}}} \sinh \left(
          \sqrt{\frac{k}{D_{AB}}} \delta
        \right) + C_2 \sqrt{\frac{k}{D_{AB}}} \cosh \left(
          \sqrt{\frac{k}{D_{AB}}} \delta
        \right) \nonumber \\
        \implies \quad C_2 &= C_{A0} \tanh \left(
          \sqrt{\frac{k}{D_{AB}}} \delta
        \right)
      \end{align}
      代入通解得濃度分布
      \begin{equation}
        C_A = C_{A0} \left[
          \cosh \left(
            \sqrt{\frac{k}{D_{AB}}} z
          \right) + \tanh \left(
            \sqrt{\frac{k}{D_{AB}}} \delta
          \right) \sinh \left(
            \sqrt{\frac{k}{D_{AB}}} z
          \right)
        \right]
      \end{equation}
      再代入Fick's 1st Law得通量
      \begin{align}
        N_A &= -D_{AB}\frac{d C_A}{dz} \nonumber\\
        &= 
        -D_{AB} C_{A0} \sqrt{\frac{k}{D_{AB}} \left[
          \sinh \left(
            \sqrt{\frac{k}{D_{AB}}} z
          \right) + \tanh \left(
            \sqrt{\frac{k}{D_{AB}}} \delta
          \right) \cosh \left(
            \sqrt{\frac{k}{D_{AB}}} z
          \right)
        \right]} \nonumber \\
        \therefore \quad N_A\big|_{z=0} &= 
         C_{A0} \sqrt{k\cdot D_{AB}} \tanh \left(
          \sqrt{\frac{k}{D_{AB}}} \delta
        \right) \label{eq:ch4_7_homogeneous_reaction_flux_bottom}
      \end{align}
      同樣的,若反應速率無限快,$k \rightarrow \infty$,則
      $\tanh(\infty) \rightarrow 1$,代入(\ref{eq:ch4_7_homogeneous_reaction_flux_bottom})式得
      \begin{equation}
        N_A\big|_{z=0} = C_{A0} \sqrt{k\cdot D_{AB}} \quad (k \rightarrow \infty)
      \end{equation}
    \end{enumerate}
  \end{itemize}
  \item 孔洞觸媒內部的質傳問題
\end{itemize}
\end{CJK*}
\end{document}

