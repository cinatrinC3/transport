\documentclass[../main.tex]{subfiles}
\begin{document}
\begin{CJK*}{UTF8}{bkai}
\subsection{質量輸送,Stefan Problem、Film Theory}
\begin{figure}[H]
  \centering
  \begin{tikzpicture}[>=Latex, line cap=round, line join=round, thick]
    \draw (0,6) -- (0,0) -- (2,0) -- (2,6);
    \fill[pattern=north east lines, pattern color=blue] (0,0) rectangle (2,2);
    \node[anchor=west] at (2,1) {揮發性液體A};
    \draw[->] (-1,0) -- (-1,1) node[above] {$z$};
    \draw[<->] (-0.5, 0) -- (-0.5,6) node[midway, left] {$L$};
    \draw[dashed] (-1.5,0) -- (0,0);
    \draw[dashed] (-1.5,6) -- (0,6);
    \draw[blue, dashed] (0,2) -- (2,2);
    \draw[->, red] (-2,6.5) -- (0, 6.5);
    \draw[->, red] (-2,7) -- (0,7);
    \draw[->, red] (-2,7.5) -- (0,7.5);
    \node[red, anchor=south] at (-1,7.5) {空氣B};
    \draw[decorate, decoration=snake, blue] (1.33,2) -- (1.33,5.5);
    \draw[->, blue] (1.33,5.5) -- (1.33,6) node[above] {揮發};
    \node[anchor=west] at (2,2) {$z_l(t)$: 液面};
    \draw[decorate, decoration=snake, red] (0.66,6) -- (0.66,2.5);
    \draw[->, red] (0.66,2.5) -- (0.66,2);
    \def\xShift{6}
    \def\yShift{2}
    \draw[->] (\xShift,0) -- (\xShift,5) node[above] {$X$,莫耳分率};
    \draw[->] (\xShift,0) -- (\xShift+6,0) node[right] {$z$};
    \draw[dashed] (\xShift+5,0) -- (\xShift+5,3.5);
    \node[anchor=north] at (\xShift+5,0) {$L$};
    \draw[blue, ->] (\xShift,3.8-\yShift) .. controls (\xShift+2,3.8-\yShift) and (\xShift+3.8, 3.5-\yShift) .. (\xShift+5,2.9-\yShift);
    \draw[red, <-] (\xShift, 0.5+\yShift) .. controls (\xShift+2,0.5+\yShift) and (\xShift+3.8,0.8+\yShift) .. (\xShift+5,1.4+\yShift);
    \node[anchor=east] at (\xShift, 3.8-\yShift) {$\frac{P_{\text{sat}}}{P}$};
    \node[anchor=east] at (\xShift, 0.5+\yShift) {$X_{B0}$};
    \draw[dashed] (\xShift,3.5) -- (\xShift+5,3.5);
    \node[anchor=east] at (\xShift,3.5) {1};
    \node[anchor=north east] at (\xShift,0) {0};
  \end{tikzpicture}
  \caption{Stefan Problem示意圖}
\end{figure}
\begin{itemize}
  \item 測量揮發液體擴散係數的方法\\
  液體A先吸收空氣能量、產生Vapor A\\
  Vapor A於氣相中進行質傳
  \item 液面高度為$z_l$,會隨著蒸發而降低\\
  但我們假設\fbox{quasi-steady state},假設濃度擴散的速度遠比液體下降的速度要快\\
  換句話說,令$z_l$是定值
  \item 寫出邊界條件
  \begin{align}
    x_A(z_l) &= \frac{P_{\text{sat}}}{P} \label{eq:ch4_3_stefan_bc1}\\
    x_B(z_l) &= 1 - \frac{P_{\text{sat}}}{P} \label{eq:ch4_3_stefan_bc2}\\
    x_B(L) &= X_{B0} \label{eq:ch4_3_stefan_bc3}
  \end{align}
  \item 寫出氣相中的質量平衡:
  \begin{equation}
    \cancel{\frac{\partial C_i}{\partial t}} + \nabla \cdot \vec{\bm N_i} - \cancel{R_i} = 0 \implies \nabla \cdot \vec{\bm N_i} = 0
  \end{equation}
  而這代表了
  \begin{align}
    \frac{d N_{Az}}{dz} = 0 \implies & N_{Az} = C_1\label{eq:ch4_3_stefan_N_Az}\\
    \frac{d N_{Bz}}{dz} = 0 \implies & N_{Bz} = C_2\label{eq:ch4_3_stefan_N_Bz}
  \end{align}
  將(\ref{eq:ch4_3_stefan_N_Bz})代入邊界條件,\fbox{空氣無法擴散進入液體}\\
  也就是$N_{Bz}(z_l) = 0$,得到$C_2 = 0$,所以$N_{Bz} = 0$在整個氣相中成立\\
  P.S. 這不代表就沒有$x_B$,只是沒有淨通量\\
  可以想成\fbox{空氣的擴散只是為了填補液體的蒸發(對流)},所以淨通量為0\\
  P.S. 反過來說,只要另一成分進不去擴散原,又不會反應,那就可以把它當成介質\\
  我們稱為\fbox{Stagnant Component}
  \item 根據Fick's Law,解出濃度分布:
  \begin{align}
    N_{Az}&=x_A\left(N_{Az}+\cancel{N_{Bz}}\right) - C D_{AB}\frac{dx_{A}}{dz} \nonumber\\
    (1-x_A)N_{Az} &= - C D_{AB}\frac{dx_{A}}{dz} \nonumber\\
    \frac{1}{1-x_A}dx_A &= - \frac{N_{Az}}{C D_{AB}}dz = C_3 dz \label{eq:ch4_3_stefan_diff_eq} \\
    \implies \ln(1-x_A) &= C_3 z + C_4 \nonumber\\
    \implies x_A &= 1 - e^{C_3 z + C_4} \label{eq:ch4_3_stefan_xA_distribution}
  \end{align}
  P.S. 其實這裡隱含了$N_{Az}$是常數,而因為$N_{Az}=\frac{W_A}{A}$\\
  此題截面積固定,並假設了Quasi-Steady State,所以揮發速率$W_A$也是定值
  \item 將邊界條件(\ref{eq:ch4_3_stefan_bc1}), (\ref{eq:ch4_3_stefan_bc3})代入(\ref{eq:ch4_3_stefan_xA_distribution}),
  解出常數$C_3, C_4$:
  \begin{align}
    \text{由}~z=z_l:~& \frac{P_{\text{sat}}}{P} = 1 - e^{C_3 z_l + C_4} \nonumber\\
    \implies & C_4 = \ln\left(1 - \frac{P_{\text{sat}}}{P}\right) - C_3 z_l \label{eq:ch4_3_stefan_C4}\\
    \text{由}~z=L:~& X_{B0} = 1 - e^{C_3 L + C_4} \nonumber\\
    \implies & C_3 = \frac{1}{L - z_l} \ln\left(\frac{1 - X_{B0}}{1 - \frac{P_{\text{sat}}}{P}}\right) \label{eq:ch4_3_stefan_C3}
  \end{align}
  \item 將(\ref{eq:ch4_3_stefan_C3}), (\ref{eq:ch4_3_stefan_C4})代入(\ref{eq:ch4_3_stefan_xA_distribution}),得到濃度分布:
  \begin{equation}
    \boxed{x_A = 1 - \left(1 - \frac{P_{\text{sat}}}{P}\right)^{\frac{L-z}{L-z_l}} \left(1 - X_{B0}\right)^{\frac{z - z_l}{L - z_l}}}
  \end{equation}
  \item 將(\ref{eq:ch4_3_stefan_C3})代入(\ref{eq:ch4_3_stefan_N_Az}),得到揮發速率:
  \begin{equation}
    \boxed{N_{Az} = \frac{C D_{AB}}{L - z_l} \ln\left(\frac{1 - X_{B0}}{1 - \frac{P_{\text{sat}}}{P}}\right)}
  \end{equation}
  P.S. 也可以用分壓來表示,理想氣體
  \begin{equation}
    PV = nRT \implies P = \frac{n}{V} RT = CRT  \implies C = \frac{P}{RT}
  \end{equation}
  代入:
  \begin{equation}
    N_{Az} = \frac{P D_{AB}}{RT (L - z_l)}
    \ln\left(\frac{P - P_{\text{sat}}}{P - P X_{B0}}\right)
  \end{equation}
  也可以用濃度來表示:
  \begin{equation}
    N_{Az} = \frac{D_{AB}}{L - z_l}
    \ln\left(\frac{C - C_{\text{sat}}}{C - C X_{B0}}\right)
  \end{equation}
  \item 計算流乾所需時間:
  \begin{align}
    -N_A A &= \frac{dn_A}{dt}  = \frac{d}{dt}\left(\frac{m_A}{M_A}\right) \nonumber\\
    & = \frac{d}{dt}\left(\frac{\rho_A V_A}{M_A}\right) = \frac{\rho_A}{M_A} \frac{d}{dt} V_A \nonumber\\
    & = \frac{\rho_A}{M_A} \frac{d}{dt} A z_l \nonumber\\
    &= \frac{\rho_A A}{M_A} \frac{dz_l}{dt}
  \end{align}
  代入$N_{Az}$,得到:
  \begin{equation}
    \frac{dz_l}{dt} = - \frac{M_A C D_{AB}}{\rho_A (L - z_l)} \ln\left(\frac{1 - X_{B0}}{1 - \frac{P_{\text{sat}}}{P}}\right)
  \end{equation}
  積分後得到液面高度隨時間變化:
  \begin{equation}
    \boxed{z_l(t) = L - \sqrt{(L - z_{l0})^2 - \frac{2 M_A C D_{AB}}{\rho_A} \ln\left(\frac{1 - X_{B0}}{1 - \frac{P_{\text{sat}}}{P}}\right) t}}
  \end{equation}
  另外,流光所需時間為:
  \begin{equation}
    \boxed{t_{\text{dry}} = \frac{(L - z_{l0})^2 \rho_A}{2 M_A C D_{AB} \ln\left(\frac{1 - X_{B0}}{1 - \frac{P_{\text{sat}}}{P}}\right)}}
  \end{equation}
  或進一步改寫為壓力形式(理想氣體):
  \begin{equation}
    \boxed{t_{\text{dry}} = \frac{(L - z_{l0})^2 \rho_A R T}{2 M_A P D_{AB} \ln\left(\frac{P - P_{AL}}{P - P_{\text{sat}}}\right)}}
  \end{equation}
  P.S.這代表流光所需時間與長度平方成正比\\
  另外此式能解決三種問題:
  \begin{enumerate}
    \item 已知時間,求液面流掉多少
    \item 已知液面高度,求已經過了多少時間
    \item 已知時間與液面高度,\fbox{求擴散係數}\\
    又稱為 Arnold Cell Experiment
    \begin{equation}
      D_{AB} = \frac{(L - z_{l0})^2 \rho_A R T}{2 M_A P t_{\text{dry}} \ln\left(\frac{P - P_{AL}}{P - P_{\text{sat}}}\right)}
    \end{equation}
  \end{enumerate}
  \item 針對$N_{Bz}$:
  \begin{equation}
    N_{Bz} = x_B\left(N_{Az} + \cancel{N_{Bz}}\right) - C D_{AB} \frac{dx_B}{dz} = 0
  \end{equation}
  可以得到
  \begin{equation}
    \frac{1}{x_B} dx_B = \frac{N_{Az}}{C D_{AB}} dz
  \end{equation}
  \item 積分後得到:
  \begin{align}
    \ln x_B &= \frac{N_{Az}}{C D_{AB}} z + C_5 \nonumber\\
    \implies x_B &= e^{\frac{N_{Az}}{C D_{AB}} z + C_5}
  \end{align}
  \item P.S. 將邊界條件(\ref{eq:ch4_3_stefan_bc2}), (\ref{eq:ch4_3_stefan_bc3})代入,解出常數$C_5$與驗證$N_{Az}$:
  \begin{align}
    \text{由}~z=z_l:~& 1 - \frac{P_{\text{sat}}}{P} = e^{\frac{N_{Az}}{C D_{AB}} z_l + C_5} \nonumber\\
    \implies & C_5 = \ln\left(1 - \frac{P_{\text{sat}}}{P}\right) - \frac{N_{Az}}{C D_{AB}} z_l \\
    \text{由}~z=L:~& X_{B0} = e^{\frac{N_{Az}}{C D_{AB}} L + C_5} \nonumber\\
    \implies & N_{Az} = \frac{C D_{AB}}{L - z_l} \ln\left(\frac{1 - X_{B0}}{1 - \frac{P_{\text{sat}}}{P}}\right)
  \end{align}
  \item 回顧Pseudo Steady State approximation,代表若要符合\\
  液面下降速率遠小於濃度擴散速率:
  \begin{equation}
    \left|\frac{dz_l}{dt}\right| \ll \frac{D_{AB}}{L}
  \end{equation}
  則:
  \begin{equation}
    \frac{M_A C D_{AB}}{\rho_A (L - z_l)} \ln\left(\frac{1 - X_{B0}}{1 - \frac{P_{\text{sat}}}{P}}\right) \ll \frac{D_{AB}}{L} \implies
    \frac{C}{\frac{\rho_A}{M_A}} \ll 1
  \end{equation}
  \item 將問題衍生為Multicomponent Stefan Problem\\
  假設空氣$B$是由0.79的氮氣$N_2$與0.21的氧氣$O_2$組成\\
  稱揮發性物質為$1$,氮氣為$2$,氧氣為$3$\\
  質量平衡式:
  \begin{equation}
    \cancel{\frac{\partial C_i}{\partial t}} + \nabla \cdot \vec{\bm N_i} - \cancel{R_i} = 0 \implies \nabla \cdot \vec{\bm N_i} = 0
  \end{equation}
  \item 邊界條件:
  \begin{align}
    \frac{dN_{1z}}{dz} = 0 & \implies N_{1z} = C_1 \\
    \frac{dN_{2z}}{dz} = 0 & \implies N_{2z} = C_2 = 0 \\
    \frac{dN_{3z}}{dz} = 0 & \implies N_{3z} = C_3 = 0
  \end{align}
  P.S. 會等於0原因同上,氣體無法進入液體
  \item 寫出Stefan-Maxwell方程式:
  \begin{equation}
    C_i\nabla \mu_i = \sum_{j\neq i} \frac{RT}{\mathbcal{D}_{ij}} \left(
      x_i \vec{\bm N_j} - x_j \vec{\bm N_i}
    \right)
  \end{equation}
  假設為理想氣體
  \begin{equation}
    \nabla x_i = \sum_j \frac{1}{C D_{ij}} \left(
      x_i \vec{\bm N_j} - x_j \vec{\bm N_i}
    \right)
  \end{equation}
  列出$C-1=2$個獨立方程式:
  \begin{align}
    \frac{dx_2}{dz} &= \frac{1}{C D_{21}}\left(
      x_2 N_{1z} - x_1 \cancel{N_{2z}}
    \right)+ \frac{1}{C D_{23}}\left(
      x_2 \cancel{N_{3z}} - x_3 \cancel{N_{2z}}
    \right) \nonumber\\
    &= \frac{N_{1z}}{C D_{21}} x_2 \label{eq:ch4_3_stefan_multi_dx2}\\
    \frac{dx_3}{dz} &= \frac{1}{C D_{31}}\left(
      x_3 N_{1z} - x_1 \cancel{N_{3z}}
    \right)+ \frac{1}{C D_{32}}\left(
      x_3 \cancel{N_{2z}} - x_2 \cancel{N_{3z}}
    \right) \nonumber\\
    &= \frac{N_{1z}}{C D_{31}} x_3 \label{eq:ch4_3_stefan_multi_dx3}
  \end{align}
  \item 積分(\ref{eq:ch4_3_stefan_multi_dx2}), (\ref{eq:ch4_3_stefan_multi_dx3}):
  \begin{align}
    \ln x_2 &= \frac{N_{1z}}{C D_{21}} z + C_6 \nonumber\\
    \implies x_2 &= C_6e^{\frac{N_{1z}}{C D_{21}} z} \\
    \ln x_3 &= \frac{N_{1z}}{C D_{31}} z + C_7 \nonumber\\
    \implies x_3 &= C_7 e^{\frac{N_{1z}}{C D_{31}} z }
  \end{align}
  由於$x_2(L) = x_{2L}=0.79$,$x_3(L) = x_{3L}=0.21$,所以:
  \begin{align}
    x_2 &= 0.79 e^{\frac{N_{1z}}{C D_{21}} (z - L)} \\
    x_3 &= 0.21 e^{\frac{N_{1z}}{C D_{31}} (z - L)}
  \end{align}
  \item 由於$x_1 + x_2 + x_3 = 1$,所以:
  \begin{equation}
    x_1 = 1 - 0.79 e^{\frac{N_{1z}}{C D_{21}} (z - L)} - 0.21 e^{\frac{N_{1z}}{C D_{31}} (z - L)}
  \end{equation}
  而同時
  \begin{equation}
    x_1(z_l) = \frac{P_{\text{sat}}}{P} = 1 - 0.79 e^{\frac{N_{1z}}{C D_{21}} (z_l - L)} - 0.21 e^{\frac{N_{1z}}{C D_{31}} (z_l - L)}
  \end{equation}
  無法解析解,只能用數值方法求解$N_{1z}$
  \item 液體中的質量平衡:
  \begin{equation}
    \frac{\rho_A}{M_A} \frac{dz}{dt} = -N_{1z}(z_l(t))
  \end{equation}
  以及液面高度隨時間變化:
  \begin{equation}
    z_l(t) = z_{l0} - \frac{M_A}{\rho_A} \int_0^t N_{1z}(z_l(t)) dt
  \end{equation}
  邊界條件:
  \begin{equation}
    z_l(0) = z_{l0}
  \end{equation}
  \item  跟熱量與動量傳輸不同,質量傳輸在介面之間是允許濃度跳躍的\\
  而連續的方程式是在介面處兩處化學能相同:
  \begin{equation}
    \mu_A^{\text{liq}} (z_l) = \mu_A^{\text{gas}} (z_l)
  \end{equation}
  而由此開展開來,若液體為純液體,氣相為理想氣體,則得到
  \begin{align}
    \mu_A^{\text{liq}} (z_l) &= \mu_A^{\circ,\text{liq}} \nonumber\\
    \mu_A^{\text{gas}} (z_l) &= \mu_A^{\ast} + RT \ln\left(X_A(z_l)P\right) 
  \end{align}
  合併後可得:
  \begin{equation}
    x_A(z_l) = \frac{P_{\text{sat}}}{P},~C_A^\text{gas}(z_l) = \frac{P_1^{\text{sat}}}{RT}
\end{equation}
\end{itemize}
\end{CJK*}
\end{document}