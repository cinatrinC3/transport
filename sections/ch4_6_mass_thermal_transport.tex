\documentclass[../main.tex]{subfiles}
\begin{document}
\begin{CJK*}{UTF8}{bkai}
\subsection{質量輸送,質量與熱一起輸送}
\begin{itemize}
  \item 微觀下多分子的能量平衡式:\\
  Multicomponent Microscopic Energy Balance\\
  會連結質量傳輸的部分
  \begin{equation}
    \underbrace{\frac{\partial}{\partial t}\left(
      \rho \hat U + \frac{1}{2}\rho\left|\vec{\bm u}\right|^2
    \right)}_{E_{\text{acc}}} = \underbrace{-\nabla\cdot \vec{\bm e}}_{E_{\text{in}}-E_{\text{out}}}
     + \underbrace{\rho\left(\vec{\bm v}\cdot \vec{\bm g}\right)}_{E_{\text{gen}}}
  \end{equation}
  其中$\vec{\bm e}$為總能量通量:
  \begin{equation}
    \boxed{\vec{\bm e} = \left(\rho \hat U + \frac{1}{2}\rho\left|\vec{\bm u}\right|^2\right)\vec{\bm u} 
    + \vec{\bm q} + \underbrace{ \overline{\overline{\bm \pi}} \cdot \vec{\bm u}}_{P\bm \delta\bm\delta 
    + \overline{\overline{\bm \tau}}\cdot \vec{\bm u}}}
  \end{equation}
  對單物質系統
  \begin{equation}
    \vec{\bm q} = -k\nabla T
  \end{equation}
  對多物質系統
  \begin{equation}
    \vec{\bm q} = -k\nabla T + \sum_{i=1}^N \hat H_i \vec{\bm J}_i + \vec{\bm q}^{(T)}
  \end{equation}
  其中
  \begin{align}
    \hat H_i &= \left(\frac{\partial H}{\partial n_i}\right)_{T,P,n_j,j\neq i}, \quad \text{Molar Enthalpy of species }i\\
    \vec{\bm J}_i &= \text{Diffusive Molar Flux of species }i\\
    \vec{\bm q}^{(T)} &= \text{Duforer, Thermal Diffusion soret effect}
  \end{align}
  所以$\vec{\bm e}$又可以寫成其他形式
  \begin{equation}
    \boxed{\vec{\bm e} = \underbrace{\rho \hat U\vec{\bm u}}_{
      \sum (C_i\hat H_i) \vec{\bm u}
    } + \frac{1}{2}\rho\left|\vec{\bm u}\right|^2\vec{\bm u}
    + \left(-k\nabla T + \sum_{i=1}^N \hat H_i \vec{\bm J}_i + \vec{\bm q}^{(T)}\right)
    + P\bm \delta\bm\delta \cdot \vec{\bm u} + \overline{\overline{\bm \tau}}\cdot \vec{\bm u}}
  \end{equation}
  假設省略動能項$\frac{1}{2}\rho\left|\vec{\bm u}\right|^2\vec{\bm u}$
  、Duforer effect$\vec{\bm q}^{(T)}$\\
  耗散項$\overline{\overline{\bm \tau}} : \left(\nabla \vec{\bm u}\right)$、
  變形項$P\bm \delta\bm\delta \cdot \vec{\bm u}$\\
  則微觀能量方程式變成:
  \begin{align}
    \vec {\bm e} &= -k\nabla T + \sum_{i=1}^N \hat H_i \vec{\bm J}_i + \sum_i (C_i \hat H_i) \vec{\bm u}\\
     \quad  &= -k\nabla T +\sum_{i=1}^N\left[
      \hat H_i \left(\vec{\bm J}_i + C_i \vec{\bm u}\right)
    \right] \nonumber\\
    &= \boxed{-k\nabla T + \sum_{i=1}^N \hat H_i \vec{\bm N}_i}
  \end{align}
  若沒有重力,穩態下的微觀能量平衡,就跟Equation of Continuity一樣
  \begin{equation}
    \boxed{\nabla \cdot \vec{\bm e} = 0}
  \end{equation}
  \item  對流質量傳輸產生的條件:
    \begin{enumerate}
      \item 某成分有濃度差
      \item 有流動的連續介質
    \end{enumerate}
  \item 二元對流的質量傳輸
    \begin{equation}
      \boxed{N_A = k_c\cdot \Delta C_A = k_c\left(C_{A_1}-C_{A_2}\right)=k_c\cdot C\cdot\left(y_{A_1}-y_{A_2}\right)}
    \end{equation}
    $k_c$是質量傳輸係數\\
    也可以寫成壓力單位下:
    \begin{equation}
      \boxed{N_A = k_c\Delta C_A = k_p\Delta p_A = k_y\Delta y_A}
    \end{equation}
    $k_p$是壓力單位下的傳輸係數,$k_y$是莫爾單位下的傳輸係數\\
    P.S. $k_y = k_p\cdot P = k_c C$\\
    同熱對流的$h$,質量傳輸係數$k_{c,p,y}$也很難獲得\\
    因為會受容器的形狀、流速、流動方式、流體的物性(密度、黏度、熱容、熱導係數)、質傳方式而影響\\
    通常使用質量傳輸經驗式,計算,再代入獲得$k_{c,p,y}$
  \item Evaporation of a Water Film Into Air
  \begin{figure}[H]
    \centering
    \begin{tikzpicture}[>=Latex, line cap=round, line join=round, thick]
      \draw (0,0) rectangle (6,-0.3);
      \draw[decorate, decoration=snake, blue] (0,2) -- (6,2);
      \draw [->] (-0.5,0) -- (-0.5,4.5) node[above] {$z$};
      \node[anchor=west] at (6,0) {$z=-L$};
      \node[anchor=west] at (6,2) {$z=0$};
      \draw[blue, ->] (5,2) -- (5,3) node[above] {Evaporation};
      \node[blue] at(3,1) {Water (A)};
      \node[red] at (3,2.5) {Air (B)};
      \draw[red,->] (-2.3,2.5) -- (-0.9,2.5);
      \draw[red,->] (-2.3,3) -- (-0.9,3);
      \draw[red,->] (-2.3,3.5) -- (-0.9,3.5);
      \node at (3,4) {$RH = \frac{P_\infty}{P_A^{\text{sat}}(T_\infty)}$};
      \node[anchor=east] at (-0.5,0) {$T_\infty$};
      \node[anchor=east] at (-0.5,2) {$T_0$};
      \node[anchor=east] at (-0.5,4) {$T_\infty$};
      \def\xShift{8.5}
      \draw[->] (\xShift,0) -- (\xShift,5) node[above] {$T$};
      \draw[->] (\xShift,0) -- (\xShift+6.5,0) node[right] {$z$};
      \draw[dashed] (\xShift,2) -- (\xShift+2,2);
      \node[anchor=west] at (\xShift+2,2) {Wet Bulb Temp};
      \draw[dashed] (\xShift+2,0) -- (\xShift+2,2);
      \draw (\xShift,4) -- (\xShift+2,2) .. controls (\xShift+3,4) and (\xShift+5,4) .. (\xShift+6,4);
      \node[anchor=east] at (\xShift,4) {$T_\infty$};
      \node[anchor=east] at (\xShift,2) {$T_0$};
      \node[anchor=north] at (\xShift,0) {$-L$};
      \node[anchor=north] at (\xShift+6,0) {$\infty$};
      \node[anchor=north] at (\xShift+2,0) {$0$};
    \end{tikzpicture}
    \caption{Evaporation of a Water Film Into Air}
  \end{figure}
  \begin{enumerate}
    \item Heat Transfer in Liquid (Pure A)
    \begin{equation}
      \nabla \cdot \vec{\bm e}= 0 =\nabla^2 T \implies T(z) = C_1 z + C_2
    \end{equation}
    邊界條件:
    \begin{align}
      T(-L) &= T_\infty\\
      T(0) &= T_0
    \end{align}
    解得:
    \begin{equation}
      T(z) = \left(\frac{T_0 - T_\infty}{L}\right) z + T_0
    \end{equation}
    \item 在交界面處($z=0$),能量守恆:
    \begin{equation}
      e_z(0) = -k\left(
        \frac{\partial T(0)}{\partial z}
      \right) + N_{Az}^{(V)}(0)\hat H_A^{\circ (L)}
      = -k\left(
        \frac{T_0-T_\infty}{L}
      \right) + N_{Az}^{(V)}(0)\hat H_A^{\circ (L)} \label{eq:ch4_6_mass_thermal_ez_interface}
    \end{equation}
    其中$N_{Az}^{(V)}(0)$為A流入氣相,Flux of A into vapor phase\\
    $\hat H_A^{\circ (L)}$為A在液相的Molar Enthalpy
    \item 氣相中的熱傳與質傳
    \begin{equation}
      N_{Az}^{(V)} = x_A\left(
        N_{Az}^{(V)} + N_{Bz}^{(V)}
      \right)-C D_{AB}^{(V)}\frac{dx_A}{dz}
    \end{equation}
    由於在交界面,空氣無法進入液體,所以$N_{Bz}^{(V)}(0) = 0$
    \begin{align}
      N_{Az}^{(V)}(0) &= x_A(0)\left[
        N_{Az}^{(V)}(0) +\cancel{N_{Bz}^{(V)}(0)}
      \right] - C D_{AB}^{(V)}\left(
        \frac{dx_A}{dz}\right)_{z=0} \nonumber\\
      (1-x_A(0))N_{Az}^{(V)}(0) &= - C D_{AB}^{(V)}\left(
        \frac{dx_A}{dz}\right)_{z=0} \equiv k_x \left(
          x_A(0) - x_A(\infty)
        \right) \label{eq:ch4_6_mass_thermal_N_Az_interface}
    \end{align}
    P.S. $k_x$為Mass Transfer Coefficient\\
    另外
    \begin{equation}
      x_A(0) = \frac{P_A^{\text{sat}}(T_0)}{P}, \quad 
      x_A(\infty) = \frac{P_A^{\text{sat}}(T_\infty)}{P} RH
    \end{equation}
    \item 將(\ref{eq:ch4_6_mass_thermal_N_Az_interface})代入(\ref{eq:ch4_6_mass_thermal_ez_interface}):
    \begin{align}
      N_{Az}^{(V)}(0) &= \frac{k_x\left(
        x_A(0) - x_A(\infty)
      \right)}{1-x_A(0)} \nonumber\\
      &= k_x x_A(0) \left(
        1-\frac{x_A(\infty)}{x_A(0)}
      \right)\cdot \frac{1}{1 - x_A(0)} \nonumber\\
      &= k_x \frac{P_A^{\text{sat}}(T_0)}{P} \left[
        \frac{1 - RH \cdot \frac{P_A^{\text{sat}}(T_\infty)}{P_A^{\text{sat}}(T_0)}
      }{1 - \frac{P_A^{\text{sat}}(T_0)}{P}}
      \right] \label{eq:ch4_6_mass_thermal_N_Az_final}
    \end{align}
    \item 以對流的能量平衡式描述:
    \begin{equation}
      e_z^{(L)}(0) = e_z^{(V)}(0)
    \end{equation}
    氣相中的能量通量:
    \begin{equation}
      e_z^{(V)}(0) = N_{Az}^{(V)}(0) \hat H_A^{\circ (V)} + \cancel{N_{Bz}^{(V)}(0) \hat H_B^{\circ (V)}}
      + h(T_0 - T_\infty)
    \end{equation}
    其中$h$為Heat Transfer Coefficient\\
    解出$ N_{Az}^{(V)}(0)$:
    \begin{align}
      \frac{k(T_\infty - T_0)}{L} + N_{Az}^{(V)}(0)\hat H_A^{\circ (L)} 
      &=h(T_0 - T_\infty)+ N_{Az}^{(V)}(0) \hat H_A^{\circ (V)} \nonumber\\
      N_{Az}^{(V)}(0) \left(\hat H_A^{\circ (V)} - \hat H_A^{\circ (L)}\right)
      &= \frac{k(T_\infty - T_0)}{L} + h(T_0 - T_\infty) \nonumber\\
      N_{Az}^{(V)}(0) &= \frac{\frac{k(T_\infty - T_0)}{L} + h(T_0 - T_\infty)}
      {\hat H_A^{\circ (V)} - \hat H_A^{\circ (L)}} \nonumber\\
      &= \frac{h\left(
        1+\frac{k}{hL}
      \right)\left(
        T_\infty - T_0
      \right)}{\Delta \hat H_{A,\text{vap}}^{\circ}}
    \end{align}
    \item 結合兩種$N_{Az}^{(V)}(0)$的解:
    \begin{equation}
      \boxed{
        k_x \frac{P_A^{\text{sat}}(T_0)}{P} \left[
          \frac{1 - RH \cdot \frac{P_A^{\text{sat}}(T_\infty)}{P_A^{\text{sat}}(T_0)}
        }{1 - \frac{P_A^{\text{sat}}(T_0)}{P}}
        \right] = \frac{h\left(
          1+\frac{k}{hL}
        \right)\left(
          T_\infty - T_0
        \right)}{\Delta \hat H_{A,\text{vap}}^{\circ}}}
    \end{equation}
    假設當$\frac{P_A^{\text{sat}}(T_0)}{P} \ll 1$時,化簡為:
    \begin{equation}
      \boxed{
        \frac{\Delta H_{A,\text{vap}}^{\circ} P_A^{\text{sat}}(T_0)\left[
          1- RH \cdot \frac{P_A^{\text{sat}}(T_\infty)}{P_A^{\text{sat}}(T_0)}
        \right]}{P(T_\infty - T_0)} = \frac{h\left(
          1+\frac{k}{hL}\right)}{k_x}}
    \end{equation}
    代表只要測量出$h,k_x,k$,就可以測量出$T_0$,稱為\fbox{Wet Bulb Temperature}
  \end{enumerate}
  \item Two-Films Theory
  \begin{itemize}
    \item 假設有一液相與一氣相,兩項有一混合的接處面\\
    液相有$A$濃度分布$x_{A,l}$,到交界面時濃度$x_{A,i}$,氣相有$A$濃度分布$y_{A,g}$,到交界面時濃度$y_{A,i}$\\
    假設$A$在氣相與液相濃度皆很稀薄,可符合Henry's Law:
    \begin{equation}
      y_{A,i} = Hx_{A,i}
    \end{equation}
    定義$x_A^*$是與$y_{A,g}$平衡後的莫爾分率、\\
    $y_A^*$是與$x_{A,l}$平衡後的莫爾分率,也可以說是遠離交界面的氣液濃度
    \begin{enumerate}
      \item Henry's Law:
        \begin{align}
          y_{A,i} &= Hx_{A,i} \\
          y_A^* &= Hx_{A,l}\\
          y_{A,g} &= Hx_A^*
        \end{align}
      \item Local gas-phase:
        \begin{equation}
          N_A = k'_y\left(y_{A,g}-y_{A,i}\right)
        \end{equation}
        $k'_y$是氣相中Equimolar的質量傳輸係數
      \item Local liquid-phase:
        \begin{equation}
          N_A = k'_x\left(x_{A,i}-x_{A,l}\right)
        \end{equation}
        $k'_x$是液相中Equimolar的質量傳輸係數
      \item Overall gas-phase:
        \begin{equation}
          N_A = K'_y\left(y_{A,g}-y_A^*\right) \label{eq060106}
        \end{equation}
        $K'_y$是整體的氣相Equimolar的質量傳輸係數
      \item Overall liquid-phase:
        \begin{equation}
          N_A = K'_x\left(x_A^*-x_{A,l}\right)\label{eq060107}
        \end{equation}
       $K'_x$是整體的液相Equimolar的質量傳輸係數
       \item 改討論在整體氣相與液相的質量傳輸\\
       由(\ref{eq060201})和(\ref{eq060202}),代入(\ref{eq060106})和(\ref{eq060107})可得到:
      \begin{align}
        N_A &= k'_y\left(y_{A,g}-y_{A,i}\right) = \frac{y_{A,g}-y_{A,i}}{\frac{1}{k'_y}}\label{eq060301}\\
        N_A &= k'_x\left(x_{A,i}-x_{A,l}\right) = k'_x\frac{x_{A,i}-x_{A,l}}{y_{A,i}-y_A^*}\left(y_{A,i}-y_A^*\right)
      \end{align}
      定義$m'$是:
      \begin{equation}
        m' = \frac{y_{A,i}-y_A^*}{x_{A,i}-x_{A,l}}
      \end{equation}
      第二式可改寫為
      \begin{equation}
        N_A = k_x'\frac{1}{m'}\left(y_{A,i}-y_A^*\right) = \frac{y_{A,i}-y_A^*}{\frac{m'}{k_x'}}\label{eq060304}
      \end{equation}
      \item 合併(\ref{eq060301})和(\ref{eq060304})可得到:
      \begin{equation}
        N_A = \frac{\left(y_{A,g}-y_{A,i}\right)+\left(y_{A,i}-y_A^*\right)}{\frac{1}{k'_y}+\frac{m'}{k_x'}} = 
        \frac{y_{A,g}-y_A^*}{\frac{1}{k'_y}+\frac{m'}{k_x'}}
      \end{equation}
      \item 再跟(\ref{eq060106})比較,可得到:
      \begin{equation}
        \frac{1}{K'_y} = \frac{1}{k'_y}+\frac{m'}{k_x'}
      \end{equation}
      此為從兩相質量傳輸係數,得到整體氣相平衡的質量傳輸係數的方法
      \item 反著第一步,再做一次:
      \begin{align}
        N_A &= k'_y\left(y_{A,g}-y_{A,i}\right) = k'_y\frac{y_{A,g}-y_{A,i}}{x_A^*-x_{A,i}}\left(x_A^*-x_{A,l}\right)\\
        N_A &= k'_x\left(x_{A,i}-x_{A,l}\right) = \frac{x_{A,i}-x_{A,l}}{\frac{1}{k'_x}}\label{eq060308}
      \end{align}
      定義$m''$是:
      \begin{equation}
        m'' = \frac{y_{A,g}-y_{A,i}}{x_A^*-x_{A,i}}
      \end{equation}
      第一式可改寫為
      \begin{equation}
        N_A = k_y'm''\left(x_A^*-x_{A,i}\right) = \frac{x_A^*-x_{A,i}}{\frac{1}{m''k'_y}}\label{eq060310}
      \end{equation}
      \item 合併(\ref{eq060308})和(\ref{eq060310})可得到:
      \begin{equation}
        N_A = \frac{\left(x_A^*-x_{A,i}\right)+\left(x_{A,i}-x_{A,l}\right)}{\frac{1}{k'_x}+\frac{1}{m''k'_y}} = 
        \frac{x_A^*-x_{A,l}}{\frac{1}{k'_x}+\frac{1}{m''k'_y}}
      \end{equation}
      \item 再跟(\ref{eq060107})比較,可得到:
      \begin{equation}
        \frac{1}{K'_x} = \frac{1}{k'_x}+\frac{1}{m''k'_y}
      \end{equation}
      此為從兩相質量傳輸係數,得到整體液相平衡的質量傳輸係數的方法
    \end{enumerate}
    \item 若為Equilmolar Counterdiffusion,(A可擴散到B,B可擴散到A)
    \begin{enumerate}
      \item 在兩相中:
        \begin{align}
          N_A &= k'_y\left(y_{A,g}-y_{A,i}\right)\label{eq060201}\\
          N_A &= k'_x\left(x_{A,i}-x_{A,l}\right)\label{eq060202}
        \end{align}
      \item 合併兩式:
        \begin{equation}
          k'_y\left(y_{A,g}-y_{A,i}\right) = k'_x\left(x_{A,i}-x_{A,l}\right)
        \end{equation}
        \item 將實驗數據或算式,可得到不同時候平衡時的$y_{A,g},~x_{A,l}$
        \item 做出$y_A-x_A$的圖,將上步驟的數據畫於圖上,此為平衡線
        \item 已知$y_{A,g},~x_{A,l}$,可在此圖上點出此為置,令為點$P$
        \item 上式移項後得到:
        \begin{equation}
          -\frac{k'_x}{k'_y} = \frac{y_{A,g}-y_{A,i}}{x_{A,i}-x_{A,l}}
        \end{equation}
        \item 從點$P$出發,做過點$P$斜率為$-\frac{k'_x}{k'_y}$的直線\\
        與平衡線交於點$M(x_i,y_i)$,$M(x_i,y_i)$即為交界面的濃度
        \item 而得知交界面的兩相濃度後,即可計算非勻相反應情形
      \end{enumerate}
      \item 若$B$是Stagnant,也就是$B$不會擴散到$A$,則:
      \begin{enumerate}
        \item 根據Fick's Law:
      \begin{equation}
        N_A = -D_{AB} C\cdot \frac{dy_A}{dz}+y_A\left(N_A+\cancel{N_B}\right)
      \end{equation}
      \item 也就是:
      \begin{equation}
        N_A = -\frac{D_{AB}C}{L}\ln\left(\frac{1-y_{A,g}}{1-y_{A,l}}\right)
      \end{equation}
      \item 代入$N_A$式子的氣相那一條(\ref{eq060201}),經過化簡後可得到:
      \begin{equation}
        N_A =  k'_y\cdot \frac{\ln\left(\frac{1-y_{A,i}}{1-y_{A,g}}\right)}{\left(1-y_{A,i}\right)-\left(1-y_{A,g}\right)}
        \left(y_{A,g}-y_{A,l}\right) 
      \end{equation}
      \item 將常數換掉後可寫成這樣:
      \begin{equation}
        N_A = k_y\left(y_{A,g}-y_{A,l}\right)
      \end{equation}
      其中定義$k_y$是氣相,但氣相不傳給液相的質量傳輸係數
      \begin{equation}
        k_y = \frac{k_y'}{\left(1-y_A\right)_{i,M}}
      \end{equation}
      並定義$\left(1-y_A\right)_{i,M}$是
      \begin{equation}
        \left(1-y_A\right)_{i,M} = \frac{\left(1-y_{A,i}\right)-\left(1-y_{A,g}\right)}{\ln\left(\frac{1-y_{A,i}}{1-y_{A,g}}\right)}\label{eq060210}
      \end{equation}
      \item 同樣地,代入$N_A$式子的液相那一條(\ref{eq060202}),經過化簡後可得到:
      \begin{equation}
        N_A = k'_x\cdot \frac{\ln\left(\frac{1-x_{A,i}}{1-x_{A,l}}\right)}{\left(1-x_{A,i}\right)-\left(1-x_{A,l}\right)}
        \left(x_{A,i}-x_{A,l}\right)
      \end{equation}
      \item 將常數換掉後可寫成這樣:
      \begin{equation}
        N_A = k_x\left(x_{A,i}-x_{A,l}\right)
      \end{equation}
      其中定義$k_x$是液相,但氣相不傳給液相質量傳輸係數
      \begin{equation}
        k_x = \frac{k_x'}{\left(1-x_A\right)_{i,M}}
      \end{equation}
      並定義$\left(1-x_A\right)_{i,M}$是
      \begin{equation}
        \left(1-x_A\right)_{i,M} = \frac{\left(1-x_{A,i}\right)-\left(1-x_{A,l}\right)}{\ln\left(\frac{1-x_{A,i}}{1-x_{A,l}}\right)}\label{eq060214}
      \end{equation}
      \item 兩式移項後得到:
      \begin{equation}
        -\frac{k_x}{k_y} = \frac{y_{A,g}-y_{A,i}}{x_{A,i}-x_{A,l}} =
        \frac{\frac{-k'_x}{\left(1-x_A\right)_{i,M}}}{\frac{k'_y}{\left(1-y_A\right)_{i,M}}}\label{eq060215}
      \end{equation}
      \item 將實驗數據或算式,可得到不同時候平衡時的$y_{A,g},~x_{A,l}$
      \item 做出$y_A-x_A$的圖,將上步驟的數據畫於圖上,此為平衡線
      \item 已知$y_{A,g},~x_{A,l}$,可在此圖上點出此為置,令為點$P$
      \item 首先,令$k_x = k_x'$,$k_y = k_y'$
      \item 做過點$P$斜率為$-\frac{k_x}{k_y}$的直線,與平衡線交於點$M(x_i,y_i)$,$M(x_i,y_i)$
      \item 將得到的$x_i,y_i$代入(\ref{eq060214})和(\ref{eq060210}),即可得到$\left(1-x_A\right)_{i,M}$和$\left(1-y_A\right)_{i,M}$
      \item 將$k'_x,k'_y$代入(\ref{eq060215}),可得到另一$k_x,k_y$
      \item 重複步驟12,直到$k_x,k_y$不再改變
      \item 得到正確的交點$M(x_i,y_i)$
      \item 若為Equilmolar Counterdiffusion,(A可擴散到B,B可擴散到A)則:
      \item 若$B$是Stagnant,也就是$B$不會擴散到$A$,則會有完全與上述相同的式子\\
      但$k'_x,k'_y$會被$k_x,k_y$取代\\
      P.S. $k_x,k_y$ 的定義分別是:
      \begin{align}
        k_x &= \frac{k_x'}{\left(1-x_A\right)_{i,M}} \nonumber\\
        k_y &= \frac{k_y'}{\left(1-y_A\right)_{i,M}} \nonumber\\
        \left(1-x_A\right)_{i,M} &= \frac{\left(1-x_{A,i}\right)-\left(1-x_{A,l}\right)}{\ln\left(\frac{1-x_{A,i}}{1-x_{A,l}}\right)} \nonumber\\
        \left(1-y_A\right)_{i,M} &= \frac{\left(1-y_{A,i}\right)-\left(1-y_{A,g}\right)}{\ln\left(\frac{1-y_{A,i}}{1-y_{A,g}}\right)} \nonumber
      \end{align}
      最後會得到
      \begin{equation}
        \frac{1}{K_y} = \frac{1}{k_y}+\frac{m'}{k_x},\quad \frac{1}{K_x} = \frac{1}{k_x}+\frac{1}{m''k_y}
      \end{equation}
    \end{enumerate}
  \end{itemize}
\end{itemize}
\end{CJK*}
\end{document}


