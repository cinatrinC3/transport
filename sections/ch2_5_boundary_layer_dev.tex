\documentclass[../main.tex]{subfiles}
\begin{document}
\begin{CJK*}{UTF8}{bkai}
\subsection{邊界層發展、紊流}
\label{sec:ch2_5_boundary_layer_dev}
\begin{itemize}
  \item Hydrodynamic Boundary Layer,流體在固體表面附近會形成一層薄薄的邊界層\\
  過了邊界層,流體的速度會趨近於自由流速
  \begin{figure}[H]
    \centering
    \begin{tikzpicture}[>=Latex, line cap=round, line join=round, thick]
      \draw (2,0) -- (12,0);
      \draw[blue] (0.5,0) -- (0.5,4);
      \draw[blue, dashed] (1.5,0) -- (1.5,4);
      \draw[->, blue, dashed] (0.5,0) -- (1.5,0);
      \draw[->, blue, dashed] (0.5,0.5) -- (1.5,0.5);
      \draw[->, blue, dashed] (0.5,1) -- (1.5,1);
      \draw[->, blue, dashed] (0.5,1.5) -- (1.5,1.5);
      \draw[->, blue, dashed] (0.5,2) -- (1.5,2);
      \draw[->, blue, dashed] (0.5,2.5) -- (1.5,2.5);
      \draw[->, blue, dashed] (0.5,3) -- (1.5,3);
      \draw[->, blue, dashed] (0.5,3.5) -- (1.5,3.5);
      \draw[->, blue, dashed] (0.5,4) -- (1.5,4);
      \node[anchor=south, blue] at (1,4) {U};
      \draw[->] (13,0) -- (13,1) node[above] {$y$};
      \draw[->] (13,0) -- (14,0) node[right] {$x$};
      \draw[red] (2,0) .. controls (5,3) and (10,3.72) .. (15.6,4.24);
      \path[name path=B] (2,0) .. controls (5,3) and (10,3.72) .. (15.6,4.24);
      \path[name path=C] (11,0) -- (11,4);
      \path[name intersections={of=B and C, by=D}];
      \draw[red, dashed,<->] (11,0) -- (D) node[midway, right] {$\delta(x)$};
      \node at (9.5, 3.8) {Free Stream};
      \draw[dashed, blue] (4,0) ..controls (4.8,0) and (5, 2.6) .. (5,4);
      \path[name path=P1] (4,0) ..controls (4.8,0) and (5, 2.6) .. (5,4);
      \draw[blue] (4,0) -- (4,4);
      \path[name path=Pa] (4,0.5) -- (5,0.5);
      \path[name intersections={of=P1 and Pa, by=I1}];
      \draw[->, blue, dashed] (4,0.5) -- (I1);
      \path[name path=Pb] (4,1) -- (5,1);
      \path[name intersections={of=P1 and Pb, by=I2}];
      \draw[->, blue, dashed] (4,1) -- (I2);
      \path[name path=Pc] (4,1.5) -- (5,1.5);
      \path[name intersections={of=P1 and Pc, by=I3}];
      \draw[->, blue, dashed] (4,1.5) -- (I3);
      \path[name path=Pd] (4,2) -- (5,2);
      \path[name intersections={of=P1 and Pd, by=I4}];
      \draw[->, blue, dashed] (4,2) -- (I4);
      \path[name path=Pe] (4,2.5) -- (5,2.5);
      \path[name intersections={of=P1 and Pe, by=I5}];
      \draw[->, blue, dashed] (4,2.5) -- (I5);
      \path[name path=Pf] (4,3) -- (5,3);
      \path[name intersections={of=P1 and Pf, by=I6}];
      \draw[->, blue, dashed] (4,3) -- (I6);
      \path[name path=Pg] (4,3.5) -- (5,3.5);
      \path[name intersections={of=P1 and Pg, by=I7}];
      \draw[->, blue, dashed] (4,3.5) -- (I7);
      \path[name path=Ph] (4,4) -- (5,4);
      \path[name intersections={of=P1 and Ph, by=I8}];
      \draw[->, blue, dashed] (4,4) -- (I8);
      \draw[dashed, blue] (7,0) ..controls (7.6,0) and (8,3.4) .. (8,4);
      \path[name path=P2] (7,0) ..controls (7.6,0) and (8,3.4) .. (8,4);
      \path[name path=Qa] (7,0.5) -- (8,0.5);
      \path[name intersections={of=P2 and Qa, by=J1}];
      \draw[->, blue, dashed] (7,0.5) -- (J1);
      \path[name path=Qb] (7,1) -- (8,1);
      \path[name intersections={of=P2 and Qb, by=J2}];
      \draw[->, blue, dashed] (7,1) -- (J2);
      \path[name path=Qc] (7,1.5) -- (8,1.5);
      \path[name intersections={of=P2 and Qc, by=J3}];
      \draw[->, blue, dashed] (7,1.5) -- (J3);
      \path[name path=Qd] (7,2) -- (8,2);
      \path[name intersections={of=P2 and Qd, by=J4}];
      \draw[->, blue, dashed] (7,2) -- (J4);
      \path[name path=Qe] (7,2.5) -- (8,2.5);
      \path[name intersections={of=P2 and Qe, by=J5}];
      \draw[->, blue, dashed] (7,2.5) -- (J5);
      \path[name path=Qf] (7,3) -- (8,3);
      \path[name intersections={of=P2 and Qf, by=J6}];
      \draw[->, blue, dashed] (7,3) -- (J6);
      \path[name path=Qg] (7,3.5) -- (8,3.5);
      \path[name intersections={of=P2 and Qg, by=J7}];
      \draw[->, blue, dashed] (7,3.5) -- (J7);
      \path[name path=Qh] (7,4) -- (8,4);
      \path[name intersections={of=P2 and Qh, by=J8}];
      \draw[->, blue, dashed] (7,4) -- (J8);
      \draw[blue] (7,0) -- (7,4);
    \end{tikzpicture}
    \caption{Boundary Layer Development}
    \label{fig:ch2_5_boundary_layer_development}
  \end{figure}
  \begin{itemize}
    \item 假設:
    \begin{enumerate}
      \item Steady State
      \item Incompressible Flow
      \item Newtonian Fluid
      \item 平板無限長且無厚度
      \item 無壓力梯度(平行流)
      \item 二維流動
      \item Isothermal
      \item No Slip
      \item Potential Flow在邊界層外(不受重力影響)
    \end{enumerate}
    \item 無因次化及Goldilocks Argument:\\
    還是先假設有一個長度,只是這個長度相較於$\delta$來說很大:
    \begin{equation}
      \frac{\delta}{L} \ll 1
    \end{equation}
    各方向上的特徵長度
    \begin{align}
      x^\ast &= \frac{x}{L}\\
      y^\ast &= \frac{y}{\delta}\\
      u_x^\ast &= \frac{u_x}{U} \\
      u_y^\ast &= \frac{u_y}{\boxed{V}}
    \end{align}
    而$V$目前不知道,透過代入連續方程式,可以觀察得到關係\\
    P.S.這邊不能假設$u_y=0$或任何常數,不然Boundary Layer就不會長了\\
    代入連續方程式:
    \begin{equation}
      \frac{\partial u_x}{\partial x} + \frac{\partial u_y}{\partial y} = 0
    \end{equation}
    變成:
    \begin{align}
      &\frac{U}{L} \frac{\partial u_x^\ast}{\partial x^\ast} + \frac{V}{\delta} \frac{\partial u_y^\ast}{\partial y^\ast} = 0
      \nonumber\\
      &\boxed{\frac{U\delta}{VL}} \frac{\partial u_x^\ast}{\partial x^\ast} + \frac{\partial u_y^\ast}{\partial y^\ast} = 0
    \end{align}
    因此Goldilocks Argument告訴我們\\
    如果$\frac{U\delta}{VL} \gg 1$,則$\frac{\partial u_y^\ast}{\partial y^\ast}$會趨近於0\\
    表示垂直方向上沒有速度變化,表示沒有邊界層\\
    如果$\frac{U\delta}{VL} \ll 1$,則$\frac{\partial u_x^\ast}{\partial x^\ast}$會趨近於0\\
    表示水平方向上沒有速度變化,表示沒有邊界層\\
    而Goldilocks Argument告訴我們,必須要差不多是1,才能有邊界層的發展\\
    因此可以得到:
    \begin{equation}
      \frac{U\delta}{VL} \approx 1 \implies V \sim \frac{U\delta}{L}
    \end{equation}
    \item 觀察動量方程式:
    \begin{equation}
      \rho\left(
        v_x \frac{\partial u_x}{\partial x} +
        v_y \frac{\partial u_x}{\partial y}
      \right) = \mu\left[
        \frac{\partial^2 u_x}{\partial x^2} +
        \frac{\partial^2 u_x}{\partial y^2}
      \right]
    \end{equation}
    代入無因次化的變數:
    \begin{align}
      &\rho
        \frac{U^2}{L} u_x^\ast \frac{\partial u_x^\ast}{\partial x^\ast} +
       \rho \frac{UV}{\delta} u_y^\ast \frac{\partial u_x^\ast}{\partial y^\ast}  = 
        \mu \frac{U}{L^2} \frac{\partial^2 u_x^\ast}{\partial {x^\ast}^2} +
        \mu \frac{U}{\delta^2} \frac{\partial^2 u_x^\ast}{\partial {y^\ast}^2}
      \nonumber\\
      \implies&\rho
        \frac{U^2}{L} u_x^\ast \frac{\partial u_x^\ast}{\partial x^\ast} +
       \rho \frac{U^2 \delta}{\delta L} u_y^\ast \frac{\partial u_x^\ast}{\partial y^\ast}  = 
        \mu \frac{U}{L^2} \frac{\partial^2 u_x^\ast}{\partial {x^\ast}^2} +
        \mu \frac{U}{\delta^2} \frac{\partial^2 u_x^\ast}{\partial {y^\ast}^2}
      \nonumber\\
     (\times \delta^2)\implies &\rho
        \frac{U^2\delta^2}{L} \left(
          u_x^\ast \frac{\partial u_x^\ast}{\partial x^\ast} +
          u_y^\ast \frac{\partial u_x^\ast}{\partial y^\ast}
        \right) =
        \mu U \left(
          \frac{\delta^2}{L^2} \frac{\partial^2 u_x^\ast}{\partial {x^\ast}^2} +
          \frac{\partial^2 u_x^\ast}{\partial {y^\ast}^2}
        \right)
    \end{align}
    由於$\frac{\delta}{L} \ll 1$,因此可以忽略掉右邊的第一項
    \begin{equation}
      \frac{U^2\delta^2}{L} \left(
          u_x^\ast \frac{\partial u_x^\ast}{\partial x^\ast} +
          u_y^\ast \frac{\partial u_x^\ast}{\partial y^\ast}
        \right) =
        \mu U \frac{\partial^2 u_x^\ast}{\partial {y^\ast}^2}
    \end{equation}
    把變數全部扔到左邊
    \begin{equation}
      \frac{\rho U\delta^2}{\mu L}\left(
        u_x^\ast \frac{\partial u_x^\ast}{\partial x^\ast} +
        u_y^\ast \frac{\partial u_x^\ast}{\partial y^\ast}
      \right) = \frac{\partial^2 u_x^\ast}{\partial {y^\ast}^2}
    \end{equation}
  \item 又有了Goldilocks Argument:\\
  如果$\frac{\rho U\delta^2}{\mu L} \gg 1$,則代表慣性力遠大於黏滯力\\
  等式右側的$\frac{\partial^2 u_x^\ast}{\partial {y^\ast}^2}$會很大\\
  代表流體不想水平的待著,會違反No slip\\
  如果$\frac{\rho U\delta^2}{\mu L} \ll 1$,則代表黏滯力遠大於慣性力\\
  會沒有Free Stream的現象,邊界層無法發展\\
  因此必須要差不多是1,才能有邊界層的發展\\
  因此可以得到:
  \begin{equation}
    \frac{\rho U\delta^2}{\mu L} \approx 1 \implies \boxed{\delta(x) \sim \sqrt{\frac{\mu L}{\rho U}} \propto \sqrt{L}}
  \end{equation}
  也就是說,上圖的紅色線,大概會是$\delta(x) = C\sqrt{x}$的形式\\
  另外,我們若將其同除以$L$,會發現邊界層發展速率,與雷諾數的關係
  \begin{equation}
    \boxed{\frac{\delta(x)}{L} \sim \sqrt{\frac{\mu}{\rho U L}} = \text{Re}_L^{-\frac{1}{2}}} \label{eq:ch2_5_boundary_layer_dev_delta_Re}
  \end{equation}
  \item 回到一開始的問題,因為實際上沒有$L$這個長度,因此只能用$x$來表示
  \begin{enumerate}
    \item Governing Equation:
    \begin{align}
      \text{Continuity:}\quad & \frac{\partial u_x}{\partial x} + \frac{\partial u_y}{\partial y} = 0\nonumber\\
      \text{X-Motion:} \quad & \rho\left(
        u_x \frac{\partial u_x}{\partial x} +
        u_y \frac{\partial u_x}{\partial y}
      \right) = \mu \frac{\partial^2 u_x}{\partial y^2}
    \end{align}
    \item Boundary Conditions:
    \begin{align}
      &u_x(x,0) = 0 \\
      &u_x(x,\infty) = U \\
      &u_x(0,y) = U\\
      &u_y(x,0) = 0\\
      &u_y(x,\infty) = 0\\
      &u_y(0,y) = 0
    \end{align}
    \item 想要找到一個$\eta$,使得可以將偏微分方程式轉成常微分方程式\\
    這個$\eta$如同similarity approach一樣,希望能是$x$與$y$的函數\\
    但因為$x,y$都是同樣單位,所以就沒必要做無因次化了\\
    由剛剛的Goldilocks Argument,可以猜測$\eta$的形式為:
    \begin{equation}
      \eta = \frac{y}{\delta(x)} = y\left(
        \frac{\mu x}{\rho U}
      \right)^{-\frac{1}{2}}
    \end{equation}
    定義: \label{sec:ch2_5_boundary_layer_dev_definition_F}
    \begin{equation}
      \frac{u_x}{U} = F'(\eta)
    \end{equation}
    這邊定義$F'(\eta)$是為了之後方便積分而已\\
    因為可以由Equation of Continuity,將$u_y$換成$u_x$的函數\\
    可是Equation of Continuity是說兩個變數的微分相同,所以定$F'(\eta)$會比較方便
    \item 先從Equation of Motion 把$u_x$算出來:\\
    一樣記得要把多出來的東西想辦法換成$\eta$的函數
    \begin{itemize}
      \item $\frac{\partial u_x}{\partial x}$:
      \begin{align}
        \frac{\partial u_x}{\partial x} &= U \frac{\partial F'(\eta)}{\partial x} \nonumber\\
        &= U \frac{d F'(\eta)}{d \eta} \frac{\partial \eta}{\partial x} \nonumber\\
        &= U \left[
          F''(\eta) - \frac{1}{2}y \left(
            \frac{\mu x}{\rho U}
          \right)^{-\frac{1}{2}}x^{-1}
        \right] \nonumber\\
        &= -\frac{U}{2x} \eta F''(\eta)
      \end{align}
      然後這條可以拿去Equation of Continuity算$u_y$
      \begin{align}
        \frac{\partial u_y}{\partial y} &= -\frac{\partial u_x}{\partial x} \nonumber\\
        &= \frac{U}{2x} \eta F''(\eta)
      \end{align}
      積分得到:(由於$\eta = \frac{y}{\delta(x)}$,記得要換出來($\delta(x)d\eta = dy$))
      \begin{align}
        u_y &= \int \frac{U}{2x} \eta F''(\eta) dy \nonumber\\
        &= \frac{U}{2x}\delta(x) \int \eta F''(\eta) d\eta + G(x) \nonumber\\
        &= \frac{U}{2x}\delta(x) \left(
          \eta F'(\eta) - \int F'(\eta) d\eta
        \right) + G(x)
      \end{align}
      這裡就是為什麼要定$F'(\eta)$的原因,因為這樣積分會比較簡單
      \begin{equation}
        \boxed{u_y = \frac{U}{2x}\delta(x) \left(\eta F'(\eta) - F(\eta)\right) + G(x)}
      \end{equation}
      這裡有個支線任務,我們能帶入邊界條件$u_y(x,0) = 0$來求出$G(x)$
      \begin{align}
        u_y(x,0) &= 0 \nonumber\\
        \implies 0 &= \frac{U}{2x}\delta(x) \left(\eta F'(\eta) - F(\eta)\right)+ G(x) \nonumber\\
        \implies G(x) &= -\frac{U}{2x}\delta(x) \left(\eta F'(\eta) - F(\eta)\right)
      \end{align}
      而根據No slip的Boundary Condition,$F'(0)=0$
      \begin{equation}
        G(x) = \frac{U}{2x}\delta(x) F(0) = \frac{F(0)}{2}\sqrt{\frac{\mu U}{\rho x}}
      \end{equation}
      寫進去$u_y$:
      \begin{equation}
        u_y = \frac{U}{2x}\delta(x) \left(\eta F'(\eta) - F(\eta) + F(0)\right)
      \end{equation}
      至於$F(0)$的話,我們把他認為是\fbox{Arbitrary Constant}\\
      所以就讓$F(0)=0$,原話如下:
      \begin{quote}
        With no loss in generality given the functional form of $f$ can compenstate
      \end{quote}
      所以$G(x)$就消失了!
      \begin{equation}
        \boxed{u_y = \frac{U}{2x}\delta(x) \left(\eta F'(\eta) - F(\eta)\right)}
      \end{equation}
      \item $\frac{\partial u_x}{\partial y}$:
      \begin{align}
        \frac{\partial u_x}{\partial y} &= U \frac{\partial F'(\eta)}{\partial y} \nonumber\\
        &= U \frac{d F'(\eta)}{d \eta} \frac{\partial \eta}{\partial y} \nonumber\\
        &= U F''(\eta) \left(
          \frac{\mu x}{\rho U}
        \right)^{-\frac{1}{2}} \nonumber\\
        &= U F''(\eta) \frac{1}{\delta(x)} \nonumber\\
        &= \frac{U}{y} \eta F''(\eta)
      \end{align}
      \item $\frac{\partial^2 u_x}{\partial y^2}$:
      \begin{align}
        \frac{\partial^2 u_x}{\partial y^2} &= \frac{\partial}{\partial y} \left(
          \frac{\partial u_x}{\partial y}
        \right) \nonumber\\
        &= \frac{\partial}{\partial y} \left(
          \frac{U}{\delta(x)} F''(\eta)
        \right) \nonumber\\
        &= \frac{U}{\delta(x)} \frac{d F''(\eta)}{d \eta} \frac{\partial \eta}{\partial y} \nonumber\\
        & = \frac{U}{\delta(x)^2} F'''(\eta) \nonumber\\
        &= \frac{U\eta^2}{y^2} F'''(\eta)
      \end{align}
    \end{itemize}
    \item 把以上三個結果,還有$u_y$,代回Equation of Motion:
      \begin{align}
        & \rho\left[
          u_x \frac{\partial u_x}{\partial x} +
          u_y \frac{\partial u_x}{\partial y}
        \right] = \mu \frac{\partial^2 u_x}{\partial y^2} \nonumber\\
        \implies & \rho \left[
          U F'(\eta) \left(
            -\frac{U}{2x} \eta F''(\eta)
          \right) +
          \frac{U}{2x}\delta(x) \left(
            \eta F'(\eta) - F(\eta)
          \right) \frac{U}{\delta(x)} F''(\eta)
        \right] = \mu \frac{U}{\delta(x)^2} F'''(\eta)
      \end{align}
      由於一直寫$F(\eta)$很麻煩,所以簡寫成$F$\\
      $\delta(x)$用$\left(
        \frac{\mu x}{\rho U}
      \right)^{\frac{1}{2}}$代入\\
      $\eta$用$y\left(
        \frac{\mu x}{\rho U}
      \right)^{-\frac{1}{2}}$代入
      \begin{align}
         & \rho \left[
          U F' \left(
            -\frac{U}{2x} \eta F''
          \right) +
          \frac{U}{2x}\delta(x) \left(
            \eta F' - F
          \right) \frac{U}{\delta(x)} F''
        \right] = \mu \frac{U}{\delta(x)^2} F''' \nonumber\\
        (\div U) \implies & \rho \left[
          F' \left(
            -\frac{U}{2x} \eta F''
          \right) +
          \frac{1}{2x}\delta(x) \left(
            \eta F' - F
          \right) \frac{U}{\delta(x)} F''
        \right] = \mu \frac{1}{\delta(x)^2} F''' \nonumber\\
        (\text{換}\eta,\delta) \implies & \rho \left[
          F' \left(
            -\frac{U}{2x} y\left(
              \frac{\mu x}{\rho U}
            \right)^{-\frac{1}{2}} F''
          \right) \right. \nonumber\\
          &\phantom{\rho\quad}\left. +
          \frac{1}{2x}\left(
            \frac{\mu x}{\rho U}
          \right)^{\frac{1}{2}} \left(
            y\left(
              \frac{\mu x}{\rho U}
            \right)^{-\frac{1}{2}} F' - F
          \right) \frac{U}{\left(
            \frac{\mu x}{\rho U}
          \right)^{\frac{1}{2}}} F''
        \right] = \mu \frac{1}{\left(
          \frac{\mu x}{\rho U}
        \right)} F''' \nonumber\\
        \implies & \rho \left[
          F' \left(
            -\frac{U}{2x} y\left(
              \frac{\rho U}{\mu x}
            \right)^{\frac{1}{2}} F''
          \right) +
          \frac{1}{2}\left(
            \frac{\mu}{\rho U x}
          \right)^{\frac{1}{2}} \left(
            y\left(
              \frac{\rho U}{\mu x}
            \right)^{\frac{1}{2}} F' - F
          \right)\left(
            \frac{\rho U^3}{\mu x}
          \right)^{\frac{1}{2}} F''
        \right] \nonumber\\
        &= \mu \frac{\rho U}{\mu x} F''' \nonumber\\
        \implies & \rho \left[ F' \left(
            -\frac{1}{2} y\left(
              \frac{\rho U^3}{\mu x^3}
            \right)^{\frac{1}{2}} F''
          \right) +
          \frac{U}{2x} \left(
            \left(
              \frac{\rho U y^2}{\mu x}
            \right)^{\frac{1}{2}} F' - F
          \right) F''
        \right] = \frac{\rho U}{x} F'''\nonumber
      \end{align}
      再整理(分頁用)
      \begin{align}
        (\div \mu)\implies &  F' \left(
            -\frac{1}{2} y\left(
              \frac{\rho U^3}{\mu x^3}
            \right)^{\frac{1}{2}} F''
          \right) +
          \frac{U}{2x} \left(
            \left(
              \frac{\rho U y^2}{\mu x}
            \right)^{\frac{1}{2}} F' - F
          \right) F''= \frac{U}{x} F''' \nonumber\\
       (\div \frac{U}{x}) \implies &  F' \left(
            -\frac{1}{2} y\left(
              \frac{\rho U}{\mu x}
            \right)^{\frac{1}{2}} F''
          \right) +
          \frac{1}{2} \left(
            \left(
              \frac{\rho U y^2}{\mu x}
            \right)^{\frac{1}{2}} F' - F
          \right) F''= F''' \nonumber\\
        \implies & 
        \cancel{-\frac{1}{2}\left(
          \frac{\rho Uy^2}{\mu x}
        \right)^{\frac{1}{2}} F' F'' +
        \frac{1}{2}\left(
          \frac{\rho Uy^2}{\mu x}
        \right)^{\frac{1}{2}} F' F''} -
        \frac{1}{2} F F'' = F''' \nonumber\\
        \implies & F''' + \frac{1}{2} F F'' = 0\nonumber\\
        &\boxed{2F''' + F F'' = 0}
      \end{align}
      很開心對吧?但這個是Blasius Equation,到現在沒人解出來哦!
    \item  加上邊界條件:
    \begin{align}
      &F'(0) = 0 \\
      &F'(\infty) = 1 \\
      &F(0) = 0
    \end{align}
    \item Blasius Equation的數值解:
    \begin{itemize}
      \item 定義邊界層厚度$\delta$為$u_x = 0.99U$的位置\\
      厚度:
      \begin{equation}
        y = \delta(x) \approx 5\sqrt{\frac{\mu x}{\rho U}}
      \end{equation}
      \item Drag Coefficient:
      \begin{equation}
        F_x = -2 w\int_0^L \tau_{yx} dx
      \end{equation}
      其中$\tau_{yx} = \mu \left(\frac{\partial u_x}{\partial y}+ \frac{\partial u_y}{\partial x}\right)$\\
      由於邊界層很薄,$\frac{\partial u_x}{\partial y} \gg \frac{\partial u_y}{\partial x}$
      或者寫成特徵長度模式:
      \begin{equation}
        -\mu\left(\frac{U}{\delta}+\cancel{\frac{V}{L}}\right),L\gg \delta 
      \end{equation}
      因此可以忽略掉$\frac{\partial u_y}{\partial x}$\\
      所以:
      \begin{align}
        \tau_{yx} &= -\mu \frac{\partial u_x}{\partial y} \nonumber\\
        &= -\mu \frac{\partial}{\partial y} \left(
          U F'(\eta)
        \right) \nonumber\\
        &= -\mu U F''(\eta) \frac{\partial \eta}{\partial y} \nonumber\\
        &= -\mu U F''(\eta) \left(
          \frac{\mu x}{\rho U}
        \right)^{-\frac{1}{2}} \nonumber\\
        &= -\mu U F''(\eta) \frac{1}{\delta(x)} \nonumber\\
        &= -\mu U F''(0) \sqrt{\frac{\rho U}{\mu x}} \label{eq:ch2_5_boundary_layer_dev_shear_stress}
      \end{align}
      另外,由於$\eta$是正比於$y$的,所以可以假想$F(\eta)$也是正比於$y$的\\
      或者說在邊界層內,速度是線性增加的\\
      我們假定速率分布是$u_x= \beta(x)y$\\
      那麼:
      \begin{align}
        \beta(x) &= \left[
          \frac{\partial u_x}{\partial y}
        \right]_{y=0} \nonumber\\
        & = \left[
          \frac{\partial u_x}{\partial \eta} \frac{\partial \eta}{\partial y}
        \right]_{\eta = 0} \nonumber\\
        &= U F''(0) \left(
          \frac{\partial \eta}{\partial y}
        \right)_{\eta=0} \nonumber\\
        &= U F''(0) \sqrt{\frac{\rho U}{\mu x}}
      \end{align}
      而解析解的數值發現$F''(0) = 0.332$
      \begin{equation}
        \beta(x) = 0.332 U \sqrt{\frac{\rho U}{\mu x}} \implies \boxed{u_x = 0.332 U \sqrt{\frac{\rho U}{\mu x}} y} \label{eq:ch2_5_boundary_layer_dev_velocity_profile}
      \end{equation}
      而將速度分布代入Equation of Continuity,可以換出$u_y$:
      \begin{align}
        \frac{\partial u_x}{\partial x} + \frac{\partial u_y}{\partial y} &= 0 \nonumber\\
        \implies \frac{\partial u_y}{\partial y} &= -\frac{\partial u_x}{\partial x} \nonumber\\
        &= -\frac{\partial}{\partial x} \left(
          0.332 \sqrt{\frac{\rho U^3}{\mu x}} y
        \right) \nonumber\\
       &= -0.332 \sqrt{\frac{\rho U^3}{\mu}} y \left(
        -\frac{1}{2} x^{-\frac{3}{2}}
        \right) \nonumber\\
       &= 0.166 \sqrt{\frac{\rho U^3}{\mu x^3}} y
      \end{align}
      積分得到:
      \begin{equation}
        u_y = 0.166 \sqrt{\frac{\rho U^3}{\mu x^3}} \frac{y^2}{2} + H(x) = 0.083 \sqrt{\frac{\rho U^3}{\mu x^3}} y^2 + H(x)
      \end{equation}
      由於$u_y(x,0) = 0$,所以$H(x) = 0$
      \begin{equation}
        \boxed{u_y = 0.083 \sqrt{\frac{\rho U^3}{\mu x^3}} y^2} \label{eq:ch2_5_boundary_layer_dev_velocity_profile_u_y}
      \end{equation}
      \item Drag Force:\\
      將(\ref{eq:ch2_5_boundary_layer_dev_shear_stress})代入Drag Force:
      \begin{align}
        F_x &= -2w \int_0^L -\mu U \sqrt{\frac{\rho U}{\mu x}} F''(0) dx \nonumber\\
        &= 2w \mu U F''(0) \sqrt{\frac{\rho U}{\mu}} \int_0^L x^{-\frac{1}{2}} dx \nonumber\\
        &= 2w \mu U F''(0) \sqrt{\frac{\rho U}{\mu}} \left[
          2L^{\frac{1}{2}}
        \right] \nonumber\\
        &= 4w F''(0) \sqrt{\rho \mu U^3 L} \propto \sqrt{L}
      \end{align}
      解析數值:
      \begin{equation}
        F_x = 1.328 \sqrt{w^2 \rho \mu U^3 L}
      \end{equation}
      P.S. Scalmg 做實驗說:
      \begin{equation}
        F_x \approx 4.0 \sqrt{w^2 \rho \mu U^3 L}
      \end{equation}
      就是跟我們的有個只差係數的誤差
      \item Drag Coefficient
      \begin{align}
        C_D &= \frac{F_X}{\text{A}\cdot \text{KE}} \nonumber\\
        &= \frac{1.328 \sqrt{w^2 \rho \mu U^3 L}}{2wL \cdot \frac{1}{2}\rho U^2} \nonumber\\
        &= \frac{1.328}{\sqrt{\text{Re}_x}} 
      \end{align}
      P.S. 會是$2wL$是因為又假設有上板了
      \item 衍生,管中的Entrance Length:\\
      假設一個圓管,定義Fully Developed的時候是邊界層互相接觸的時候\\
      也就是邊界層$\delta = \frac{D}{2}$的時候\\
      我們想要找到要進去多深的$x$,稱作Entrance Length $L_e$\\
      而解析的結果說明:
      \begin{equation}
        \delta(L_e) = 5 \sqrt{\frac{\mu L_e}{\rho U}} = \frac{D}{2}
      \end{equation}
      解出來:
      \begin{equation}
        D^2 = 100 \frac{\mu L_e}{\rho U} \implies L_e = \frac{\rho U D^2}{100 \mu} = 0.01 \text{Re}_D D
      \end{equation}
      或者寫成:
      \begin{equation}
        \boxed{\frac{L_e}{D} = 0.01 \text{Re}_D}
      \end{equation}
      這能充分預估一個直徑為$D$的管子,遇上雷諾數Re的流體\\
      需要約1\%Re的長度,才能讓流體進入Fully Developed狀態
    \end{itemize}
  \end{enumerate}
  \end{itemize}
  \item 紊流的經驗假設:
  \begin{itemize}
    \item 紊流的整體Shear stress,為$r$為水力半徑時的Shear stress\\
    注意半徑為$R$的圓管的水力半徑$R_h = \frac{\pi R^2}{2\pi R} = \frac{R}{2}$
    \item 邊界層的發展:
    \begin{figure}[H]
      \centering
      \begin{tikzpicture}[>=Latex, line cap=round, line join=round, thick]
        \draw (-1,0) -- (14,0);
        \draw (0,0) .. controls (2,2) and (3,2) .. (6,2);
        \draw (6,2) .. controls (8,4) and (9,4) .. (10,4);
        \draw[dashed] (6,2) .. controls (8,0.5) and (9,0.5) .. (10,0.5);
        \node at (8,1.7) {Sub Layer};
        \draw (10,4) -- (14,4);
        \node at (4, 1) {層流邊界層};
        \node at (12, 2) {紊流邊界層};
        \draw[<->] (0,-0.5) -- (6,-0.5) node[midway, fill=white] {$\delta_{laminar}$};
        \draw[<->] (6,-0.5) -- (10,-0.5) node[midway, fill=white] {$\delta_{transition}$};
        \draw[<->] (10,-0.5) -- (14,-0.5) node[midway, fill=white] {$\delta_{turbulent}$};
        \draw[red, dashed] (0,0) .. controls (1,0.5) and (2,0.5) .. (6,0.5) -- (14,0.5);
        \draw[->, blue] (-2,1) -- (0,1);
        \draw[->, blue] (-2,2) -- (0,2);
        \draw[->, blue] (-2,3) -- (0,3);
        \draw[->, blue] (-2,4) -- (0,4);
        \draw[->, blue] (-2,5) -- (0,5);
        \node[anchor=south, red] at (8,0) {\tiny Stagnant Fluid(靜止層)};
        \node[anchor=south, red] at (12,0) {\tiny 純傳導};
        \node[anchor=south, teal] at (12,0.5) {\tiny 對流+傳導};
        \draw[->, red] (5.5,0) -- (5.5,0.5) node[midway, left] {$q_k$};
        \draw[teal, decorate, decoration={snake, amplitude=3pt, segment length=4pt}] 
          (5.5,0.5) -- (5.5,1.2);
        \draw[->, teal] (5.5,1.2) -- (5.5,1.5) node[anchor=south]{$q_c$};
        \draw[dashed] (6,-1) -- (6,5);
        \draw[dashed] (10,-1) -- (10,5);
      \end{tikzpicture}
      \caption{紊流邊界層發展示意圖}
    \end{figure}
  \end{itemize}
\end{itemize}
\end{CJK*}
\end{document}