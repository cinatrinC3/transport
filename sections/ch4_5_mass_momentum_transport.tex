\documentclass[../main.tex]{subfiles}
\begin{document}
\begin{CJK*}{UTF8}{bkai}
\subsection{質量輸送,質量與動量一起傳輸、Penetration Theory}
\begin{itemize}
  \item Mass transfer for a rotating disk
  \begin{enumerate}
    \item Limiting current $\rightarrow$ diffusion dominated
    \item dilute species in aquerous solution
    \item uniforming accessible的$v_z(z)$在旋轉下的轉換:
    \begin{equation}
      v_z(z) = \sqrt{\nu\omega} H(\eta),\quad \eta = z\sqrt{\frac{\omega}{\nu}}
    \end{equation}
    \item 寫出物種平衡式:
    \begin{equation}
      \cancel{\frac{\partial C_i}{\partial t}} + \nabla \cdot \vec{\bm N_i} - \cancel{R_i} = 0
    \end{equation}
    以及:
    \begin{equation}
      \vec{\bm N_i} = C_i \vec{\bm v} - D_{is} \nabla C_i
    \end{equation}
    代入後得到:
    \begin{equation}
      \vec{\bm v} \cdot \nabla C_i = D_{is} \nabla^2 C_i
    \end{equation}
    \item 令$\vec{\bm v}\approx v_z {\bm\delta}_z$ only,且$v_z(z)$\\
    P.S.$v_r=r\omega F(\xi)$,$v_\theta=r\omega G(\xi)$\\
    列出最簡化形式的物種平衡式(Dilute):
    \begin{equation}
      v_z(z)\frac{d C_i}{dz} = D_{is}\frac{d^2 C_i}{dz^2}
    \end{equation}
    \item Boundary conditions:
    \begin{align}
      C_i(\infty) &= C_\infty\\
      C_i(0) &= C_0 \to 0 \text{limiting current}
    \end{align}
    \item 無因次化濃度:
    \begin{equation}
      \theta_i = \frac{C_i(z)-C_0}{C_\infty -C_0}
    \end{equation}
    代入後:
    \begin{equation}
      v_z(z)\frac{d\theta_i}{dz}=D_{is}\frac{d^2\theta_i}{dz^2}
    \end{equation}
    邊界條件改為:
    \begin{align}
      \theta_i(\infty) &= 1\\
      \theta_i(0) &=0
    \end{align}
    代入uniforming accessible的$v_z(z)$:
    \begin{equation}
      v_z(z)= \sqrt{\nu\omega} H(\eta), \quad \eta = z\sqrt{\frac{\omega}{\nu}}, \quad
      d\eta = dz\sqrt{\frac{\omega}{\nu}}
    \end{equation}
    代入後:
    \begin{equation}
      \sqrt{\nu\omega}H(\eta)\sqrt{\frac{\omega}{\nu}}\frac{d\theta_i}{d\eta} 
      = D_{is}\frac{\omega}{\nu}\frac{d^2\theta_i}{d\eta^2}
    \end{equation}
    移項:
    \begin{equation}
      H(\eta)\frac{d\theta_i}{d\eta} = \frac{D_{is}}{\nu}\frac{d^2\theta_i}{d\eta^2}
    \end{equation}
    \item 定義Schmidt number!
    \begin{equation}
      \text{Sc} = \frac{\nu}{D_{is}}
    \end{equation}
    \item 寫出ODE:
    \begin{equation}
      \frac{d^2\theta_i}{d\eta^2} - \text{Sc} H(\eta)\frac{d\theta_i}{d\eta} = 0
    \end{equation}
    利用Reduction of order求解:
    \begin{equation}
      \theta_i(\eta) = \frac{
        \int_0^\eta e^{\text{Sc}\int_0^q H(\tilde\eta)d\tilde\eta}dq
      }{
        \int_0^\infty e^{\text{Sc}\int_0^q H(\tilde\eta)d\tilde\eta}dq
      }
    \end{equation}
    \item 令$P = \frac{d\theta_i}{d\eta}$,則:
    \begin{equation}
      P' - \text{Sc} H(\eta) P = 0
    \end{equation}
    \item 解出$P$:
    \begin{align}
      \frac{dP}{P} &=\text{Sc}H(\eta)d\eta\\
      \ln P\big|_0^\eta &= \int_0^\eta \text{Sc}H(\tilde \eta)d\tilde\eta\\
      \ln P(\eta) = \int_0^\eta \text{Sc}H(\tilde \eta)d\tilde\eta + \ln P(0)\\
      P(\eta) = P(0) e^{\int_0^\eta \text{Sc}H(\tilde \eta)d\tilde\eta}
    \end{align}
    \item 積分$P$得到$\theta_i$:
    \begin{align}
      \theta_i(\eta)\bigg|_{\theta_i(0)}^{\theta_i(\eta)} &= \int_0^\eta P(0) 
      e^{\int_0^q \text{Sc}H(\tilde \eta)d\tilde\eta} dq\\
      \theta_i(\eta) &= P(0) \int_0^\eta e^{\int_0^q \text{Sc}H(\tilde \eta)d\tilde\eta} dq
    \end{align}
    \item 利用邊界條件$\theta_i(\infty)=1$
    \begin{equation}
      \theta(\infty) =1 \implies P(0) =\frac{1}{
        \int_0^\infty e^{\int_0^q \text{Sc}H(\tilde \eta)d\tilde\eta} dq
      }
    \end{equation}
    \item 計算Current / flux:
    \begin{align}
      -N_{iz} &= D_{is} \frac{dC_i(0)}{dz}\nonumber\\
      & = D_{is}\left(C_\infty - C_0\right) \frac{d\theta_i(0)}{d\eta}\frac{d\eta}{dz}\nonumber\\
      &= D_{is}\left(C_\infty - C_0\right) \frac{P(0)}{\sqrt{\frac{\nu}{\omega}}}\frac{d\theta_i(0)}{d\eta}\nonumber\\
      &= D_{is}\left(C_\infty - C_0\right)\sqrt{\nu\omega} \cdot P(0)\nonumber\\
      \frac{N}{\sqrt{\nu\omega}\left(C_\infty - C_0\right)} &=\frac{P(0)}{\text{Sc}} \label{eq:ch4_5_mass_momentum_Nondim_flux}
    \end{align}
    \item 用Newton's law of viscosity求出$W$:
    \begin{equation}
      W = k_cA_{\text{char}}\Delta C = k_c A_{\text{char}} (C_\infty - C_0) = N A_{\text{char}}
    \end{equation}
    代表:
    \begin{equation}
      k_c (C_\infty - C_0) = N
    \end{equation}
    代回(\ref{eq:ch4_5_mass_momentum_Nondim_flux}):
    \begin{equation}
      \frac{k_c (C_\infty - C_0)}{\sqrt{\nu\omega}(C_\infty - C_0)} = 
      \boxed{\frac{k_c}{\sqrt{\nu\omega}}} = \frac{P(0)}{\text{Sc}} 
      = \frac{1}{\text{Sc}\int_0^\infty e^{\int_0^q \text{Sc}H(\tilde \eta)d\tilde\eta} dq}
    \end{equation}
    而$\frac{k_c}{\sqrt{\nu\omega}}$即為Stanton number
    \item 針對$H(\eta)$做近似:\\
    在Sc很大時,可以假設質傳的boundary layer很薄
    \begin{equation}
      H(\eta) \approx -0.5102\eta^2 + \frac{1}{3}\eta^3 + \frac{-0.616}{6}\eta^4
    \end{equation}
    若保留第一項而已:
    \begin{align}
      \text{St} &= \frac{1}{\text{Sc}\int_0^\infty e^{\int_0^q \text{Sc}H(\tilde \eta)d\tilde\eta} dq} \nonumber\\
      &\approx \frac{1}{\text{Sc}}\int_0^\infty e^{\int_0^q -0.5102\text{Sc}\cdot \tilde\eta^2 d\tilde\eta} dq \nonumber\\
      &= \frac{1}{\text{Sc}}\int_0^\infty e^{-0.1701\text{Sc}\cdot q^3} dq
    \end{align}
    讓$a=0.1701$,$t=a\text{Sc}\cdot q^3$,則:
    \begin{align}
      q &= \left(\frac{t}{a\text{Sc}}\right)^{\frac{1}{3}}\\
      dq &= \frac{1}{3}\left(\frac{1}{a\text{Sc}}\right)^{\frac{1}{3}} t^{-\frac{2}{3}} dt
    \end{align}
    改寫為:
    \begin{align}
      \int_0^\infty e^{-a\text{Sc}\cdot q^3} dq &= 
      \frac{1}{3}\left(a\text{Sc}\right)^{-\frac{1}{3}}\int_0^\infty t^{-\frac{2}{3}} e^{-t} dt \nonumber\\
      &= \frac{1}{3}\left(a\text{Sc}\right)^{-\frac{1}{3}} \Gamma\left(\frac{1}{3}\right)
    \end{align}
    最後得到:
    \begin{equation}
      \boxed{\text{St}=0.6204 \text{Sc}^{-\frac{2}{3}}}
    \end{equation}
    \item 如果Sc很小時,代表$\eta$可以很大,則$H(\eta)\approx 0.88447$是個常數\\
    而此時:
    \begin{equation}
      \text{St} \approx 0.88447
    \end{equation}
    \item 若以St為縱軸,Sc為橫軸繪圖,可以得到下圖:
    \begin{figure}[H]
      \centering
      \begin{tikzpicture}[>=Latex, line cap=round, line join=round, thick]
        \begin{loglogaxis}[
            width=10cm, height=8cm, xmin=0.05, xmax=5000, ymin=0.004, ymax=2,
            axis lines = left, axis line style = {thick, -latex}, % Arrows at ends
            xlabel = {$\text{Sc} = \nu/D$}, ylabel = {$\text{St}=\frac{k_c}{\sqrt{\omega \nu}}$},
            every axis x label/.style={at={(ticklabel* cs:1.05)}, anchor=west},
            every axis y label/.style={at={(ticklabel* cs:1.05)}, anchor=south},
            xtick={0.1, 1, 10, 100, 1000},
            xticklabels={0.1, 1, 10, 100, $10^3$},
            ytick={0.01, 0.1, 1},
            yticklabels={0.01, 0.1, 1},
            grid=none,
          ]
          \addplot [gray, domain=0.05:200, samples=2] {0.88447};
          \node at (axis cs: 80, 1.2) {\bfseries 0.88447};
          \addplot [red, thick, domain=20:5000, samples=10] {0.6204 * x^(-2/3)};
          \node[red, anchor=west] (eqlabel) at (axis cs: 150, 0.08) {$\bm 0.6204 Sc^{-\frac{2}{3}}$};
          \draw[red, ->, thick] (eqlabel.west) -- (axis cs: 80, 0.04);
          \draw[ultra thick, black] 
              (axis cs: 0.1, 0.88447).. controls (axis cs: 1, 0.88447) and (axis cs: 20, 0.06) .. 
              (axis cs: 2000, {0.6204 * 2000^(-2/3)});
          \draw[thin, black] (axis cs: 800, 0.004) -- (axis cs: 800, 0.015);
          \end{loglogaxis}
        \end{tikzpicture}
      \caption{Stanton number as a function of Schmidt number}
      \end{figure}
  \end{enumerate}
  \item 滲透理論 (Penetration Theory)
  \begin{figure}[H]
    \centering
    \begin{tikzpicture}[>=Latex, line cap=round, line join=round, thick]
      \draw (0,0) -- (0,6);
      \draw [<->] (-0.6, 0 )-- (-0.6, 6) node [midway, fill=white] {$L$};
      \fill[pattern=north east lines] (0,0) rectangle (-0.3, 6);
      \draw[dashed] (3,0) -- (3,6);
      \node at (5,3.5) {Gas A};
      \draw[->] (3,6) -- (2,6) node[left] {$x$};
      \draw[dashed] (3,6) -- (4,6);
      \draw[->] (3.5, 6) -- (3.5,5) node[below] {$z$};
      \draw[blue] (0,4) -- (3,4);
      \draw[blue, dashed] (0,4) .. controls (0,3) and (2,3) .. (3,3);
      \path[name path=A] (0,4) .. controls (0,3) and (2,3) .. (3,3);
      \path[name path=Ba] (0.5,4) -- (0.5,2);
      \path[name path=Bb] (1,4) -- (1,2);
      \path[name path=Bc] (1.5,4) -- (1.5,2);
      \path[name path=Bd] (2,4) -- (2,2);
      \path[name path=Be] (2.5,4) -- (2.5,2);
      \path[name intersections={of=A and Ba, by=Ca}];
      \path[name intersections={of=A and Bb, by=Cb}];
      \path[name intersections={of=A and Bc, by=Cc}];
      \path[name intersections={of=A and Bd, by=Cd}];
      \path[name intersections={of=A and Be, by=Ce}];
      \draw[blue, ->] (0.5,4) -- (Ca);
      \draw[blue, ->] (1,4) -- (Cb);
      \draw[blue, ->] (1.5,4) -- (Cc);
      \draw[blue, ->] (2,4) -- (Cd);
      \draw[blue, ->] (2.5,4) -- (Ce);
      \node[anchor=south] at (1.5, 4) {$u_z(x)$};
      \draw[<->] (2,0.5) -- (3,0.5) node[midway, below] {$\delta$};
      \draw[red,dashed] (2,0) -- (2,6);
      \draw[red] (0,2) -- (3,2);
      \node[anchor=south] at (1.5, 2) {$C_A(x)$};
      \draw[red, dashed] (0,2) -- (2,2) ..controls (3,2) and (3,1.5) .. (3,1);
      \node[anchor=west] at (3,1) {$C_{A0}$};
      \draw[->,blue] (1.5,7) -- (1.5,6.5);
      \draw[->,blue] (2,7) -- (2,6.5);
      \draw[->,blue] (2.5,7) -- (2.5,6.5);
      \draw[->,blue] (0.5,7) -- (0.5,6.5);
      \draw[->,blue] (1,7) -- (1,6.5);
      \draw[dashed, blue] (0,6.5) rectangle (3,7);
      \draw[dashed] (-1,0) -- (0,0);
      \draw[dashed] (-1,6) -- (0,6);
      \draw[red, ->] (4, 2.5) -- (2.5,2.5);
      \node[anchor=west] at (4,2.5) {$N_A(z)\big|_{x=0}$};
    \end{tikzpicture}
    \caption{Penetration Theory示意圖,吸收塔}
  \end{figure}
  \begin{enumerate}
    \item Laminar flow, Penetration distance $\delta$ is small, S.S.\\
    \fbox{$\text{Pe}_M$ is large},z方向對流遠大於z方向擴散
    \item 先求速率分佈:\\
    Laminar flow in z direction:
    \begin{equation}
      u_z(z) =  u_{\text{max}}\left[
        1-\left(\frac{x}{\delta}\right)^2
      \right]
    \end{equation}
    P.S. 這邊的$\delta$不是質傳的Penetration distance,而是流體力學的Boundary layer thickness\\
    但不過沒關係,因為題目說Penertration distance很小\\
    所以直接刪掉視為在穿透區間內$u_z$為常數即可
    \begin{equation}
      u_z(z) \approx u_{\text{max}}
    \end{equation}
    \item 求$C_A(x,z)$\\
    帶入Equation of Continuity:
    \begin{equation}
      \cancel{\frac{\partial C_A}{\partial t}} + \vec{\bm u}\cdot\nabla C_A = D_{AB}\nabla^2 C_A
      + \cancel{R_A}
    \end{equation}
    2D:
    \begin{equation}
      \cancel{u_x}\frac{\partial C_A}{\partial x} + \underbrace{u_z \frac{\partial C_A}{\partial z}}_{z\text{方向質量對流}} = D_{AB}
      \left(
        \underbrace{\frac{\partial^2 C_A}{\partial x^2}}_{\text{x方向擴散}} 
        + \underbrace{\cancelto{\text{Pe}_M}{\frac{\partial^2 C_A}{\partial z^2}}}_{z\text{方向擴散}}
      \right)
    \end{equation}
    得到微分方程:
    \begin{equation}
      u_{\text{max}}\frac{\partial C_A}{\partial z} = D_{AB}\frac{\partial^2 C_A}{\partial x^2}
      \label{eq:ch4_5_mass_momentum_penetration_PDE}
    \end{equation}
    邊界條件:
    \begin{align}
      C_A(x,0) &= 0 \\
      C_A(0,z) &= C_{A0} \\
      C_A(\delta,z) &=0
    \end{align}
    而由於後兩個邊界條件是非齊次的,所以不能用分離變數法\\
    故使用結合變數法:
    \begin{equation}
      C_A(x,z) \to C_A(\eta)
    \end{equation}
    省略因次分析後(跟流體力學時的方法相同),見(\ref{sec:ch2_5_boundary_layer_dev})\\
    但建議就背起來,令
    \begin{equation}
      \boxed{\eta= ax^bz^c  = \frac{x}{2\sqrt{\frac{D_{AB}}{u_{\text{max}}}z}}}
    \end{equation}
    代入(\ref{eq:ch4_5_mass_momentum_penetration_PDE})後,得到ODE:
    \begin{align}
      \frac{\partial C_A}{\partial z} &= 
      \frac{d C_A}{d\eta} \frac{\partial \eta}{\partial z} \nonumber\\
      &= \frac{d C_A}{d\eta}\cdot\frac{\partial}{\partial z}\left(
        \frac{x}{2\sqrt{\frac{D_{AB}}{u_{\text{max}}}z}}
      \right) \nonumber\\
      &= \frac{d C_A}{d\eta}\frac{x}{2\sqrt{\frac{D_{AB}}{u_{\text{max}}}}}\cdot
       \left(-\frac{1}{2}z^{-\frac{3}{2}}\right) \nonumber\\
      &= -\frac{\eta}{2z}\frac{d C_A}{d\eta}
    \end{align}
    另一邊:
    \begin{align}
      \frac{\partial C_A}{\partial x} &=
      \frac{d C_A}{d\eta} \frac{\partial \eta}{\partial x} \nonumber\\
      &= \frac{d C_A}{d\eta}\cdot \frac{\partial}{\partial x}\left(
        \frac{x}{2\sqrt{\frac{D_{AB}}{u_{\text{max}}}z}}
      \right) \nonumber\\
      &= \frac{d C_A}{d\eta}\cdot \frac{1}{2\sqrt{\frac{D_{AB}}{u_{\text{max}}}z}} \cancelto{1}{\frac{\partial x}{\partial x}} \nonumber\\
      &= \frac{d C_A}{d\eta}\cdot \frac{1}{2\sqrt{\frac{D_{AB}}{u_{\text{max}}}z}}
    \end{align}
    再次求二次導數:
    \begin{align}
      \frac{\partial^2 C_A}{\partial x^2} &=
      \frac{\partial}{\partial x}\left(
        \frac{d C_A}{d\eta}\cdot \frac{1}{2\sqrt{\frac{D_{AB}}{u_{\text{max}}}z}}
      \right) \nonumber\\
      &= \frac{1}{2\sqrt{\frac{D_{AB}}{u_{\text{max}}}z}} \cdot
      \frac{\partial}{\partial x}\left(
        \frac{d C_A}{d\eta}
      \right) \nonumber\\
      &= \frac{1}{2\sqrt{\frac{D_{AB}}{u_{\text{max}}}z}} \cdot
      \frac{d^2 C_A}{d\eta^2} \cdot \frac{\partial \eta}{\partial x} \nonumber\\
      &= \frac{1}{2\sqrt{\frac{D_{AB}}{u_{\text{max}}}z}} \cdot
      \frac{d^2 C_A}{d\eta^2} \cdot \frac{1}{2\sqrt{\frac{D_{AB}}{u_{\text{max}}}z}} \nonumber\\
      &= \frac{1}{4\frac{D_{AB}}{u_{\text{max}}}z} \cdot \frac{d^2 C_A}{d\eta^2}
    \end{align}
    代入(\ref{eq:ch4_5_mass_momentum_penetration_PDE}):
    \begin{align}
      u_{\text{max}}\left(-\frac{\eta}{2z}\frac{d C_A}{d\eta}\right) &=
      D_{AB}\left(\frac{1}{4\frac{D_{AB}}{u_{\text{max}}}z} \cdot \frac{d^2 C_A}{d\eta^2}\right) \nonumber\\
      -\frac{u_{\text{max}}\eta}{2z}\frac{d C_A}{d\eta} &=
      \frac{u_{\text{max}}}{4z} \cdot \frac{d^2 C_A}{d\eta^2}
    \end{align}
    移項後得到ODE:
    \begin{equation}
      \frac{d^2 C_A}{d\eta^2} + 2\eta \frac{d C_A}{d\eta} = 0
    \end{equation}
    \item 解ODE:\\
    令 $P = \frac{d C_A}{d\eta}$:
    \begin{align}
      \frac{dP}{d\eta} + 2\eta P &= 0\\
      \frac{dP}{P} &= -2\eta d\eta\\
      \ln P\big|_{C_1}^\eta &= -\eta^2\\
      \ln P(\eta) - \ln P(C_1) &= -\eta^2\\
      \ln P(\eta) = -\eta^2 + \ln P(C_1)\\
      P(\eta) = C_2 e^{-\eta^2}
    \end{align}
    一整個ERF的感覺:\\
    代回$P$:
    \begin{align}
      \frac{d C_A}{d\eta} &= C_2 e^{-\eta^2}\\
      C_A(\eta) &= C_2 \int_0^\eta e^{-\tilde\eta^2} d\tilde\eta + C_3
    \end{align}
    換掉Boundary condition:\\
    P.S. 第三個Boundary condition,先換成無限的形式:
    \begin{equation}
      C_A(\delta,z) = 0 \to C_A(\infty, z) = 0
    \end{equation}
    再利用:$\eta = \frac{x}{2\sqrt{\frac{D_{AB}}{u_{\text{max}}}z}}$\\
    當$x=\delta$時,$\eta\to\infty$\\
    故:
    \begin{align}
      C_A(0) &= C_{A0}\\
      C_A(\infty) &=  0
    \end{align}
    利用$C_A(0) = C_{A0}$求出$C_3$:
    \begin{equation}
      C_A(0) = C_2 \int_0^0 e^{-\tilde\eta^2} d\tilde\eta + C_3 = C_{A0} \implies C_3 = C_{A0}
    \end{equation}
    再利用$C_A(\infty) =0$求出$C_2$:
    \begin{align}
      C_A(\infty) &= C_2 \int_0^\infty e^{-\tilde\eta^2} d\tilde\eta + C_{A0} = 0\\
      C_2 \cdot \frac{\sqrt{\pi}}{2} + C_{A0} &= 0 \nonumber\\
      \implies C_2 = -\frac{2}{\sqrt{\pi}} C_{A0}
    \end{align}
    代入$C_2$,最後得到:
    \begin{align}
      C_A(\eta) &= C_{A0}\left[
        1 - \frac{2}{\sqrt{\pi}}\int_0^\eta e^{-\tilde\eta^2} d\tilde\eta
      \right] \nonumber\\
      &=C_{A0}\left[
        1 - \text{erf}(\eta)
      \right] \nonumber\\
      &= \boxed{C_{A0} \text{erfc}(\eta)}
    \end{align}
    \item 代入$\eta$:
    \begin{equation}
      C_A(x,z) = C_{A0} \text{erfc}\left(
        \frac{x}{2\sqrt{\frac{D_{AB}}{u_{\text{max}}}z}}
      \right) \label{eq:ch4_5_mass_momentum_penetration_concentration}
    \end{equation}
    為濃度分佈式
    \item 求Molar Flux,$N_A(z)\big|_{x=0}$\\
    代入Fick's first law:
    \begin{equation}
      N_A = -D_{AB}\frac{\partial C_A}{\partial x} + y_A(N_A+N_B)
    \end{equation}
    但因為Penetration distance很小,A的濃度很低,故可以忽略對流項,$y_A\to 0$\\
    故:
    \begin{equation}
      N_A = -D_{AB}\frac{\partial C_A}{\partial x}
    \end{equation}
    代入(\ref{eq:ch4_5_mass_momentum_penetration_concentration}):\\
    記得Error function的微分會搞Leibniz rule
    \begin{align}
      \frac{d}{dx}\text{erf}(\eta) &= \frac{d}{dx}\left[
        \frac{\int_0^\eta e^{-\tilde\eta^2} d\tilde\eta}{\int_0^\infty e^{-\tilde\eta^2} d\tilde\eta}
      \right] \nonumber\\
      &= \frac{\sqrt{\pi}}{2} \cdot \frac{d}{dx}\left[
        \int_0^\eta e^{-\tilde\eta^2} d\tilde\eta
      \right] \nonumber\\
      &= \frac{\sqrt{\pi}}{2} \cdot \left[
        \int_0^\eta \frac{\partial}{\partial x} e^{-\tilde\eta^2} d\tilde\eta + 
        \frac{\partial \eta}{\partial x} e^{-\eta^2}\big|_{\tilde\eta=\eta}
        - \frac{\partial 0}{\partial x} e^{-0^2}\big|_{\tilde\eta=0}
      \right] 
    \end{align}
    當$x=0$時,$\eta=0$,故
    \begin{align}
      &\frac{\sqrt{\pi}}{2} \cdot \left[
        \int_0^\eta \frac{\partial}{\partial x} e^{-\tilde\eta^2} d\tilde\eta + 
        \frac{\partial \eta}{\partial x} e^{-\eta^2}\big|_{\tilde\eta=\eta}
        - \frac{\partial 0}{\partial x} e^{-0^2}\big|_{\tilde\eta=0}
      \right]_{\eta=0} \nonumber\\
      =& \frac{\sqrt{\pi}}{2} \cdot \left[
        \int_0^0 \frac{\partial}{\partial x} e^{-\tilde\eta^2} d\tilde\eta + 
        \frac{\partial \eta}{\partial x} e^{-\eta^2}\big|_{\tilde\eta=0}
        - \frac{\partial 0}{\partial x} e^{-0^2}\big|_{\tilde\eta=0}
      \right] \nonumber\\
      =& \frac{\sqrt{\pi}}{2} \cdot \frac{\partial \eta}{\partial x}\bigg|_{\eta=0} \nonumber\\
      =& \frac{\sqrt{\pi}}{2} \cdot \frac{1}{2\sqrt{\frac{D_{AB}}{u_{\text{max}}}z}}
    \end{align}
    代入$N_A$:
    \begin{align}
      N_A(z)\big|_{x=0} &= -D_{AB}\frac{\partial C_A}{\partial x}\bigg|_{x=0} \nonumber\\
      &= -D_{AB} C_{A0} \cdot \frac{d}{dx}\text{erfc}(\eta)\bigg|_{x=0} \nonumber\\
      &= -D_{AB} C_{A0} \cdot \frac{d}{dx}\left[
        1- \text{erf}(\eta)
      \right]_{x=0} \nonumber\\
      &= D_{AB} C_{A0} \frac{2}{\sqrt{\pi}}\left[
        \frac{1}{2\sqrt{\frac{D_{AB}}{u_{\text{max}}}z}}
      \right] \nonumber\\
      &= \boxed{C_{A0} \sqrt{\frac{D_{AB} u_{\text{max}}}{\pi z}} }
    \end{align}
    \item 求經驗質傳係數 $k_c$
    \begin{equation}
      N_A(z)\big|_{x=0} = k_c (C_{A0}-0) \implies \boxed{k_c = \frac{N_A(z)\big|_{x=0}}{C_{A0}} 
      = \sqrt{\frac{D_{AB} u_{\text{max}}}{\pi z}}}
    \end{equation}
    此為\fbox{滲透定律,Penetration theory}:
    \begin{equation}
      k_c \propto \sqrt{D_{AB}}
    \end{equation}
    \item 積分求Total Molar transfer rate $W_A$:\\
    剛剛是求出了在任意在$x=0$上的$yz$平面上的\fbox{點}的Molar flux $N_A(z)\big|_{x=0}$\\
    現在要積分整個表面才能得到Total Molar transfer rate\\
    假設每一層z方向的有橫向寬度$W$
    \begin{equation}
      W_A = \int_0^W\int_0^L N_A(z)\big|_{x=0} dz dy
    \end{equation}
    代入$N_A(z)\big|_{x=0}$:
    \begin{align}
      W_A &= \int_0^W\int_0^L C_{A0} \sqrt{\frac{D_{AB} u_{\text{max}}}{\pi z}} dz dy \nonumber\\
      &= C_{A0} \sqrt{\frac{D_{AB} u_{\text{max}}}{\pi}} \int_0^W dy \int_0^L z^{-\frac{1}{2}} dz \nonumber\\
      &= C_{A0} \sqrt{\frac{D_{AB} u_{\text{max}}}{\pi}} \cdot W \cdot 2L^{\frac{1}{2}} \nonumber\\
      &= \boxed{2 C_{A0} W \sqrt{\frac{D_{AB} u_{\text{max}} L}{\pi}}}
    \end{align}
  \end{enumerate}
\end{itemize}
\end{CJK*}
\end{document}