\documentclass[../main.tex]{subfiles}
\begin{document}
\begin{CJK*}{UTF8}{bkai}
\subsection{質量輸送,Equimolar counterdiffusion}
\begin{figure}[H]
  \centering
  \begin{tikzpicture}[>=Latex, line cap=round, line join=round, thick]
    \fill[white, draw=black, ultra thick] (0,-0.3) rectangle (10,0.3);
    \fill[white, draw=black] (0,0) circle (2.5);
    \fill[white, draw=black] (10,0) circle (2.5);
    \draw[dashed] (2.5, 0) -- (2.5, -3);
    \draw[dashed] (7.5, 0) -- (7.5, -3);
    \fill[white] (0,-0.3) rectangle (10,0.3);
    \node at (0,2) {Pure A};
    \node at (10,2) {Pure B};
    \draw[<->] (2.5,-2) -- (7.5,-2) node[midway, fill=white] {$L$};
    \fill[pattern=north east lines, pattern color=blue] (5.75, -0.3) rectangle (6.25, 0.3);
    \draw[dashed, blue] (5.75, -0.3) -- (5.75, 0.3);
    \draw[dashed, blue] (6.25, -0.3) -- (6.25, 0.3);
    \node[anchor=east, teal] at (5.25,0) {\small $N_A$};
    \node[anchor=west, red] at (6.75,0) {\small $N_B$};
    \draw[->, teal] (5.25,0.15) -- (6.75,0.15);
    \draw[->, red] (6.75,-0.15) -- (5.25,-0.15);
    \node[anchor=south] at (6, 0.3) {C.V.};
    \draw[->] (2.5, -1) -- (3.5, -1) node[right] {$z$};
  \end{tikzpicture}
  \caption{Equimolar counterdiffusion示意圖}
\end{figure}
\begin{itemize}
  \item 假設A、B兩種物質在管道兩端,在$t_0$時開始擴散
  \item 擴散前後兩容器$P,V,T$皆不變,理想氣體,代表\fbox{Equimolar counterdiffusion}
  \begin{equation}
    N_A = -N_B
  \end{equation}
  \item寫出出Fick's law
  \begin{equation}
    N_A = -D_{AB}\frac{dC_A}{dz} + y_A \cancelto{0}{(N_A+N_B)} \implies N_A = -D_{AB}\frac{dC_A}{dz}
  \end{equation}
  \item 兩邊積分
  \begin{equation}
    N_A\int_0^L dz = -D_{AB}\int_{C_{A1}}^{C_{A2}} dC_A \implies N_A L = -D_{AB}\left(C_{A2}-C_{A1}\right)
  \end{equation}
  \item 得到$N_A(z)$為常數
  \begin{equation}
    N_A = \frac{D_{AB}}{L}\left(C_{A1}-C_{A2}\right)
  \end{equation}
  也可以換成:
  \begin{align}
    N_A &= \frac{D_{AB}}{L} C\left(
      y_{A1}-y_{A2}
    \right) \nonumber\\
    N_A &= \frac{P D_{AB}}{R T L} \left(
      y_{A1}-y_{A2}
    \right) \nonumber\\
    &= \frac{D_{AB}}{RTL}\left(
      P_{A1}-P_{A2}
    \right)
  \end{align}
  \item 算出濃度分布$C_A(z)$\\
  透過Equation of Continuity
  \begin{align}
    \cancelto{\text{S.S.}}{\frac{\partial C_A}{\partial t}} + \nabla\cdot N_A &= \cancel{R_A}\nonumber\\
    \frac{dN_A}{dz} &= 0 \nonumber\\
    \frac{d}{dz}\left(-
      D_{AB}\frac{dC_A}{dz}
    \right) &= 0 \nonumber\\
    \implies \quad \frac{d^2 C_A}{dz^2} &= 0\nonumber\\
    \implies \quad C_A(z) &= C_1 + C_2 z
  \end{align}
  代入邊界條件:
  \begin{align}
    C_A(0) &= C_{A1} = C_1 \nonumber\\
    C_A(L) &= C_{A2} = C_1 + C_2 L \nonumber\\
    \implies \quad C_2 &= \frac{C_{A2}-C_{A1}}{L}
  \end{align}
  最後得到:
  \begin{equation}
    \boxed{C_A(z) = C_{A1} + \left(
      \frac{C_{A2}-C_{A1}}{L}
    \right) z}
  \end{equation}
  P.S. $N_B=-N_A$:
  \begin{equation}
    C_B(z) = C_{B1} + \left(
      \frac{C_{B2}-C_{B1}}{L}
    \right) z
  \end{equation}
 \end{itemize}
\end{CJK*}
\end{document}