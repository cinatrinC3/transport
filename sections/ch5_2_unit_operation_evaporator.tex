\documentclass[../main.tex]{subfiles}
\begin{document}
\begin{CJK*}{UTF8}{bkai}
\subsection{蒸發器 Evaporator}
藉由加熱使溶劑因沸騰而汽化,以去除溶劑達到溶液純化的目的
\begin{itemize}
  \item 蒸發器的種類
  \begin{enumerate}
    \item 橫管式蒸發器:
    \begin{itemize}
      \item 水蒸氣於管內流動、溶液於管外流動
      \item 自然對流的加熱方式,加熱效果較差,高黏度流體效果更差
      \item 不適用易產生泡沫或結垢的流體
      \item 構造簡單、價格低廉、適合小規模生產
    \end{itemize}
    \item 垂直短管式蒸發器
    \begin{itemize}
      \item 水蒸氣於管外流動,溶液於管內流動,濃縮液則由底部流出
      \item 自然對流的加熱方式,可加裝擋板以消除泡沫,並增加對流效果
      \item 適用於易產生泡沫的流體,但結垢仍然不行
      \item 構造簡單,價格不貴
    \end{itemize}
    \item 垂直長管式蒸發器
    \begin{itemize}
      \item 水蒸氣於管外流動,溶液於管內流動,濃縮液則由底部流出
      \item 自然對流的加熱方式,但因為很長,熱傳效果更好
      \item 溶液只要經管一次,即可完成蒸發的目的,大幅縮短加熱時間
      \item 適用於易產生泡沫的流體、對溫度敏感的流體、同樣不適合用在易結垢的
      \item 不適合用在會有鹽類析出、黏度高的流體
    \end{itemize}
    \item 強制循環式蒸發器
    \begin{itemize}
      \item 水蒸氣於管外流動,溶液於管內流動,濃縮液則由底部流出
      \item 加裝循環泵浦作為融業的強制循環、強制對流、熱傳效果更好、大幅縮短加熱時間
      \item 適用於易產生泡沫的流體、對溫度敏感的流體、黏度高的流體
      \item 由於氣液飛濺的情形嚴重、泵浦的動力消耗、成本較高
    \end{itemize}
    \item 夾套式蒸發器(攪膜蒸發器)
  \end{enumerate}
  而根據操作也可以歸類成:
  \begin{enumerate}
    \item 單效蒸發器(Single-effect evaporator)
    \item 多效蒸發器(multiple-effect evaporator)
    \begin{enumerate}
      \item 順向多效蒸發器
      \item 逆向多效蒸發器
    \end{enumerate}
  \end{enumerate}
  兩者的比較
  \begin{center}
  \begin{tabular}{|c|c|c|}
    \hline   & 單效 & 多效 \\\hline
    操作成本 &  相同 & 相同 \\\hline
    蒸發總量 & 少 & 多\\\hline
    經濟效益 & 小 & 大\\\hline
    設備成本 & 少 & 多\\\hline
    操作性  & 易 & 難\\\hline
    物料損失 & 少 & 多\\\hline
    最終濃度 & 小 & 大\\\hline
    溫度 & $T_1$ & $T_1>T_2>\cdots>T_n$\\\hline
    壓力 & $P_1$ & $P_1>P_2>\cdots>P_n$ \\\hline
  \end{tabular}
  \end{center}
  \item 蒸發器的名詞介紹
  \begin{itemize}
  \item 蒸發量(Capacity),$V$:\\
    單位時間所去除溶劑的質量
    \item 蒸氣使用量(Steam Consumption),$S$:
    單位時間,蒸發器使用(消耗)的蒸氣質量
    \item 經濟效益(Economy),$\eta$:
    每使用單位質量蒸氣,所能去除溶劑的質量
    \begin{equation}
      \eta = \frac{V}{S}
    \end{equation}
  \end{itemize}
  \item 單效蒸發器的計算範例\\
  假設一單效蒸發器,進料$F$,狀態$T_F,X_F,h_F$,經過蒸氣$S$加熱後\\
  變為蒸氣$V$,狀態$T_1,X_V,H_V$,留下濃縮液$L$,狀態$T_1,X_L,h_L$\\
  蒸氣$S$通入前狀態$T_S,H_S$,通入後凝結變成液體,溫度不變,狀態$T_S,h_S$
  \begin{itemize}
    \item 整體質量平衡:
    \begin{equation}
      F= L+V
    \end{equation}
    \item 溶質質量平衡:
    \begin{equation}
      FX_F=LX_L
    \end{equation}
    \item 熱平衡:
    \begin{equation}
      F\cdot h_F + S\cdot \lambda = L\cdot h_K + V\cdot H_V
    \end{equation}
    其中$\lambda$為蒸氣的潛熱
    \begin{equation}
      \lambda = H_S -h_S
    \end{equation}
    \item 蒸發器的熱傳方程
    \begin{equation}
      Q = S\cdot(H_S -h_S) = S\cdot \lambda = UA(T_S-T_1)
    \end{equation}
  \end{itemize}
\end{itemize}
\end{CJK*}
\end{document}
