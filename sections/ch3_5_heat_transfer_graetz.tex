\documentclass[../main.tex]{subfiles}
\begin{document}
\begin{CJK*}{UTF8}{bkai}
\subsection{熱量+動量輸送,黏性摩擦, Graetz Problem, Leveque Method}
\begin{itemize}
  \item 從固體變成流體後,會多了動量輸送的影響\\
  也就是會出現Governing Equation中的,來自流體壓降所反應的不可逆的能量上升(\ref{eq:viscous_dissipation})
  \begin{equation}
    \overline{\overline{\bm \tau}} : \nabla \vec {\bm v} =\frac{2}{3}\mu \sum_{i=1}^{3}\sum_{j=1}^{3}\left[
          \left(
          \frac{\partial u_i}{\partial x_i} + \frac{\partial u_j}{\partial x_j}
        \right) - \frac{2}{3}\delta_{ij} \nabla \cdot \vec{\bm u}
        \right]^2 + \kappa \left(
          \nabla \cdot \vec{\bm u}
        \right)^2 = \mu \bm \Phi_u + \kappa \bm \Phi_b
  \end{equation}
  \begin{figure}[H]
    \centering
    \begin{tikzpicture}[>=Latex, line cap=round, line join=round, thick]
      \draw[->] (0,0) -- (0,1) node[above] {$x$};
      \draw[->] (0,0) -- (1,0) node[right] {$z$};
      \draw (0,-3) -- (10,-3);
      \draw (0,3) -- (10,3);
      \draw [<->] (9.5,-3) -- (9.5,3) node[midway, right] {$2B$};
      \draw[dashed, blue] (2,3) .. controls (4,0) .. (2,-3);
      \draw[dashed, blue] (2,3) -- (2,-3);
      \draw[dashed, red] (5,3) -- (5,-3);
      \draw[dashed, red] (5.5,3) .. controls (7,0) .. (5.5,-3);
      \node[anchor=south, red] at (5.5,3) {$T_0$};
      \node[blue, anchor=south] at (2,3) {$u_z(x,z,t)$};
      \path[name path=vo] (2,3) .. controls (4,0) .. (2,-3);
      \path[name path=ha] (2,2.5) -- (8,2.5);
      \path[name path=hb] (2,2) -- (8,2);
      \path[name path=hc] (2,1.5) -- (8,1.5);
      \path[name path=hd] (2,1) -- (8,1);
      \path[name path=he] (2,0.5) -- (8,0.5);
      \path[name path=hf] (2,0) -- (8,0);
      \path[name path=hg] (2,-0.5) -- (8,-0.5);
      \path[name path=hh] (2,-1) -- (8,-1);
      \path[name path=hi] (2,-1.5) -- (8,-1.5);
      \path[name path=hj] (2,-2) -- (8,-2);
      \path[name path=hk] (2,-2.5) -- (8,-2.5);
      \path[name intersections={of=vo and ha, by=ya}];
      \draw[blue,dashed, ->] (2,2.5) -- (ya);
      \path[name intersections={of=vo and hb, by=yb}];
      \draw[blue,dashed, ->] (2,2) -- (yb);
      \path[name intersections={of=vo and hc, by=yc}];
      \draw[blue,dashed, ->] (2,1.5) -- (yc);
      \path[name intersections={of=vo and hd, by=yd}];
      \draw[blue,dashed, ->] (2,1) -- (yd);
      \path[name intersections={of=vo and he, by=ye}];
      \draw[blue,dashed, ->] (2,0.5) -- (ye);
      \path[name intersections={of=vo and hf, by=yf}];
      \draw[blue,dashed, ->] (2,0) -- (yf);
      \path[name intersections={of=vo and hg, by=yg}];
      \draw[blue,dashed, ->] (2,-0.5) -- (yg);
      \path[name intersections={of=vo and hh, by=yh}];
      \draw[blue,dashed, ->] (2,-1) -- (yh);
      \path[name intersections={of=vo and hi, by=yi}];
      \draw[blue,dashed, ->] (2,-1.5) -- (yi);
      \path[name intersections={of=vo and hj, by=yj}];
      \draw[blue,dashed, ->] (2,-2) -- (yj);
      \path[name intersections={of=vo and hk, by=yk}];
      \draw[blue,dashed, ->] (2,-2.5) -- (yk);
      \path[name path=temp] (5.5,3) .. controls (7,0) .. (5.5,-3);
      \path[name intersections={of=temp and ha, by=ta}];
      \draw[red,dashed, ->] (5,2.5) -- (ta);
      \path[name intersections={of=temp and hb, by=tb}];
      \draw[red,dashed, ->] (5,2) -- (tb);
      \path[name intersections={of=temp and hc, by=tc}];
      \draw[red,dashed, ->] (5,1.5) -- (tc);
      \path[name intersections={of=temp and hd, by=td}];
      \draw[red,dashed, ->] (5,1) -- (td);
      \path[name intersections={of=temp and he, by=te}];
      \draw[red,dashed, ->] (5,0.5) -- (te);
      \path[name intersections={of=temp and hf, by=tf}];
      \draw[red,dashed, ->] (5,0) -- (tf);
      \path[name intersections={of=temp and hg, by=tg}];
      \draw[red,dashed, ->] (5,-0.5) -- (tg);
      \path[name intersections={of=temp and hh, by=th}];
      \draw[red,dashed, ->] (5,-1) -- (th);
      \path[name intersections={of=temp and hi, by=ti}];
      \draw[red,dashed, ->] (5,-1.5) -- (ti);
      \path[name intersections={of=temp and hj, by=tj}];
      \draw[red,dashed, ->] (5,-2) -- (tj);
      \path[name intersections={of=temp and hk, by=tk}];
      \draw[red,dashed, ->] (5,-2.5) -- (tk);
      \node[anchor=west, red] at (7,0) {$T(x,z,t)$};
      \draw[->] (4.8, 0) -- (4.8, 3) node[midway, left] {$q_c$};
    \end{tikzpicture}
    \caption{穩定流體中的熱傳示意圖}
  \end{figure}
  \item 假設最簡單的狀況\\
  流體已經發展完全,為牛頓流體,且溫度分布也發展完全\\
  邊界條件,已知壁面溫度維持在$T_0$,如同浸泡在冷卻液當中的金屬管內流體
  \begin{equation}
    u_z(x,z,t) \to u_z(x) = U_{\text{max}}\left[1-\left(\frac{x}{B}^2\right)\right],\quad T(x,z,t) \to T(x)
  \end{equation}
  \begin{enumerate}
    \item 由已知的流場,\fbox{寫出熱平衡方程式中的Dissipation項}
    \begin{equation}
      -\overline{\overline{\bm \tau}} : \nabla \vec {\bm v} = \mu \left(
        \frac{\partial u_z}{\partial x}
      \right)^2 = \mu \left(-\frac{2U_{\text{max}}}{B^2}x\right)^2 = \frac{4\mu U_{\text{max}}^2}{B^4} x^2
    \end{equation}
    P.S. 因為已經發展完全,$\vec{\bm u}= u_z$,而$u_z(x)$ only\\
    因此整個剪力的9項中,只會有\fbox{z方向流動造成x方向剪力}的$\tau_{xz}$
    \item 寫出熱平衡的方程式:
    \begin{align}
      \rho \hat C_p \left(
        \cancelto{S.S.}{\frac{\partial T}{\partial t}} + \vec{\bm v} \cdot \nabla T
      \right) &= k \nabla^2 T + \cancel{\dot q} - \overline{\overline{\bm \tau}} : \nabla \vec {\bm v} \nonumber\\
      \rho \hat C_p \left(
        u_z \cancelto{\text{只有T(x)}}{\frac{\partial T}{\partial z}}
      \right) &= k \left( \frac{\partial^2 T}{\partial x^2} 
      + \cancelto{\text{只有T(x)}}{\frac{\partial^2 T}{\partial z^2}}\right) + \frac{4\mu U_{\text{max}}^2}{B^4} x^2 \nonumber\\
      0 &= \boxed{k \frac{d^2 T}{d x^2} + \frac{4\mu U_{\text{max}}^2}{B^4} x^2 } \label{eq:simple_graetz_ODE}
    \end{align}
    \item 邊界條件:
    \begin{align}
      T(B) &= T_0 \label{eq:simple_graetz_bc1}\\
      \frac{dT}{dx}\bigg|_{x=0} &= 0,\quad(\text{對稱連續}) \label{eq:simple_graetz_bc2}
    \end{align}
    \item 移項(\ref{eq:simple_graetz_ODE}),並積分:
    \begin{align}
      \frac{d^2 T}{d x^2} &= -\frac{4\mu U_{\text{max}}^2}{k B^4} x^2 \nonumber\\
      \frac{d T}{d x} &= -\frac{4\mu U_{\text{max}}^2}{k B^4} \cdot \frac{x^3}{3} + C_1 
    \end{align}
    \item 代入(\ref{eq:simple_graetz_bc2})求$C_1$:
    \begin{equation}
      0 = -\frac{4\mu U_{\text{max}}^2}{k B^4} \cdot \frac{0^3}{3} + C_1 \Rightarrow C_1 = 0
    \end{equation}
    \item 再次積分:
    \begin{align}
      T(x) &= -\frac{4\mu U_{\text{max}}^2}{k B^4} \cdot \frac{x^4}{12} + C_2 \nonumber\\
      &= -\frac{\mu U_{\text{max}}^2}{3 k B^4} x^4 + C_2
    \end{align}
    \item 代入(\ref{eq:simple_graetz_bc1})求$C_2$:
    \begin{equation}
      T_0 = -\frac{\mu U_{\text{max}}^2}{3 k B^4} B^4 + C_2 \Rightarrow C_2 = T_0 + \frac{\mu U_{\text{max}}^2}{3 k} 
    \end{equation}
    \item 最後解為:
    \begin{equation}
      \boxed{T(x) = T_0 + \frac{\mu U_{\text{max}}^2}{3 k} \left(
        1 - \frac{x^4}{B^4}
      \right)}
    \end{equation}
    \item 可看出溫度最大值發生在$x=0$
    \begin{equation}
      T_{\text{max}} = T(x=0) = T_0 + \frac{\mu U_{\text{max}}^2}{3 k}
    \end{equation}
    \item 而將$T_0$移到等號左邊,變成$\Delta T$,然後再同除$\Delta T$,就會出現無因次群
    \begin{equation}
      \Delta T = \frac{\mu U_{\text{max}}^2}{3 k} \implies \frac{\mu U_{\text{max}}^2}{k \Delta T} = 3
    \end{equation}
    Brinkman number:
    \begin{equation}
      \boxed{\text{Br} = \frac{\mu U_{\text{max}}^2}{k \Delta T} = \frac{\text{黏性耗散產生的熱量}}{\text{導熱能力}}}
    \end{equation}
  \end{enumerate}
  \item Graetz Problem,同樣情境,一個泡在冷卻液中的金屬管\\
  但我想要知道我需要設計多長的金屬管,才會使溫度從$T_0$下降到$T_1$\\
  也就是說,$T$不只是$r$的函數,還是$z$的函數\\
  而這時,我們先假設$Br\ll 1$,因為是個金屬管,導熱能力很好,$\overline{\overline{\bm \tau}}:\nabla \vec{\bm v}$可以忽略\\
  P.S.上一題不能忽略,不然就變成固體散熱了($k\nabla^2T = 0$)\\
  而上一題也能想成是熱交換平衡後,流體內的溫度分布
  \begin{figure}[H]
    \centering
    \begin{tikzpicture}[>=Latex, line cap=round, line join=round, thick]
      \draw (-4,8) -- (-3,8);
      \draw (-3,7.5) rectangle (-2,8.5);
      \draw (-2,8) -- (0,8) -- (0, 7.5);
      \draw [decorate, decoration={coil,aspect=0.3,segment length=1.5mm,amplitude=3mm}] (0,7.5) -- (0,5.25);
      \draw (0,5.25)--(0,5) -- (3,5) -- (3,8) -- (5,8); 
      \draw (-0.5,7.5) -- (-0.5,4.5) -- (3.5,4.5) -- (3.5, 7.5);
      \fill[pattern=crosshatch dots, pattern color=blue] (-0.5,4.5) rectangle (3.5,7);
      \draw [blue, dashed] (-0.5,7) -- (3.5,7);
      \node[anchor=south] at (-0.5,8) {$T_0$};
      \node[anchor=south] at (4,8) {$\approx T_1$};
      \node[blue, anchor=south] at (1.5, 7.5) {$T_1$};
      \node at (-2.5, 8) {裝置};
      \draw[->] (0,1) -- (0,4) node[above] {$r$};
      \draw[dashed] (0,-1.5) -- (0,1);
      \draw[->] (0,3.5) -- (1,3.5) node[right] {$z$};
      \draw (-4,-1) -- (6,-1);
      \draw (-4,3) -- (6,3);
      \draw[dashed] (-4,1) -- (6,1);
      \draw [<->] (-0.3, 1) -- (-0.3,3) node[midway, left] {$R$};
      \fill[pattern=north east lines, draw=black] (-4,3) rectangle (0,3.2);
      \fill[pattern=north east lines, draw=black] (-4,-1) rectangle (0,-1.2);
      \fill[pattern=north west lines, pattern color=blue, draw=blue] (0,3) rectangle (6,3.2);
      \fill[pattern=north west lines, pattern color=blue, draw=blue] (0,-1) rectangle (6,-1.2);
      \node[anchor=south east] at (0,3.2) {$T_0$};
      \node[anchor=south east] at (6,3.2) {$T_1$};
      \draw[dashed, domain=-1:3, smooth, variable=\y] plot ({-0.25*(\y-1)^2 + 1}, \y);
      \foreach \y in {-0.75, -0.25, 0.25,0.75,1.25, 1.75, 2.25, 2.75} {
        \draw[->, dashed] (0, \y) -- ({-0.25*(\y-1)^2 + 1}, \y);
      }
      \node[anchor=south west] at (1,1) {$2\overline{u_z}\left[1-\left(\frac{r}{R}\right)^2\right]$};
    \end{tikzpicture}
    \caption{Graetz 問題示意圖}
  \end{figure}
  \begin{enumerate}
    \item 假設流體進入熱交換前,溫度為均勻的$T_0$,流場已完全發展
    \begin{align}
      u_z(r) = 2\overline{u_z}\left[1-\left(\frac{r}{R}\right)^2\right] 
    \end{align}
    Steady State, $k$, $\rho$, $\hat C_p$皆為常數\\
    $\text{Br}\ll 1$,忽略Dissipation項
    \item 寫出熱平衡的方程式
    \begin{equation}
      \rho \hat C_p \left(
        \vec{\bm v} \cdot \nabla T
      \right) = k \nabla^2 T
    \end{equation}
    展開我們已知的項,$T(r,z)$ only
    \begin{align}
      \rho \hat C_p \left(
        u_z \frac{\partial T}{\partial z}
      \right) &= k \left( \frac{1}{r} \frac{\partial}{\partial r} \left( r \frac{\partial T}{\partial r} \right)
      + \frac{\partial^2 T}{\partial z^2}\right) \nonumber\\
      2 \rho \hat C_p \left(
        \overline{u_z}\left[1-\left(\frac{r}{R}\right)^2\right] \frac{\partial T}{\partial z}
      \right) &= k \left[
        \frac{1}{r} \frac{\partial}{\partial r} \left( r \frac{\partial T}{\partial r} \right)+
        \frac{\partial^2 T}{\partial z^2}
        \right] \label{eq:graetz_heat_equation}
    \end{align}
    \item 邊界條件:
    \begin{align}
      T(R,z<0) &= T_0 \label{eq:graetz_bc1}\\
      T(R,z\ge0) &= T_1 \label{eq:graetz_bc2}\\
      \frac{\partial T}{\partial r}(0,z) &= 0 \quad (\text{對稱連續}) \label{eq:graetz_bc3}\\
      T(r,0+) &= T(r,0-) \quad (\text{入口連續}) \label{eq:graetz_bc4} \\
      \frac{\partial T}{\partial z}(r,0+) &= \frac{\partial T}{\partial z}(r,0-) \quad (\text{入口連續}) \label{eq:graetz_bc5}
    \end{align}
    P.S. 最後一個邊界條件,又稱為Danckwerts Boundary Condition
    \item 無因次化(\ref{eq:graetz_heat_equation}),並定義無因次群\\
    溫度很明顯會用$T_0,T_1$作為特徵溫度:
    \begin{equation}
      \theta = \frac{T - T_1}{T_0 - T_1}
    \end{equation}
    長度方面,$R$是橫向特徵長度
    \begin{equation}
      r^\ast = \frac{r}{R}
    \end{equation}
    至於$z$的方向,則觀察(\ref{eq:graetz_heat_equation})的常數項係數\\
    左邊忽視$\left(1-\frac{r^2}{R^2}\right)$,丟到右邊後會有
    \begin{equation}
      \frac{k}{2 \rho \hat C_p \overline{u_z}}
    \end{equation}
    再分母再補上一個$R$,就會變成無因次的\fbox{Peclet Number}的倒數
    \begin{equation}
       \boxed{\text{Pe} = \frac{2\rho \hat C_P\left<u_z\right>R}{k} = \frac{\text{軸向熱對流}}{\text{熱傳導}}}
    \end{equation}
    而我們再補上一個$R$,就能讓單位變成長度,作為$z$的特徵長度
    \begin{equation}
      \boxed{z^\ast = \frac{z}{R} \cdot\frac{1}{\text{Pe}} = \frac{z k}{2 \rho \hat C_P \left<u_z\right> R^2}}
    \end{equation}
    P.S. Peclet Number, Reynolds Number, Prandtl Number的關係為
    \begin{equation}
      \text{Pe} = \text{Re} \cdot \text{Pr}
    \end{equation}
    \item 最後將(\ref{eq:graetz_heat_equation})無因次化:
    \begin{align}
      2\rho \hat C_P \overline{u_z} \left[
        1 - (r^\ast)^2
      \right] \frac{T_0 - T_1}{R\text{Pe}} \frac{\partial \theta}{\partial z^\ast} &= k \left[
        \frac{1}{r^\ast R} \frac{\partial}{\partial r^\ast} \left(
          r^\ast \frac{T_0 - T_1}{R} \frac{\partial \theta}{\partial r^\ast}
        \right) + \frac{T_0 - T_1}{R^2\text{Pe}^2} \frac{\partial^2 \theta}{\partial z^{\ast 2}}
      \right] \nonumber\\
      \div (T_0 - T_1) \Rightarrow 2\rho \hat C_P \overline{u_z} \left[
        1 - (r^\ast)^2
      \right] \frac{1}{R\text{Pe}}\frac{\partial \theta}{\partial z^\ast} &= \frac{k}{R^2} \left[
        \frac{1}{r^\ast} \frac{\partial}{\partial r^\ast} \left(
          r^\ast \frac{\partial \theta}{\partial r^\ast}
        \right) + \frac{1}{\text{Pe}^2} \frac{\partial^2 \theta}{\partial z^{\ast 2}}
      \right] \nonumber\\
      \Rightarrow  \left[
        1 - (r^\ast)^2
      \right] \frac{1}{\cancel{R}\text{Pe}}\frac{\partial \theta}{\partial z^\ast} &= \frac{k}{2\rho \hat C_P \overline{u_z} R}\frac{1}{\cancel{R}} \left[
        \frac{1}{r^\ast} \frac{\partial}{\partial r^\ast} \left(
          r^\ast \frac{\partial \theta}{\partial r^\ast}
        \right) + \frac{1}{\text{Pe}^2} \frac{\partial^2 \theta}{\partial z^{\ast 2}}
      \right] \nonumber\\
      \Rightarrow  \left[
        1 - (r^\ast)^2
      \right]\frac{1}{\text{Pe}} \frac{\partial \theta}{\partial z^\ast} &= \frac{1}{\text{Pe}} \left[
        \frac{1}{r^\ast} \frac{\partial}{\partial r^\ast} \left(
          r^\ast \frac{\partial \theta}{\partial r^\ast}
        \right) + \frac{1}{\text{Pe}^2} \frac{\partial^2 \theta}{\partial z^{\ast 2}}
      \right]  \nonumber\\
      \div \frac{1}{\text{Pe}}\Rightarrow  \left[
        1 - (r^\ast)^2
      \right] \frac{\partial \theta}{\partial z^\ast} &= \left[
        \frac{1}{r^\ast} \frac{\partial}{\partial r^\ast} \left(
          r^\ast \frac{\partial \theta}{\partial r^\ast}
        \right) + \frac{1}{\text{Pe}^2} \frac{\partial^2 \theta}{\partial z^{\ast 2}}
      \right]   \label{eq:graetz_nondim_heat_equation}
    \end{align}
    由於$\text{Pe}=\text{Re}\cdot \text{Pr}$,所以假設流速很快,所以$\text{Re}\gg 1$\\
    而且液體的$\text{Pr}$通常在10左右,因此$\text{Pe}\gg 1$,可以忽略最後一項
    \begin{equation}
      \boxed{\left[
        1 - (r^\ast)^2
      \right] \frac{\partial \theta}{\partial z^\ast} = \left[
        \frac{1}{r^\ast} \frac{\partial}{\partial r^\ast} \left(
          r^\ast \frac{\partial \theta}{\partial r^\ast}
        \right)
      \right]}   \label{eq:graetz_simplified_nondim_heat_equation}
    \end{equation}
    \item 邊界條件無因次化(省略那些$z<0$的部分):
    \begin{align}
      \theta(r^\ast,0) &= 1 \label{eq:graetz_nondim_bc1}\\
      \theta(1,z^\ast) &= 0 \label{eq:graetz_nondim_bc2}\\
      \frac{\partial \theta}{\partial r^\ast}(0,z^\ast) &= 0 \quad (\text{對稱連續}) \label{eq:graetz_nondim_bc3}
    \end{align}
    可以觀察到(\ref{eq:graetz_nondim_bc2})和(\ref{eq:graetz_nondim_bc3})都是齊次邊界條件而且與$z^\ast$無關\\
    而(\ref{eq:graetz_nondim_bc1})則是非齊次邊界條件且與$z^\ast$有關\\
    所以這又是一個標準的S-L問題
    \item 分離變數,假設解為:
    \begin{equation}
      \theta(r^\ast, z^\ast) = R(r^\ast) Z(z^\ast)
    \end{equation}
    代入(\ref{eq:graetz_simplified_nondim_heat_equation}):
    \begin{align}
      \left[
        1 - (r^\ast)^2
      \right] R \frac{dZ}{d z^\ast} &= Z \left[
        \frac{1}{r^\ast} \frac{d}{d r^\ast} \left(
          r^\ast \frac{d R}{d r^\ast}
        \right)
      \right] \nonumber\\
      \Rightarrow \frac{1}{Z} \frac{dZ}{d z^\ast} &= \frac{1}{R} \cdot \frac{1}{\left[
        1 - (r^\ast)^2
      \right]} \left[
        \frac{1}{r^\ast} \frac{d}{d r^\ast} \left(
          r^\ast \frac{d R}{d r^\ast}
        \right)
      \right] = -\lambda^2
    \end{align}
    \item 兩邊分別得到兩個ODE:
    \begin{align}
      \frac{dZ}{d z^\ast} + \lambda^2 Z &= 0 \label{eq:graetz_z_ODE}\\
      \frac{d}{d r^\ast} \left(
          r^\ast \frac{d R}{d r^\ast}
        \right) + \left[
        1 - (r^\ast)^2
      \right] r^\ast \lambda^2 R &= 0 \label{eq:graetz_r_ODE}
    \end{align}
    \item 解(\ref{eq:graetz_z_ODE}):
    \begin{equation}
      Z(z^\ast) = C e^{-\lambda^2 z^\ast}
    \end{equation}
    \item 解(\ref{eq:graetz_r_ODE})
    \begin{equation}
      \frac{d}{d r^\ast} \left(
          r^\ast \frac{d R}{d r^\ast}
        \right) + \left[
        1 - (r^\ast)^2
      \right] r^\ast \lambda^2 R = 0
    \end{equation}
    寫成S-L形式:
    \begin{equation}
      \frac{d}{dx}\left[
        p(x)\frac{dy}{dx}
      \right] + q(x)y + \lambda^\ast r(x) y = 0
    \end{equation}
    比較係數可得到
    \begin{equation}
      p(r^\ast) = r^\ast,\quad q(r^\ast) = 0, r(r^\ast) = \left[
        1 - (r^\ast)^2
      \right] r^\ast,\quad \lambda^\ast = \lambda^2
    \end{equation}
    因此通解可以寫成:
    \begin{align}
      \theta(r^\ast, z^\ast) = Ce^{-\lambda^2 z^\ast}\left[
        AR(r^\ast; \lambda) + BS(r^\ast; \lambda)
      \right]
    \end{align}
    其中這些解我們稱為Graetz EigenFunctions,若對其微分\\
    S會發散,R會收斂
    \item 代入第三個邊界條件(\ref{eq:graetz_nondim_bc3}),可得$B=0$\\
    合併$A,C$為一常數$D$:
    \begin{equation}
      \theta(r^\ast, z^\ast) = D e^{-\lambda^2 z^\ast} R(r^\ast; \lambda)
    \end{equation}
    \item 代入第二個邊界條件(\ref{eq:graetz_nondim_bc2}),可得特徵基底:\\
    因為$z^\ast$不影響邊界條件,因此要令$R(1;\lambda) = 0$
    \begin{equation}
      R(1; \lambda) = 0 \Rightarrow \lambda_n\implies \Phi_n = R(r^\ast; \lambda_n)
    \end{equation}
    這一步驟的解,就如之前畫$\cot$和$ax$的交點一樣,是無法求出數值表示的\\
    而如果可以表示出數值,就跟Fourier Series一樣\\
    $R(1;\lambda n)=0=cos\theta$的解為$\theta_n = (2n-1)\frac{\pi}{2}$\\
    這一步驟就是在做這件事情而已\\
    這時解為:
    \begin{equation}
      \theta_n(r^\ast, z^\ast) = \sum_{n=1}^{\infty} D_n e^{-\lambda_n^2 z^\ast} R(r^\ast; \lambda_n)
    \end{equation}
    \item 最後代入第一個邊界條件(\ref{eq:graetz_nondim_bc1}),利用正交性求出$D_n$:
    \begin{align}
      1 = \theta(r^\ast,0) &= \sum_{n=1}^{\infty} D_n R(r^\ast; \lambda_n) \nonumber\\
      D_n&=\frac{
        \int f(x)r(x)\phi_n(x)dx
      }{\int r(x)\phi_n^2(x) dx} \nonumber\\
      &= \frac{\int_0^1 1 \cdot \left[
        1 - (r^\ast)^2
      \right] r^\ast R(r^\ast; \lambda_n) dr^\ast}{\int_0^1 \left[
        1 - (r^\ast)^2
      \right] r^\ast R^2(r^\ast; \lambda_n) dr^\ast}
    \end{align}
    而由於$R(r^\ast; \lambda_n)$是被Normalized過的,因此分母為1\\
    最後解為:
    \begin{equation}
      \boxed{\theta(r^\ast, z^\ast) = \sum_{n=1}^{\infty} \left[
        \int_0^1 \left(
          1 - (r^\ast)^2
        \right) r^\ast R(r^\ast; \lambda_n) dr^\ast
      \right] e^{-\lambda_n^2 z^\ast} R(r^\ast; \lambda_n)}
    \end{equation}
    \item 用Adiabatic Mixing Cup的溫度($T_{\text{AMC}}$)來找到$\lambda_n$\\
    實驗上,其實就是找到任一個截面的平均溫度,利用讓流體流進一個絕熱的杯子\\
    等流體在裡面熱平衡後,測量溫度,就是平均溫度\\
    不過這裡的\fbox{截面平均溫度是加權除過流速}的,因為流速快的部分,進去杯子的量也多\\
    而以數學上來寫,則為
    \begin{align}
      T_{\text{AMC}}(z) &= \frac{4\pi \overline u_z \int_0^R T(r,z) \left[
        1 - \left(\frac{r}{R}\right)^2
      \right] r dr}{
        \overline u_z \cdot \pi R^2
      } \nonumber\\
      &= \frac{4 \int_0^R T(r,z) \left[
        1 - \left(\frac{r}{R}\right)^2
      \right] r dr}{R^2} \nonumber\\
      \theta_{\text{AMC}} &= \frac{T_{\text{AMC}}-T_1}{T_0-T_1}\nonumber\\
       &= 4\int_0^1 \theta(r^\ast, z^\ast) \left[
        1 - (r^\ast)^2
      \right] r^\ast dr^\ast \nonumber\\
      &= 4 \sum_{n=1}^{\infty} D_n e^{-\lambda_n^2 z^\ast} \underbrace{\left[
        \int_0^1 \left(
          1 - (r^\ast)^2
        \right) r^\ast R(r^\ast; \lambda_n) dr^\ast
      \right]}_{=D_n} \nonumber\\
      &= 4 \sum_{n=1}^{\infty} D_n^2 e^{-\lambda_n^2 z^\ast} \label{eq:graetz_amc_temperature}
    \end{align}
    進一步令$4D_n^2 = M_n$,則(\ref{eq:graetz_amc_temperature})變成
    \begin{equation}
      \boxed{\theta_{\text{AMC}} = \sum_{n=1}^{\infty} M_n e^{-\lambda_n^2 z^\ast}}
    \end{equation}
    也就是說,我們跟換不同的$\lambda_n$,就能測量出不同的$\theta_{\text{AMC}}$\\
    進而反推出$M_n$,或者說$D_n$,最後就能得到完整的溫度分布\\
    而數值解提供了我們一系列的$\lambda_n,M_n$
    \begin{table}[H]
      \centering
      \begin{tabular}{c|c|c}
        $n$ & $\lambda_n$ & $M_n$ \\
        \hline
        1 & 2.704 & 0.819\\
        2 & 6.679 & 0.098\\
        3 & 10.673 & 0.033\\
        4 & 14.671 & 0.015\\
        5 & 18.669 & 0.009
      \end{tabular}
      \caption{Graetz Problem 特徵值與係數}
    \end{table}
    也就是說,假設我們希望$\theta_{\text{AMC}}=0.1$,也就是100度被0度淨泡冷卻到10度\\
    那我們可以算出:
    \begin{equation}
      \theta_{\text{AMC}} = 0.1 = 0.819 e^{-2.704^2 z^\ast} + 0.098 e^{-6.679^2 z^\ast}\cdots
    \end{equation}
    假設只取第一項就好,那就能算出$z^\ast\approx 0.2876$\\
    代回去$z^\ast$的定義,就能算出需要多長的管子
    \begin{equation}
      L = z^\ast \cdot \frac{2 \rho \hat C_P \overline{u_z} R^2}{k} = 
      0.2876 \cdot \frac{2 \rho \hat C_P \overline{u_z} R^2}{k} = 0.2876 R \text{Pe}
    \end{equation}
    如果以管子直徑($D=2R$)來當作特徵長度,則
    \begin{align}
      \boxed{\frac{L}{D} = 0.144 \text{Pe}}
    \end{align}
    也就是說,假設今天Pe算出來是1000,那我們需要的管長就是$L/D = 144$倍管直徑\\
    但這個解析解只能用在想要知道到達某個溫度所需長度的情況\\
    如果想要知道整個的溫度分布$\theta(z^\ast)$,換句話說,可能在靠近入口處\\
    $\theta_{\text{AMC}}=0.99$,之類的,\\
    就會需要接近無窮多項的數值解才能逼近的算出該位置($z^\ast$)\\
    因此這樣的近似是不夠好的\\
    這原因也能從我們的邊界條件看出\\
    因為我們既要求$\theta(1,z^\ast)=0$,又要求$\theta(r^\ast,0)=1$\\
    這\fbox{兩個條件在交界處($r^\ast=1,z^\ast=0$)會產生不連續點}\\
    因此需要無窮多項的解才能逼近這個不連續點\\
    這也是為什麼要介紹下面章節的,Leveque的方法\\
    但在這之前,還有一點要介紹
    \item 以巨觀下的熱傳係數$h$、Nusselt Number來描述Graetz Problem\\
    可以將整根管子當成一個系統\\
    則能寫出巨觀的能量平衡式:
    \begin{align}
      \left<h\right> 2\pi RL (T_i - T_0) =&\rho \hat C_P 2\left<u_z\right> \int_0^R\left[
        1 - \left(\frac{r}{R}\right)^2
      \right]\left(T(r,L)-T^\dagger \right)2\pi r dr \nonumber\\
      &-\rho \hat C_P 2\left<u_z\right> \int_0^R\left[
        1 - \left(\frac{r}{R}\right)^2
      \right](T_0 - T^\dagger)2\pi r dr
    \end{align}
    其中$T_i$為流體進入管子的溫度,$T^\dagger$為假想的參考溫度,積分後會消掉\\
    第一項是在右端口計算流體繼續傳下去的熱量($T(r,L)$)\\
    因為只是管子離開交換區間,並不是流體離開管子\\
    第二項是流體進入管子時帶入的熱量($T(r,0)=T_0$)\\
    將上式同除$\pi$,並同除$k$,湊出Nusselt Number:
    \begin{equation}
      \boxed{\text{Nu} = \frac{\left<h\right>L}{k}} \label{eq:graetz_nusselt_number}
    \end{equation}
    因此:
    \begin{align}
      \left<h\right> 2\pi RL (T_i - T_0) &=\rho \hat C_P 2\left<u_z\right> \int_0^R\left[
        1 - \left(\frac{r}{R}\right)^2
      \right]\left(T(r,L)-T^\dagger \right)2\pi r dr \nonumber\\
      &\phantom{=} -\rho \hat C_P 2\left<u_z\right> \int_0^R\left[
        1 - \left(\frac{r}{R}\right)^2
      \right](T_0 - T^\dagger)2\pi r dr \nonumber\\
      \left<h\right> 2\pi RL (T_i - T_0) &=4\pi\rho \hat C_P \left<u_z\right> \int_0^R\left[
        1 - \left(\frac{r}{R}\right)^2
      \right]\left(T(r,L)-T_0\right) r dr \nonumber\\
      \frac{2R\left<h\right>}{k}L (T_i - T_0) &=\frac{4\rho \hat C_P\left<u_z\right>}{k} \int_0^R\left[
        1 - \left(\frac{r}{R}\right)^2
      \right]\left(T(r,L)-T_0\right) r dr \nonumber\\
      \text{Nu} (T_i - T_0) &= \frac{4\rho \hat C_P\left<u_z\right>}{k}\frac{R^2}{L}\int_0^1\left[
        1 - (r^\ast)^2
      \right]\left(T(r^\ast,L)-T_0\right) r^\ast dr^\ast \nonumber\\
      &= \frac{4\rho \hat C_P\left<u_z\right> R^2}{k L} (T_i - T_0) \int_0^1\left[
        1 - (r^\ast)^2
      \right]\left(1-\theta(r^\ast, z^\ast) \right) r^\ast dr^\ast \nonumber\\
      \text{Nu} &= \frac{2 R\rho \hat C_P\left<u_z\right>}{k} \cdot \frac{2R}{L}\cdot
      \int_0^1\left(
        1 - (r^\ast)^2
      \right)\left(1-\theta(r^\ast, z^\ast) \right) r^\ast dr^\ast
    \end{align}
    這裡又出現了一個合併的無因次群\\
    就是Graetz Number:
    \begin{equation}
      \boxed{\text{Gz} = \text{Pe} \cdot \frac{2R}{L} = \frac{4 \rho \hat C_P \left<u_z\right> R^2}{k L}}
    \end{equation}
    而將$\theta(r^\ast, z^\ast)$,以平均的$\theta_{\text{AMC}}$來近似\\
    將積分積出可得:
    \begin{equation}
      \text{Nu} = \text{Gz} \left[
        \frac{1}{4} - \frac{1}{4} \sum_{n=1}^{\infty} M_n e^{-\lambda_n^2 z^\ast_L}
      \right]
    \end{equation}
    而若將$z^\ast_L$的定義代入:
    \begin{equation}
      z^\ast_L = \frac{L k}{2 \rho \hat C_P \left<u_z\right> R^2} = \frac{2}{\text{Gz}}
    \end{equation}
    最後可得:
    \begin{equation}
      \boxed{\text{Nu} = \text{Gz} \left[
        \frac{1}{4} - \frac{1}{4} \sum_{n=1}^{\infty} M_n e^{-\frac{2\lambda_n^2}{\text{Gz}}}
      \right]}
    \end{equation}
    從這裡更明顯能看出,隨著L越來越小\\
    Nusselt Number與$L$無關,但Graetz Number反比於$L$\\
    故需要更多項的級數相加才能讓等式成立
  \end{enumerate}
  \item Leveque 方法,用在Graetz Problem的入口區域
  \begin{figure}[H]
    \centering
    \begin{tikzpicture}[>=Latex, line cap=round, line join=round, thick]
      \fill[pattern=north east lines, draw=black] (-4,2) rectangle (0,2.2);
      \fill[pattern=north east lines, draw=black] (-4,-2) rectangle (0,-2.2);
      \fill[pattern=north west lines, pattern color=blue, draw=blue] (0,2) rectangle (4,2.2);
      \fill[pattern=north west lines, pattern color=blue, draw=blue] (0,-2) rectangle (4,-2.2);
      \draw (-4,2) -- (4,2);
      \draw (-4,-2) -- (4,-2);
      \draw [->] (0,-2) -- (0,-1) node[above] {$y$};
      \draw [dashed] (0, -3) -- (0,-2);
      \def\yShift{2}
      \draw [dashed] (0,0-\yShift) .. controls (1.5,0.5-\yShift) and (2.6, 0.6-\yShift) .. (4,0.7-\yShift);
      \fill [pattern=crosshatch dots] (0,0-\yShift) .. controls (1.5,0.5-\yShift) and (2.6, 0.6-\yShift) .. (4,0.7-\yShift) -- (4,0-\yShift) -- cycle;
      \draw [dashed] (0,0+\yShift) .. controls (1.5,-0.5+\yShift) and (2.6, -0.6+\yShift) .. (4,-0.7+\yShift);
      \fill [pattern=crosshatch dots] (0,0+\yShift) .. controls (1.5,-0.5+\yShift) and (2.6, -0.6+\yShift) .. (4,-0.7+\yShift) -- (4,0+\yShift) -- cycle;
      \node [anchor=south west] at (4,-2) {$\delta_T(z)$};
      \node at (2,0) {Thermal Boundary Layer};
      \node[anchor=north] at (4, 1.3) {$T_0$};
      \node[anchor=south] at (4, -1.3) {$T_0$};
      \node[anchor=north west] at (0, -2.2) {$T_1$};
      \node[anchor=north east] at (0, -2.2) {$T_0$};
      \draw[<->] (-1,-2) -- (-1,2) node[midway, left] {$2R$};
      \draw[->] (-4,2.7) -- (-3,2.7) node [right] {$z$};
    \end{tikzpicture}
    \caption{Leveque 方法示意圖}
  \end{figure}
  坐標軸改以$y=R-r$,而$y/R\ll 1$
  \begin{enumerate}
    \item 能量平衡式:
    \begin{align}
      &2\rho \hat C_P \overline u_z(y)\left[
        1 -\left(\frac{r}{R}\right)^2
      \right]\frac{\partial T}{\partial z} = k\frac{1}{r}\frac{\partial}{\partial r}\left(r\frac{\partial T}{\partial r}\right) \nonumber\\
      \Rightarrow&2\rho \hat C_P \overline u_z(y)\left[
        1 -\left(\frac{R-y}{R}\right)^2
      \right]\frac{\partial T}{\partial z} =
       k\frac{1}{R-y}\frac{\partial}{\partial y}\left((R-y)\frac{\partial T}{\partial y}\right)\nonumber\\
      (R-y\to R)\Rightarrow&2\rho \hat C_P \overline u_z(y)\left[
        1-\frac{R^2}{R^2} + \frac{2Ry}{R^2} - \cancel{\frac{y^2}{R^2}}
      \right]\frac{\partial T}{\partial z} = k \frac{1}{R} \frac{\partial }{\partial y}\left(
        R\frac{\partial T}{\partial y}
      \right) \nonumber\\
      \Rightarrow & 4\rho \hat  C_P \overline u_z(y)\frac{y}{R}\frac{\partial T}{\partial z}
      = k\frac{\partial^2 T}{\partial y^2} \label{eq:leveque_energy_equation}
    \end{align}
    \item 邊界條件:
    \begin{align}
      T(y,0) &= T_0 \\
      T(0,z) &= T_1 \\
      T(y>\delta_T,z) &= T_0,\quad T(\infty, z)=T_0
    \end{align}
    由於沒有任兩個邊界條件是獨立且齊次的,所以不能用S-L方法\\
    改採用Similarity Solution
    \item 令$\eta = \frac{y}{\delta_T(z)}$\\
    至於$\delta_T(z)$是什麼,則要做因次分析
    也就是說能量平衡式(\ref{eq:leveque_energy_equation}):
    \begin{align}
      4\rho \hat  C_P \overline u_z\frac{\delta_T}{R}\frac{T}{z} = k\frac{T}{\delta_T^2}
    \end{align}
    而兩邊因次要相等,因此:
    \begin{align}
      &\frac{4\rho \hat C_P \overline u_z T}{R z k} = \frac{1}{\delta_T^3} \nonumber\\
      \Rightarrow &\delta_T = \left(
        \frac{k R z}{4\rho \hat C_P \overline u_z}
      \right)^{1/3}
    \end{align} 
    而為了方便計算,定義特徵長度$\delta_T(z)$為:
    \begin{equation}
      \boxed{\delta_T(z) = \left(\frac{9 kRz}{4 \rho \hat C_P \overline u_z}\right)^{1/3}}
    \end{equation}
    \item 將$\eta=\frac{y}{\delta_T(z)}$代入(\ref{eq:leveque_energy_equation})\\
    並順便也將溫度無因次化:
    \begin{equation}
      \theta = \frac{T - T_1}{T_0 - T_1},\quad \frac{\partial T}{\partial z} = (T_0 - T_1) \frac{\partial\theta}{\partial z}
      ,\quad \frac{\partial^2 T}{\partial y^2} = (T_0 - T_1) \frac{\partial^2 \theta}{\partial y^2}
    \end{equation}
    故
    \begin{align}
     & 4\rho \hat  C_P \overline u_z \frac{y}{R}\frac{\partial T}{\partial z} = k \frac{\partial^2 T}{\partial y^2} \nonumber\\
    \Rightarrow& 4\rho \hat  C_P \overline u_z \left(\frac{y}{R}\right)(T_0 - T_1) \frac{\partial \theta}{\partial z} 
      = k (T_0 - T_1) \frac{\partial^2 \theta}{\partial y^2} \nonumber\\
      \Rightarrow& 4\rho \hat  C_P \overline u_z \left(\frac{y}{R}\right) \frac{\partial \theta}{\partial z} 
      = k \frac{\partial^2 \theta}{\partial y^2} \nonumber\\
      \Rightarrow& \frac{\partial \theta}{\partial z} = \frac{k R}{4\rho \hat C_P \overline u_z y} \frac{\partial^2 \theta}{\partial y^2} \nonumber\\
      \Rightarrow& \frac{\partial \theta}{\partial z} = \frac{k R}{4\rho \hat C_P \overline u_z \delta_T \eta} \frac{\partial^2 \theta}{\partial y^2} 
    \end{align}
    而上式的每個偏微分分別是
    \begin{align}
      \frac{\partial \theta}{\partial z} &= \frac{d\theta}{d\eta}{d\eta}{dz} \nonumber\\
      &= \frac{d\theta}{d\eta} \cdot \left[
        -\frac{1}{3}y\left(
          \frac{4\rho \hat C_P \overline u_z}{9 k R}
        \right)^{1/3} z^{-4/3}
      \right] \nonumber\\
      &= \frac{d\theta}{d\eta} \cdot \left[
        -\frac{1}{3}\underbrace{y\left(
          \frac{4\rho \hat C_P \overline u_z}{9 k R z}
        \right)^{1/3}}_{\eta} z^{-1}
      \right]\nonumber\\
      &= \boxed{-\frac{1}{3} \eta z^{-1} \frac{d\theta}{d\eta}} \\
      \frac{\partial \theta}{\partial y} &= 
      \frac{d\theta}{d\eta} \frac{d\eta}{dy} \nonumber\\
      &= \frac{d\theta}{d\eta} \cdot \left[
        \frac{4\rho \hat C_P \overline u_z}{9 k R z}
      \right]^{1/3} \nonumber\\
      \frac{\partial^2 \theta}{\partial y^2} &=
      \frac{\partial}{\partial y}\left(
        \frac{\partial \theta}{\partial y}
      \right) \nonumber\\
      &= \frac{\partial}{\partial y} \left[
        \frac{d\theta}{d\eta} \cdot \left(
          \frac{4\rho \hat C_P \overline u_z}{9 k R z}
        \right)^{1/3}
      \right] \nonumber\\
      &= \boxed{\left(
          \frac{4\rho \hat C_P \overline u_z}{9 k R z}
        \right)^{2/3} \frac{d^2 \theta}{d\eta^2}}
    \end{align}
    代入能量平衡式:
    \begin{align}
      &4\rho \hat  C_P \overline u_z \left(\frac{y}{R}\right) \frac{\partial \theta}{\partial z} 
      = k \frac{\partial^2 \theta}{\partial y^2} \nonumber\\
      \Rightarrow& 4\rho \hat  C_P \overline u_z \left(\frac{\delta_T \eta}{R}\right) \left(
        -\frac{1}{3} \frac{\eta}{z}\right) \frac{d\theta}{d\eta} = k \left(
          \frac{4\rho \hat C_P \overline u_z}{9 k R z}
        \right)^{2/3} \frac{d^2 \theta}{d\eta^2} \nonumber\\
      \Rightarrow& -\frac{4}{3}\frac{\rho \hat  C_P \overline u_z}{R}\eta \frac{d\theta}{d\eta}
      = k\left(
        \frac{4\rho \hat C_P \overline u_z}{9 k R}
      \right)^{2/3} \frac{z^{\frac{1}{3}}}{y} \frac{d^2 \theta}{d\eta^2} \nonumber\\
      \Rightarrow& -\frac{4}{3}\frac{\rho \hat  C_P \overline u_z}{R}\eta \frac{d\theta}{d\eta}
      = \cancel{k}\left(
        \frac{4\rho \hat C_P \overline u_z}{9 \cancel{k} R}
      \right)\left(
        \frac{4\rho \hat C_P \overline u_z}{9 k R}
      \right)^{-\frac{1}{3}} \frac{z^{\frac{1}{3}}}{y} \frac{d^2 \theta}{d\eta^2} \nonumber\\
      \Rightarrow& -\frac{1}{3}\eta \frac{d\theta}{d\eta} = \frac{1}{9}
      \underbrace{\left(
        \frac{4\rho \hat C_P \overline u_z}{9 k R}
      \right)^{-\frac{1}{3}} \frac{z^{\frac{1}{3}}}{y}}_{\frac{1}{\eta}} \frac{d^2 \theta}{d\eta^2} \nonumber\\
      \Rightarrow& -3\eta \frac{d\theta}{d\eta} = 
      \frac{1}{\eta} \frac{d^2 \theta}{d\eta^2} \nonumber\\
      \Rightarrow& -3\eta^2 \frac{d\theta}{d\eta} = 
       \frac{d^2 \theta}{d\eta^2} \label{eq:leveque_ODE}
    \end{align}
    所以最後的ODE變成:
    \begin{equation}
      \frac{d^2 \theta}{d\eta^2} + 3\eta^2 \frac{d\theta}{d\eta} = 0
    \end{equation}
    邊界條件變成:
    \begin{align}
      \theta(\infty) &= 1\\
      \theta(0) &= 0
    \end{align}
    \item 解ODE(\ref{eq:leveque_ODE}):\\
    方法同之前解流體的邊界層類似(\ref{sec:ch2_5_boundary_layer_dev_definition_F})\\
    先令$F'=\frac{d\theta}{d\eta}$,則
    \begin{align}
      \frac{dF'}{d\eta} + 3\eta^2 F' &= 0 \nonumber\\
      \Rightarrow \frac{dF'}{F'} &= -3\eta^2 d\eta \nonumber\\
      \Rightarrow \ln F' &= -\eta^3 + C_1 \nonumber\\
      \Rightarrow F' &= C_2 e^{-\eta^3} \nonumber\\
      \Rightarrow \frac{d\theta}{d\eta} &= C_2 e^{-\eta^3} \nonumber\\
      \Rightarrow \theta &= C_2 \int_0^\eta e^{-x^3} dx + C_3
    \end{align}
    代入邊界條件:
    \begin{align}
      \theta(0) &= 0 \Rightarrow C_3 = 0 \\
      \theta(\infty) &= 1 \Rightarrow C_2 \int_0^\infty e^{-x^3} dx = 1 
    \end{align}
    令$x=\eta^3$,則$dx = 3\eta^2 d\eta$,$\eta = x^{1/3}$\\
    因此:
    \begin{equation}
      1 = C_2 \int_0^\infty \frac{1}{3}e^{-x} x^{-\frac{2}{3}} dx
    \end{equation}
    而$\int_0^\infty e^{-x} x^{n-1} dx = \Gamma(n)$\\
    因此:
    \begin{equation}
      1 = C_2 \cdot \frac{1}{3} \Gamma\left(
        \frac{1}{3}
      \right)
    \end{equation}
    又因為$\Gamma$函數具有下列性質:
    \begin{align}
      \Gamma(n+1) &= n\Gamma(n) \\
      \Gamma(n+1) &= n! \quad (n\in \mathbb{N})
    \end{align}
    因此:
    \begin{equation}
       \frac{1}{3} \Gamma\left(
        \frac{1}{3}\right) = \Gamma\left(
          \frac{4}{3}
        \right)
    \end{equation}
    故:
    \begin{equation}
      C_2 = \frac{1}{\Gamma\left(
        \frac{4}{3}
      \right)}
    \end{equation}
    最後解為:
    \begin{equation}
      \boxed{\theta(\eta) = \frac{1}{\Gamma\left(
        \frac{4}{3}
      \right)} \int_0^\eta e^{-x^3} dx} \label{eq:leveque_solution}
    \end{equation}
    \item 令$\theta=0.99$,也就是溫度差下降到1\%時,為Boundary Layer的厚度\\
    代入(\ref{eq:leveque_solution}),可得$\theta(1.4) \approx 0.99$\\
    也就是當$\eta = 1.4$時,溫度差下降到1\%\\
    而此時的$y$,為$\delta_T$,因此:
    \begin{align}
      &\eta = y\left(
        \frac{4\rho \hat C_P \overline u_z}{9 k R z}
      \right)^{\frac{1}{3}} \nonumber\\
      \Rightarrow & 1.4 = \delta_T \left(
        \frac{4\rho \hat C_P \overline u_z}{9 k R z}
      \right)^{\frac{1}{3}} \nonumber\\
      \Rightarrow & \delta_T = 1.4 \left(
        \frac{9 k R z}{4\rho \hat C_P \overline u_z}
      \right)^{\frac{1}{3}} \nonumber\\
       & \phantom{\delta_T} = 2.912 \text{Pe}^{-\frac{1}{3}}\left(
        \frac{z}{2R}
      \right)^{\frac{1}{3}} \quad \boxed{\propto z^{\frac{1}{3}}}
    \end{align}
    也就是說那條曲線大概是$z^{1/3}$的函數\\
    P.S. 回顧流體的邊界層厚度是$z^{1/2}$\\
    可見熱傳邊界層發展得比流體邊界層還要慢
    \item 同樣的,可以用Leveque的方法來求出Nusselt Number\\
    計算管面上的徑向熱通量守恆\\
    因為Leveque方法假設熱傳邊界層很薄\\
    因此可以近似認為整個管面都是在熱傳邊界層內\\
    而溫差都為$T_0 - T_1$
    \begin{equation}
      h(T_1 - T_0) 2\pi R dz = -k \left.
        2\pi R \frac{\partial T}{\partial r}
      \right|_{r=R} dz = k \left.
        2\pi R \frac{\partial T}{\partial y}
      \right|_{y=0} dz
    \end{equation}
    無因次化,並湊出Nusselt Number:
    \begin{align}
      &h(T_1 - T_0) 2\pi R dz = k \left.
        2\pi R \frac{\partial T}{\partial y}
      \right|_{y=0} dz \nonumber\\
      \Rightarrow & h(T_1 - T_0)  dz = k \left.
         \frac{\partial T}{\partial y}
      \right|_{y=0} dz \nonumber\\
      \Rightarrow & h (T_0 - T_1) = -k (T_0 - T_1) \left.
         \frac{\partial \theta}{\partial y}
      \right|_{y=0} \nonumber\\
      \Rightarrow & \frac{h}{k} = - \left.
         \frac{\partial \theta}{\partial y}
      \right|_{y=0} \nonumber\\
      \times (2R) \Rightarrow & \text{Nu} = -2R \left.
         \frac{\partial \theta}{\partial y}
      \right|_{y=0}
    \end{align}
    代入 $\theta(\eta)$:
    \begin{align}
      \text{Nu} &= -2R \left.
         \frac{\partial \theta}{\partial y}
      \right|_{y=0} \nonumber\\
      &= -2R \left.
        \frac{d\theta}{d\eta} \frac{d\eta}{dy}
      \right|_{y=0} \nonumber\\
      &= -2R \left.
        \frac{1}{\Gamma\left(
          \frac{4}{3}
        \right)} e^{-\eta^3} \cdot \left(
          \frac{4\rho \hat C_P \overline u_z}{9 k R z}
        \right)^{\frac{1}{3}}
      \right|_{y=0} \nonumber\\
      &= -2R \cdot \frac{1}{\Gamma\left(
          \frac{4}{3}
        \right)} e^{-(0)^3} \cdot \left(
          \frac{4\rho \hat C_P \overline u_z}{9 k R z}
        \right)^{\frac{1}{3}} \nonumber\\
      &= \frac{2R}{\Gamma\left(
          \frac{4}{3}
        \right)} \left(
          \frac{4\rho \hat C_P \overline u_z}{9 k R z}
        \right)^{\frac{1}{3}} \nonumber\\
      &= \frac{2}{\Gamma\left(
          \frac{4}{3}
        \right)} \left(
          \frac{4\rho \hat C_P \overline u_z R^2}{9 k z}
        \right)^{\frac{1}{3}} \nonumber\\
      &= \frac{2}{\Gamma\left(
          \frac{4}{3}
        \right)} \left(
          \frac{\text{Pe}}{9 (z/2R)}
        \right)^{\frac{1}{3}}
    \end{align}
    最後得到:
    \begin{equation}
      \boxed{\text{Nu} = 1.077 \text{Pe}^{\frac{1}{3}} \left(
        \frac{2R}{z}
      \right)^{\frac{1}{3}}} = 1.077 \text{Gz}^{\frac{1}{3}}
    \end{equation}
    \item 將剛剛獲得的Nusselt Number積分出管長L的平均Nusselt Number
    \begin{align}
      \left<\text{Nu}\right>_L &= \frac{1}{L} \int_0^L \text{Nu} dz \nonumber\\
      &= 1.077\left( \text{Pe} 2R\right)^{\frac{1}{3}}\frac{1}{L}  \int_0^L z^{-\frac{1}{3}} dz \nonumber\\
      &= 1.077\left( \text{Pe} 2R\right)^{\frac{1}{3}}\frac{1}{L} \cdot\frac{3}{2} L^{\frac{2}{3}} \nonumber\\
      &= 1.615 \left( \frac{2R}{L} \text{Pe} \right)^{\frac{1}{3}} L^{\frac{2}{3}} \nonumber\\
      &= 1.615 \left(
        \text{Pe}\cdot \frac{2R}{L}
      \right)^{\frac{1}{3}} \nonumber\\
      &= 1.615 \text{Gz}^{\frac{1}{3}} \label{eq:leveque_average_nusselt_number}
    \end{align}
    \item 邊界層發展完全的深度$\delta_T/R = 1$,$L_{\text{entry}}$為:
    \begin{align}
      &1 = \frac{\delta_T}{R} \nonumber\\
      \Rightarrow & 1 = 2.912 \text{Pe}^{-\frac{1}{3}} \left(
        \frac{L_{\text{entry}}}{2R}
      \right)^{\frac{1}{3}} \nonumber\\
      \Rightarrow & \left(\frac{L_{\text{entry}}}{2R}\right)^{\frac{1}{3}} = \frac{1}{2.912} \text{Pe}^{\frac{1}{3}} \nonumber\\
      \Rightarrow & \frac{L_{\text{entry}}}{2R} = \left(
        \frac{1}{2.912}
      \right)^3 \text{Pe} \nonumber\\
      \Rightarrow & \boxed{L_{\text{entry}} = 0.0405 \text{Pe} \cdot 2R}
    \end{align}
    P.S. 也就是說假設Pe是1000的話,入口約略是40倍管直徑
  \end{enumerate}
  \item 假設流場與溫度場一起發展,回顧之前討論流體的邊界層發展(\ref{fig:ch2_5_boundary_layer_development})\\
  以及上個假設中所觀察到,溫度場的邊界層發展速度較流體層慢($z^{1/3}$ vs. $z^{1/2}$)\\
  因此可以畫出下圖:
  \begin{figure}[H]
    \centering
    \begin{tikzpicture}[>=Latex, line cap=round, line join=round, thick]
      \fill[pattern=north east lines, draw=black] (-1,0) rectangle (2,-0.2);
      \fill[pattern=north west lines, pattern color=red, draw=red] (2,0) rectangle (12,-0.2);
      \draw[dashed] (2,0) -- (2,-1);
      \draw[->] (2,-0.7) -- (1,-0.7) node[left] {$T_\infty$};
      \draw[->] (2,-0.7) -- (3,-0.7) node[right] {$T_0$};
      \draw[blue] (0.5,0) -- (0.5,4);
      \draw[blue, dashed] (1.5,0) -- (1.5,4);
      \draw[->, blue, dashed] (0.5,0.5) -- (1.5,0.5);
      \draw[->, blue, dashed] (0.5,1) -- (1.5,1);
      \draw[->, blue, dashed] (0.5,1.5) -- (1.5,1.5);
      \draw[->, blue, dashed] (0.5,2) -- (1.5,2);
      \draw[->, blue, dashed] (0.5,2.5) -- (1.5,2.5);
      \draw[->, blue, dashed] (0.5,3) -- (1.5,3);
      \draw[->, blue, dashed] (0.5,3.5) -- (1.5,3.5);
      \draw[->, blue, dashed] (0.5,4) -- (1.5,4);
      \node[anchor=south, blue] at (1,4) {U};
      \draw[->] (13,0) -- (13,1) node[above] {$y$};
      \draw[->] (13,0) -- (14,0) node[right] {$x$};
      \draw[blue] (2,0) .. controls (5,3) and (10,3.72) .. (15.6,4.24);
      \path[name path=B] (2,0) .. controls (5,3) and (10,3.72) .. (15.6,4.24);
      \path[name path=C] (9,0) -- (9,4);
      \path[name intersections={of=B and C, by=D}];
      \draw[blue, dashed,<->] (9,0) -- (D) node[midway, right] {$\delta(x)$};
      \path[name path=E] (9.5,0) -- (9.5,4);
      \path[name intersections={of=B and E, by=F}];
      \draw[->, blue, dotted] (F) -- +(0,0.5);
      \node[anchor=south east, blue] at (F) {$U$};
      \draw[red] (2,0) .. controls (5,2.1) and (10,2.6) .. (15.6,3);
      \path[name path=Ba] (2,0) .. controls (5,2.1) and (10,2.6) .. (15.6,3);
      \path[name path=Ca] (11,0) -- (11,4);
      \path[name intersections={of=Ba and Ca, by=Da}];
      \draw[red, dashed,<->] (11,0) -- (Da) node[midway, right] {$\delta_T(x)$};
      \path[name path=Ea] (11.5,0) -- (11.5,4);
      \path[name intersections={of=Ba and Ea, by=Fa}];
      \draw[->, red, dotted] (Fa) -- +(0,0.5);
      \node[anchor=south east, red] at (Fa) {$T_\infty$};
      \draw[dashed, blue] (4,0) ..controls (4.8,0) and (5, 2.6) .. (5,4);
      \path[name path=P1] (4,0) ..controls (4.8,0) and (5, 2.6) .. (5,4);
      \draw[blue] (4,0) -- (4,4);
      \path[name path=Pa] (4,0.5) -- (5,0.5);
      \path[name intersections={of=P1 and Pa, by=I1}];
      \draw[->, blue, dashed] (4,0.5) -- (I1);
      \path[name path=Pb] (4,1) -- (5,1);
      \path[name intersections={of=P1 and Pb, by=I2}];
      \draw[->, blue, dashed] (4,1) -- (I2);
      \path[name path=Pc] (4,1.5) -- (5,1.5);
      \path[name intersections={of=P1 and Pc, by=I3}];
      \draw[->, blue, dashed] (4,1.5) -- (I3);
      \path[name path=Pd] (4,2) -- (5,2);
      \path[name intersections={of=P1 and Pd, by=I4}];
      \draw[->, blue, dashed] (4,2) -- (I4);
      \path[name path=Pe] (4,2.5) -- (5,2.5);
      \path[name intersections={of=P1 and Pe, by=I5}];
      \draw[->, blue, dashed] (4,2.5) -- (I5);
      \path[name path=Pf] (4,3) -- (5,3);
      \path[name intersections={of=P1 and Pf, by=I6}];
      \draw[->, blue, dashed] (4,3) -- (I6);
      \path[name path=Pg] (4,3.5) -- (5,3.5);
      \path[name intersections={of=P1 and Pg, by=I7}];
      \draw[->, blue, dashed] (4,3.5) -- (I7);
      \path[name path=Ph] (4,4) -- (5,4);
      \path[name intersections={of=P1 and Ph, by=I8}];
      \draw[->, blue, dashed] (4,4) -- (I8);
      \draw[dashed, blue] (7,0) ..controls (7.6,0) and (8,3.4) .. (8,4);
      \path[name path=P2] (7,0) ..controls (7.6,0) and (8,3.4) .. (8,4);
      \path[name path=Qa] (7,0.5) -- (8,0.5);
      \path[name intersections={of=P2 and Qa, by=J1}];
      \draw[->, blue, dashed] (7,0.5) -- (J1);
      \path[name path=Qb] (7,1) -- (8,1);
      \path[name intersections={of=P2 and Qb, by=J2}];
      \draw[->, blue, dashed] (7,1) -- (J2);
      \path[name path=Qc] (7,1.5) -- (8,1.5);
      \path[name intersections={of=P2 and Qc, by=J3}];
      \draw[->, blue, dashed] (7,1.5) -- (J3);
      \path[name path=Qd] (7,2) -- (8,2);
      \path[name intersections={of=P2 and Qd, by=J4}];
      \draw[->, blue, dashed] (7,2) -- (J4);
      \path[name path=Qe] (7,2.5) -- (8,2.5);
      \path[name intersections={of=P2 and Qe, by=J5}];
      \draw[->, blue, dashed] (7,2.5) -- (J5);
      \path[name path=Qf] (7,3) -- (8,3);
      \path[name intersections={of=P2 and Qf, by=J6}];
      \draw[->, blue, dashed] (7,3) -- (J6);
      \path[name path=Qg] (7,3.5) -- (8,3.5);
      \path[name intersections={of=P2 and Qg, by=J7}];
      \draw[->, blue, dashed] (7,3.5) -- (J7);
      \path[name path=Qh] (7,4) -- (8,4);
      \path[name intersections={of=P2 and Qh, by=J8}];
      \draw[->, blue, dashed] (7,4) -- (J8);
      \draw[blue] (7,0) -- (7,4);
    \end{tikzpicture}
    \caption{Boundary Layer Development with both Velocity and Temperature Fields}
  \end{figure}
  \begin{enumerate}
    \item 假設Newtonnian, incompressible, steady flow\\
    有個虛擬的長度$L\gg \delta >\delta_T$\\
    並假設在邊界層內\fbox{流速與$y$成正比},雖然實際上是二次關係\\
    但因為邊界層很薄,可以假想都受到相同的$\tau_{yx}$剪應力影響
    \begin{equation}
      \tau_{yx} = \mu \frac{\partial u_x}{\partial y} \implies u_x(y) = \frac{\tau_{yx}}{\mu} y \label{eq:graetz_linear_velocity_profile}
    \end{equation}
    又由於當發展到$\delta(x)$時,流速要等於$U$,所以那個$y$前面的係數會是$\frac{U}{\delta(x)}$
    \begin{equation}
      u_x(y) = \frac{U}{\delta(x)} y
    \end{equation}
    \item 無因次特徵量:\\
    溫度的無因次化
    \begin{equation}
      T^\ast = \frac{T - T_0}{T_\infty - T_0}
    \end{equation}
    $y$方向的長度特徵選擇溫度的邊界厚度
    \begin{equation}
      y_T^\ast = \frac{y}{\delta_T(x)}
    \end{equation}
    而且會與用流體邊界厚度的有以下比例關係
    \begin{equation}
      y^\ast = \frac{y}{\delta(x)} = \frac{\delta_T(x)}{\delta(x)} y_T^\ast
    \end{equation}
    $x$方向速度的無因次化,則利用剛剛的(\ref{eq:graetz_linear_velocity_profile})假設\\
    可以換成$y_T^\ast$來消除一個變數
    \begin{equation}
      u_x^\ast = \frac{u_x}{U} = \frac{y}{\delta(x)} = \frac{\delta_T(x)}{\delta(x)} y_T^\ast \label{eq:graetz_ux_dimensionless}
    \end{equation}
    $y$方向速度的無因次化\\
    同之前流場的邊界層一樣,透過Goldilock Argument選擇
    \begin{equation}
      u_y^\ast = \frac{u_y}{V},\quad V = U \frac{\delta(x)}{L}
    \end{equation}
    再利用Equation of Continuity,可以換成$y_T^\ast$\\
    連續方程式:
    \begin{equation}
      \frac{\partial u_x}{\partial x} = \frac{\partial u_y}{\partial y}
    \end{equation}
    無因次化:
    \begin{align}
      & \frac{U}{L} \frac{\partial u^\ast}{\partial x^\ast} = \frac{V}{\delta_T} \frac{\partial u_y^\ast}{\partial y_T^\ast} \nonumber\\
      \Rightarrow & \frac{U}{L} \frac{\partial u^\ast}{\partial x^\ast}  = \frac{U \delta}{L \delta_T} \frac{\partial u_y^\ast}{\partial y_T^\ast} \nonumber\\
      \Rightarrow & \frac{\partial u^\ast}{\partial x^\ast}  = \frac{\delta}{\delta_T} \frac{\partial u_y^\ast}{\partial y_T^\ast}
    \end{align}
    等號左邊利用(\ref{eq:graetz_ux_dimensionless})代入:
    \begin{align}
      & \frac{\partial }{\partial x^\ast} \left(
        \frac{\delta_T}{\delta} y_T^\ast
      \right) = \frac{\delta}{\delta_T} \frac{\partial u_y^\ast}{\partial y_T^\ast} \nonumber\\
      \Rightarrow & y_T^\ast \frac{\partial }{\partial x^\ast} \left(
        \frac{\delta_T}{\delta}
      \right) = \frac{\delta}{\delta_T} \frac{\partial u_y^\ast}{\partial y_T^\ast} \nonumber\\
      \Rightarrow & \frac{\partial u_y^\ast}{\partial y_T^\ast} = \frac{\delta_T^2}{\delta^2} y_T^\ast \frac{\partial }{\partial x^\ast} \left(
        \frac{\delta_T}{\delta}
      \right)
    \end{align}
    而我們假定兩個邊界層的\fbox{發展比例是固定的}\\
    將上式積分:
    \begin{equation}
      u_y^\ast = \frac{\delta_T^2}{\delta^2}y_T^{\ast 2} \label{eq:graetz_uy_dimensionless}
    \end{equation}
    \item 最後將能量方程式無因次化:
    \begin{align}
      &\rho \hat C_P \left(
        u_x \frac{\partial T}{\partial x} + u_y \frac{\partial T}{\partial y}
      \right) = k\left(
        \frac{\partial^2 T}{\partial x^2} + \frac{\partial^2 T}{\partial y^2}
      \right)
      \nonumber\\
      \Rightarrow & \rho \hat C_P \left[
        \left(
          u_x^\ast U
        \right) \left(
          \frac{T_\infty - T_0}{L} \frac{\partial T^\ast}{\partial x^\ast}
        \right) + \left(
          u_y^\ast V
        \right) \left(
          \frac{T_\infty - T_0}{\delta_T} \frac{\partial T^\ast}{\partial y_T^\ast}
        \right)
      \right] = \nonumber\\
      & k \left[
        \frac{T_\infty - T_0}{L^2} \frac{\partial^2 T^\ast}{\partial x^{\ast 2}} +
        \frac{T_\infty - T_0}{\delta_T^2} \frac{\partial^2 T^\ast}{\partial y_T^{\ast 2}}
      \right]
    \end{align}
    因為$\delta_T \ll L$,所以可以忽略掉$\frac{\partial^2 T^\ast}{\partial x^{\ast 2}}$項\\
    並將$u_x^\ast, u_y^\ast$用(\ref{eq:graetz_ux_dimensionless}), (\ref{eq:graetz_uy_dimensionless})代入:
    \begin{equation}
      \rho \hat C_P \left[
        \left(
          \frac{\delta_T}{\delta} y_T^\ast U
        \right) \left(
          \frac{T_\infty - T_0}{L} \frac{\partial T^\ast}{\partial x^\ast}
        \right) + \left(
          \frac{\delta_T^2}{\delta^2} y_T^{\ast 2} V
        \right) \left(
          \frac{T_\infty - T_0}{\delta_T} \frac{\partial T^\ast}{\partial y_T^\ast}
        \right)
      \right] =  k \left[
        \frac{T_\infty - T_0}{\delta_T^2} \frac{\partial^2 T^\ast}{\partial y_T^{\ast 2}}
      \right]
    \end{equation}
    再簡化,同除以$(T_\infty - T_0)$:
    \begin{align}
      &\rho \hat C_P \left[
        \frac{\delta_T}{\delta} y_T^\ast U \cdot \frac{1}{L} \frac{\partial T^\ast}{\partial x^\ast} +
        \frac{\delta_T^2}{\delta^2} y_T^{\ast 2} U \frac{\delta}{L} \cdot \frac{1}{\delta_T} \frac{\partial T^\ast}{\partial y_T^\ast}
      \right] = \frac{k}{\delta_T^2} \left(\frac{\partial^2 T^\ast}{\partial y_T^{\ast 2}}\right) \nonumber\\
      \Rightarrow & \rho \hat C_P U \left[
        \frac{\delta_T}{\delta} y_T^\ast \frac{1}{L} \frac{\partial T^\ast}{\partial x^\ast} +
        \frac{\delta_T}{\delta} y_T^{\ast 2} \frac{1}{L} \frac{\partial T^\ast}{\partial y_T^\ast}
      \right] =  \frac{k}{\delta_T^2} \left(\frac{\partial^2 T^\ast}{\partial y_T^{\ast 2}}\right) \nonumber\\
      \Rightarrow & \rho \hat C_P U \frac{\delta_T}{\delta} \frac{1}{L} \left[
        y_T^\ast \frac{\partial T^\ast}{\partial x^\ast} +
        y_T^{\ast 2} \frac{\partial T^\ast}{\partial y_T^\ast}
      \right] =  \frac{k}{\delta_T^2} \left(\frac{\partial^2 T^\ast}{\partial y_T^{\ast 2}}\right) \nonumber\\
      \Rightarrow & \frac{\rho \hat C_P U \delta_T^3}{k L\delta} \left[
        y_T^\ast \frac{\partial T^\ast}{\partial x^\ast} +
        y_T^{\ast 2} \frac{\partial T^\ast}{\partial y_T^\ast}
      \right] = \frac{\partial^2 T^\ast}{\partial y_T^{\ast 2}} \label{eq:graetz_energy_equation_dimensionless}
    \end{align}
    \item 同樣,利用Goldilock Argument,判定左邊那陀係數應該接近1\\
    因此可以得到兩邊界層的比例關係:
    \begin{equation}
      \frac{\rho \hat C_P U \delta_T^3}{k L\delta} = 1 
    \end{equation}
    以$\delta_T$來看:
    \begin{equation}
      \delta_T = \left(
        \frac{kL \delta}{ \rho \hat C_P U}
      \right)^{\frac{1}{3}} =  \left(
        \frac{kL}{ \rho \hat C_P U\delta^2}
      \right)^{\frac{1}{3}} \delta
    \end{equation}
    同除$\delta$,得到:
    \begin{equation}
      \boxed{\frac{\delta_T}{\delta} = \left(
        \frac{kL}{\rho \hat C_P U\delta^2}
        \right)^{\frac{1}{3}}} \label{eq:graetz_deltaT_over_delta}
    \end{equation}
    \item 更進一步,將上式結合流場邊界層得出的(\ref{eq:ch2_5_boundary_layer_dev_delta_Re}):
    \begin{equation}
      \frac{\delta}{L} \sim \text{Re}_L^{-\frac{1}{2}} = \left(
        \frac{\mu}{\rho U L}
      \right)^{\frac{1}{2}} \implies \delta^2 \sim \frac{\mu L}{\rho U}
    \end{equation}
    代入(\ref{eq:graetz_deltaT_over_delta}):
    \begin{align}
      & \frac{\delta_T}{\delta} = \left(
        \frac{kL}{\rho \hat C_P U} \cdot \frac{\rho U}{\mu L}
      \right)^{\frac{1}{3}} \nonumber\\
      \Rightarrow & \frac{\delta_T}{\delta} = \left(
        \frac{k}{\hat C_P \mu}
      \right)^{\frac{1}{3}} \nonumber\\
      \Rightarrow & \boxed{\frac{\delta_T}{\delta} = \text{Pr}^{-\frac{1}{3}}}
    \end{align}
    也就是說當Prandtl Number越大,熱傳邊界層相對於流體邊界層就越薄\\
    衍生,如果將$\frac{\delta_T}{L}$表示出來:
    \begin{align}
      & \frac{\delta_T}{L} = \frac{\delta_T}{\delta} \cdot \frac{\delta}{L} \nonumber\\
      \Rightarrow & \boxed{\frac{\delta_T}{L} = \text{Re}_L^{-\frac{1}{2}} \text{Pr}^{-\frac{1}{3}}}
    \end{align}
    然後把熱傳係數相關的Nusselt Number也表示出來(見\ref{eq:graetz_nusselt_number})
    \begin{align}
      & \text{Nu}_L = \frac{hL}{k}, \quad h = \frac{k}{\delta_T} \nonumber\\
      \Rightarrow & \text{Nu}_L = \frac{k}{\delta_T}\cdot \frac{L}{k} = \frac{L}{\delta_T} \nonumber\\
      \Rightarrow & \boxed{\text{Nu}_L = \text{Re}_L^{\frac{1}{2}} \text{Pr}^{\frac{1}{3}}}
    \end{align}
    \item 不過剛剛也只是假設兩個邊界層發展比例固定,而產生的因次關係推導\\
    解題上還需回到只利用流體邊界層內流體是線性的這個假設\\
    不能將全部都以$y_T^\ast$來表示
    \begin{equation}
      u_x \frac{\partial T}{\partial x} + u_y \frac{\partial T}{\partial y} = \alpha \frac{\partial^2 T}{\partial y^2}
    \end{equation}
    但這個式子其實裡面有$u(x,y)$,又有$T(x,y)$,是兩個PDE耦合在一起的系統\\
    需要想辦法將$u$消掉,只剩下$T$的PDE\\
    於是我們假設獨立性,而先解流體邊界層,得到$u(x,y)$後,再代入能量方程式中\\
    至於流體邊界層的解法,已經在之前章節解過見(\ref{sec:ch2_5_boundary_layer_dev_definition_F})\\
    並令$u_x = U F'(\eta_H)$\\
    其中$\eta_H = y\cdot \sqrt{\frac{\rho U}{\mu x}}$,是Hydrodynamic Boundary Layer的無因次化變數\\
    可以看(\ref{sec:ch2_5_boundary_layer_dev_definition_F})\\
    當時用解析解解出(\ref{eq:ch2_5_boundary_layer_dev_velocity_profile})
    \begin{equation}
      u_x = 0.332 U \sqrt{\frac{\rho U}{\mu x}} y
    \end{equation}
    以及將$u_x$代入連續方程式得到的$u_y$(\ref{eq:ch2_5_boundary_layer_dev_velocity_profile_u_y})
    \begin{equation}
      u_y = 0.083 \sqrt{\frac{\rho U^3}{\mu x^3}} y^2
    \end{equation}
    代入能量方程式:
    \begin{align}
      & U F'(\eta_H) \frac{\partial T}{\partial x} + 0.083 \sqrt{\frac{\rho U^3}{\mu x^3}} y^2 \frac{\partial T}{\partial y} = \alpha \frac{\partial^2 T}{\partial y^2} \nonumber\\
      \Rightarrow & U \left(
        0.332 \sqrt{\frac{\rho U}{\mu x}} y
      \right) \frac{\partial T}{\partial x} + 0.083 \sqrt{\frac{\rho U^3}{\mu x^3}} y^2 \frac{\partial T}{\partial y} = \alpha \frac{\partial^2 T}{\partial y^2} \nonumber\\
      \Rightarrow & 0.332 U \sqrt{\frac{\rho U}{\mu x}} y \frac{\partial T}{\partial x} + 0.083 \sqrt{\frac{\rho U^3}{\mu x^3}} y^2 \frac{\partial T}{\partial y} = \alpha \frac{\partial^2 T}{\partial y^2}
    \end{align}
    至此,我們成功的消除速度項,而變成$T(x,y)$的偏微分方程式\\
    而邊界條件為:
    \begin{align}
      &T(0,y) = T_\infty \nonumber\\
      &T(x,\infty) = T_\infty, \quad T(x,\delta_T(x))=T_\infty \nonumber\\
      &T(x,0) = T_0
    \end{align}
    其中前兩個邊界條件可以合併再一起\\
    但可以發現剩下兩個邊界條件在$y=0$處是矛盾的,但沒關係\\
    我們還是可以再用一次跟流場一樣的Similarity Solution來解這個PDE\\
    定義無因次化變數:
    \begin{equation}
      \eta_T =\frac{y}{\delta_T(x)}, \quad \theta = \frac{T - T_0}{T_\infty - T_0}
    \end{equation}
    則省略那些($T_\infty - T_0$),畢竟代進去也會消掉,或者就直接想成能量平衡式
    \begin{equation}
      u_x \frac{\partial \theta}{\partial x} + u_y \frac{\partial \theta}{\partial y} = \alpha \frac{\partial^2 \theta}{\partial y^2}
    \end{equation}
    而每個偏微分項:
    \begin{align}
      \frac{\partial \theta}{\partial x} & = \frac{\partial \theta}{\partial \eta_T}{\partial \eta_T}{\partial x}
      = -\frac{y}{\delta_T^2(x)}\frac{d\delta_T}{dx}\frac{\partial \theta}{\partial \eta_T} \nonumber\\
      \frac{\partial \theta}{\partial y} & = \frac{\partial \theta}{\partial \eta_T}{\partial \eta_T}{\partial y}
      = \frac{1}{\delta_T(x)} \frac{\partial \theta}{\partial \eta_T} \nonumber\\
      \frac{\partial^2 \theta}{\partial y^2} & = \frac{\partial }{\partial y} \left(
        \frac{1}{\delta_T(x)} \frac{\partial \theta}{\partial \eta_T}
      \right) = \frac{1}{\delta_T^2(x)} \frac{\partial^2 \theta}{\partial \eta_T^2}
    \end{align}
    把$u_x, u_y$以及上面三個偏微分代入能量方程式:
    \begin{align}
      &0.332 \left(
        \frac{\rho U^3}{\mu x}
      \right)^\frac{1}{2} y \cdot \left(
        -\frac{y}{\delta_T^2(x)} \frac{d\delta_T}{dx} \frac{\partial \theta}{\partial \eta_T}
      \right) + 0.083 \left(
        \frac{\rho U^3}{\mu x^3}
      \right)^\frac{1}{2} y^2 \cdot \left(
        \frac{1}{\delta_T(x)} \frac{\partial \theta}{\partial \eta_T}
      \right) \nonumber\\
      &\quad = \alpha \cdot \left(
        \frac{1}{\delta_T^2(x)} \frac{\partial^2 \theta}{\partial \eta_T^2}
      \right)
    \end{align}
    再將$y = \eta_T \delta_T(x)$代入:
    \begin{align}
      & -0.332 \left(
        \frac{\rho U^3}{\mu x}
      \right)^\frac{1}{2} \eta_T^2 \cancel{\delta_T^2(x) }\cdot \cancel{\frac{1}{\delta_T^2(x)} }\frac{d\delta_T}{dx} \frac{\partial \theta}{\partial \eta_T}
      + 0.083 \left(
        \frac{\rho U^3}{\mu x^3}
      \right)^\frac{1}{2} \eta_T^2 \delta_T(x) \cdot \frac{1}{\delta_T(x)} \frac{\partial \theta}{\partial \eta_T} \nonumber\\
      & \quad = \alpha \cdot \left(
        \frac{1}{\delta_T^2(x)} \frac{\partial^2 \theta}{\partial \eta_T^2}
      \right)
    \end{align}
    目前這就已經是2階ODE了,但其實沒有沒被微分的$\theta$\\
    所以同樣的方法,令$F'(\eta_T)=\frac{\partial \theta}{\partial \eta}$\\
    但為了怕跟上面的$F$搞混,我們叫他$F_T$好了,把$\frac{\alpha}{\delta_T^2(x)}$移到左邊\\
    然後把全部丟到右邊變成:
    \begin{equation}
      F_T' + \left[
        0.332\sqrt{
          \frac{\rho U^3}{\mu x}
        }\frac{\eta_T^2\delta_T^2(x)}{\alpha} \frac{d\delta_T}{dx} F_T
      \right]
      - \left[
        0.083 \sqrt{
          \frac{\rho U^3}{\mu x^3}
        } \frac{\eta_T^2 \delta_T^3(x)}{\alpha} F_T
      \right] = 0 \label{eq:graetz_ODE_before_simplify}
    \end{equation}
    然後,還記得在上一個Leveque的假設中有(\ref{eq:leveque_ODE}):
    \begin{equation}
      \frac{d^2 \theta}{d\eta^2} + 3\eta^2 \frac{d\theta}{d\eta} = 0
    \end{equation}
    所以如果那兩坨中刮號內的係數相減能是常數3的話,那就再好不過了!\\
    或者說,如果是圓管的話,那她大概就是3了\\
    那接著我們來要要怎麼讓他變成成3?\\
    觀察兩個係數,0.332是4倍的0.083\\
    我們如果又右邊的係數共同項拿出來叫做$\xi$的話
    \begin{equation}
      \xi = 0.083 \sqrt{
        \frac{\rho U^3}{\mu x}
      } \frac{\eta_T^2}{\alpha}
    \end{equation}
    那麼左邊的係數就會是$4 \xi \delta_T^3 \frac{d\delta_T}{dx}$\\
    因此我們要
    \begin{equation}
      4 \xi \delta_T^3 \frac{d\delta_T}{dx} - \xi\frac{\delta_T^2}{x} = 3
    \end{equation}
    % Proof that the left term will be constant 6 and the right term is 3
    令$f(x)=\delta_T^3(x)$,則$f'(x)=3\delta_T^2 \frac{d\delta_T}{dx}$\\
    或者說:
    \begin{equation}
      \frac{1}{3}f'(x) = \delta_T^2 \frac{d\delta_T}{dx}
    \end{equation}
    代入上式:
    \begin{align}
      & 4 \xi \delta_T^2\delta_T' = \frac{4}{3}\xi f'(x) \nonumber\\
      \Rightarrow & \xi \frac{\delta_T^3}{x} = \xi\frac{f}{x}
    \end{align}
    又由於$\xi$當中$x$在分母,故$\propto x^{-0.5}$
    \begin{equation}
      \xi \frac{f}{x} \propto x^{-0.5} \frac{f}{x}= x^{-1.5} f \implies f \propto x^{1.5}
    \end{equation}
    因此令$f(x) = C x^{1.5}$,則$f'(x) = 1.5 C x^{0.5}$\\
    代入上式:
    \begin{equation}
      4\xi \delta_T^2\delta_T' = \frac{4}{3}\xi (1.5 C x^{0.5}) = 2\xi C x^{0.5}
    \end{equation}
    以及:
    \begin{equation}
      \xi \frac{\delta_T^3}{x} = \xi \frac{C x^{1.5}}{x} = \xi C x^{0.5}
    \end{equation}
    可以看到他們的比例是2:1\\
    而兩個相減又要是3,所左邊會是6,右邊是3\\
    也就是說回到一開始的ODE:
    \begin{equation}
      \left[
        0.332\sqrt{
          \frac{\rho U^3}{\mu x}
        }\frac{\delta_T^2(x)}{\alpha} \frac{d\delta_T}{dx}
      \right] = 6, \quad
      \left[
        0.083 \sqrt{
          \frac{\rho U^3}{\mu x^3}
        } \frac{\delta_T^3(x)}{\alpha}
      \right] = 3
    \end{equation}
   而右邊那項,是可以把$\delta_T(x)$解出來的
   % Try to make 3.306 Pr^(-1/3) Re_x^(-1/2)x
   \begin{align}
    & 0.083 \sqrt{
          \frac{\rho U^3}{\mu x^3}
        } \frac{\delta_T^3(x)}{\alpha} 
       = 3 \nonumber\\
    \Rightarrow & \delta_T^3(x) = \frac{3}{0.083} \alpha\sqrt{
          \frac{\mu x^3}{\rho U^3}
        } \nonumber\\
    \Rightarrow & \delta_T(x) = \sqrt[3]{
      \frac{3}{0.083}
    }\cdot \alpha^{\frac{1}{3}} \left(
      \frac{\mu}{\rho U^3}
    \right)^{\frac{1}{6}} x^{\frac{1}{2}} \nonumber\\
    \Rightarrow & \delta_T(x) = 3.306 \cdot \alpha^{\frac{1}{3}} \left(
      \frac{\mu}{\rho U^3}
    \right)^{\frac{1}{6}} x^{\frac{1}{2}} \nonumber\\
    \Rightarrow & \boxed{\delta_T(x) = 3.306 \cdot \text{Pr}^{-\frac{1}{3}} \text{Re}_x^{-\frac{1}{2}} x} \label{eq:graetz_deltaT_exact_solution}
   \end{align}
   \item 最後把相似於Leveque假設的ODE解出\\
   假設等於3時,還原$F_T$,得到ODE:
  \begin{equation}
    \frac{d^2 \theta}{d\eta_T^2} + 3\eta_T^2 \frac{d\theta}{d\eta_T} = 0
  \end{equation}
  邊界條件:
  \begin{align}
    & \theta(0) = 1 \nonumber\\
    & \theta(\infty) = 0
  \end{align}
  解出來的解為:
  \begin{equation}
    \theta(\infty) = 1 = A \int_0^\infty e^{-\eta_T^3} d\eta_T = A\Gamma\left(\frac{4}{3}\right)
  \end{equation}
  所以$A = \frac{1}{\Gamma(4/3)}$
  \begin{equation}
    \boxed{\theta(\eta_T) = \frac{1}{\Gamma(4/3)} \int_0^{\eta_T} e^{-\eta_T^3} d\eta_T} \label{eq:graetz_temperature_profile}
  \end{equation}
  \item 衍生,熱邊界層深度與厚度\\
  假設定義熱邊界層厚度為$\theta(\eta_T) = 0.99$,則由(\ref{eq:graetz_temperature_profile})可知
  \begin{equation}
    0.99 = \frac{1}{\Gamma(4/3)} \int_0^{\eta_{T}} e^{-x^3} dx \implies \eta_T=1.4037
  \end{equation}
  再將$\eta_T$代入(\ref{eq:graetz_deltaT_exact_solution}),求得熱邊界層厚度:
  \begin{align}
    & \delta_T(x) = \eta_T \cdot \left(
      3.306 \cdot \text{Pr}^{-\frac{1}{3}} \text{Re}_x^{-\frac{1}{2}} x
    \right) \nonumber\\
    \Rightarrow & \boxed{\delta_T(x) = 4.629 \text{Pr}^{-\frac{1}{3}} \text{Re}_x^{-\frac{1}{2}} x} \label{eq:graetz_deltaT_final}
  \end{align}
  \item 衍生,熱傳係數$h$與Nusselt Number的推算\\
  由截面的熱平衡,冷卻,吸熱,所以是正值
  \begin{equation}
    h(T_\infty -T_0) = +k\frac{\partial T}{\partial y}\bigg|_{y=0}
  \end{equation}
  解得:
  \begin{align}
    h &= \frac{k\frac{\partial T}{\partial y}\bigg|_{y=0}}{T_\infty - T_0} \nonumber\\
    &= k\left(\frac{\partial \theta}{\partial y}\right)_{y=0} \nonumber\\
    &= k\left(
      \frac{\partial \theta}{\partial \eta_T} \cdot \frac{\partial \eta_T}{\partial y}
    \right)_{\eta=0} \nonumber\\
    &= k\left(
      \frac{1}{\delta_T(x)} \frac{\partial \theta}{\partial \eta_T}
    \right)_{\eta=0}
  \end{align}
  代入(\ref{eq:graetz_temperature_profile})和(\ref{eq:graetz_deltaT_exact_solution}):
  \begin{align}
    h &= k\left(
      \frac{1}{x\cdot 3.306 \cdot \text{Pr}^{-\frac{1}{3}} \text{Re}_x^{-\frac{1}{2}}} 
      \cdot \frac{1}{\Gamma(4/3)} \cdot \cancelto{1}{ e^{-\eta_T^3}\bigg|_{\eta_T=0}}
    \right) \nonumber\\
    &= 0.3387 \cdot \frac{k}{x} \cdot \text{Pr}^{\frac{1}{3}} \text{Re}_x^{\frac{1}{2}}
  \end{align}
  將$\frac{k}{x}$移到左邊,並乘以2,得到Nusselt Number(假設是圓管,故$D=2x$)
  \begin{align}
    & \frac{hx}{k} = 0.3387 \cdot \text{Pr}^{\frac{1}{3}} \text{Re}_x^{\frac{1}{2}} \nonumber\\
    \Rightarrow & \frac{2hx}{k} = 0.6775 \cdot \text{Pr}^{\frac{1}{3}} \text{Re}_x^{\frac{1}{2}} \nonumber\\
    \Rightarrow & \boxed{\text{Nu}_x = 0.6775 \cdot \text{Pr}^{\frac{1}{3}} \text{Re}_x^{\frac{1}{2}}}
  \end{align} 
  當時解Leveque假設的結果(\ref{eq:leveque_average_nusselt_number})比較:
  \begin{equation}
    \left<\text{Nu}\right>_L=0.615 \text{Gz}^{\frac{1}{3}} 
  \end{equation}
  其中$Gz$的定義:
  \begin{equation}
    \text{Gz} = \text{Pe} \cdot \frac{2R}{L}
  \end{equation}
  而$\text{Pe} = \text{Re}\cdot \text{Pr}$\\
  顧可以看到兩者在雷諾數的指數關係與本題的是不同的($\frac{1}{2}$ vs. $\frac{1}{3}$)\\
  因為Leveque假設是針對平均熱傳係數來算的,而本題是針對局部熱傳係數來算的
\end{enumerate}
\end{itemize}
\end{CJK*}
\end{document}