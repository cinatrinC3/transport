\documentclass[../main.tex]{subfiles}
\begin{document}
\begin{CJK*}{UTF8}{bkai}
\subsection{熱量+動量輸送,黏性摩擦, Graetz Problem, Leveque Method}
\begin{itemize}
  \item 從固體變成流體後,會多了動量輸送的影響\\
  也就是會出現Governing Equation中的,來自流體壓降所反應的不可逆的能量上升(\ref{eq:viscous_dissipation})
  \begin{equation}
    \overline{\overline{\bm \tau}} : \nabla \vec {\bm v} =\frac{2}{3}\mu \sum_{i=1}^{3}\sum_{j=1}^{3}\left[
          \left(
          \frac{\partial u_i}{\partial x_i} + \frac{\partial u_j}{\partial x_j}
        \right) - \frac{2}{3}\delta_{ij} \nabla \cdot \vec{\bm u}
        \right]^2 + \kappa \left(
          \nabla \cdot \vec{\bm u}
        \right)^2 = \mu \bm \Phi_u + \kappa \bm \Phi_b
  \end{equation}
  \begin{figure}[H]
    \centering
    \begin{tikzpicture}[>=Latex, line cap=round, line join=round, thick]
      \draw[->] (0,0) -- (0,1) node[above] {$x$};
      \draw[->] (0,0) -- (1,0) node[right] {$z$};
      \draw (0,-3) -- (10,-3);
      \draw (0,3) -- (10,3);
      \draw [<->] (9.5,-3) -- (9.5,3) node[midway, right] {$2B$};
      \draw[dashed, blue] (2,3) .. controls (4,0) .. (2,-3);
      \draw[dashed, blue] (2,3) -- (2,-3);
      \draw[dashed, red] (5,3) -- (5,-3);
      \draw[dashed, red] (5.5,3) .. controls (7,0) .. (5.5,-3);
      \node[anchor=south, red] at (5.5,3) {$T_0$};
      \node[blue, anchor=south] at (2,3) {$u_z(x,z,t)$};
      \path[name path=vo] (2,3) .. controls (4,0) .. (2,-3);
      \path[name path=ha] (2,2.5) -- (8,2.5);
      \path[name path=hb] (2,2) -- (8,2);
      \path[name path=hc] (2,1.5) -- (8,1.5);
      \path[name path=hd] (2,1) -- (8,1);
      \path[name path=he] (2,0.5) -- (8,0.5);
      \path[name path=hf] (2,0) -- (8,0);
      \path[name path=hg] (2,-0.5) -- (8,-0.5);
      \path[name path=hh] (2,-1) -- (8,-1);
      \path[name path=hi] (2,-1.5) -- (8,-1.5);
      \path[name path=hj] (2,-2) -- (8,-2);
      \path[name path=hk] (2,-2.5) -- (8,-2.5);
      \path[name intersections={of=vo and ha, by=ya}];
      \draw[blue,dashed, ->] (2,2.5) -- (ya);
      \path[name intersections={of=vo and hb, by=yb}];
      \draw[blue,dashed, ->] (2,2) -- (yb);
      \path[name intersections={of=vo and hc, by=yc}];
      \draw[blue,dashed, ->] (2,1.5) -- (yc);
      \path[name intersections={of=vo and hd, by=yd}];
      \draw[blue,dashed, ->] (2,1) -- (yd);
      \path[name intersections={of=vo and he, by=ye}];
      \draw[blue,dashed, ->] (2,0.5) -- (ye);
      \path[name intersections={of=vo and hf, by=yf}];
      \draw[blue,dashed, ->] (2,0) -- (yf);
      \path[name intersections={of=vo and hg, by=yg}];
      \draw[blue,dashed, ->] (2,-0.5) -- (yg);
      \path[name intersections={of=vo and hh, by=yh}];
      \draw[blue,dashed, ->] (2,-1) -- (yh);
      \path[name intersections={of=vo and hi, by=yi}];
      \draw[blue,dashed, ->] (2,-1.5) -- (yi);
      \path[name intersections={of=vo and hj, by=yj}];
      \draw[blue,dashed, ->] (2,-2) -- (yj);
      \path[name intersections={of=vo and hk, by=yk}];
      \draw[blue,dashed, ->] (2,-2.5) -- (yk);
      \path[name path=temp] (5.5,3) .. controls (7,0) .. (5.5,-3);
      \path[name intersections={of=temp and ha, by=ta}];
      \draw[red,dashed, ->] (5,2.5) -- (ta);
      \path[name intersections={of=temp and hb, by=tb}];
      \draw[red,dashed, ->] (5,2) -- (tb);
      \path[name intersections={of=temp and hc, by=tc}];
      \draw[red,dashed, ->] (5,1.5) -- (tc);
      \path[name intersections={of=temp and hd, by=td}];
      \draw[red,dashed, ->] (5,1) -- (td);
      \path[name intersections={of=temp and he, by=te}];
      \draw[red,dashed, ->] (5,0.5) -- (te);
      \path[name intersections={of=temp and hf, by=tf}];
      \draw[red,dashed, ->] (5,0) -- (tf);
      \path[name intersections={of=temp and hg, by=tg}];
      \draw[red,dashed, ->] (5,-0.5) -- (tg);
      \path[name intersections={of=temp and hh, by=th}];
      \draw[red,dashed, ->] (5,-1) -- (th);
      \path[name intersections={of=temp and hi, by=ti}];
      \draw[red,dashed, ->] (5,-1.5) -- (ti);
      \path[name intersections={of=temp and hj, by=tj}];
      \draw[red,dashed, ->] (5,-2) -- (tj);
      \path[name intersections={of=temp and hk, by=tk}];
      \draw[red,dashed, ->] (5,-2.5) -- (tk);
      \node[anchor=west, red] at (7,0) {$T(x,z,t)$};
      \draw[->] (4.8, 0) -- (4.8, 3) node[midway, left] {$q_c$};
    \end{tikzpicture}
    \caption{穩定流體中的熱傳示意圖}
  \end{figure}
  \item 假設最簡單的狀況\\
  流體已經發展完全,為牛頓流體,且溫度分布也發展完全\\
  邊界條件,已知壁面溫度維持在$T_0$,如同浸泡在冷卻液當中的金屬管內流體
  \begin{equation}
    u_z(x,z,t) \to u_z(x) = U_{\text{max}}\left[1-\left(\frac{x}{B}^2\right)\right],\quad T(x,z,t) \to T(x)
  \end{equation}
  \begin{enumerate}
    \item 由已知的流場,\fbox{寫出熱平衡方程式中的Dissipation項}
    \begin{equation}
      -\overline{\overline{\bm \tau}} : \nabla \vec {\bm v} = \mu \left(
        \frac{\partial u_z}{\partial x}
      \right)^2 = \mu \left(-\frac{2U_{\text{max}}}{B^2}x\right)^2 = \frac{4\mu U_{\text{max}}^2}{B^4} x^2
    \end{equation}
    P.S. 因為已經發展完全,$\vec{\bm u}= u_z$,而$u_z(x)$ only\\
    因此整個剪力的9項中,只會有\fbox{z方向流動造成x方向剪力}的$\tau_{xz}$
    \item 寫出熱平衡的方程式:
    \begin{align}
      \rho \hat C_p \left(
        \cancelto{S.S.}{\frac{\partial T}{\partial t}} + \vec{\bm v} \cdot \nabla T
      \right) &= k \nabla^2 T + \cancel{\dot q} - \overline{\overline{\bm \tau}} : \nabla \vec {\bm v} \nonumber\\
      \rho \hat C_p \left(
        u_z \cancelto{\text{只有T(x)}}{\frac{\partial T}{\partial z}}
      \right) &= k \left( \frac{\partial^2 T}{\partial x^2} 
      + \cancelto{\text{只有T(x)}}{\frac{\partial^2 T}{\partial z^2}}\right) + \frac{4\mu U_{\text{max}}^2}{B^4} x^2 \nonumber\\
      0 &= \boxed{k \frac{d^2 T}{d x^2} + \frac{4\mu U_{\text{max}}^2}{B^4} x^2 } \label{eq:simple_graetz_ODE}
    \end{align}
    \item 邊界條件:
    \begin{align}
      T(B) &= T_0 \label{eq:simple_graetz_bc1}\\
      \frac{dT}{dx}\bigg|_{x=0} &= 0,\quad(\text{對稱連續}) \label{eq:simple_graetz_bc2}
    \end{align}
    \item 移項(\ref{eq:simple_graetz_ODE}),並積分:
    \begin{align}
      \frac{d^2 T}{d x^2} &= -\frac{4\mu U_{\text{max}}^2}{k B^4} x^2 \nonumber\\
      \frac{d T}{d x} &= -\frac{4\mu U_{\text{max}}^2}{k B^4} \cdot \frac{x^3}{3} + C_1 
    \end{align}
    \item 代入(\ref{eq:simple_graetz_bc2})求$C_1$:
    \begin{equation}
      0 = -\frac{4\mu U_{\text{max}}^2}{k B^4} \cdot \frac{0^3}{3} + C_1 \Rightarrow C_1 = 0
    \end{equation}
    \item 再次積分:
    \begin{align}
      T(x) &= -\frac{4\mu U_{\text{max}}^2}{k B^4} \cdot \frac{x^4}{12} + C_2 \nonumber\\
      &= -\frac{\mu U_{\text{max}}^2}{3 k B^4} x^4 + C_2
    \end{align}
    \item 代入(\ref{eq:simple_graetz_bc1})求$C_2$:
    \begin{equation}
      T_0 = -\frac{\mu U_{\text{max}}^2}{3 k B^4} B^4 + C_2 \Rightarrow C_2 = T_0 + \frac{\mu U_{\text{max}}^2}{3 k} 
    \end{equation}
    \item 最後解為:
    \begin{equation}
      \boxed{T(x) = T_0 + \frac{\mu U_{\text{max}}^2}{3 k} \left(
        1 - \frac{x^4}{B^4}
      \right)}
    \end{equation}
    \item 可看出溫度最大值發生在$x=0$
    \begin{equation}
      T_{\text{max}} = T(x=0) = T_0 + \frac{\mu U_{\text{max}}^2}{3 k}
    \end{equation}
    \item 而將$T_0$移到等號左邊,變成$\Delta T$,然後再同除$\Delta T$,就會出現無因次群
    \begin{equation}
      \Delta T = \frac{\mu U_{\text{max}}^2}{3 k} \implies \frac{\mu U_{\text{max}}^2}{k \Delta T} = 3
    \end{equation}
    Brinkman number:
    \begin{equation}
      \boxed{\text{Br} = \frac{\mu U_{\text{max}}^2}{k \Delta T} = \frac{\text{黏性耗散產生的熱量}}{\text{導熱能力}}}
    \end{equation}
  \end{enumerate}
  \item Graetz Problem,同樣情境,一個泡在冷卻液中的金屬管\\
  但我想要知道我需要設計多長的金屬管,才會使溫度從$T_0$下降到$T_1$\\
  也就是說,$T$不只是$r$的函數,還是$z$的函數\\
  而這時,我們先假設$Br\ll 1$,因為是個金屬管,導熱能力很好,$\overline{\overline{\bm \tau}}:\nabla \vec{\bm v}$可以忽略\\
  P.S.上一題不能忽略,不然就變成固體散熱了($k\nabla^2T = 0$)\\
  而上一題也能想成是熱交換平衡後,流體內的溫度分布
  \begin{figure}[H]
    \centering
    \begin{tikzpicture}[>=Latex, line cap=round, line join=round, thick]
      \draw (-4,8) -- (-3,8);
      \draw (-3,7.5) rectangle (-2,8.5);
      \draw (-2,8) -- (0,8) -- (0, 7.5);
      \draw [decorate, decoration={coil,aspect=0.3,segment length=1.5mm,amplitude=3mm}] (0,7.5) -- (0,5.25);
      \draw (0,5.25)--(0,5) -- (3,5) -- (3,8) -- (5,8); 
      \draw (-0.5,7.5) -- (-0.5,4.5) -- (3.5,4.5) -- (3.5, 7.5);
      \fill[pattern=crosshatch dots, pattern color=blue] (-0.5,4.5) rectangle (3.5,7);
      \draw [blue, dashed] (-0.5,7) -- (3.5,7);
      \node[anchor=south] at (-0.5,8) {$T_0$};
      \node[anchor=south] at (4,8) {$\approx T_1$};
      \node[blue, anchor=south] at (1.5, 7.5) {$T_1$};
      \node at (-2.5, 8) {裝置};
      \draw[->] (0,1) -- (0,4) node[above] {$r$};
      \draw[dashed] (0,-1.5) -- (0,1);
      \draw[->] (0,3.5) -- (1,3.5) node[right] {$z$};
      \draw (-4,-1) -- (6,-1);
      \draw (-4,3) -- (6,3);
      \draw[dashed] (-4,1) -- (6,1);
      \draw [<->] (-0.3, 1) -- (-0.3,3) node[midway, left] {$R$};
      \fill[pattern=north east lines, draw=black] (-4,3) rectangle (0,3.2);
      \fill[pattern=north east lines, draw=black] (-4,-1) rectangle (0,-1.2);
      \fill[pattern=north west lines, pattern color=blue, draw=blue] (0,3) rectangle (6,3.2);
      \fill[pattern=north west lines, pattern color=blue, draw=blue] (0,-1) rectangle (6,-1.2);
      \node[anchor=south east] at (0,3.2) {$T_0$};
      \node[anchor=south east] at (6,3.2) {$T_1$};
      \draw[dashed, domain=-1:3, smooth, variable=\y] plot ({-0.25*(\y-1)^2 + 1}, \y);
      \foreach \y in {-0.75, -0.25, 0.25,0.75,1.25, 1.75, 2.25, 2.75} {
        \draw[->, dashed] (0, \y) -- ({-0.25*(\y-1)^2 + 1}, \y);
      }
      \node[anchor=south west] at (1,1) {$2\overline{u_z}\left[1-\left(\frac{r}{R}\right)^2\right]$};
    \end{tikzpicture}
    \caption{Graetz 問題示意圖}
  \end{figure}
  \begin{enumerate}
    \item 假設流體進入熱交換前,溫度為均勻的$T_0$,流場已完全發展
    \begin{align}
      u_z(r) = 2\overline{u_z}\left[1-\left(\frac{r}{R}\right)^2\right] 
    \end{align}
    Steady State, $k$, $\rho$, $\hat C_p$皆為常數\\
    $\text{Br}\ll 1$,忽略Dissipation項
    \item 寫出熱平衡的方程式
    \begin{equation}
      \rho \hat C_p \left(
        \vec{\bm v} \cdot \nabla T
      \right) = k \nabla^2 T
    \end{equation}
    展開我們已知的項,$T(r,z)$ only
    \begin{align}
      \rho \hat C_p \left(
        u_z \frac{\partial T}{\partial z}
      \right) &= k \left( \frac{1}{r} \frac{\partial}{\partial r} \left( r \frac{\partial T}{\partial r} \right)
      + \frac{\partial^2 T}{\partial z^2}\right) \nonumber\\
      2 \rho \hat C_p \left(
        \overline{u_z}\left[1-\left(\frac{r}{R}\right)^2\right] \frac{\partial T}{\partial z}
      \right) &= k \left[
        \frac{1}{r} \frac{\partial}{\partial r} \left( r \frac{\partial T}{\partial r} \right)+
        \frac{\partial^2 T}{\partial z^2}
        \right] \label{eq:graetz_heat_equation}
    \end{align}
    \item 邊界條件:
    \begin{align}
      T(R,z<0) &= T_0 \label{eq:graetz_bc1}\\
      T(R,z\ge0) &= T_1 \label{eq:graetz_bc2}\\
      \frac{\partial T}{\partial r}(0,z) &= 0 \quad (\text{對稱連續}) \label{eq:graetz_bc3}\\
      T(r,0+) &= T(r,0-) \quad (\text{入口連續}) \label{eq:graetz_bc4} \\
      \frac{\partial T}{\partial z}(r,0+) &= \frac{\partial T}{\partial z}(r,0-) \quad (\text{入口連續}) \label{eq:graetz_bc5}
    \end{align}
    P.S. 最後一個邊界條件,又稱為Danckwerts Boundary Condition
    \item 無因次化(\ref{eq:graetz_heat_equation}),並定義無因次群\\
    溫度很明顯會用$T_0,T_1$作為特徵溫度:
    \begin{equation}
      \theta = \frac{T - T_1}{T_0 - T_1}
    \end{equation}
    長度方面,$R$是橫向特徵長度
    \begin{equation}
      r^\ast = \frac{r}{R}
    \end{equation}
    至於$z$的方向,則觀察(\ref{eq:graetz_heat_equation})的常數項係數\\
    左邊忽視$\left(1-\frac{r^2}{R^2}\right)$,丟到右邊後會有
    \begin{equation}
      \frac{k}{2 \rho \hat C_p \overline{u_z}}
    \end{equation}
    再分母再補上一個$R$,就會變成無因次的\fbox{Peclet Number}的倒數
    \begin{equation}
       \boxed{\text{Pe} = \frac{2\rho \hat C_P\left<u_z\right>R}{k} = \frac{\text{軸向熱對流}}{\text{熱傳導}}}
    \end{equation}
    而我們再補上一個$R$,就能讓單位變成長度,作為$z$的特徵長度
    \begin{equation}
      \boxed{z^\ast = \frac{z}{R} \cdot\frac{1}{\text{Pe}} = \frac{z k}{2 \rho \hat C_P \left<u_z\right> R^2}}
    \end{equation}
    P.S. Peclet Number, Reynolds Number, Prandtl Number的關係為
    \begin{equation}
      \text{Pe} = \text{Re} \cdot \text{Pr}
    \end{equation}
    \item 最後將(\ref{eq:graetz_heat_equation})無因次化:
    \begin{align}
      2\rho \hat C_P \overline{u_z} \left[
        1 - (r^\ast)^2
      \right] \frac{T_0 - T_1}{R\text{Pe}} \frac{\partial \theta}{\partial z^\ast} &= k \left[
        \frac{1}{r^\ast R} \frac{\partial}{\partial r^\ast} \left(
          r^\ast \frac{T_0 - T_1}{R} \frac{\partial \theta}{\partial r^\ast}
        \right) + \frac{T_0 - T_1}{R^2\text{Pe}^2} \frac{\partial^2 \theta}{\partial z^{\ast 2}}
      \right] \nonumber\\
      \div (T_0 - T_1) \Rightarrow 2\rho \hat C_P \overline{u_z} \left[
        1 - (r^\ast)^2
      \right] \frac{1}{R\text{Pe}}\frac{\partial \theta}{\partial z^\ast} &= \frac{k}{R^2} \left[
        \frac{1}{r^\ast} \frac{\partial}{\partial r^\ast} \left(
          r^\ast \frac{\partial \theta}{\partial r^\ast}
        \right) + \frac{1}{\text{Pe}^2} \frac{\partial^2 \theta}{\partial z^{\ast 2}}
      \right] \nonumber\\
      \Rightarrow  \left[
        1 - (r^\ast)^2
      \right] \frac{1}{\cancel{R}\text{Pe}}\frac{\partial \theta}{\partial z^\ast} &= \frac{k}{2\rho \hat C_P \overline{u_z} R}\frac{1}{\cancel{R}} \left[
        \frac{1}{r^\ast} \frac{\partial}{\partial r^\ast} \left(
          r^\ast \frac{\partial \theta}{\partial r^\ast}
        \right) + \frac{1}{\text{Pe}^2} \frac{\partial^2 \theta}{\partial z^{\ast 2}}
      \right] \nonumber\\
      \Rightarrow  \left[
        1 - (r^\ast)^2
      \right]\frac{1}{\text{Pe}} \frac{\partial \theta}{\partial z^\ast} &= \frac{1}{\text{Pe}} \left[
        \frac{1}{r^\ast} \frac{\partial}{\partial r^\ast} \left(
          r^\ast \frac{\partial \theta}{\partial r^\ast}
        \right) + \frac{1}{\text{Pe}^2} \frac{\partial^2 \theta}{\partial z^{\ast 2}}
      \right]  \nonumber\\
      \div \frac{1}{\text{Pe}}\Rightarrow  \left[
        1 - (r^\ast)^2
      \right] \frac{\partial \theta}{\partial z^\ast} &= \left[
        \frac{1}{r^\ast} \frac{\partial}{\partial r^\ast} \left(
          r^\ast \frac{\partial \theta}{\partial r^\ast}
        \right) + \frac{1}{\text{Pe}^2} \frac{\partial^2 \theta}{\partial z^{\ast 2}}
      \right]   \label{eq:graetz_nondim_heat_equation}
    \end{align}
    由於$\text{Pe}=\text{Re}\cdot \text{Pr}$,所以假設流速很快,所以$\text{Re}\gg 1$\\
    而且液體的$\text{Pr}$通常在10左右,因此$\text{Pe}\gg 1$,可以忽略最後一項
    \begin{equation}
      \boxed{\left[
        1 - (r^\ast)^2
      \right] \frac{\partial \theta}{\partial z^\ast} = \left[
        \frac{1}{r^\ast} \frac{\partial}{\partial r^\ast} \left(
          r^\ast \frac{\partial \theta}{\partial r^\ast}
        \right)
      \right]}   \label{eq:graetz_simplified_nondim_heat_equation}
    \end{equation}
    \item 邊界條件無因次化(省略那些$z<0$的部分):
    \begin{align}
      \theta(r^\ast,0) &= 1 \label{eq:graetz_nondim_bc1}\\
      \theta(1,z^\ast) &= 0 \label{eq:graetz_nondim_bc2}\\
      \frac{\partial \theta}{\partial r^\ast}(0,z^\ast) &= 0 \quad (\text{對稱連續}) \label{eq:graetz_nondim_bc3}
    \end{align}
    可以觀察到(\ref{eq:graetz_nondim_bc2})和(\ref{eq:graetz_nondim_bc3})都是齊次邊界條件而且與$z^\ast$無關\\
    而(\ref{eq:graetz_nondim_bc1})則是非齊次邊界條件且與$z^\ast$有關\\
    所以這又是一個標準的S-L問題
    \item 分離變數,假設解為:
    \begin{equation}
      \theta(r^\ast, z^\ast) = R(r^\ast) Z(z^\ast)
    \end{equation}
    代入(\ref{eq:graetz_simplified_nondim_heat_equation}):
    \begin{align}
      \left[
        1 - (r^\ast)^2
      \right] R \frac{dZ}{d z^\ast} &= Z \left[
        \frac{1}{r^\ast} \frac{d}{d r^\ast} \left(
          r^\ast \frac{d R}{d r^\ast}
        \right)
      \right] \nonumber\\
      \Rightarrow \frac{1}{Z} \frac{dZ}{d z^\ast} &= \frac{1}{R} \cdot \frac{1}{\left[
        1 - (r^\ast)^2
      \right]} \left[
        \frac{1}{r^\ast} \frac{d}{d r^\ast} \left(
          r^\ast \frac{d R}{d r^\ast}
        \right)
      \right] = -\lambda^2
    \end{align}
    \item 兩邊分別得到兩個ODE:
    \begin{align}
      \frac{dZ}{d z^\ast} + \lambda^2 Z &= 0 \label{eq:graetz_z_ODE}\\
      \frac{d}{d r^\ast} \left(
          r^\ast \frac{d R}{d r^\ast}
        \right) + \left[
        1 - (r^\ast)^2
      \right] r^\ast \lambda^2 R &= 0 \label{eq:graetz_r_ODE}
    \end{align}
    \item 解(\ref{eq:graetz_z_ODE}):
    \begin{equation}
      Z(z^\ast) = C e^{-\lambda^2 z^\ast}
    \end{equation}
    \item 解(\ref{eq:graetz_r_ODE})
    \begin{equation}
      \frac{d}{d r^\ast} \left(
          r^\ast \frac{d R}{d r^\ast}
        \right) + \left[
        1 - (r^\ast)^2
      \right] r^\ast \lambda^2 R = 0
    \end{equation}
    寫成S-L形式:
    \begin{equation}
      \frac{d}{dx}\left[
        p(x)\frac{dy}{dx}
      \right] + q(x)y + \lambda^\ast r(x) y = 0
    \end{equation}
    比較係數可得到
    \begin{equation}
      p(r^\ast) = r^\ast,\quad q(r^\ast) = 0, r(r^\ast) = \left[
        1 - (r^\ast)^2
      \right] r^\ast,\quad \lambda^\ast = \lambda^2
    \end{equation}
    因此通解可以寫成:
    \begin{align}
      \theta(r^\ast, z^\ast) = Ce^{-\lambda^2 z^\ast}\left[
        AR(r^\ast; \lambda) + BS(r^\ast; \lambda)
      \right]
    \end{align}
    其中這些解我們稱為Graetz EigenFunctions,若對其微分\\
    S會發散,R會收斂
    \item 代入第三個邊界條件(\ref{eq:graetz_nondim_bc3}),可得$B=0$\\
    合併$A,C$為一常數$D$:
    \begin{equation}
      \theta(r^\ast, z^\ast) = D e^{-\lambda^2 z^\ast} R(r^\ast; \lambda)
    \end{equation}
    \item 代入第二個邊界條件(\ref{eq:graetz_nondim_bc2}),可得特徵基底:\\
    因為$z^\ast$不影響邊界條件,因此要令$R(1;\lambda) = 0$
    \begin{equation}
      R(1; \lambda) = 0 \Rightarrow \lambda_n\implies \Phi_n = R(r^\ast; \lambda_n)
    \end{equation}
    這一步驟的解,就如之前畫$\cot$和$ax$的交點一樣,是無法求出數值表示的\\
    而如果可以表示出數值,就跟Fourier Series一樣\\
    $R(1;\lambda n)=0=cos\theta$的解為$\theta_n = (2n-1)\frac{\pi}{2}$\\
    這一步驟就是在做這件事情而已\\
    這時解為:
    \begin{equation}
      \theta_n(r^\ast, z^\ast) = \sum_{n=1}^{\infty} D_n e^{-\lambda_n^2 z^\ast} R(r^\ast; \lambda_n)
    \end{equation}
    \item 最後代入第一個邊界條件(\ref{eq:graetz_nondim_bc1}),利用正交性求出$D_n$:
    \begin{align}
      1 = \theta(r^\ast,0) &= \sum_{n=1}^{\infty} D_n R(r^\ast; \lambda_n) \nonumber\\
      D_n&=\frac{
        \int f(x)r(x)\phi_n(x)dx
      }{\int r(x)\phi_n^2(x) dx} \nonumber\\
      &= \frac{\int_0^1 1 \cdot \left[
        1 - (r^\ast)^2
      \right] r^\ast R(r^\ast; \lambda_n) dr^\ast}{\int_0^1 \left[
        1 - (r^\ast)^2
      \right] r^\ast R^2(r^\ast; \lambda_n) dr^\ast}
    \end{align}
    而由於$R(r^\ast; \lambda_n)$是被Normalized過的,因此分母為1\\
    最後解為:
    \begin{equation}
      \boxed{\theta(r^\ast, z^\ast) = \sum_{n=1}^{\infty} \left[
        \int_0^1 \left(
          1 - (r^\ast)^2
        \right) r^\ast R(r^\ast; \lambda_n) dr^\ast
      \right] e^{-\lambda_n^2 z^\ast} R(r^\ast; \lambda_n)}
    \end{equation}
    \item 用Adiabatic Mixing Cup的溫度($T_{\text{AMC}}$)來找到$\lambda_n$\\
    實驗上,其實就是找到任一個截面的平均溫度,利用讓流體流進一個絕熱的杯子\\
    等流體在裡面熱平衡後,測量溫度,就是平均溫度\\
    不過這裡的\fbox{截面平均溫度是加權除過流速}的,因為流速快的部分,進去杯子的量也多\\
    而以數學上來寫,則為
    \begin{align}
      T_{\text{AMC}}(z) &= \frac{4\pi \overline u_z \int_0^R T(r,z) \left[
        1 - \left(\frac{r}{R}\right)^2
      \right] r dr}{
        \overline u_z \cdot \pi R^2
      } \nonumber\\
      &= \frac{4 \int_0^R T(r,z) \left[
        1 - \left(\frac{r}{R}\right)^2
      \right] r dr}{R^2} \nonumber\\
      \theta_{\text{AMC}} &= \frac{T_{\text{AMC}}-T_1}{T_0-T_1}\nonumber\\
       &= 4\int_0^1 \theta(r^\ast, z^\ast) \left[
        1 - (r^\ast)^2
      \right] r^\ast dr^\ast \nonumber\\
      &= 4 \sum_{n=1}^{\infty} D_n e^{-\lambda_n^2 z^\ast} \underbrace{\left[
        \int_0^1 \left(
          1 - (r^\ast)^2
        \right) r^\ast R(r^\ast; \lambda_n) dr^\ast
      \right]}_{=D_n} \nonumber\\
      &= 4 \sum_{n=1}^{\infty} D_n^2 e^{-\lambda_n^2 z^\ast} \label{eq:graetz_amc_temperature}
    \end{align}
    進一步令$4D_n^2 = M_n$,則(\ref{eq:graetz_amc_temperature})變成
    \begin{equation}
      \boxed{\theta_{\text{AMC}} = \sum_{n=1}^{\infty} M_n e^{-\lambda_n^2 z^\ast}}
    \end{equation}
    也就是說,我們跟換不同的$\lambda_n$,就能測量出不同的$\theta_{\text{AMC}}$\\
    進而反推出$M_n$,或者說$D_n$,最後就能得到完整的溫度分布\\
    而數值解提供了我們一系列的$\lambda_n,M_n$
    \begin{table}[H]
      \centering
      \begin{tabular}{c|c|c}
        $n$ & $\lambda_n$ & $M_n$ \\
        \hline
        1 & 2.704 & 0.819\\
        2 & 6.679 & 0.098\\
        3 & 10.673 & 0.033\\
        4 & 14.671 & 0.015\\
        5 & 18.669 & 0.009
      \end{tabular}
      \caption{Graetz Problem 特徵值與係數}
    \end{table}
    也就是說,假設我們希望$\theta_{\text{AMC}}=0.1$,也就是100度被0度淨泡冷卻到10度\\
    那我們可以算出:
    \begin{equation}
      \theta_{\text{AMC}} = 0.1 = 0.819 e^{-2.704^2 z^\ast} + 0.098 e^{-6.679^2 z^\ast}\cdots
    \end{equation}
    假設只取第一項就好,那就能算出$z^\ast\approx 0.2876$\\
    代回去$z^\ast$的定義,就能算出需要多長的管子
    \begin{equation}
      L = z^\ast \cdot \frac{2 \rho \hat C_P \overline{u_z} R^2}{k} = 
      0.2876 \cdot \frac{2 \rho \hat C_P \overline{u_z} R^2}{k} = 0.2876 R \text{Pe}
    \end{equation}
    如果以管子直徑($D=2R$)來當作特徵長度,則
    \begin{align}
      \boxed{\frac{L}{D} = 0.144 \text{Pe}}
    \end{align}
    也就是說,假設今天Pe算出來是1000,那我們需要的管長就是$L/D = 144$倍管直徑\\
    但這個解析解只能用在想要知道到達某個溫度所需長度的情況\\
    如果想要知道整個的溫度分布$\theta(z^\ast)$,換句話說,可能在靠近入口處\\
    $\theta_{\text{AMC}}=0.99$,之類的,\\
    就會需要接近無窮多項的數值解才能逼近的算出該位置($z^\ast$)\\
    因此這樣的近似是不夠好的\\
    這原因也能從我們的邊界條件看出\\
    因為我們既要求$\theta(1,z^\ast)=0$,又要求$\theta(r^\ast,0)=1$\\
    這\fbox{兩個條件在交界處($r^\ast=1,z^\ast=0$)會產生不連續點}\\
    因此需要無窮多項的解才能逼近這個不連續點\\
    這也是為什麼要介紹下面章節的,Leveque的方法\\
    但在這之前,還有一點要介紹
    \item 以巨觀下的熱傳係數$h$、Nusselt Number來描述Graetz Problem\\
    可以將整根管子當成一個系統\\
    則能寫出巨觀的能量平衡式:
    \begin{align}
      \left<h\right> 2\pi RL (T_i - T_0) =&\rho \hat C_P 2\left<u_z\right> \int_0^R\left[
        1 - \left(\frac{r}{R}\right)^2
      \right]\left(T(r,L)-T^\dagger \right)2\pi r dr \nonumber\\
      &-\rho \hat C_P 2\left<u_z\right> \int_0^R\left[
        1 - \left(\frac{r}{R}\right)^2
      \right](T_0 - T^\dagger)2\pi r dr
    \end{align}
    其中$T_i$為流體進入管子的溫度,$T^\dagger$為假想的參考溫度,積分後會消掉\\
    第一項是在右端口計算流體繼續傳下去的熱量($T(r,L)$)\\
    因為只是管子離開交換區間,並不是流體離開管子\\
    第二項是流體進入管子時帶入的熱量($T(r,0)=T_0$)\\
    將上式同除$\pi$,並同除$k$,湊出Nusselt Number:
    \begin{equation}
      \text{Nu} = \frac{\left<h\right>L}{k}
    \end{equation}
    因此:
    \begin{align}
      \left<h\right> 2\pi RL (T_i - T_0) &=\rho \hat C_P 2\left<u_z\right> \int_0^R\left[
        1 - \left(\frac{r}{R}\right)^2
      \right]\left(T(r,L)-T^\dagger \right)2\pi r dr \nonumber\\
      &\phantom{=} -\rho \hat C_P 2\left<u_z\right> \int_0^R\left[
        1 - \left(\frac{r}{R}\right)^2
      \right](T_0 - T^\dagger)2\pi r dr \nonumber\\
      \left<h\right> 2\pi RL (T_i - T_0) &=4\pi\rho \hat C_P \left<u_z\right> \int_0^R\left[
        1 - \left(\frac{r}{R}\right)^2
      \right]\left(T(r,L)-T_0\right) r dr \nonumber\\
      \frac{2R\left<h\right>}{k}L (T_i - T_0) &=\frac{4\rho \hat C_P\left<u_z\right>}{k} \int_0^R\left[
        1 - \left(\frac{r}{R}\right)^2
      \right]\left(T(r,L)-T_0\right) r dr \nonumber\\
      \text{Nu} (T_i - T_0) &= \frac{4\rho \hat C_P\left<u_z\right>}{k}\frac{R^2}{L}\int_0^1\left[
        1 - (r^\ast)^2
      \right]\left(T(r^\ast,L)-T_0\right) r^\ast dr^\ast \nonumber\\
      &= \frac{4\rho \hat C_P\left<u_z\right> R^2}{k L} (T_i - T_0) \int_0^1\left[
        1 - (r^\ast)^2
      \right]\left(1-\theta(r^\ast, z^\ast) \right) r^\ast dr^\ast \nonumber\\
      \text{Nu} &= \frac{2 R\rho \hat C_P\left<u_z\right>}{k} \cdot \frac{2R}{L}\cdot
      \int_0^1\left(
        1 - (r^\ast)^2
      \right)\left(1-\theta(r^\ast, z^\ast) \right) r^\ast dr^\ast
    \end{align}
    這裡又出現了一個合併的無因次群\\
    就是Graetz Number:
    \begin{equation}
      \boxed{\text{Gz} = \text{Pe} \cdot \frac{2R}{L} = \frac{4 \rho \hat C_P \left<u_z\right> R^2}{k L}}
    \end{equation}
    而將$\theta(r^\ast, z^\ast)$,以平均的$\theta_{\text{AMC}}$來近似\\
    將積分積出可得:
    \begin{equation}
      \text{Nu} = \text{Gz} \left[
        \frac{1}{4} - \frac{1}{4} \sum_{n=1}^{\infty} M_n e^{-\lambda_n^2 z^\ast_L}
      \right]
    \end{equation}
    而若將$z^\ast_L$的定義代入:
    \begin{equation}
      z^\ast_L = \frac{L k}{2 \rho \hat C_P \left<u_z\right> R^2} = \frac{2}{\text{Gz}}
    \end{equation}
    最後可得:
    \begin{equation}
      \boxed{\text{Nu} = \text{Gz} \left[
        \frac{1}{4} - \frac{1}{4} \sum_{n=1}^{\infty} M_n e^{-\frac{2\lambda_n^2}{\text{Gz}}}
      \right]}
    \end{equation}
    從這裡更明顯能看出,隨著L越來越小\\
    Nusselt Number與$L$無關,但Graetz Number反比於$L$\\
    故需要更多項的級數相加才能讓等式成立
  \end{enumerate}
  \item Leveque 方法,用在Graetz Problem的入口區域
  \begin{figure}[H]
    \centering
    \begin{tikzpicture}[>=Latex, line cap=round, line join=round, thick]
      \fill[pattern=north east lines, draw=black] (-4,2) rectangle (0,2.2);
      \fill[pattern=north east lines, draw=black] (-4,-2) rectangle (0,-2.2);
      \fill[pattern=north west lines, pattern color=blue, draw=blue] (0,2) rectangle (4,2.2);
      \fill[pattern=north west lines, pattern color=blue, draw=blue] (0,-2) rectangle (4,-2.2);
      \draw (-4,2) -- (4,2);
      \draw (-4,-2) -- (4,-2);
      \draw [->] (0,-2) -- (0,-1) node[above] {$y$};
      \draw [dashed] (0, -3) -- (0,-2);
      \def\yShift{2}
      \draw [dashed] (0,0-\yShift) .. controls (1.5,0.5-\yShift) and (2.6, 0.6-\yShift) .. (4,0.7-\yShift);
      \fill [pattern=crosshatch dots] (0,0-\yShift) .. controls (1.5,0.5-\yShift) and (2.6, 0.6-\yShift) .. (4,0.7-\yShift) -- (4,0-\yShift) -- cycle;
      \draw [dashed] (0,0+\yShift) .. controls (1.5,-0.5+\yShift) and (2.6, -0.6+\yShift) .. (4,-0.7+\yShift);
      \fill [pattern=crosshatch dots] (0,0+\yShift) .. controls (1.5,-0.5+\yShift) and (2.6, -0.6+\yShift) .. (4,-0.7+\yShift) -- (4,0+\yShift) -- cycle;
      \node [anchor=south west] at (4,-2) {$\delta_T(z)$};
      \node at (2,0) {Thermal Boundary Layer};
      \node[anchor=north] at (4, 1.3) {$T_0$};
      \node[anchor=south] at (4, -1.3) {$T_0$};
      \node[anchor=north west] at (0, -2.2) {$T_1$};
      \node[anchor=north east] at (0, -2.2) {$T_0$};
      \draw[<->] (-1,-2) -- (-1,2) node[midway, left] {$2R$};
      \draw[->] (-4,2.7) -- (-3,2.7) node [right] {$z$};
    \end{tikzpicture}
    \caption{Leveque 方法示意圖}
  \end{figure}
  坐標軸改以$y=R-r$,而$y/R\ll 1$
  \begin{enumerate}
    \item 能量平衡式:
    \begin{align}
      &\rho \hat C_P \overline u_z(y)\left[
        1 -\left(\frac{r}{R}\right)^2
      \right]\frac{\partial T}{\partial z} = k\frac{1}{r}\frac{\partial}{\partial r}\left(r\frac{\partial T}{\partial r}\right) \nonumber\\
      \Rightarrow&\rho \hat C_P \overline u_z(y)\left[
        1 -\left(\frac{R-y}{R}\right)^2
      \right]\frac{\partial T}{\partial z} =
       k\frac{1}{R-y}\frac{\partial}{\partial y}\left((R-y)\frac{\partial T}{\partial y}\right)\nonumber\\
      (R-y\to R)\Rightarrow&\rho \hat C_P \overline u_z(y)\left[
        1-\frac{R^2}{R^2} + \frac{2Ry}{R^2} - \cancel{\frac{y^2}{R^2}}
      \right]\frac{\partial T}{\partial z} = k \frac{1}{R} \frac{\partial }{\partial y}\left(
        R\frac{\partial T}{\partial y}
      \right) \nonumber\\
      \Rightarrow & \rho \hat  C_P \overline u_z(y)\frac{2y}{R}\frac{\partial T}{\partial z}
      = k\frac{\partial^2 T}{\partial y^2}
    \end{align}
  \end{enumerate}
\end{itemize}
\end{CJK*}
\end{document}