\documentclass[../main.tex]{subfiles}
\begin{document}
\begin{CJK*}{UTF8}{bkai}
\subsection{熱交換器}
\begin{itemize}
  \item 利用熱傳導與熱對流的方式,使兩個\fbox{不混合}的流體\\
  將熱量由高溫傳到低溫,達到加熱或冷卻的目的\\
  定義$T_{H,i},T_{H,o},T_{C,i},T_{C,o}$分別是熱交換器中
  \begin{equation}
    \underbrace{T_{H,i}}_{\text{高溫進入溫度}},\quad
    \underbrace{T_{H,o}}_{\text{高溫離開溫度}},\quad
    \underbrace{T_{C,i}}_{\text{低溫進入溫度}},\quad
    \underbrace{T_{C,o}}_{\text{低溫離開溫度}}
  \end{equation}
  \item 解題方式有兩種:\\
  差別在於有沒有經驗或實驗數據來求解熱交換器的效率
  \begin{enumerate}
    \item Log Mean Temperature Difference(LMTD) Method\\
    ($T_{H,i},T_{H,o},T_{C,i},T_{C,o}$)的四個參數中,\fbox{要知道其中3個}才能使用
    \begin{equation}
      \text{LMTD} = \frac{\Delta T_1-\Delta T_2}{\ln\left(\frac{\Delta T_1}{\Delta T_2}\right)}
    \end{equation}
    注意:$1,2$指的是交換器的位置,無關高低溫
    \item $\varepsilon-\text{NTU}$ Mehod\\
    Effectiveness - Number of Transfer Units(NTU) Method\\
    只需知道$(T_{H,i},T_{C,i})$,也就是兩流體進入交換器的溫度\\
    配合\fbox{實驗或經驗圖表}找到效率(Effectiveness, $\varepsilon$),即可求得
  \end{enumerate}
  \item 依照結構可分為:
  \begin{enumerate}
    \item 套管式熱交換器 Double pipe heat exchangers\\
    (又稱 Single-pass heat exchangers)\\
    套管式熱交換器,依照流體流動方式可分為:
    \begin{enumerate}
      \item Counterflow heat exchanger(反向流熱交換器)
      \begin{figure}[H]
        \centering
        \begin{tikzpicture}[>=Latex, line cap=round, line join=round, thick]
          \draw(0,0) arc(-90:-270:0.5 and 1) -- (6,2) arc(90:270:0.5 and 1) -- cycle;
          \draw (6,1) ellipse (0.5 and 1);
          \draw (6,1) ellipse (0.2 and 0.4);
          \draw[dashed] (0,0.6) arc(-90:-270:0.2 and 0.4) -- (6,1.4) arc(90:270:0.2 and 0.4) -- cycle;
          \draw[dashed] (0,1.4) arc(90:-90:0.2 and 0.4);
          \draw[dashed] (0,2) arc(90:-90:0.5 and 1);
          \draw[->, red] (-2,1) -- (0,1);
          \node[anchor=east, red] at (-2,1) {$T_{H,i}$};
          \draw[->,red] (6,1) -- (8,1);
          \node[anchor=west, red] at (8,1) {$T_{H,o}$};
          \draw[->,blue] (0,1.7) -- (-2,1.7);
          \node[anchor=east, blue] at (-2,1.7) {$T_{C,o}$};
          \draw[->,blue] (8,1.7) -- (6,1.7);
          \node[anchor=west, blue] at (8,1.7) {$T_{C,i}$};
          \draw[ultra thick, ->, red] (3,1) -- (3,1.7) node[midway, right] {\large $\bm q$};
          \node[anchor=north] at (0,0) {$\mcirc{1}$};
          \node[anchor=north] at (6,0) {$\mcirc{2}$};
          \def\yShift{-5}
          \draw(0,\yShift+4) -- (0,\yShift) -- (6,\yShift) -- (6,\yShift+4);
          \draw[red, ->] (0,\yShift+3.7) -- (6,\yShift+2.5);
          \draw[blue, ->] (6,\yShift+0.5) -- (0, \yShift+2.3);
          \node[anchor=north] at (0,\yShift) {$\mcirc{1}$};
          \node[anchor=north] at (6,\yShift) {$\mcirc{2}$};
          \node[anchor=east, red] at (0,\yShift+3.7) {$T_{H,i}$};
          \node[anchor=east, blue] at (0,\yShift+2.3) {$T_{C,o}$};
          \node[anchor=west, red] at (6,\yShift+2.5) {$T_{H,o}$};
          \node[anchor=west, blue] at (6,\yShift+0.5) {$T_{C,i}$};
        \end{tikzpicture}
        \caption{Counterflow heat exchanger示意圖}
      \end{figure}
      \item Co-current flow heat exchanger(同向流熱交換器)
      \begin{figure}[H]
        \centering
        \begin{tikzpicture}[>=Latex, line cap=round, line join=round, thick]
          \draw(0,0) arc(-90:-270:0.5 and 1) -- (6,2) arc(90:270:0.5 and 1) -- cycle;
          \draw (6,1) ellipse (0.5 and 1);
          \draw (6,1) ellipse (0.2 and 0.4);
          \draw[dashed] (0,0.6) arc(-90:-270:0.2 and 0.4) -- (6,1.4) arc(90:270:0.2 and 0.4) -- cycle;
          \draw[dashed] (0,1.4) arc(90:-90:0.2 and 0.4);
          \draw[dashed] (0,2) arc(90:-90:0.5 and 1);
          \draw[->, red] (-2,1) -- (0,1);
          \node[anchor=east, red] at (-2,1) {$T_{H,i}$};
          \draw[->,red] (6,1) -- (8,1);
          \node[anchor=west, red] at (8,1) {$T_{H,o}$};
          \draw[->,blue] (-2,1.7) -- (0,1.7);
          \node[anchor=east, blue] at (-2,1.7) {$T_{C,i}$};
          \draw[<-,blue] (8,1.7) -- (6,1.7);
          \node[anchor=west, blue] at (8,1.7) {$T_{C,o}$};
          \draw[ultra thick, ->, red] (3,1) -- (3,1.7) node[midway, right] {\large $\bm q$};
          \node[anchor=north] at (0,0) {$\mcirc{1}$};
          \node[anchor=north] at (6,0) {$\mcirc{2}$};
          \def\yShift{-5}
          \draw(0,\yShift+4) -- (0,\yShift) -- (6,\yShift) -- (6,\yShift+4);
          \draw[red, ->] (0,\yShift+3.7) -- (6,\yShift+2.8);
          \draw[blue, ->] (0,\yShift+0.5) -- (6, \yShift+2.3);
          \node[anchor=north] at (0,\yShift) {$\mcirc{1}$};
          \node[anchor=north] at (6,\yShift) {$\mcirc{2}$};
          \node[anchor=east, red] at (0,\yShift+3.7) {$T_{H,i}$};
          \node[anchor=east, blue] at (0,\yShift+0.5) {$T_{C,i}$};
          \node[anchor=west, red] at (6,\yShift+2.8) {$T_{H,o}$};
          \node[anchor=west, blue] at (6,\yShift+2.3) {$T_{C,o}$};
        \end{tikzpicture}
        \caption{Co-current flow heat exchanger示意圖}
      \end{figure}
    \end{enumerate}
    \begin{equation}
      q = \dot m_H C_{p,H}\left(T_{H,i}-T_{H,o}\right) = \dot m_C C_{p,C}\left(T_{C,o}-T_{C,i}\right)
    \end{equation}
    而流向只影響$T_{H,o},T_{C,o},T_{H,i},T_{C,i}$的加減關係,適用相同式子
    \item 殼管式熱交換器 Shell-and-tube heat exchanger\\
    會使用(X-Y Shell and Tube Heat Exchanger)的方式來表示\\
    其中$X$代表Shell的數量,$Y$代表Tube的數量
    \begin{equation}
      q = \dot m_H C_{p,H}\left(T_{H,i}-T_{H,o}\right) = \dot m_C C_{p,C}\left(T_{C,o}-T_{C,i}\right)
    \end{equation}
    一樣的式子,但這裡熱傳的常數是被乘上一個修正因子Correction Factor(F)後的\\
    而$F$則會根據不同的Shell-and-tube heat exchanger有不同的(F-Y)關係圖\\
    關係圖上也會因為不同的$\frac{T_{H,i}-T_{H,o}}{T_{C,i}-T_{C,o}}$數值而有不同的線
  \end{enumerate}
  \item 解題方式有兩種:\\
  差別在於有沒有經驗或實驗數據來求解熱交換器的效率
  \begin{enumerate}
    \item Log Mean Temperature Difference(LMTD) Method\\
    ($T_{H,i},T_{H,o},T_{C,i},T_{C,o}$)的四個參數中,\fbox{要知道其中3個}才能使用
    舉例來說\\
    Double pipe heat exchanger,且為Counterflow heat exchanger,則:
    \begin{equation}
      \Delta T_1 = T_{H,i}-T_{C,o},\quad \Delta T_2 = T_{H,o}-T_{C,i}
    \end{equation}
    Double pipe heat exchanger,且為Co-current flow heat exchanger,則:
    \begin{equation}
      \Delta T_1 = T_{H,i}-T_{C,i},\quad \Delta T_2 = T_{H,o}-T_{C,o}
    \end{equation}
    而只要對Double pipe heat exchanger,都適用:
    \begin{equation}
      q = U\cdot A\cdot \text{LMTD}
    \end{equation}
    而對Shell-and-tube heat exchanger:
    \begin{equation}
      q = F_G\cdot UA \text{LMTD}
    \end{equation}
    有一種特殊情況是,當$\Delta T_1 = \Delta T_2$時,這時$\text{LMTD}$會變成$\frac{0}{0}$的未定義狀態\\
    這時要使用L'Hopital's Rule
    \begin{equation}
      \text{LMTD} = \frac{\Delta T_1-\Delta T_2}{\ln\left(\frac{\Delta T_1}{\Delta T_2}\right)} = 
      \Delta T_2 \cdot \frac{\frac{\Delta T_1}{\Delta T_2}-1}{\ln\left(\frac{\Delta T_1}{\Delta T_2}\right)}
    \end{equation}
    令$\frac{\Delta T_1}{\Delta T_2} = x$,則根據L'Hopital's Rule:
    \begin{equation}
      \lim_{x\rightarrow 1}\left(\Delta T_2\cdot\frac{x-1}{\ln x} \right) =  \Delta T_2 x = \Delta \frac{T_1}{T_2} = \Delta T_1
    \end{equation}
    故此時$\Delta T_1 = \Delta T_2$
    \item $\varepsilon-\text{NTU}$ Mehod\\
    Effectiveness - Number of Transfer Units(NTU) Method\\
    只需知道$(T_{H,i},T_{C,i})$,也就是兩流體進入交換器的溫度\\
    配合\fbox{實驗或經驗圖表}找到效率(Effectiveness, $\varepsilon$),即可求得\\
    $\varepsilon$的定義:
    \begin{equation}
      \varepsilon = \frac{\text{熱交換器的真實熱傳速率}}{\text{具有無限大熱傳面積的熱傳速率}}
    \end{equation}
    $\varepsilon$會是$f(\text{NTU},\frac{C_\text{min}}{C_\text{max}})$的函數\\
    定義NTU:
    \begin{equation}
      \text{NTU}=\frac{UA}{C_\text{min}}
    \end{equation}
    故在$A=\infty$時,會有最大的
    \begin{equation}
      q(A=\infty) = C_{\text{min}}\left(T_{H,i}-T_{C,i}\right) 
    \end{equation}
    其中$C_\text{min}$是兩個介質中的Minimum heat capacity\\
    也就是:
    \begin{equation}
      \begin{cases}
         C_H > C_C \Rightarrow C_H=C_\text{max},~C_C=C_\text{min} \\
         C_C > C_H \Rightarrow C_C=C_\text{max},~C_H=C_\text{min}
      \end{cases}
    \end{equation}
    得到:
    \begin{equation}
      q_{\text{act}}= \varepsilon\cdot C_{\text{min}}\left(T_{H,i}-T_{C,i}\right)
    \end{equation}
  \end{enumerate}
\end{itemize}
\end{CJK*}
\end{document}

