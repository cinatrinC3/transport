\documentclass[../main.tex]{subfiles}
\begin{document}
\begin{CJK*}{UTF8}{bkai}
\subsection{數學知識——向量、張量、純量}
\begin{itemize}
  \item 純量,\fbox{階數0}:\\
  只有大小,沒有方向的量,例如:溫度、壓力、密度、黏度
  \item 向量,\fbox{階數1}:\\
  有大小與方向的量,例如:速度、加速度、力\\
  當寫作$\vec {\bm u}_x$時,表示向量$\vec {\bm u}$在$x$方向的分量,代表$u_x \bm \delta_x$\\
  而當寫作$u_x$時,就是純量,代表向量$\vec {\bm u}$在$x$方向的大小\\
  向量$\vec {\bm u}$有兩種表式方式,可寫成分量形式:
  \begin{equation}
    \vec {\bm u} = u_x \bm \delta_x + u_y \bm \delta_y + u_z \bm \delta_z
  \end{equation}
  或者矩陣形式:
  \begin{equation}
    \vec {\bm u} = \begin{bmatrix}
      u_x \\
      u_y \\
      u_z 
    \end{bmatrix}
  \end{equation}
  \item 張量(化工只會碰到二階張量),\fbox{階數2}:\\
  有大小與方向的量,且有兩個方向,例如:應力張量、慣性張量\\
  例如 Shear stress: $\overline{\overline{\bm \tau}}$ (\ref{eq:shear_stress_tensor})
  \begin{equation}
    \overline{\overline{\bm \tau}} = \begin{bmatrix}
      \tau_{xx} & \tau_{xy} & \tau_{xz} \\
      \tau_{yx} & \tau_{yy} & \tau_{yz} \\
      \tau_{zx} & \tau_{zy} & \tau_{zz}
    \end{bmatrix}
  \end{equation}
  由於出於對稱性的考量,故定義張量的大小為
  \begin{equation}
    \left|\overline{\overline {\bm \tau}}\right| = \sqrt{\frac{1}{2}\left(
      \overline{\overline {\bm \tau}} : \overline{\overline {\bm \tau^\dagger}}  
    \right)} = \sqrt{\frac{1}{2}\sum_{i}\sum_{j} \tau_{ij}\tau_{ij}}
  \end{equation}uv
  並定義張量的跡(Trace)為:
  \begin{equation}
    \text{Tr}(\overline{\overline {\bm \tau}}) = \sum_{i} \tau_{ii} = \tau_{xx} + \tau_{yy} + \tau_{zz}
  \end{equation}
  \item 量之間有四種乘法:
  \begin{itemize}
    \item 內積(Dot product),\fbox{總階數下降2}:\\
    向量降階為純量
    \begin{equation}
      \vec {\bm v} \cdot \vec {\bm w} = |\vec {\bm v}||\vec {\bm w}|\cos\theta
    \end{equation}
    張量內積向量降階為向量
    \begin{equation}
      \overline{\overline {\bm \tau}} \cdot \vec {\bm u} = 
      \begin{bmatrix}
        \tau_{xx}u_x + \tau_{xy}u_y + \tau_{xz}u_z\\
        \tau_{yx}u_x + \tau_{yy}u_y + \tau_{yz}u_z \\
        tau_{zx}u_x + \tau_{zy}u_y + \tau_{zz}u_z
      \end{bmatrix}
    \end{equation}
    或寫成單位向量形式
    \begin{align}
      \overline{\overline {\bm \tau}} \cdot \vec {\bm u} =&\phantom{+} (\tau_{xx}u_x + \tau_{xy}u_y + \tau_{xz}u_z) \bm \delta_x \nonumber \\
      &+ (\tau_{yx}u_x + \tau_{yy}u_y + \tau_{yz}u_z) \bm \delta_y \nonumber \\
      &+ (tau_{zx}u_x + tau_{zy}u_y + tau_{zz}u_z) \bm \delta_z
    \end{align}
    同個向量內積自己後開根號會等於該向量的大小
    \begin{equation}
      \sqrt{\vec {\bm u} \cdot \vec {\bm u}} = |\vec {\bm u}| \implies \boxed{\vec {\bm u} \cdot \vec {\bm u} = |\vec {\bm u}|^2}
    \end{equation}
    \item 外積(Cross product),\fbox{總階數下降1}:\\
    向量外積向量仍為向量(1+1-1=1)
    \begin{equation}
      \vec {\bm u}_x \times \vec {\bm u}_y = |\vec {\bm u}_x||\vec {\bm u}_y|\sin\theta \cdot \bm \delta_z
    \end{equation}
    而座標軸則為右手定則下的方向\\
    同個向量外積自己會等於0
    \begin{equation}
      \vec {\bm u}_x \times \vec {\bm u}_x = 0
    \end{equation}
    \item 直積,\fbox{總階數直接相加}\\
    向量乘向量變張量(1+1=2)\\
    例如:Inertial Stress: $\rho \vec {\bm u} \vec {\bm u}$ (\ref{eq:inertial_stress_tensor})
    \begin{equation}
      \rho \vec {\bm u} \vec {\bm u} = \begin{bmatrix}
        \rho u_x u_x & \rho u_x u_y & \rho u_x u_z \\
        \rho u_y u_x & \rho u_y u_y & \rho u_y u_z \\
        \rho u_z u_x & \rho u_z u_y & \rho u_z u_z
      \end{bmatrix}
    \end{equation}
    或寫成單位向量形式
    \begin{align}
      \rho \vec {\bm u} \vec {\bm u} =&\phantom{+} \rho u_x u_x \bm \delta_x \bm \delta_x + \rho u_x u_y \bm \delta_x \bm \delta_y + \rho u_x u_z \bm \delta_x \bm \delta_z \nonumber \\
      &+ \rho u_y u_x \bm \delta_y \bm \delta_x + \rho u_y u_y \bm \delta_y \bm \delta_y + \rho u_y u_z \bm \delta_y \bm \delta_z \nonumber \\
      &+ \rho u_z u_x \bm \delta_z \bm \delta_x + \rho u_z u_y \bm \delta_z \bm \delta_y + \rho u_z u_z \bm \delta_z \bm \delta_z \label{eq:inertial_stress_tensor_unit}
    \end{align}
    \item Double dot product,\fbox{總階數下降4}:\\
    張量跟張量相乘降階至純量(2+2-4=0)
    \begin{equation}
      \overline{\overline {\bm \sigma}} : \overline{\overline {\bm \tau}} = \sum_{i}\sum_{j} \sigma_{ij}\tau_{ji} \label{eq:double_dot_product}
    \end{equation}
    簡單說就是把$3\times 3$的兩個矩陣後項轉置後每一項相乘後加總起來\\
    之所以要轉置,是讓非方陣也能作內積(其實平常向量內積,也會把一個轉過來)
  \end{itemize}
  \item 量之間的加減法,\fbox{階數必須相等}就能做加減法
  \item 量之間的交換律:
  \begin{itemize}
    \item 純量與純量直積,可交換,$rs = sr$
    \item 純量與向量直積,可交換,$\phi \vec {\bm v} = \vec {\bm v} \phi$
    \item 向量與向量內積,可交換,$\vec {\bm v} \cdot \vec {\bm w} = \vec {\bm w} \cdot \vec {\bm v}$
    \item \fbox{向量與向量外積不可交換},會差一個方向的負號
      \begin{equation}
        \boxed{\vec {\bm v} \times \vec {\bm w} = -(\vec {\bm w} \times \vec {\bm v})}
      \end{equation}
  \end{itemize}
  \item 量之間的分配律
  \begin{itemize}
    \item 純量與純量有分配律,$r(s+t) = rs + rt$
    \item 純量與向量有分配律,$\phi(\vec {\bm v} + \vec {\bm w}) = \phi \vec {\bm v} + \phi \vec {\bm w}$
    \item 向量與向量內積有分配律,$\vec {\bm v} \cdot (\vec {\bm w} + \vec {\bm C}) = \vec {\bm v} \cdot \vec {\bm w} + \vec {\bm v} \cdot \vec {\bm C}$
    \item 向量與向量外積有分配律,$(\vec {\bm v} + \vec {\bm w}) \times \vec {\bm C} = \vec {\bm v} \times \vec {\bm C} + \vec {\bm w} \times \vec {\bm C}$
  \end{itemize}
  \item 量之間的結合律
  \begin{itemize}
    \item 純量與純量有結合律,$r(st) = (rs)t$
    \item 純量與向量有結合律,$\phi(\psi \vec {\bm v}) = (\phi \psi)\vec {\bm v}$
    \item \fbox{向量與向量內積沒有結合律}
      \begin{equation}
        \boxed{\vec {\bm v} \cdot (\vec {\bm w} \cdot \vec {\bm C}) \neq (\vec {\bm v} \cdot \vec {\bm w}) \cdot \vec {\bm C}}
      \end{equation}
    \item \fbox{向量與向量外積沒有結合律}
      \begin{equation}
        \boxed{\vec {\bm v} \times (\vec {\bm w} \times \vec {\bm C}) \neq (\vec {\bm v} \times \vec {\bm w}) \times \vec {\bm C}}
      \end{equation}
  \end{itemize}
  \item 向量之間的關係:
  \begin{itemize}
    \item 向量三重積: ($1 + (-2) + (1 + 1 - 1) = 0$)\\
    會變純量,照循環順序$u\to v\to w$不可相反
    \begin{equation}
      \vec {\bm u} \cdot (\vec {\bm v} \times \vec {\bm w}) = \vec {\bm v} \cdot (\vec {\bm w} \times \vec {\bm u}) = \vec {\bm w} \cdot (\vec {\bm u} \times \vec {\bm v})
    \end{equation}
    \item 向量三重積: ($1+ (-1) + (1+ 1 -1) =1$)\\
    會變向量
    \begin{equation}
      \vec {\bm u} \times (\vec {\bm v} \times \vec {\bm w}) = \vec {\bm v} (\vec {\bm u} \cdot \vec {\bm w}) - \vec {\bm w} (\vec {\bm u} \cdot \vec {\bm v})
    \end{equation}
    \item 向量四重積: ($(1+(-1)+1) + (-2) + (1+(-1)+1) = 0$)\\
    會變純量
    \begin{equation}
      (\vec {\bm u} \times \vec {\bm v}) \cdot (\vec {\bm w} \times \vec {\bm s}) = (\vec {\bm u} \cdot \vec {\bm w})(\vec {\bm v} \cdot \vec {\bm s}) - (\vec {\bm u} \cdot \vec {\bm s})(\vec {\bm v} \cdot \vec {\bm w})
    \end{equation}
    \item 向量四重積: ($(1+(-1)+1) + (-1) + (1+(-1)+1) = 1$)\\
    會變向量
    \begin{equation}
      (\vec {\bm u} \times \vec {\bm v}) \times (\vec {\bm w} \times \vec {\bm s}) = (\vec {\bm u} \cdot (\vec {\bm v} \times \vec {\bm s}))\vec {\bm w} - (\vec {\bm u} \cdot (\vec {\bm v} \times \vec {\bm w}))\vec {\bm s}
    \end{equation}
  \end{itemize}
\end{itemize}
\subsection{數學知識——向量、張量、純量的微分運算}
\begin{itemize}
  \item $\nabla$算子:
    \begin{equation}
      \nabla = \frac{\partial }{\partial x}\bm \delta_x + \frac{\partial }{\partial y}\bm \delta_y + \frac{\partial }{\partial z}\bm \delta_z
    \end{equation}
    $\nabla$算子在\fbox{升階}的時候,會變成$\boxed{3\times 1}$的矩陣
    \begin{equation}
      \begin{bmatrix}
        \frac{\partial }{\partial x}\\
        \frac{\partial }{\partial y}\\
        \frac{\partial }{\partial z}
      \end{bmatrix}
    \end{equation}
    而\fbox{降階}的時候,會變成$\boxed{1\times 3}$的矩陣
    \begin{equation}
      \begin{bmatrix}
        \frac{\partial }{\partial x} & \frac{\partial }{\partial y} & \frac{\partial }{\partial z}
      \end{bmatrix}
    \end{equation}
    但在沒有升降階的旋度下,沒有矩陣表示法,只有以行列式值,與外積對象合併的表示法\\
    計算出行列式值後,以單位向量表示\\
    不管什麼座標都永遠不會變,只是在不同座標下,\\
    $\frac{\partial }{\partial x},\frac{\partial }{\partial y},\frac{\partial }{\partial z}$要換成對應的座標微分形式
    \begin{enumerate}
      \item Cartesian coordinate:
      \begin{equation}
        \nabla = \frac{\partial }{\partial x}\bm \delta_x + \frac{\partial }{\partial y}\bm \delta_y + \frac{\partial }{\partial z}\bm \delta_z
      \end{equation}
      \item Cylindrical coordinate:
      \begin{equation}
        \nabla = \frac{\partial }{\partial r}\bm \delta_r + \frac{1}{r}\frac{\partial }{\partial \theta}\bm \delta_{\theta} + \frac{\partial }{\partial z}\bm \delta_z
      \end{equation}
      \item Spherical coordinate:
      \begin{equation}
        \nabla = \frac{\partial }{\partial r}\bm \delta_r + \frac{1}{r}\frac{\partial }{\partial \theta}\bm \delta_{\theta} 
        + \frac{1}{r\sin\theta}\frac{\partial }{\partial \phi}\bm \delta_{\phi} 
      \end{equation}
    \end{enumerate}
  \item 向量有四種微分運算
    \begin{itemize}
      \item Gradient(梯度):\fbox{朝最大變化量移動的方向}\\
      \fbox{一次微分,升階},純量微分為向量
      \begin{equation}
        \nabla \phi = \frac{\partial \phi}{\partial x}\bm \delta_x + \frac{\partial \phi}{\partial y}\bm \delta_y + \frac{\partial \phi}{\partial z}\bm \delta_z
      \end{equation}
      \item Divergence(散度):\fbox{單位體積內流出的淨流量}\\
      \fbox{一次微分,降階},向量微分為純量\\
      純量無法做Divergence\\
      \fbox{向量的Divergence}:
      \begin{align}
        \nabla \cdot \vec {\bm v} &= \left(
          \frac{\partial }{\partial x}\bm \delta_x + \frac{\partial }{\partial y}\bm \delta_y + \frac{\partial }{\partial z}\bm \delta_z
        \right)\cdot \left(
          v_x \bm \delta_x + v_y \bm \delta_y + v_z \bm \delta_z
        \right) \nonumber \\
        &= \frac{\partial v_x}{\partial x} \cancel{\bm \delta_x \cdot \bm \delta_x} + 
        \frac{\partial v_y}{\partial y} \cancel{\bm \delta_y \cdot \bm \delta_y} + 
        \frac{\partial v_z}{\partial z} \cancel{\bm \delta_z \cdot \bm \delta_z} \nonumber \\  
        &= \frac{\partial v_x}{\partial x} + \frac{\partial v_y}{\partial y} + \frac{\partial v_z}{\partial z}
      \end{align}
      \fbox{張量的Divergence}:
      \begin{align}
        \nabla \cdot \overline{\overline {\bm \tau}} &= 
        \begin{bmatrix}
          \frac{\partial }{\partial x} & \frac{\partial }{\partial y} & \frac{\partial }{\partial z}
        \end{bmatrix} 
        \cdot \begin{bmatrix}
          \tau_{xx} & \tau_{xy} & \tau_{xz} \\
          \tau_{yx} & \tau_{yy} & \tau_{yz} \\
          \tau_{zx} & \tau_{zy} & \tau_{zz}
        \end{bmatrix} \nonumber \\
        &= \begin{bmatrix}
          \frac{\partial \tau_{xx}}{\partial x} + \frac{\partial \tau_{yx}}{\partial y} + \frac{\partial \tau_{zx}}{\partial z} \\
          \frac{\partial \tau_{xy}}{\partial x} + \frac{\partial \tau_{yy}}{\partial y} + \frac{\partial \tau_{zy}}{\partial z} \\
          \frac{\partial \tau_{xz}}{\partial x} + \frac{\partial \tau_{yz}}{\partial y} + \frac{\partial \tau_{zz}}{\partial z}
        \end{bmatrix}
      \end{align}
      或寫成單位向量形式:
      \begin{align}
        &\nabla \cdot \overline{\overline {\bm \tau}} \nonumber\\
        &=\left(
          \frac{\partial}{\partial x} \bm \delta_x + \frac{\partial}{\partial y} \bm \delta_y + \frac{\partial}{\partial z} \bm \delta_z
        \right) \cdot \left(
          \tau_{xx} \bm \delta_x \bm \delta_x + \tau_{xy} \bm \delta_x \bm \delta_y + \tau_{xz} \bm \delta_x \bm \delta_z \right. \nonumber \\
        &\phantom{=\left(
          \frac{\partial}{\partial x} \bm \delta_x + \frac{\partial}{\partial y} \bm \delta_y + \frac{\partial}{\partial z} \bm \delta_z
        \right) \cdot } + \bm \delta_y \bm \delta_x \tau_{yx} + \tau_{yy} \bm \delta_y \bm \delta_y + \tau_{yz} \bm \delta_y \bm \delta_z \nonumber \\
        &\phantom{=\left(
          \frac{\partial}{\partial x} \bm \delta_x + \frac{\partial}{\partial y} \bm \delta_y + \frac{\partial}{\partial z} \bm \delta_z
        \right) \cdot }\left. + \tau_{zx} \bm \delta_z \bm \delta_x + \tau_{zy} \bm \delta_z \bm \delta_y + \tau_{zz} \bm \delta_z \bm \delta_z\right) \nonumber \\
        \tiny &=\left(
          \frac{\partial \tau_{xx}}{\partial x} + \frac{\partial \tau_{yx}}{\partial y} + \frac{\partial \tau_{zx}}{\partial z}
        \right) \bm \delta_x + \left(
          \frac{\partial \tau_{xy}}{\partial x} + \frac{\partial \tau_{yy}}{\partial y} + \frac{\partial \tau_{zy}}{\partial z}
        \right) \bm \delta_y + \left(
          \frac{\partial \tau_{xz}}{\partial x} + \frac{\partial \tau_{yz}}{\partial y} + \frac{\partial \tau_{zz}}{\partial z}
        \right) \bm \delta_z
      \end{align}
      \item Laplacian: ($\nabla^2 = \nabla \cdot (\nabla)$)\\
      先做Gradient再做Divergence,純量操作完後仍為純量,向量操作完後仍為向量\\
      二次微分,階數不變
      \fbox{純量的Laplacian}:
      \begin{align}
        &\nabla^2\phi = \nabla \cdot (\nabla \phi) \nonumber \\
        &= \left(
          \frac{\partial }{\partial x}\bm \delta_x + \frac{\partial }{\partial y}\bm \delta_y + \frac{\partial }{\partial z}\bm \delta_z
        \right) \cdot \left(
          \frac{\partial \phi}{\partial x}\bm \delta_x + \frac{\partial \phi}{\partial y}\bm \delta_y + \frac{\partial \phi}{\partial z}\bm \delta_z
        \right) \nonumber \\
        &= \frac{\partial^2 \phi}{\partial x^2}\bm \delta_x \cdot \bm \delta_x 
        + \frac{\partial^2 \phi}{\partial y^2}\bm \delta_y \cdot \bm \delta_y
        + \frac{\partial^2 \phi}{\partial z^2}\bm \delta_z \cdot \bm \delta_z \nonumber \\
        &= \frac{\partial^2 \phi}{\partial x^2} + \frac{\partial^2 \phi}{\partial y^2} + \frac{\partial^2 \phi}{\partial z^2}
      \end{align}
      \fbox{向量的Laplacian}:
      \begin{align}
        \nabla^2 \vec {\bm u} &= \nabla \cdot (\nabla \vec {\bm u}) \nonumber \\
        &= \left(
          \frac{\partial }{\partial x}\bm \delta_x + \frac{\partial }{\partial y}\bm \delta_y + \frac{\partial }{\partial z}\bm \delta_z
        \right) \cdot \left(
          \frac{\partial u_x}{\partial x}\bm \delta_x + \frac{\partial u_y}{\partial y}\bm \delta_y + \frac{\partial u_z}{\partial z}\bm \delta_z
        \right) \nonumber \\
        &= \left(
          \frac{\partial^2 u_x}{\partial x^2} + \frac{\partial^2 u_x}{\partial y^2} + \frac{\partial^2 u_x}{\partial z^2}
        \right) \bm \delta_x + \left(
          \frac{\partial^2 u_y}{\partial x^2} + \frac{\partial^2 u_y}{\partial y^2} + \frac{\partial^2 u_y}{\partial z^2}
        \right) \bm \delta_y \nonumber \\
        &\phantom{=\frac{\partial^2 u_x}{\partial x^2} +} 
        + \left(
          \frac{\partial^2 u_z}{\partial x^2} + \frac{\partial^2 u_z}{\partial y^2} + \frac{\partial^2 u_z}{\partial z^2}
        \right) \bm \delta_z
      \end{align}
      \item Curl(旋度):\fbox{單位面積上的旋轉強度}\\
      \fbox{一次微分,階數不變},向量微分為向量
      \begin{equation}
        \nabla \times \vec {\bm u} = 
        \begin{vmatrix}
          \bm \delta_x & \bm \delta_y & \bm \delta_z \\
          \frac{\partial }{\partial x} & \frac{\partial }{\partial y} & \frac{\partial }{\partial z} \\
          u_x & u_y & u_z
        \end{vmatrix} \label{eq:curl_definition}
      \end{equation}
      行列式展開後:
      \begin{equation}
        \nabla \times \vec {\bm u} =\left(
          \frac{\partial u_z}{\partial y} - \frac{\partial u_y}{\partial z}
        \right) \bm \delta_x + \left(
          \frac{\partial u_x}{\partial z} - \frac{\partial u_z}{\partial x}
        \right) \bm \delta_y + \left(
          \frac{\partial u_y}{\partial x} - \frac{\partial u_x}{\partial y}
        \right) \bm \delta_z
      \end{equation}
    \end{itemize}
  \item 分配律,注意階數相等
    \begin{itemize}
      \item Divergence,$\nabla \cdot (\rho \vec {\bm u})$\\
        可看出$\nabla$會分別對每個括弧內變數作用,先觀察總階數\\
        括弧內為純量(階數0)乘向量(階數1),總階數為1\\
        而Divergence會使階數降1,最後結果為純量(階數0)
        \begin{equation}
          \nabla \cdot (\rho \vec {\bm u}), \quad \text{階數: -1 + (0 + 1) = 0}
        \end{equation}
        故分配進去每項階數數必須都為0\\
        提取$\rho$作用於$\vec {\bm u}$時,由於$\rho$(階數0),故要使$\vec {\bm u}$(階數1)降為0,故作Divergence
        \begin{equation}
          \rho (\nabla \cdot \vec {\bm u}), \quad \text{階數: 0 + (-1 + 1) = 0}
        \end{equation}
        提取$\vec {\bm u}$作用於$\rho$時,由於$\vec {\bm u}$(階數1),故需作\fbox{內積}來降階為0\\
        並且內積的對象要是階數1(1+1-2=0),故$\rho$(階數0)需作Gradient(升階1)來達成
        \begin{equation}
          \vec {\bm u} \cdot (\nabla \rho), \quad \text{階數: 1 -2 (+1 + 0) = 0}
        \end{equation}
        綜合以上兩項,得到分配律:
        \begin{equation}
          \nabla \cdot (\rho \vec {\bm u})  = \rho (\nabla \cdot \vec {\bm u}) + \vec {\bm u} \cdot (\nabla \rho)
        \end{equation}
      \item 推論牛頓黏度定律的向量表示法:$\tau_{yx} = -\mu \frac{\partial u_x}{\partial y}$\\
        觀察左邊$\tau$為張量(階數2)\\
        右邊$\mu$為純量(階數0)\\
        微分,故有$\nabla$算子\\
        $u$為向量(階數1)\\
        故需使$\nabla$算子與$u$結合後,階數變為2\\
        故為Gradient(升階1)與向量直積(階數1+1=2)的組合
        \begin{equation}
          \overline{\overline {\bm \tau}} = -\mu (\nabla \vec {\bm u})
        \end{equation}
        將等號兩邊同時做Divergence運算,得到
        \begin{equation}
          \nabla \cdot \overline{\overline {\bm \tau}} = -\nabla \cdot \left(\mu \nabla \vec {\bm u} \right)
        \end{equation}
        由於$\mu$為純量,故可提取出來
        \begin{equation}
          \nabla \cdot \overline{\overline {\bm \tau}} = -\mu \nabla \cdot( \nabla \vec {\bm u})
        \end{equation}
        而$\nabla \cdot \nabla$為Laplacian\\
        證明,當為牛頓流體,且黏度固定,流體不可壓縮時
        \begin{equation}
          -\nabla \cdot \overline{\overline {\bm \tau}} = \mu \nabla^2 \vec {\bm u}
        \end{equation}
        P.S. 不可壓縮的假設在一開始牛頓流體時就用了,否則牛頓流體定律應為:
        \begin{equation}
          \overline{\overline {\bm \tau}} = -\mu \left[ \nabla \vec {\bm u} + (\nabla \vec {\bm u})^T\right]
           + \left(
            \frac{2}{3}\mu - \kappa 
           \right)\left(
            \nabla \cdot \vec {\bm u}
           \right)\bm \delta           
        \end{equation}
        因為不可壓縮,故$\nabla \cdot \vec {\bm u} = 0$,故省略最後一項\\
        且$\nabla \vec {\bm u}$為對稱矩陣,故省略轉置項
    \end{itemize}
\end{itemize}
\subsection{單位向量的運算}
由上個章節可以看出,不同的張量、向量、純量做不同的乘除、合併都會有不同的計算方式\\
而若能讓數值的部分回歸數值,交給單位向量以及一些符號來處理向量、張量的運算\\
會讓計算上更為簡潔,也可以輕鬆證明一些向量、張量的恆等式
\begin{itemize}
  \item 單位向量與單位向量的運算:
    \begin{itemize}
      \item 單位向量的內積:\\
        單位向量表示方向,且互相垂直且大小為1,滿足以下關係:
          \begin{align}
            \bm \delta_x \cdot \bm \delta_y &= 0, \quad \bm \delta_x \cdot \bm \delta_x = 1 \nonumber \\
            \bm \delta_x \times \bm \delta_y &= \bm \delta_z, \quad \bm \delta_x \times \bm \delta_x = 0
          \end{align}
        引入Kronecker delta符號$\delta_{ij}$來表示單位向量內積關係:
        \begin{equation}
          \boxed{\bm \delta_i \cdot \bm \delta_j = \delta_{ij} = \begin{cases}
            1 & i=j \\
            0 & i \neq j
          \end{cases}}
        \end{equation}
      \item 單位向量的外積:\\
        引入Levi-Civita符號$\varepsilon_{ijk}$來表示單位向量外積關係:
        \begin{equation}
          \boxed{\bm \delta_i \times \bm \delta_j = \varepsilon_{ijk} \bm \delta_k}
        \end{equation}
        其中$\varepsilon_{ijk}$定義為:
        \begin{equation}
          \varepsilon_{ijk} = \begin{cases}
            +1 & (i,j,k)\text{為}(1,2,3),(2,3,1),(3,1,2)\text{的循環排列}\\
            -1 & (i,j,k)\text{為}(3,2,1),(1,3,2),(2,1,3)\text{的反循環排列}\\
            0 & \text{若有任兩個指標相同}
          \end{cases}
        \end{equation}
        如果要寫的不直觀的話,也可以寫成:
        \begin{equation}
          \varepsilon_{ijk} = \frac{1}{2}(i-j)(j-k)(k-i)
        \end{equation}
        例如$\varepsilon_{123} = \frac{1}{2}(1-2)(2-3)(3-1) = +1$\\
        例如$\varepsilon_{132} = \frac{1}{2}(1-3)(3-2)(2-1) = -1$\\
        例如$\varepsilon_{112} = \frac{1}{2}(1-1)(1-2)(2-1) = 0$
      \item 單位向量直積$\implies$二階張量,dyadic product:
        \begin{equation}
          \bm \delta_i \bm \delta_j \neq \bm \delta_j \bm \delta_i
        \end{equation}
        不代表任何意義上的乘法,只是表示方向的組合,只是一個二階張量的其中一項\\
        整體如同(\ref{eq:inertial_stress_tensor_unit})所示
        \begin{equation}
          \overline{\overline{\bm \tau}} = \sum_{i=1}^3\sum_{j=1}^3 \bm \delta_i \bm \delta_j\tau_{ij} 
        \end{equation}
      \item 單位張量:\\
        所謂\fbox{單位向量、張量}的想法是,\\
        任何非零向量、純量、張量,與單位向量、張量做運算後\\
        都不會改變原本向量、純量、張量的數值大小\\
        因此單位張量其實就是$3\times 3$的單位矩陣
        \begin{equation}
          \sum_{i=1}^3 \bm\delta_i \bm \delta_i
        \end{equation}
        但這樣太容易和內積搞混,畢竟粗體不好在非電腦媒介呈現\\
        故以Kronecker delta符號,配合兩個$\sum$號來表示
        \begin{equation}
          \boxed{\bm \delta = \sum_{i=1}^3\sum_{j=1}^3 \bm \delta_i \bm \delta_j \delta_{ij} }
        \end{equation}
        若張量是由兩個向量直積產生的,\fbox{則相乘的轉置等於相反的直積}
        \begin{equation}
          (\vec {\bm v} \vec {\bm w})^T = \vec {\bm w} \vec {\bm v}
        \end{equation}
      \item 單位張量與單位張量、向量的各種運算\\
      有個很簡單的規則,在不改變撰寫順序下,把張量拆開\\
      其中一邊去跟想跟張量運算內外積包在一起作運算。\\
      舉例來說:$\bm\delta_i\bm\delta_j \cdot \vec {\bm v}$\\
      把張量拆開成$\bm\delta_i$與$\bm\delta_j$兩部分\\
      並且把$\vec {\bm v}$跟他最靠近的$\bm\delta_j$包在一起作(不影響順序)
      \begin{equation}
        \bm\delta_i\bm\delta_j \cdot \vec {\bm v} = \bm\delta_i (\bm\delta_j \cdot \vec {\bm v})
      \end{equation}
      \begin{enumerate}
        \item 單位張量與單位張量作Double dot product:
          \begin{equation}
            \boxed{
              \left(\bm\delta_i\bm\delta_j\right):\left(\bm\delta_m\bm\delta_n\right) =
              \left(\bm\delta_j \cdot \bm\delta_k\right) \left(
                \bm\delta_i \cdot \bm\delta_n
              \right) = \delta_{jm}\delta_{in}
            }
          \end{equation}
          Proof:\\
          根據運算邏輯,Double dot product將可寫成:
          \begin{equation}
            \overline{\overline{\bm \sigma}} : \overline{\overline{\bm \tau}} = 
            \sum_{i=1}^3\sum_{j=1}^3 \sum_{m=1}^3 \sum_{n=1}^3
            \left(\bm \delta_i \bm \delta_j : \bm \delta_m \bm \delta_n\right)\sigma_{ij} \tau_{mn}
          \end{equation}
          根據Double dot product的定義,為兩張量各項內積後再加總(\ref{eq:double_dot_product})
          \begin{equation}
            \overline{\overline{\bm \sigma}} : \overline{\overline{\bm \tau}} =
            \sum_{i=1}^3\sum_{j=1}^3 \sigma_{ij} \tau_{ij}
          \end{equation}
          比較後可得出:
          \begin{equation}
            \bm \delta_i \bm \delta_j : \bm \delta_m \bm \delta_n = \delta_{jm}\delta_{in}
          \end{equation}
          又由於($\bm \delta_j \cdot \bm \delta_k = \delta_{jm}$)以及($\bm \delta_i \cdot \bm \delta_n = \delta_{in}$)\\
          故可得出:
          \begin{equation}
            \bm \delta_i \bm \delta_j : \bm \delta_m \bm \delta_n = 
            \left(\bm \delta_j \cdot \bm \delta_k\right) \left(
              \bm \delta_i \cdot \bm \delta_n
            \right) = \delta_{jm}\delta_{in}
          \end{equation}
          另外,也可以很上述很簡單的規則得到:
          \begin{align}
             \bm \delta_i \bm \delta_j : \bm \delta_m \bm \delta_n &=
              \bm \delta_i \left(
                \bm \delta_j : \bm \delta_m \bm \delta_n
              \right) \nonumber \\
              &= \bm \delta_i \left(
                \left(\bm \delta_j \cdot \bm \delta_m\right) \bm \delta_n
              \right) \nonumber \\
              &= \left(\bm \delta_j \cdot \bm \delta_m\right) \left(\bm \delta_i \cdot \bm \delta_n\right) \nonumber \\
              &= \delta_{jm}\delta_{in}
          \end{align}
        \item 單位張量與單位向量作Dot product:
          \begin{equation}
            \boxed{
              \left(\bm\delta_i\bm\delta_j\right) \cdot \bm \delta_k =
              \bm \delta_i \left(\bm\delta_j \cdot \bm \delta_k\right) = \bm \delta_i\delta_{jk}}
          \end{equation}
          反過來也可以
          \begin{equation}
            \boxed{
              \bm \delta_i \cdot \left(\bm\delta_j\bm\delta_k\right) =
              \left(\bm \delta_i \cdot \bm\delta_j\right) \bm \delta_k = \delta_{ij}\bm \delta_k}
          \end{equation}
        \item 單位張量與單位張量的內積
        \begin{equation}
          \boxed{
            \left(\bm\delta_i\bm\delta_j\right) \cdot \left(\bm\delta_m\bm\delta_n\right) =
            \left(\bm\delta_j \cdot \bm\delta_m\right) \left(\bm\delta_i \bm\delta_n\right) 
            =\sum_{i=1}^3 \delta_{jm}\bm \delta_i \bm \delta_n
          }
        \end{equation}
        \item 單位張量與單位向量的外積:
        \begin{equation}
          \boxed{
            \left(\bm\delta_i\bm\delta_j\right) \times \bm \delta_k =
            \bm \delta_i \times \left(\bm\delta_j \times \bm \delta_k\right) = 
            \sum_{i=1}^3\varepsilon_{jkl}\bm \delta_i \bm \delta_l
          }
        \end{equation}
      \end{enumerate}
      \item $\varepsilon_{ijk}$與$\delta_{ij}$的關係:\\
        有兩大定律,這兩個定律可以很快速的將一堆$\sum$號的式子減少
        \begin{enumerate}
          \item 兩個$\varepsilon$相乘,底數後面兩個一樣的話,可以寫成$\delta$的形式
            \begin{equation}
              \sum_{j=1}^3\sum_{k=1}^3 \varepsilon_{ijk}\varepsilon_{hjk} = 2\delta_{ih} \label{eq:epsilon_delta_relation}
            \end{equation}
            Proof:
            \begin{align}
              \sum_{j=1}^3\sum_{k=1}^3 \varepsilon_{ijk}\varepsilon_{hjk} &= 
              \cancel{\varepsilon_{i11}\varepsilon_{h11}} + \varepsilon_{i12}\varepsilon_{h12} + \varepsilon_{i13}\varepsilon_{h13} \nonumber \\
              &\quad + \varepsilon_{i21}\varepsilon_{h21} + \cancel{\varepsilon_{i22}\varepsilon_{h22}} + \varepsilon_{i23}\varepsilon_{h23} \nonumber \\
              &\quad + \varepsilon_{i31}\varepsilon_{h31} + \varepsilon_{i32}\varepsilon_{h32} + \cancel{\varepsilon_{i33}\varepsilon_{h33}} \nonumber \\
              &= \varepsilon_{i12}\varepsilon_{h12} + \varepsilon_{i13}\varepsilon_{h13} + \varepsilon_{i21}\varepsilon_{h21} \nonumber \\
              &\quad + \varepsilon_{i23}\varepsilon_{h23} + \varepsilon_{i31}\varepsilon_{h31} + \varepsilon_{i32}\varepsilon_{h32}
            \end{align}
            觀察可看出,當$i\neq h$時,由於每一個配對的後兩個指標都相同$\varepsilon_{i31}\varepsilon_{h31}$\\
            故一定有一項會重複,每一項都會是0\\
            更進一步地\\
            第一和第三項要求$i=h=3$才會有值,且一個為$1^2$,另一個為$(-1)^2$,相加後為2\\
            第二和第五項要求$i=h=2$才會有值,且一個為$1^2$,另一個為$(-1)^2$,相加後為2\\
            第四和第六項要求$i=h=1$才會有值,且一個為$1^2$,另一個為$(-1)^2$,相加後為2\\
            故總結可得:
            \begin{equation}
              \sum_{j=1}^3\sum_{k=1}^3 \varepsilon_{ijk}\varepsilon_{hjk} = 2\delta_{ih}
            \end{equation}
          \item 兩個$\varepsilon$相乘,底數最後一個一樣的話,可以寫成$\delta$的形式
            \begin{equation} 
              \boxed{\sum_{k=1}^3 \varepsilon_{ijk}\varepsilon_{mnk} = 
              \begin{vmatrix}
                \delta_{im} & \delta_{in} \\
                \delta_{jm} & \delta_{jn}
              \end{vmatrix} =\delta_{im}\delta_{jn} - \delta_{in}\delta_{jm}} \label{eq:epsilon_delta_relation_2}
            \end{equation}
            P.S. 行列式是\fbox{方便記憶},由於是($i,j$)乘上($m,n$)的組合\\
            故可想成($2\times 1$)乘($1\times 2$)來變成($2\times 2$)的矩陣,也就是
            \begin{equation}
              \begin{bmatrix}
                i \\ j
              \end{bmatrix}
              \begin{bmatrix}
                m & n
              \end{bmatrix}
              = \begin{bmatrix}
                im & in \\
                jm & jn
              \end{bmatrix}
            \end{equation}
            Proof:
            \begin{equation}
              \sum_{k=1}^3 \varepsilon_{ijk}\varepsilon_{mnk} = 
              \varepsilon_{ij1}\varepsilon_{mn1} + \varepsilon_{ij2}\varepsilon_{mn2} + \varepsilon_{ij3}\varepsilon_{mn3} 
            \end{equation}
            當$i,j$與$m,n$有一個不相同,則必定涵蓋了第三個數字\\
            由於每一項都強制了$i,j,m,n$須從兩數字中選出,故每一項皆為0\\
            而$\delta_{im}\delta_{jn}$要求$i=m$且$j=n$才會有值,$\delta_{in}\delta_{jm}$要求$i=n$且$j=m$才會有值\\
            故兩者皆為0\\ 
            當$i=m$且$j=n$時,但$i\neq j$,只有一項\\
            第三個標不是$i$也不是$j$的會有值,且為$1^2=1$,其他兩項皆為0\\
            而此時$ \delta_{im}\delta_{jn} - \delta_{in}\delta_{jm} = \delta_{mm}\delta_{nn} - \delta_{mn}\delta_{nm} = 1$\\
            當$i=m$且$j=n$且$i=j$時,每一項因為前兩個標必相同,皆為0\\
            而此時$ \delta_{im}\delta_{jn} - \delta_{in}\delta_{jm} = \delta_{mm}\delta_{mm} - \delta_{mm}\delta_{mm} = 1-1 = 0$\\
            當$i=n$且$j=m$時,但$i\neq j$,只有一項,第三個標不是$i$也不是$j$的會有值\\
            且因為前兩個標相反,故$\varepsilon_{nm1}\varepsilon_{mn1}=-1$\\
            而此時$ \delta_{im}\delta_{jn} - \delta_{in}\delta_{jm} = \delta_{nm}\delta_{mn} - \delta_{nn}\delta_{mm} = -1$\\
            故總結可得:
            \begin{equation}
              \sum_{k=1}^3 \varepsilon_{ijk}\varepsilon_{mnk} = \delta_{im}\delta_{jn} - \delta_{in}\delta_{jm}
            \end{equation}
        \end{enumerate}
      \item $\varepsilon_{ijk}$可以表示行列式的值
      \begin{equation}
        \begin{vmatrix}
          a_1 & a_2 & a_3 \\
          b_1 & b_2 & b_3 \\
          c_1 & c_2 & c_3
        \end{vmatrix} = \sum_{i=1}^3\sum_{j=1}^3\sum_{k=1}^3 \varepsilon_{ijk} a_i b_j c_k
      \end{equation}
      就是把27種排列組合中,循環排列的加起來,反循環排列的減掉
  \end{itemize}
  \item 單位向量可將所有量都寫為分量形式
    \begin{itemize}
      \item 純量: $s$
      \item 向量: $\vec {\bm v}$
        \begin{equation}
          \vec {\bm v} = v_x \bm \delta_x + v_y \bm \delta_y + v_z \bm \delta_z =  \sum_{i=1}^3 v_i \bm \delta_i
        \end{equation}
        當$\sum$遇到平方時,可以再生出一個$\sum$來表示
        \begin{equation}
          \boxed{\left(\sum_{i=1}^3 a_i\right)^2 = \sum_{i=1}^3 \sum_{j=1}^3 a_i a_j}
        \end{equation}
      \item 二階張量:\\ 
        來自一個張量:$\overline{\overline  {\bm \tau}}$
        \begin{equation}
          \overline{\overline  {\bm \tau}} = \sum_{i=1}^3 \sum_{j=1}^3  \bm \delta_i \bm \delta_j \tau_{ij}
        \end{equation}
        來自兩個向量的直積:$\vec {\bm v} \vec {\bm w}$
        \begin{equation}
          \vec {\bm v} \vec {\bm w} = \sum_{i=1}^3 \sum_{j=1}^3 \bm \delta_i \bm \delta_j v_i w_j
        \end{equation} 
    \end{itemize}
  \item 由單位向量與符號來表示運算:
    \begin{itemize}
      \item 純量與向量相乘:
        \begin{equation}
          s\vec {\bm v} = \sum_{i=1}^3 \bm \delta_i (s v_i)
        \end{equation}
      \item 向量內積:
        \begin{align}
          \vec {\bm v} \cdot \vec {\bm w} &= \sum_{i=1}^3 \sum_{j=1}^3 (\bm \delta_i \cdot \bm \delta_j) v_i w_j \nonumber\\
          &= \sum_{i=1}^3 \sum_{j=1}^3 \delta_{ij} v_i w_j \nonumber\\
          &= \sum_{i=1}^3 v_i w_i
        \end{align}
      \item 向量外積:
        \begin{align}
          \vec {\bm v} \times \vec {\bm w} &= \sum_{i=1}^3 \sum_{j=1}^3 (\bm \delta_i \times \bm \delta_j) v_i w_j \nonumber\\
          &= \sum_{i=1}^3 \sum_{j=1}^3 \sum_{k=1}^3 \varepsilon_{ijk} \bm \delta_k v_i w_j
        \end{align}
      \item 向量三重積:
        \begin{align}
          \vec {\bm u} \cdot \left[\vec {\bm v} \times \vec {\bm w}\right] &= \sum_{i=1}^3 u_i \bm \delta_i \cdot
          \left[
            \sum_{j=1}^3 \sum_{k=1}^3 \varepsilon_{jkl} \bm \delta_l v_k w_j
          \right] \nonumber\\
          &= \sum_{i=1}^3 \sum_{j=1}^3 \sum_{k=1}^3 \sum_{l=1}^3 \varepsilon_{jkl} (\bm \delta_i \cdot \bm \delta_l) u_i v_j w_k,\quad (l=1) \nonumber\\
          &= \sum_{i=1}^3 \sum_{j=1}^3 \sum_{k=1}^3 \varepsilon_{jki} u_i v_j w_k \nonumber\\
          &= \sum_{i=1}^3 \sum_{j=1}^3 \sum_{k=1}^3 \varepsilon_{ijk} u_i v_j w_k \nonumber\\
          &= \begin{vmatrix}
            u_1 & u_2 & u_3 \\
            v_1 & v_2 & v_3 \\
            w_1 & w_2 & w_3
          \end{vmatrix}
        \end{align}
      \item 向量直積$\implies$二階張量\\
        P.S. $\vec{\bm v} \vec{\bm w} \neq \vec{\bm w} \vec{\bm v}$
        \begin{equation}
          \vec {\bm v} \vec {\bm w} = \sum_{i=1}^3 \sum_{j=1}^3 \bm \delta_i \bm \delta_j v_i w_j
        \end{equation}
      \item 張量的轉置:\\
        如果一張量$\overline{\overline {\bm \tau}}$的表示為
        \begin{equation}
          \overline{\overline {\bm \tau}} = \sum_{i=1}^3 \sum_{j=1}^3 \bm \delta_i \bm \delta_j \tau_{ij}
        \end{equation}
        則其轉置$\overline{\overline {\bm \tau}}^\dagger$的表示為
        \begin{equation}
          \overline{\overline {\bm \tau}}^\dagger = \sum_{i=1}^3 \sum_{j=1}^3 \bm \delta_i \bm \delta_j \tau_{ji}
        \end{equation}
      \item 張量與張量作Double dot product
        \begin{align}
          \overline{\overline {\bm \sigma}} : \overline{\overline {\bm \tau}} &=
          \left(\sum_{i=1}^3 \sum_{j=1}^3 \bm \delta_i \bm \delta_j \sigma_{ij}\right) :
          \left(\sum_{m=1}^3 \sum_{n=1}^3 \bm \delta_m \bm \delta_n \tau_{mn}\right) \nonumber\\
          &= \sum_{i=1}^3 \sum_{j=1}^3 \sum_{m=1}^3 \sum_{n=1} \left(
            \bm \delta_i \bm \delta_j : \bm\delta_m\bm\delta_n
          \right)\sigma_{ij}\tau_{mn} \nonumber\\
          &= \sum_{i=1}^3 \sum_{j=1}^3 \sum_{m=1}^3 \sum_{n=1}^3 \delta_{jm}\delta_{in} \sigma_{ij}\tau_{mn} \nonumber\\
          &= \sum_{i=1}^3 \sum_{j=1}^3 \sigma_{ij}\tau_{ji}
        \end{align}
        P.S. 如果有一個是張量$\overline{\overline {\bm \tau}}$,另一個是兩個向量直積產生的張量$\vec {\bm v} \vec {\bm w}$,則
        \begin{equation}
          \overline{\overline {\bm \tau}} : \left(\vec {\bm v} \vec {\bm w}\right) = \sum_{i=1}^3 \sum_{j=1}^3 \tau_{ij} v_j w_i
        \end{equation}
      \item 張量內積向量:
        \begin{align}
          \overline{\overline {\bm \tau}} \cdot \vec {\bm v} &=
          \left(
            \sum_{i=1}^3 \sum_{j=1}^3 \bm \delta_i \bm \delta_j \tau_{ij}
          \right) \cdot
          \left(
            \sum_{k=1}^3 \bm \delta_k v_k
          \right) \nonumber\\
          &= \sum_{i=1}^3 \sum_{j=1}^3 \sum_{k=1}^3 \left(
            \bm \delta_i \bm \delta_j \cdot \bm \delta_k
          \right) \tau_{ij} v_k \nonumber\\
          &= \sum_{i=1}^3 \sum_{j=1}^3 \sum_{k=1}^3 \bm \delta_i (\bm \delta_j \cdot \bm \delta_k)  \tau_{ij} v_k \nonumber\\
          &= \sum_{i=1}^3 \sum_{j=1}^3 \sum_{k=1}^3 \bm \delta_i \delta_{jk}  \tau_{ij} v_k \nonumber\\
          &= \sum_{i=1}^3 \sum_{j=1}^3  \bm \delta_i  \tau_{ij} v_j 
        \end{align}
        也能看出$\overline{\overline {\bm \tau}} \cdot \vec {\bm v} \neq \vec {\bm v} \cdot \overline{\overline {\bm \tau}}$\\
        也就是把$i,j$交換\\
        而若進一步把$\bm\delta_i$往前移
        \begin{align}
          \overline{\overline {\bm \tau}} \cdot \vec {\bm v} &=
          \sum_{i=1}^3 \bm \delta_i \left(
            \sum_{j=1}^3  \tau_{ij} v_j 
          \right)
        \end{align}
        則可以看出,若$\overline{\overline {\bm \tau}}$是對稱的,
        則$\overline{\overline {\bm \tau}} \cdot \vec {\bm v} = \vec {\bm v} \cdot \overline{\overline {\bm \tau}}$
      \item 張量外積向量:
        \begin{align}
          \overline{\overline {\bm \tau}} \times \vec {\bm v} &=
          \left(
            \sum_{i=1}^3 \sum_{j=1}^3 \bm \delta_i \bm \delta_j \tau_{ij}
          \right) \times
          \left(
            \sum_{k=1}^3 \bm \delta_k v_k
          \right) \nonumber\\
          &= \sum_{i=1}^3 \sum_{j=1}^3 \sum_{k=1}^3 \left(
            \bm \delta_i \bm \delta_j \times \bm \delta_k
          \right) \tau_{ij} v_k \nonumber\\
          &= \sum_{i=1}^3 \sum_{j=1}^3 \sum_{k=1}^3 \bm \delta_i (\bm \delta_j \times \bm \delta_k)  \tau_{ij} v_k \nonumber\\
          &= \sum_{i=1}^3 \sum_{j=1}^3 \sum_{k=1}^3 \bm \delta_i \varepsilon_{jkl} \bm \delta_l  \tau_{ij} v_k \nonumber\\
          &= \sum_{i=1}^3 \sum_{j=1}^3 \sum_{k=1}^3  \sum_{l=1}^3\bm \delta_i  \bm \delta_l  \varepsilon_{jkl}  \tau_{ij} v_k \label{eq:tensor_cross_vector}
        \end{align}
        一樣可以將$\bm \delta_i$和$\bm \delta_l$往前移
        \begin{align}
          \overline{\overline {\bm \tau}} \times \vec {\bm v} &=
          \sum_{i=1}^3 \sum_{l=1}^3 \bm \delta_i  \bm \delta_l \left(
            \sum_{j=1}^3 \sum_{k=1}^3 \varepsilon_{jkl}  \tau_{ij} v_k 
          \right)
        \end{align}
    \end{itemize}
  \item 由單位向量符號來證明各種運算性質:\\
  證明時可以養成,先寫$\sum$再寫$\varepsilon_{ijk}$,再寫$\delta_{ij}$,再寫$\bm \delta_{i}$\\
  最後再把剩下的非單位向量擺在後面\\
  解題步驟如下:
  \begin{enumerate}
    \item 將所有量都展開為有單位向量的形式
    \item 除了單位向量外的,全部擺到最後面,而所有$\sum$都可以乘在一起\\
    $\delta$與$\varepsilon$,會防止最後項數不是$3^n$的情況發生
    \item 所有的運算操作都只會對單位向量進行,按照上述的運算規則進行
    \item 最後把產出的$\delta$與$\varepsilon$,將一些指標與$\sum$號消掉\\
    P.S.$\varepsilon$可以旋轉,也可以透過(\ref{eq:epsilon_delta_relation})與(\ref{eq:epsilon_delta_relation_2})式\\
    轉化為$\delta$的形式
  \end{enumerate}
  \begin{itemize}
    \item 證明 Identity of Lagrange:
      \begin{equation}
        \left(
          [\vec {\bm v} \times \vec {\bm w}] \cdot [\vec {\bm v} \times \vec {\bm w}]
        \right) + \left( \vec {\bm v} \cdot \vec {\bm w} \right)^2 = v^2w^2
      \end{equation}
      Proof:\\
      從左邊第一項開始:
      \begin{align}
        \left(
          [\vec {\bm v} \times \vec {\bm w}] \cdot [\vec {\bm v} \times \vec {\bm w}]
        \right) & = \left(
          [\sum_{i=1}^3 \sum_{j=1}^3 \sum_{k=1}^3 \varepsilon_{ijk} \bm \delta_k v_i w_j] \cdot
          [\sum_{m=1}^3 \sum_{n=1}^3 \sum_{p=1}^3 \varepsilon_{mnp} \bm \delta_p v_m w_n]
        \right) \nonumber\\
        &= \sum_{i=1}^3 \sum_{j=1}^3 \sum_{k=1}^3 \sum_{m=1}^3 \sum_{n=1}^3 \sum_{p=1}^3
        \varepsilon_{ijk} \varepsilon_{mnp} (\bm \delta_k \cdot \bm \delta_p) v_i w_j v_m w_n \nonumber\\
        &= \sum_{i=1}^3 \sum_{j=1}^3 \sum_{k=1}^3 \sum_{m=1}^3 \sum_{n=1}^3 \sum_{p=1}^3
        \varepsilon_{ijk} \varepsilon_{mnp} \delta_{kp} v_i w_j v_m w_n
      \end{align}
      由於出現,$\delta_{kp}$,故$p=k$,並除掉$\sum_{p=1}^3$,得到:
      \begin{equation}
        \left(
          [\vec {\bm v} \times \vec {\bm w}] \cdot [\vec {\bm v} \times \vec {\bm w}]
        \right) = \sum_{i=1}^3 \sum_{j=1}^3 \sum_{k=1}^3 \sum_{m=1}^3 \sum_{n=1}^3
        \varepsilon_{ijk} \varepsilon_{mnk} v_i w_j v_m w_n 
      \end{equation}
      觀察到,$\varepsilon_{ijk} \varepsilon_{mnk}$,有一個共同項,因此使用(\ref{eq:epsilon_delta_relation_2})式\\
      P.S. 注意指標順序有時可能與定義不同,順循環則不須變號,反循環則須變號
      \begin{align}
        \left(
          [\vec {\bm v} \times \vec {\bm w}] \cdot [\vec {\bm v} \times \vec {\bm w}]
        \right) &= \sum_{i=1}^3 \sum_{j=1}^3 \sum_{m=1}^3 \sum_{n=1}^3
        \left(
          \delta_{im}\delta_{jn} - \delta_{in}\delta_{jm}
        \right) v_i w_j v_m w_n \nonumber\\
        &= \sum_{i=1}^3 \sum_{j=1}^3 \sum_{m=1}^3 \sum_{n=1}^3 \delta_{im}\delta_{jn} v_i w_j v_m w_n
        - \sum_{i=1}^3 \sum_{j=1}^3 \sum_{m=1}^3 \sum_{n=1}^3 \delta_{in}\delta_{jm} v_i w_j v_m w_n \nonumber\\
        &= \sum_{i=1}^3 \sum_{j=1}^3 v_i w_j v_i w_j - \sum_{i=1}^3 \sum_{j=1}^3 v_i w_j v_j w_i \nonumber\\
        &= \sum_{i=1}^3 \sum_{j=1}^3 v_i^2 w_j^2 - \sum_{i=1}^3 \sum_{j=1}^3 v_i v_j w_i w_j 
      \end{align} 
      接著看左邊第二項:
      \begin{align}
        \left( \vec {\bm v} \cdot \vec {\bm w} \right)^2 &= \left(
          \sum_{i=1}^3 v_i w_i
        \right)^2 \nonumber\\
        &= \sum_{i=1}^3 \sum_{j=1}^3 v_i w_i v_j w_j
      \end{align}
      將兩項相加:
      \begin{align}
        \left(
          [\vec {\bm v} \times \vec {\bm w}] \cdot [\vec {\bm v} \times \vec {\bm w}]
        \right) + \left( \vec {\bm v} \cdot \vec {\bm w} \right)^2 &= 
        \left(
          \sum_{i=1}^3 \sum_{j=1}^3 v_i^2 w_j^2 - \sum_{i=1}^3 \sum_{j=1}^3 v_i v_j w_i w_j
        \right) + \left(
          \sum_{i=1}^3 \sum_{j=1} v_i w_i v_j w_j
        \right) \nonumber\\
        &= \sum_{i=1}^3 \sum_{j=1}^3 v_i^2 w_j^2 \nonumber\\
        &= \left(\sum_{i=1}^3 v_i^2\right) \left(\sum_{j=1}^3 w_j^2\right) \nonumber\\
        &= v^2 w^2
      \end{align}
    \item 證明向量三重積
      \begin{equation}
        \vec {\bm u} \times (\vec {\bm v} \times \vec {\bm w}) = (\vec {\bm u} \cdot \vec {\bm w})\vec {\bm v} - (\vec {\bm u} \cdot \vec {\bm v})\vec {\bm w}
      \end{equation}
      Proof:\\
      從左邊開始:
      \begin{align}
        \vec {\bm u} \times (\vec {\bm v} \times \vec {\bm w}) &= \sum_{m=1}^3 \bm \delta_m u_m \times
        \left[
          \sum_{i=1}^3 \sum_{j=1}^3 \sum_{k=1}^3 \varepsilon_{ijk} \bm \delta_k v_i w_j
        \right] \nonumber\\
        &= \sum_{m=1}^3 \sum_{i=1}^3 \sum_{j=1}^3 \sum_{k=1}^3
        \varepsilon_{ijk} (\bm \delta_m \times \bm \delta_k) u_m v_i w_j
      \end{align}
      當遇到單位向量外積時\\
      記得生出一個$\varepsilon$符號,前兩個為外積的標,而後面一個為新的標\\
      並多加一個$\sum$來涵蓋新的標
      \begin{align}
        \vec {\bm u} \times (\vec {\bm v} \times \vec {\bm w}) &= \sum_{m=1}^3 \sum_{i=1}^3 \sum_{j=1}^3 \sum_{k=1}^3 \sum_{n=1}
        \varepsilon_{ijk} \varepsilon_{mkn} \bm \delta_n u_m v_i w_j \nonumber\\
        &= \sum_{m=1}^3 \sum_{i=1}^3 \sum_{j=1}^3 \sum_{k=1}^3 \sum_{n=1}
        \varepsilon_{ijk} \varepsilon_{nmk} \bm \delta_n u_m v_i w_j
      \end{align}
      這邊換成$nmk$,或者可以是負的$mnk$,如果希望按照字母序\
      用(\ref{eq:epsilon_delta_relation_2})式:
      \begin{align}
        \vec {\bm u} \times (\vec {\bm v} \times \vec {\bm w}) &= \sum_{m=1}^3 \sum_{i=1}^3 \sum_{j=1}^3 \sum_{n=1}^3
        \left(
          \delta_{in}\delta_{jm} - \delta_{im}\delta_{jn}
        \right) \bm \delta_n u_m v_i w_j \nonumber\\
        &= \sum_{m=1}^3 \sum_{i=1}^3 \sum_{j=1}^3 \sum_{n=1}^3
        \delta_{in}\delta_{jm} \bm \delta_n u_m v_i w_j
        - \sum_{m=1}^3 \sum_{i=1}^3 \sum_{j=1}^3 \sum_{n=1}^3
        \delta_{im}\delta_{jn} \bm \delta_n u_m v_i w_j 
      \end{align}
      左邊,令$n=i$且$m=j$,右邊,令$m=i$且$n=j$:
      \begin{equation}
        \vec {\bm u} \times (\vec {\bm v} \times \vec {\bm w}) = {\color{red}{\sum_{i=1}^3}} \sum_{j=1}^3
        {\color{red}{\bm \delta_i}} u_j {\color{red}{v_i}} w_j
        - \sum_{i=1}^3 {\color{red}{\sum_{j=1}^3}}
        {\color{red}{\bm \delta_j}} u_i v_i {\color{red}{w_j}} 
      \end{equation}
      當看到$\sum$的底標配合他的單位向量的底標時,可以回復成向量$\sum_i \bm \delta_i a_i = \vec {\bm a}$:
      \begin{align}
         \vec {\bm u} \times (\vec {\bm v} \times \vec {\bm w}) &= \vec {\bm v} \left(
          \sum_{j=1}^3 u_j w_j
         \right) - \vec {\bm w} \left(
          \sum_{i=1}^3 u_i v_i
          \right) \nonumber\\
          &= (\vec {\bm u} \cdot \vec {\bm w}) \vec {\bm v} - (\vec {\bm u} \cdot \vec {\bm v}) \vec {\bm w}
      \end{align}
  \end{itemize}
  \item 用單位向量來表示微分運算:
    \begin{itemize}
      \item $\nabla$算符:
        \begin{equation}
          \boxed{\nabla = \sum_{i=1}^3 \bm \delta_i \frac{\partial}{\partial x_i}}
        \end{equation}
        注意這裡的$x$是代表在卡式座標系下的$x,y,z$
      \item Gradient (梯度):
        \begin{equation}
          \boxed{\nabla f = \sum_{i=1}^3 \bm \delta_i \frac{\partial f}{\partial x_i}}
        \end{equation}
        沒有交換律:
        \begin{equation}
          \boxed{\nabla f \neq f \nabla}
        \end{equation}
        沒有結合律:
        \begin{equation}
          \boxed{\nabla (fg) \neq (\nabla f) g}
        \end{equation}
      \item Divergence (散度):
        \begin{equation}
          \boxed{\nabla \cdot \vec {\bm u} = \sum_{i=1}^3\frac{\partial u_i}{\partial x_i}}
        \end{equation}
        Proof:
        \begin{align}
          &\nabla \cdot \vec {\bm u} \nonumber\\
         =& \left(
          \sum_{i=1}^3 \bm \delta_i \frac{\partial}{\partial x_i}
          \right) \cdot \left(
            \sum_{j=1}^3 \bm \delta_j u_j
          \right)\\
        =&\sum_{i=1}^3 \sum_{j=1}^3 \left(
          \bm \delta_i \cdot \bm \delta_j
        \right) \frac{\partial}{\partial x_i} u_j \nonumber\\
        =&\sum_{i=1}^3 \sum_{j=1}^3 \delta_{ij} \frac{\partial u_j}{\partial x_i} \nonumber\\
        =&\sum_{i=1}^3 \frac{\partial u_i}{\partial x_i}
        \end{align}
        不存在交換律、結合律
        \begin{equation}
          \boxed{\nabla \cdot \vec {\bm u} \neq \vec {\bm u} \cdot \nabla} \quad \boxed{\nabla \cdot (f \vec {\bm u}) \neq (\nabla f) \cdot \vec {\bm u}}
        \end{equation}
      \item Curl (旋度):
        \begin{equation}
          \boxed{\nabla \times \vec {\bm u} = \sum_{i=1}^3 \sum_{j=1}^3 \sum_{k=1}^3 
          \varepsilon_{ijk} \bm \delta_k \frac{\partial u_j}{\partial x_i}}
        \end{equation}
        Proof:
        \begin{align}
          \left[\nabla \times \vec {\bm u} \right] &
          = \left[
            \left(
              \sum_{i=1}^3 \bm \delta_i \frac{\partial}{\partial x_i}
            \right) \times 
            \left(
              \sum_{j=1}^3 \bm \delta_j u_j
            \right)
          \right] \nonumber\\
          &= \sum_{i=1}^3 \sum_{j=1}^3 \left(
            \bm \delta_i \times \bm \delta_j
          \right) \frac{\partial}{\partial x_i} u_j \nonumber\\
          &= \sum_{i=1}^3 \sum_{j=1}^3 \sum_{k=1}^3
          \varepsilon_{ijk} \bm \delta_k \frac{\partial u_j}{\partial x_i}
        \end{align}
        不存在交換律、結合律
        \begin{equation}
          \boxed{\nabla \times \vec {\bm u} \neq \vec {\bm u} \times \nabla} \quad 
          \boxed{\nabla \times (f \vec {\bm u}) \neq (\nabla f) \times \vec {\bm u}}
        \end{equation}
        P.S. 若展開則跟前面(\ref{eq:curl_definition})式相同
        \begin{equation}
          \nabla \times \vec {\bm u} = \begin{vmatrix}
            \bm \delta_x & \bm \delta_y & \bm \delta_z \\
            \frac{\partial}{\partial x} & \frac{\partial}{\partial y} & \frac{\partial}{\partial z} \\
            u_x & u_y & u_z
          \end{vmatrix}
        \end{equation}
      \item 對向量場取Gradient(升階為張量):
        \begin{equation}
          \boxed{\nabla \vec {\bm u} = \sum_{i=1}^3 \sum_{j=1}^3 \bm \delta_i \bm \delta_j \frac{\partial u_j}{\partial x_i}}
        \end{equation}
        Proof:
        \begin{align}
          \nabla \vec {\bm u} &=
          \left(
            \sum_{i=1}^3 \bm \delta_i \frac{\partial}{\partial x_i}
          \right) 
          \left(
            \sum_{j=1}^3 \bm \delta_j u_j
          \right) \nonumber\\
          &= \sum_{i=1}^3 \sum_{j=1}^3 \bm \delta_i \bm \delta_j
          \frac{\partial u_j}{\partial x_i}
        \end{align}
        P.S. 可以將其轉置:
        \begin{equation}
          \left(\nabla \vec {\bm u}\right)^\dagger = \sum_{i=1}^3 \sum_{j=1}^3 \bm \delta_i \bm \delta_j \frac{\partial u_i}{\partial x_j}
        \end{equation}
        可以注意到轉置前後不同:
        \begin{equation}
          \boxed{\frac{\partial u_j}{\partial x_i} \neq \frac{\partial u_i}{\partial x_j}}
        \end{equation}
      \item 對張量場取Divergence(降階為向量):
        \begin{equation}
          \boxed{\nabla \cdot \overline{\overline{\bm \tau}} = 
          \sum_{i=1}^3\sum_{k=1}^3 \bm \delta_k \frac{\partial \tau_{ik}}{\partial x_i}}
        \end{equation}
        少$j$的原因是在在計算中會降階\\
        也可以將$k$提出:
        \begin{equation}
          \nabla \cdot \overline{\overline{\bm \tau}} = \sum_{k=1}^3 \bm \delta_k \left(
            \sum_{i=1}^3 \frac{\partial \tau_{ik}}{\partial x_i}
          \right)
        \end{equation}
        也就是說,等同於分別對$x,y,z$三個方向取Divergence後,再組合成向量場
        \begin{equation}
          \nabla \cdot \overline{\overline{\bm \tau}} = \begin{pmatrix}
            \sum_{i=1}^3 \frac{\partial \tau_{i1}}{\partial x_i} \\
            \sum_{i=1}^3 \frac{\partial \tau_{i2}}{\partial x_i} \\
            \sum_{i=1}^3 \frac{\partial \tau_{i3}}{\partial x_i}
          \end{pmatrix}
        \end{equation}
        如果張量是由兩個向量直積產生的,$\overline{\overline{\bm \tau}} = s\vec {\bm v} \vec {\bm w}$\\
        可進一步改寫為:
        \begin{equation}
          \nabla \cdot (s \vec {\bm v} \vec {\bm w}) = \sum_{k=1}^3 \bm \delta_k \left(
            \sum_{i=1}^3 \frac{\partial (s v_i w_k)}{\partial x_i}
          \right)
        \end{equation}
        Proof:
        \begin{align}
          \nabla \cdot \overline{\overline{\bm \tau}} &=
          \left(
            \sum_{i=1}^3 \bm \delta_i \frac{\partial}{\partial x_i}
          \right) \cdot
          \left(
            \sum_{j=1}^3 \sum_{k=1}^3 \bm \delta_j \bm \delta_k \tau_{jk}
          \right) \nonumber\\
          &= \sum_{i=1}^3 \sum_{j=1}^3 \sum_{k=1}^3
          \left[\left(
            \bm \delta_i \cdot \bm \delta_j
          \right) \bm \delta_k\right] \frac{\partial \tau_{jk}}{\partial x_i} \nonumber\\
          &= \sum_{i=1}^3 \sum_{j=1}^3 \sum_{k=1}^3
          \delta_{ij} \bm \delta_k
          \frac{\partial \tau_{jk}}{\partial x_i} \nonumber\\
          &= \sum_{i=1}^3 \sum_{k=1}^3 \bm \delta_k
          \frac{\partial \tau_{ik}}{\partial x_i}
        \end{align}
      \item 對純量取Laplacian,$\nabla^2$: ($\nabla \cdot \nabla f$)
        \begin{equation}
          \boxed{\nabla^2 f = \sum_{i=1}^3 \frac{\partial^2 f}{\partial x_i^2}}
        \end{equation}
        Proof:
        \begin{align}
         \nabla^2 =  \nabla \cdot \nabla f &=
          \left(
            \sum_{i=1}^3 \bm \delta_i \frac{\partial}{\partial x_i}
          \right) \cdot
          \left(
            \sum_{j=1}^3 \bm \delta_j \frac{\partial f}{\partial x_j}
          \right) \nonumber\\
          &= \sum_{i=1}^3 \sum_{j=1}^3
          \left(
            \bm \delta_i \cdot \bm \delta_j
          \right) \frac{\partial}{\partial x_i} \frac{\partial f}{\partial x_j} \nonumber\\
          &= \sum_{i=1}^3 \sum_{j=1}^3
          \delta_{ij} \frac{\partial^2 f}{\partial x_i \partial x_j} \nonumber\\
          &= \sum_{i=1}^3 \frac{\partial^2 f}{\partial x_i^2}
        \end{align}
      \item 對向量場取Laplacian,$\nabla^2 \vec {\bm u}$:
        \begin{equation}
          \boxed{\nabla^2 \vec {\bm u}  = \nabla \cdot \nabla \vec {\bm u}
          = \sum_{i=1}^3\sum_{k=1}^3 \bm \delta_k \frac{\partial^2 u_k}{\partial x_i^2}}
        \end{equation}
        也可將$k$提出:
        \begin{equation}
          \nabla^2 \vec {\bm u} = \sum_{k=1}^3 \bm \delta_k \left(
            \sum_{i=1}^3 \frac{\partial^2 u_k}{\partial x_i^2}
          \right) = \sum_{k=1}^3 \bm \delta_k \nabla^2 u_k
        \end{equation}
        也就是說,等同於分別對$x,y,z$三個方向取Laplacian後,再組合成向量場
        \begin{equation}
          \nabla^2 \vec {\bm u}  = \begin{pmatrix}
            \nabla^2 u_x \\
            \nabla^2 u_y \\
            \nabla^2 u_z
          \end{pmatrix}
        \end{equation}
        不過非卡式座標系下,就沒這麼簡單了\\
        Proof:
        \begin{align}
          \nabla^2 \vec {\bm u} &=
          \left(
            \sum_{i=1}^3 \bm \delta_i \frac{\partial}{\partial x_i}
          \right) \cdot
          \left(
            \sum_{j=1}^3 \bm \delta_j \frac{\partial}{\partial x_j}
          \right)
          \left(
            \sum_{k=1}^3 \bm \delta_k u_k
          \right) \nonumber\\
          &= \sum_{i=1}^3 \sum_{j=1}^3 \sum_{k=1}^3
          \left(
            \bm \delta_i \cdot \bm \delta_j
          \right) \bm \delta_k
          \frac{\partial}{\partial x_i} \frac{\partial}{\partial x_j} u_k \nonumber\\
          &= \sum_{i=1}^3 \sum_{j=1}^3 \sum_{k=1}^3
          \delta_{ij} \bm \delta_k
          \frac{\partial^2 u_k}{\partial x_i \partial x_j} \nonumber\\
          &= \sum_{i=1}^3  \sum_{k=1}^3
          \bm \delta_k
          \frac{\partial^2 u_k}{\partial x_i^2} \nonumber\\
          &= \sum_{k=1}^3  \bm \delta_k
          \left(
            \sum_{i=1}^3
            \frac{\partial^2 u_k}{\partial x_i^2}
          \right) \nonumber\\
          &= \sum_{k=1}^3  \bm \delta_k
          \nabla^2 u_k
        \end{align}
    \end{itemize}
\end{itemize}
\subsection{時間導數}
\begin{itemize}
  \item Substantial time derivative 質點微分 $\frac{D()}{dt}$(實質導數)\\
    Lagrange提供:用移動座標觀察質點且移動座標速度等於質點速度\\
    質點與座標之間相對位置不變\\
    i.e. $f=f(t,\cancel{x},\cancel{y},\cancel{z})=f(t)$ only\\
    得到對$t$微分,稱$\frac{D()}{dt}$
    \begin{equation}
      \frac{Df}{Dt} = \frac{\partial f}{\partial t} + \vec {\bm v} \cdot \nabla f \label{eq:substantial_time_derivative}
    \end{equation}
  \item Partial time derivative 偏微分$\frac{\partial}{\partial t}$\\
    Euler提供:以固定座標來觀察質點\\
    質點與座標之間相對位置不斷改變\\
    i.e. $f=f(t,x,y,z)$\\
    得到對$t$微分,稱$\frac{\partial}{\partial t}$
  \item 如果知道函數對\fbox{單一變數}之關係$\Rightarrow$偏微分\\
      $\frac{\partial y}{\partial x}>0$,$y$隨$x$增加而增加\\
      $\frac{\partial y}{\partial x}<0$,$y$隨$x$增加而減少
  \item total time derivative 全微分 $\frac{d}{dt}$\\
    以移動座標來觀察質點,但移動座標速度不等於質點速度\\
    質點與座標之間相對位置不斷改變\\
    假設一函數$f=f(t,x,y,z)$,以$\vec {\bm w}$為參考點座標的移動速度,$\vec {\bm v}$為此函數的質點速度
    \begin{equation}
      \boxed{\frac{df}{dt} =\frac{\partial f}{\partial t}+ \vec{w}\cdot \nabla f} \label{eq:total_time_derivative}
    \end{equation}
  \item 如果要知道函數對\fbox{所有變數}之關係$\Rightarrow$全微分\\
    欲知$f(t,x,y,z)$對所有變數影響,假設以$\vec {\bm w}$ 參考點座標,$\vec {\bm v}$質點座標\\
    作全微分展開
    \begin{equation}
      df = \left(\frac{\partial f}{\partial t}\right)dt+
      \left(\frac{\partial f}{\partial x}\right)dx+
      \left(\frac{\partial f}{\partial y}\right)dy+
      \left(\frac{\partial f}{\partial z}\right)dz
    \end{equation}
    提出$dt$:
    \begin{equation}
      df = dt\left(
        \frac{\partial f}{\partial t}+\left(\frac{\partial f}{\partial x}\right)\left(\frac{\partial x}{\partial t}\right)
        +\left(\frac{\partial f}{\partial y}\right)\left(\frac{\partial y}{\partial t}\right)
        +\left(\frac{\partial f}{\partial z}\right)\left(\frac{\partial z}{\partial t}\right)
      \right)
    \end{equation}
    可以注意到,$\frac{\partial x}{\partial t},~\frac{\partial y}{\partial t},~\frac{\partial z}{\partial t}$\\
    就是參考點座標的速度分量,$w_x,w_y,w_z$\\
    將$dt$移到左側:
    \begin{equation}
      \frac{df}{dt}= \frac{\partial f}{\partial t}+
      \left(\frac{\partial f}{\partial x}\right)w_x+
      \left(\frac{\partial f}{\partial y}\right)w_y+
      \left(\frac{\partial f}{\partial z}\right)w_z
    \end{equation}
    其中因為$\nabla f$的定義為
    \begin{equation}
      \nabla f = \frac{\partial f}{\partial x} \bm\delta_x+\frac{\partial f}{\partial y}\bm\delta_y
      +\frac{\partial f}{\partial z} \bm\delta_z
    \end{equation}
    且$\vec {\bm w}$寫為:
    \begin{equation}
      \vec{\bm w} = w_x \bm \delta_x + w_y \bm \delta_y + w_z \bm \delta_z
    \end{equation}
    後面三項即為$\vec {\bm w} \cdot \nabla f$,故可寫成
    \begin{equation}
      \boxed{\frac{df}{dt} =\frac{\partial f}{\partial t}+ \vec{w}\cdot \nabla f} 
    \end{equation}
    也就是式(\ref{eq:total_time_derivative})
  \item $\frac{D()}{dt},~\frac{\partial}{\partial t},~\frac{d}{dt}$之間的關係式:
    由式(\ref{eq:total_time_derivative})可知,\\
    如果\fbox{座標移動速度等於質點速度},$\vec {\bm w} = \vec {\bm v}$,就會變為質點微分的形式(\ref{eq:substantial_time_derivative})
    \begin{equation}
      \frac{df}{dt}=\frac{\partial f}{\partial t}+ \vec{v}\cdot \nabla f
    \end{equation}
    而當\fbox{座標不移動}時,$\vec {\bm w} = \vec 0$,則會變為偏微分的形式
    \begin{equation}
      \boxed{\frac{Df}{Dt} = \frac{\partial f}{\partial t}} 
    \end{equation}
  \item 推導:
    \begin{equation}
      \frac{\partial (\rho \vec {\bm u})}{\partial t} + \nabla \cdot (\rho \vec {\bm u} \vec {\bm u}) = \rho \frac{D\vec {\bm u}}{Dt}  \label{eq:material_derivative_relation}
    \end{equation}
    Proof:
    \begin{align}
      \frac{\partial (\rho \vec {\bm u})}{\partial t} + \nabla \cdot (\boxed{\rho \vec {\bm u}} \boxed{\vec {\bm u}}) &= 
      \rho\frac{\partial \vec {\bm u}}{\partial t} + \vec {\bm u} \frac{\partial \rho}{\partial t} 
      + \rho \vec {\bm u} \cdot \nabla \vec {\bm u} + \vec {\bm u} \cdot \nabla (\rho \vec {\bm u}) \nonumber\\
      &= \rho\left(
        \frac{\partial \vec {\bm u}}{\partial t} + \vec {\bm u} \cdot \nabla \vec {\bm u}
      \right) + \vec {\bm u} \left(
        \frac{\partial \rho}{\partial t} + \nabla \cdot (\rho \vec {\bm u})
      \right)
    \end{align}  
    由質量守恆方程式:
    \begin{equation}
      \frac{\partial \rho}{\partial t} + \nabla \cdot (\rho \vec {\bm u}) = 0
    \end{equation}
    故可得:
    \begin{equation}
      \frac{\partial (\rho \vec {\bm u})}{\partial t} + \nabla \cdot (\rho \vec {\bm u} \vec {\bm u}) = \rho \left(
        \frac{\partial \vec {\bm u}}{\partial t} + \vec {\bm u} \cdot \nabla \vec {\bm u}
      \right)
    \end{equation}
    而右邊括弧內即為質點微分形式,故可寫成:
    \begin{equation}
      \frac{\partial (\rho \vec {\bm u})}{\partial t} + \nabla \cdot (\rho \vec {\bm u} \vec {\bm u}) = \rho \frac{D\vec {\bm u}}{Dt} 
    \end{equation}
\end{itemize}
\subsection{多重積分的定理}
關於積分面積、法向量的正負號其實原先應透過定義法向量$\vec {\bm n}$來決定\\
但只要透過假設是\fbox{流入-流出},流入時與法向量方向相反,流出時與法向量方向相同\\
即可決定正負號(兩邊算出的正負值一致)
\begin{itemize}
  \item Stokes theorem\\
    面積分等於環線積分
    \begin{equation}
      \oint \vec {\bm v} \cdot d\vec {\bm s} = \iint \left(\nabla \times \vec {\bm v}\right) \cdot d\vec {\bm A}
    \end{equation}
    \item Gauss theorem
    體積分等於環面積分,向量版本
    \begin{equation}
      \oiint \vec {\bm v} \cdot d\vec {\bm A} = \iiint \nabla \cdot \vec {\bm v} dV
    \end{equation}
    假設一個不規則形狀為控制體積,為了符合Mass balance
    \begin{equation}
      \text{流入} - \text{流出} = \frac{\partial }{\partial t} \iiint_{V} \rho dV
    \end{equation}
    而定義面積的法向量是垂直於控制體積的\\
    因此流入控制體積的量為:
    \begin{equation}
      \oiint_{in} \vec {\bm u}_{in} \cdot d\vec {\bm A}_{in} 
    \end{equation}
    流入方向與法向量方向相反,故內積為負值\\
    流出控制體積的量為:
    \begin{equation}
      \oiint_{out} \vec {\bm u}_{out} \cdot d\vec {\bm A}_{out} 
    \end{equation}
    流出方向與法向量方向相同,故內積為正值\\
    由於流入與流出皆為負值,故流入-流出為負值\\
    因此流入-流出可寫成:
    \begin{equation}
      \oiint_{in} \vec {\bm u}_{in} \cdot d\vec {\bm A}_{in} - \oiint_{out} \vec {\bm u}_{out} \cdot d\vec {\bm A}_{out} = -\oiint \vec {\bm u} \cdot d\vec {\bm A}
    \end{equation}
    綜合以上式子,可寫出Mass Balance方程式:
    \begin{equation}
      -\oiint \vec {\bm u} \cdot d\vec {\bm A} = \frac{\partial }{\partial t} \iiint_{V} \rho dV \label{eq:mass_balance_by_gauss}
    \end{equation}
    利用Gauss theorem,可將左邊面積分轉為體積分:
    \begin{equation}
      -\iiint \nabla \cdot \vec {\bm u} dV = \frac{\partial }{\partial t} \iiint_{V} \rho dV
    \end{equation}
    將右邊微分移到體積積分內:
    \begin{equation}
      \iiint_{V} \left(-\nabla \cdot \vec {\bm u} - \frac{\partial \rho}{\partial t}\right) dV = 0
    \end{equation}
    由於體積積分為0,故積分內的函數必為0:
    \begin{equation}
      -\nabla \cdot \vec {\bm u} - \frac{\partial \rho}{\partial t} = 0
    \end{equation}
    故可得出Equation of Continuity
  \item Green theorem\\
    體積分等於環面積分,純量版本
    \begin{equation}
      \oiint f d\vec {\bm A} = \iiint \nabla f dV
    \end{equation}
    由Momentum balance可得出Equation of Motion
    \begin{equation}
      \text{流入} - \text{流出} =\text{累積動能}
    \end{equation}
    動量為:
    \begin{equation}
      \vec {\bm P} = m \vec {\bm u} = \rho V \vec {\bm u}
    \end{equation}
    流入控制體積的量為:$\oiint_{in} \rho \vec {\bm u}_{in} \vec {\bm u}_{in} \cdot d\vec {\bm A}_{in}$\\
    流入方向與法向量方向相反,故內積為負值\\
    流出控制體積的量為:$\oiint_{out} \rho \vec {\bm u}_{out} \vec {\bm u}_{out} \cdot d\vec {\bm A}_{out}$\\
    流出方向與法向量方向相同,故內積為正值\\
    由於流入與流出皆為負值,故流入-流出為負值\\
    因此流入-流出可寫成:
    \begin{equation}
      \oiint_{in} \rho \vec {\bm u}_{in} \vec {\bm u}_{in} \cdot d\vec {\bm A}_{in} - \oiint_{out} \rho \vec {\bm u}_{out} \vec {\bm u}_{out} \cdot d\vec {\bm A}_{out} 
      = -\oiint \rho \vec {\bm u} \vec {\bm u} \cdot d\vec {\bm A}
    \end{equation}
    同高斯(\ref{eq:mass_balance_by_gauss}),利用Gauss theorem,可將左邊面積分轉為體積分:
    \begin{equation}
       -\oiint \rho \vec {\bm u} \vec {\bm u} \cdot d\vec {\bm A} = -\iiint \nabla \cdot (\rho \vec {\bm u} \vec {\bm u}) dV
    \end{equation}
    對於壓力也相同的做法,只是壓力為純量\\
    流入的壓力為,$\oiint_{in} P_{in} d\vec {\bm A}_{in}$\\
    流入方向與法向量方向相反,故內積為負值
    流出的壓力為,$\oiint_{out} P_{out} d\vec {\bm A}_{out}$\\
    流出方向與法向量方向相同,故內積為正值
    由於流入與流出皆為負值,故流入-流出為負值\\
    因此流入-流出可寫成:
    \begin{equation}
      \oiint_{in} P_{in} d\vec {\bm A}_{in} - \oiint_{out} P_{out} d\vec {\bm A}_{out} = -\oiint P d\vec {\bm A}
    \end{equation}
    此時因為壓力是純量,故利用Green theorem,可將左邊面積分轉為體積分:
    \begin{equation}
      -\oiint P d\vec {\bm A} = -\iiint \nabla P dV
    \end{equation}
    加上重力作用:
    \begin{equation}
      \iiint \rho \vec {\bm g} dV
    \end{equation}
    加上黏滯力,進入系統的黏滯力為$\oiint_{in} \overline {\overline {\bm \tau}}_{in} \cdot d\vec {\bm A}_{in}$
    流入方向與法向量方向相反,故內積為負值\\
    流出的黏滯力為$\oiint_{out} \overline{\overline{\bm \tau}}_{out} \cdot d\vec {\bm A}_{out}$\\
    流出方向與法向量方向相同,故內積為正值\\
    由於流入與流出皆為負值,故流入-流出為負值\\
    因此流入-流出可寫成:
    \begin{equation}
      \oiint_{in} \overline {\overline{\bm \tau}}_{in} \cdot d\vec {\bm A}_{in} - \oiint_{out} \overline{\overline{\bm \tau}}_{out} \cdot d\vec {\bm A}_{out} 
      = -\oiint \overline{\overline{\bm \tau}} \cdot d\vec {\bm A}
    \end{equation}
    利用Gauss theorem,可將左邊面積分轉為體積分
    \begin{equation}
      -\oiint \overline{\overline{\bm \tau}} \cdot d\vec {\bm A} = -\iiint \nabla \cdot \overline{\overline{\bm \tau}} dV
    \end{equation}
    累積的動能為:
    \begin{equation}
      \frac{\partial }{\partial t} \iiint_{V} \rho \vec {\bm u} dV
    \end{equation}
    綜合以上式子,可寫出Momentum Balance方程式:
    \begin{equation}
      -\iiint \nabla \cdot (\rho \vec {\bm u} \vec {\bm u}) dV - \iiint \nabla P dV + \iiint \rho \vec {\bm g} dV - \iiint \nabla \cdot \overline{\overline\tau} dV
      = \frac{\partial }{\partial t} \iiint_{V} \rho \vec {\bm u} dV
    \end{equation}
    將右邊微分移到體積積分內:
    \begin{equation}
      \iiint_{V} \left(
        -\nabla \cdot (\rho \vec {\bm u} \vec {\bm u}) - \nabla P + \rho \vec {\bm g} - \nabla \cdot \overline{\overline{\bm \tau}}
        - \frac{\partial }{\partial t}(\rho \vec {\bm u})
      \right) dV = 0
    \end{equation}
    由於體積積分為0,故積分內的函數必為0:
    \begin{equation}
      -\nabla \cdot (\rho \vec {\bm u} \vec {\bm u}) - \nabla P + \rho \vec {\bm g} - \nabla \cdot \overline{\overline{\bm \tau}}
      = \frac{\partial }{\partial t}(\rho \vec {\bm u})
    \end{equation}
    此即為Equation of Motion
\end{itemize}
\subsection{雜項的微積分工具}
\begin{itemize}
  \item Leibniz's rule\\
    當要對積分函數進行微分時,而且積分的上下界是變動的
    \begin{equation}
      f(x,y) = \int_{y_1(x)}^{y_2(x)} \phi(y,x) dy
    \end{equation}
    則對$x$微分時\\
    會是\fbox{函數被微分+上限被微分乘函數-下限被微分乘函數}
    \begin{equation}
      \frac{\partial f(x,y)}{\partial x} = \int_{y_1(x)}^{y_2(x)} \frac{\partial \phi(y,x)}{\partial x} dy 
      + \phi(y_2,x) \frac{dy_2(x)}{dx} 
      - \phi(y_1,x) \frac{dy_1(x)}{dx} \label{eq:leibniz_rule}
    \end{equation}
  \item $\int_0^\infty e^{-x^2} dx = \frac{\sqrt{\pi}}{2}$ \label{sec:error_function_proof}\\
    這東西是$\Gamma(\frac{3}{2})$,不過有其他證明方式\\
    令
    \begin{equation}
      \int_0^\infty e^{-x^2} dx = I
    \end{equation}
    則$I^2$可寫成:
    \begin{align}
      I^2 &= \left(\int_0^\infty e^{-x^2} dx\right)\left(\int_0^\infty e^{-y^2} dy\right) \nonumber\\
      &= \int_0^\infty \int_0^\infty e^{-(x^2+y^2)} dx dy
    \end{align}
    將其轉為\fbox{極座標}:
    \begin{align}
      I^2 &= \int_{\theta=0}^{\frac{\pi}{2}} \int_{r=0}^{\infty} e^{-r^2} r dr d\theta \nonumber\\
      &= \frac{\pi}{2} \int_0^\infty e^{-r^2} r dr
    \end{align}
    令$u=-r^2$,$du=-2r dr$,則
    \begin{align}
      I^2 &= \frac{\pi}{2} \int_{u=-\infty}^{0} e^{u} \left(-\frac{1}{2}\right) du \nonumber\\
      &= \frac{\pi}{4} \int_{-\infty}^{0} e^{u} du \nonumber\\
      &= \frac{\pi}{4} \left[e^{u}\right]_{-\infty}^{0} \nonumber\\
      &= \frac{\pi}{4} (1-0) \nonumber\\
      &= \frac{\pi}{4}
    \end{align}
    故
    \begin{equation}
      \boxed{I = \int_0^\infty e^{-x^2} dx = \frac{\sqrt{\pi}}{2}}
    \end{equation}
\end{itemize}
\end{CJK*}
\end{document}