\documentclass[../main.tex]{subfiles}
\begin{document}
\begin{CJK*}{UTF8}{bkai}
\subsection{質量傳輸概論}
質量傳輸就是有各種物質,在空間與時間下的分布與變化\\
假設空間中有一個控制體積,則在這個控制體積內,各個物質的
\begin{enumerate}
  \item Density, $\rho_i(\vec{\bm r}, t)$
  \item Velocity, $\vec{\bm u}_i(\vec{\bm r}, t)$
  \item Mass Flux, $\vec{\bm n}_i (\vec{\bm r}, t)$
  \item Rate of Production, $r_i(\vec{\bm r}, t)$
\end{enumerate}
會發生變化,其中注意密度、生成速率是純量,對各方向皆相同\\
而速度、質量通量則是向量,會隨方向改變\\
質量通量,又可以再拆分為分子擴散造成,以及對流造成\\
所謂對流,就是跟著整陀流體一起移動,以流體力學所造成的質量通量\\
而分子擴散,則是在這個流場中,物質因為濃度差異,額外產生的質量通量\\
因此質量通量,可以表示為
\begin{equation}
  \vec{\bm n}_i = \rho_i \vec{\bm u} + \vec{\bm j}_i \label{eq:mass_flux_decomposition}
\end{equation}
其中$\vec{\bm j}_i$即為分子擴散所造成的質量通量\\
\fbox{Molecular diffusion by mass gradient}\\
針對所有分密度相加,可以改為平均密度
\begin{equation}
  \sum_i \rho_i = \rho
\end{equation}
針對所有物質的擴散加總,會得到一個淨流入與淨流出的速度,定義為體積平均流速\\
\fbox{Volume Average Velocity}
\begin{equation}
  \sum_i \rho_i \vec{\bm u} = \rho \vec{\bm u} \implies \boxed{\vec{\bm u} = \frac{\sum_i \rho_i \vec{\bm u}}{\rho}}
\end{equation}
換句話說,各物質的擴散$\vec{\bm j}_i$,則是根據這個平均流速為基準額外增加或減少的質量通量\\
也因此根據此定義:
\begin{align}
  \vec{\bm j}_i &= \vec{\bm n}_i - \rho_i \vec{\bm u} \nonumber\\
  &= \rho_i (\vec{\bm u}_i - \vec{\bm u})\nonumber\\
  \sum_i \vec{\bm j}_i &= \sum_i \rho_i (\vec{\bm u}_i - \vec{\bm u})\nonumber\\
  &= \sum_i \rho_i \vec{\bm u}_i - \vec{\bm u} \sum_i \rho_i \nonumber\\
  &= \rho \vec{\bm u} - \rho \vec{\bm u} = 0
\end{align}
必然成立:
\begin{equation}
  \boxed{\sum_i \vec{\bm j}_i = 0}
\end{equation}
另外根據質量守恆,反應通量$r_i$,會滿足
\begin{equation}
  \boxed{\sum_i r_i=0}
\end{equation}
而將質量除上各物質的分子量$M_i$,即可獲得Molar Flux,會以大寫表示\\
對比(\ref{eq:mass_flux_decomposition}),Molar Flux可表示為
\begin{align}
  &\frac{n_i}{M_i} = \frac{\rho_i \vec{\bm u}}{M_i} + \frac{\vec{\bm j}_i}{M_i} \nonumber\\
  &\boxed{\vec{\bm N}_i = C_i \vec{\bm u} + \vec{\bm J}_i} \label{eq:molar_flux_decomposition}
\end{align}
由$\frac{\rho_i \vec{\bm u}}{M_i}$,定義出了$C_i$
\begin{equation}
  \frac{\rho_i}{M_i} = C_i \label{eq:species_molar_concentration}
\end{equation}
稱為\fbox{Species Molar Concentration}\\
$J_i$則稱為\fbox{Molar Diffusion Flux}\\
$N_i$則稱為\fbox{Species Molar Flux}\\
同理,對$r_i$除上$M_i$,即可獲得Molar Rate of Production,表示為
\begin{equation}
  R_i = \frac{r_i}{M_i} \label{eq:molar_rate_of_production}
\end{equation}
當然,畢竟就是反應莫爾數,因此加總會跟總反應的係數合有關
\begin{equation}
  \sum_i R_i \neq 0 (\text{Depends on Stoichiometry})
\end{equation}
不過$N_i$在定義上會因為$\sum_i \vec{\bm j}_i= 0$,而定義為
\begin{equation}
  \vec{\bm N}_i = C_i \vec{\bm u}_i \label{eq:species_molar_flux_alternate}
\end{equation}
而各物質基於此定義產生的額外流速,才會是$J_i$\\
可是如果用大寫的Molar Flux來看:
\begin{align}
  \vec{\bm N}_i &= C_i \vec{\bm u} + \vec{\bm J}_i \nonumber\\
  &= C_i \vec{\bm u}_i \nonumber\\
  \implies \vec{\bm J}_i &= C_i (\vec{\bm u}_i - \vec{\bm u})
\end{align}
由於
\begin{equation}
  \vec{\bm u} = \frac{\sum_i \rho_i \vec{\bm u}}{\sum_i \rho_i}
\end{equation}
因此:
\begin{align}
  \sum_i \vec{\bm J}_i &= \sum_i C_i (\vec{\bm u}_i - \vec{\bm u}) \nonumber\\
  &= \sum_i C_i \vec{\bm u}_i - \vec{\bm u} \sum_i C_i \nonumber\\
  &= \sum_i C_i \vec{\bm u}_i - \vec{\bm u} C \neq 0
\end{align}
也因為這件事\\
我們希望可以不需要寫成:$\sum_i C_i \vec{\bm u}_i$,這樣就需要2倍物質種類的資訊量\\
所以額外定義了\fbox{Molar Average Velocity}
\begin{equation}
  C \vec{\bm u}^\ast = \sum_i C_i \vec{\bm u}_i \implies \vec{\bm u}^\ast = \frac{\sum_i C_i \vec{\bm u}_i}{C} 
  \label{eq:molar_average_velocity}
\end{equation}
而這時:
\begin{equation}
  \boxed{\vec{\bm N}_i = C_i \vec{\bm u}^\ast + \vec{\bm J}_i^\ast} \label{eq:molar_flux_decomposition_volume_average}
\end{equation}
這使得
\begin{align}
  \vec{\bm J}_i^\ast &= C_i \vec{\bm u}_i - C_i \vec{\bm u}^\ast \nonumber\\
  &= C_i (\vec{\bm u}_i - \vec{\bm u}^\ast) \nonumber\\
  \implies &\sum_i \vec{\bm J}_i^\ast = 0
\end{align}
將動量傳輸、能量傳輸、質量傳輸的通量做比較
\begin{align}
  \text{Momentum Flux: }& \overline{\overline {\bm \Phi}} = \underbrace{\quad\rho \vec{\bm u}\vec{\bm u}\quad}_{\text{慣性力}} 
  + \underbrace{\quad\quad\overline{\overline{\bm \pi}}\quad\quad}_{\text{周遭流體作用力}} \nonumber\\
  \text{Energy Flux: }& \overline{\bm e}  = \underbrace{\quad\rho\hat E \vec{\bm u} \quad}_{\text{對流}}
  + \underbrace{\quad\vec{\bm_q}\quad}_{\text{熱傳導}} 
  + \underbrace{\quad\quad\overline{\overline{\bm \pi}}\cdot \vec{\bm u}\quad\quad}_{\text{分子運動做功}} \nonumber\\
  \text{Mass Flux: }& n_i = \underbrace{\quad\rho_i \vec{\bm u}\quad}_{\text{對流}}
  + \underbrace{\quad\vec{\bm j}_i\quad}_{\text{分子擴散}} \label{eq:transport_flux_comparison}
\end{align}
\begin{itemize}
  \item Governing Equation:
  \begin{itemize}
    \item 對一個物質的質量守恆:
    \begin{equation}
      m_{\text{in}} - m_{\text{out}} + m_{\text{gen}} = \Delta m_{\text{acc}} \label{eq:mass_conservation_general}
    \end{equation}
    故同之前熱傳與動量傳輸:
    \begin{equation}
      -\iint \vec{\bm n}_i \cdot \vec{\bm n}_sdS + \iiint r_i dV = \iiint \frac{\partial \rho_i}{\partial t} dV \label{eq:mass_conservation_integral}
    \end{equation}
    而其中$\vec{\bm n}_s$為控制體積表面的法向量\\
    同樣根據高斯定理,將表面積分轉成體積積分:
    \begin{align}
      & - \iiint \nabla \cdot \vec{\bm n}_i dV + \iiint r_i dV = \iiint \frac{\partial \rho_i}{\partial t} dV \nonumber\\ 
      \Rightarrow \quad & \iiint\left(
        \frac{\partial \rho_i}{\partial t} + \nabla \cdot \vec{\bm n}_i - r_i
      \right) dV = 0 \nonumber\\
      \Rightarrow \quad & \boxed{\frac{\partial \rho_i}{\partial t} + \nabla \cdot \vec{\bm n}_i  -r_i = 0} \\
      \text{or} \quad & \boxed{\frac{\partial \rho_i}{\partial t} + \nabla \cdot (\rho_i \vec{\bm u}) - r_i = 0} \label{eq:mass_conservation_differential}
    \end{align}
    \item 對所有物質的質量守恆:
    \begin{align}
      &\sum_i \frac{\partial \rho_i}{\partial t} + \sum_i\nabla \cdot (\rho_i \vec{\bm u}) - \cancelto{0}{\sum_i r_i} = 0 \nonumber\\
      &\boxed{\frac{\partial \rho}{\partial t}+ \nabla \cdot \rho \vec{\bm u}=0} \label{eq:total_mass_conservation}
    \end{align}
    而就會發現他就變成流力的Equation of Continuity了
    \item 寫成莫耳數的形式,有發生反應時會比較好計算
    \begin{equation}
      \frac{\partial C_i}{\partial t} + \nabla \cdot \vec{\bm N}_i - R_i = 0 \label{eq:molar_mass_conservation}
    \end{equation}
    $\vec{\bm N}_i=C_i\vec{\bm u}$\\
    P.S. 上面討論過了$\sum_i R_i \neq 0$,會跟反應的化學計量有關
    \item Fick's Law前身,如何得到$J_i, j_i$:\\
    \fbox{Maxwell-Stefan Diffusion Equation}\\
    假設了\fbox{物質$i$是受到所有非$i$的物質$j$的受力而移動}\\
    而其他非$i$的物質$j$施給$i$的力,與$i$因為這個受力而施予其他$j$的力相等\\
    (牛頓第三運動定律)\\
    並認為作用力的大小,與單位時間內的碰撞頻率有關\\
    而碰撞頻率,又跟兩物質的速度差成正比
    \begin{equation}
      \vec{\bm \varphi}_i^j= k_{ij} \left(\vec{\bm u_j - \vec{\bm u_i} }\right)
    \end{equation}
    $\varphi_i^j$為$j$對$i$的單位體積下的作用力,$k_{ij}$為Friction Coefficient\\
    而$\vec{\bm u_j - \vec{\bm u_i} }$,則為兩物質的相對速度\\
    且$k_{ij}=k_{ji}$\\
    而Stefan-Maxwell又檢設$\boxed{k_{ij}\propto C_i C_j}$\\
    而其比例常數取決於物質特性、溫度、以及整體濃度\\
    將物質特性部分寫作$\mathbcal{D}_{ij}$ 有著\fbox{單位 $\frac{m^2}{s}$}
    \begin{equation}
      k_{ij} = \frac{C_iC_j RT}{C \mathbcal{D}_{ij}},\quad C=\sum_i C_i
    \end{equation}
    P.S. 同理$\mathbcal{D}_{ij}=\mathbcal{D}_{ji}$,稱之為\fbox{Maxwell-Stefan Diffusivity}
    \item 對物質$i$,受到所有$j$的作用力總和
    \begin{equation}
      \vec{\bm \varphi}_i = \sum_{j \neq i} \vec{\bm \varphi}_i^j = \sum_{j \neq i} k_{ij} \left(\vec{\bm u_j - \vec{\bm u_i} }\right)
    \end{equation}
    $\vec{\bm \varphi}_i$,是Total Drag Force per Unit Volume on Species $i$\\
    而這個由速度現象產生的作用力來源,應該來自濃度不同導致的\fbox{化學能}不同
    \begin{equation}
      \text{Driving Force/V} = \vec{\bm d}_i = -C_i \nabla \mu_i,\quad 
      \mu_i = \left(
        \frac{\partial G}{\partial n_i}
      \right)_{T,P,n_j}
    \end{equation}
    P.S.  以單位來想的話,因為功是力乘上距離,故對能量取梯度\\
    單位會是\fbox{Force / mol},再乘上濃度的單位\fbox{mol / m$^3$}\\
    即可得到\fbox{Force / m$^3$}\\
    而因為Driving Force是物質感受周遭,而想要對外做功\\
    Drag Force則是周遭物質對物質$i$施加的阻力\\
    故兩者應該大小相等,方向相反\\
    故寫出Maxwell-Stefan Diffusion Equation:
    \begin{equation}
      \boxed{C_i \nabla \mu_i = \sum_{j\neq i}k_{ij}\left(
        \vec{\bm u_j - \vec{\bm u_i} }
      \right)} \label{eq:maxwell_stefan_diffusion}
    \end{equation}
    如果計算整體加總,會發現根據Gibbs-Duhem Equation
    \begin{equation}
      \boxed{
        SdT -VdP + \sum_i n_i d\mu_i = 0
      }
    \end{equation}
    在\fbox{等溫、等壓}下,後面那項$\sum_i n_i d\mu_i = 0$\\
    而這也剛好是Maxwell-Stefan Diffusion Equation加總後左側的結果\\
    故
    \begin{equation}
      \sum_i C_i \nabla \mu_i = 0 \implies \sum_i \sum_{j\neq i}\frac{C_i C_j RT}{C \mathbcal{D}_{ij}} \left(
        \vec{\bm u_j - \vec{\bm u_i} }
      \right) = 0
    \end{equation}
    另外根據定義$\vec{\bm N}_i = C_i \vec{\bm u}_i$
    \begin{equation}
      \vec{\bm u}_i = \frac{\vec{\bm N}_i}{C_i}
    \end{equation}
    代入(\ref{eq:maxwell_stefan_diffusion}),可以消除$C_i, C_j$
    \begin{align}
      C_i \nabla \mu_i &= \sum_{j\neq i}k_{ij}\left(
        \vec{\bm u_j - \vec{\bm u_i} }
      \right)\nonumber\\
      &=\sum_{j\neq i} \frac{C_i C_j RT}{C \mathbcal{D}_{ij}} \left( 
        \vec{\bm u}_j - \vec{\bm u}_i
      \right) \nonumber\\
      &= \sum_{j\neq i} \frac{C_i C_jRT}{C \mathbcal{D}_{ij}} \left(
        \frac{\vec{\bm N}_j}{C_j} - \frac{\vec{\bm N}_i}{C_i}
      \right) \nonumber\\
      &= \sum_{j\neq i} \frac{C_i C_jRT}{C \mathbcal{D}_{ij}} \left(
        \frac{\vec{\bm N}_jC_i - \vec{\bm N}_iC_j}{C_iC_j}
      \right)\nonumber\\
      &= \sum_{j\neq i} \frac{ RT}{C\mathbcal{D}_{ij}}
      \left(\vec{\bm N}_jC_i - \vec{\bm N}_iC_j\right)
    \end{align}
    把整體總濃度$C$移入括號內$\frac{C_i}{C} = X_i$\\
    即為物質$i$的莫耳分率
    \begin{equation}
      \boxed{C_i \nabla \mu_i = \sum_{j\neq i} \frac{ RT}{\mathbcal{D}_{ij}}
      \left(x_i \vec{\bm N}_j - x_j \vec{\bm N}_i\right)} \label{eq:maxwell_stefan_diffusion_final}
    \end{equation}
    \item 假設理想氣體
    \begin{equation}
      \mu_i = \mu_i^\circ + RT \ln (x_i P)
    \end{equation}
    P.S. 如果不是理想氣體,要將活度係數考慮進去,換掉$x_i$
    \begin{equation}
      \mu_i = \mu_i^\circ(T,P) + RT \ln (a_i) \label{eq:chemical_potential_non_ideal}
    \end{equation}
    而$\nabla \mu_i$為(定溫、定壓下)
    \begin{align}
      \nabla \mu_i &= \nabla \left(
        \mu_i^\circ + RT \ln( x_i P)
      \right) \nonumber\\
      &= RT \nabla \ln (x_i P) \nonumber\\
      &= RT \left(\frac{1}{x_i P}\cdot P \nabla x_i\right) \nonumber\\
      &= \frac{RT}{x_i} \nabla x_i \label{eq:chemical_potential_gradient_ideal_gas}
    \end{align}
    代回(\ref{eq:maxwell_stefan_diffusion_final}),可以消掉$RT$
    \begin{align}
      C_i \nabla \mu_i &= \sum_{j\neq i} \frac{ RT}{\mathbcal{D}_{ij}}
      \left(x_i \vec{\bm N}_j - x_j \vec{\bm N}_i\right) \nonumber\\
      \frac{C_i \cancel{RT}}{x_i} \nabla x_i &= \sum_{j\neq i} \frac{\cancel{RT}}{\mathbcal{D}_{ij}}
      \left(x_i \vec{\bm N}_j - x_j \vec{\bm N}_i\right) \nonumber\\
      \frac{C_i}{\frac{C_i}{C}} \nabla x_i &= \sum_{j\neq i} \frac{1}{\mathbcal{D}_{ij}}
      \left(x_i \vec{\bm N}_j - x_j \vec{\bm N}_i\right) \nonumber\\
      \implies \quad \nabla x_i &= \sum_{j\neq i} \frac{1}{C \mathbcal{D}_{ij}} \left(
        x_i \vec{\bm N}_j - x_j \vec{\bm N}_i
      \right) \label{eq:maxwell_stefan_diffusion_ideal_gas}
    \end{align}
    如果是二元系統,右邊的$(x_i \vec{\bm N}_j - x_j \vec{\bm N}_i)$可以合併成一個變數:
    \begin{align}
      x_A \vec{\bm N}_B - x_B \vec{\bm N}_A &= 
      \underbrace{x_A \vec{\bm N}_B}_A - \underbrace{x_A \vec{\bm N}_B}_B
      + \underbrace{x_A\vec{\bm N}_A}_A -\underbrace{ x_A \vec{\bm N}_A}_B \nonumber\\
      &= x_A \left(
        \vec{\bm N}_A + \vec{\bm N}_B
      \right) - \vec{\bm N}_A \left(
        x_A + x_B
      \right) \nonumber\\
      &= x_A \left(\vec{\bm N}_A + \vec{\bm N}_B\right) - \vec{\bm N}_A
    \end{align}
    \item 二元系統、理想氣體,定溫定壓\\
    物質為A,B,以A來看
    \begin{align}
      C_A \nabla \mu_A &= \frac{RT}{\mathbcal{D}_{AB}} (x_A \vec{\bm N}_B - x_B \vec{\bm N}_A) \nonumber\\
      & = \frac{RT}{\mathbcal{D}_{AB}} \left(x_A \left(\vec{\bm N}_A + \vec{\bm N}_B\right) - \vec{\bm N}_A\right) \\
      \nabla x_A &= \frac{1}{C \mathbcal{D}_{AB}} \left(
        x_A \left(\vec{\bm N}_A + \vec{\bm N}_B\right) - \vec{\bm N}_A
      \right)
    \end{align}
    \item 二元系統,非理想流體,非定溫定壓\\
    物質為A,B,以A來看\\
    化學能的關係式為(\ref{eq:chemical_potential_non_ideal})
    \begin{equation}
      \mu_i = \mu_i^\circ(T,P) + RT \ln (a_i) 
    \end{equation}
    而對$\mu_A(T,p,a_A)$做全微分,要對溫度、壓力、活度做偏微分
    \begin{align}
      d\mu_A &= \left(\frac{\partial \mu_A}{\partial T}\right)_{P,a_A} dT +
      \left(\frac{\partial \mu_A}{\partial P}\right)_{T,a_A} dP +
      \left(\frac{\partial \mu_A}{\partial a_A}\right)_{T,P} da_A \nonumber\\
      &= -S_A dT + V_A dP + RT \frac{1}{a_A} da_A
    \end{align}
    $S$為物質A的Molar Entropy,$V$為Molar Volume\\
    而對空間取梯度
    \begin{equation}
      \nabla \mu_A = -S_A \nabla T + V_A \nabla P + RT \nabla\ln (a_A)
    \end{equation}
    而前兩項可以用狀態方程算出來,移到左側就修正了\\
    故假設之後提到的$\nabla \mu_i$,都是減去這兩項之後的結果\\
    而因為等號右邊是用$x_i$表示,因此要將$a_i$換成$x_i$
    \begin{equation}
      \nabla \mu_A +S_A \nabla T - V_A \nabla P = RT \left(
        \frac{\partial (\ln a_A)}{\partial x_A}
      \right)_{T,P} \nabla x_A \label{eq:chemical_potential_gradient_non_ideal}
    \end{equation}
    代回(\ref{eq:maxwell_stefan_diffusion_final})
    \begin{equation}
      C_A RT \left(
        \frac{\partial (\ln a_A)}{\partial x_A}
      \right)_{T,P} \nabla x_A = \frac{RT}{\mathbcal{D}_{AB}} \left(
        x_A \left(\vec{\bm N}_A + \vec{\bm N}_B\right) - \vec{\bm N}_A
      \right)
    \end{equation}
    可以消掉$RT$
    \begin{equation}
      C_A \left(
        \frac{\partial (\ln a_A)}{\partial x_A}
      \right)_{T,P} \nabla x_A = \frac{1}{\mathbcal{D}_{AB}} \left(
        x_A \left(\vec{\bm N}_A + \vec{\bm N}_B\right) - \vec{\bm N}_A
      \right)
    \end{equation}
    \item 寫成以\fbox{$\vec{\bm N}_A$}的形式,會換成Fick's Law
    \begin{align}
      C_A\mathbcal{D}_{AB}\left(
        \frac{\partial (\ln a_A)}{\partial x_A}
      \right)_{T,P} \nabla x_A &= x_A \left(\vec{\bm N}_A + \vec{\bm N}_B\right) - \vec{\bm N}_A \nonumber\\
      \vec{\bm N}_A &= x_A \left(\vec{\bm N}_A + \vec{\bm N}_B\right) - \boxed{C_A}\mathbcal{D}_{AB}\left(
        \frac{\partial (\ln a_A)}{\partial x_A}
      \right)_{T,P} \nabla x_A \nonumber\\
      \vec{\bm N}_A &= x_A \left(\vec{\bm N}_A + \vec{\bm N}_B\right) - \boxed{x_AC}\mathbcal{D}_{AB}\left(
        \frac{\partial (\ln a_A)}{\partial x_A}
      \right)_{T,P} \nabla x_A \nonumber\\
      \vec{\bm N}_A &= x_A \left(\vec{\bm N}_A + \vec{\bm N}_B\right) -C \mathbcal{D}_{AB}\left(
        \frac{\partial (\ln a_A)}{\partial \ln(x_A)}
      \right)_{T,P} \nabla x_A
    \end{align}
    定義Fickian Diffusivity為
    \begin{equation}
      D_{AB} = \mathbcal{D}_{AB} \left(
        \frac{\partial (\ln a_A)}{\partial \ln(x_A)}
      \right)_{T,P}
    \end{equation}
    \fbox{$\mathbcal{D}_{AB}$ 比起 $D_{AB}$ 跟濃度更無關}\\
    最後得到Fick's Law的形式
    \begin{equation}
      \boxed{\vec{\bm N}_A = x_A \left(\vec{\bm N}_A + \vec{\bm N}_B\right) - C D_{AB} \nabla x_A} \label{eq:ficks_law_general}
    \end{equation}
    而,中間那項可以代入莫耳平均速度的定義(\ref{eq:molar_average_velocity})\\
    讓$\sum N_i = \vec{\bm u}^\ast \sum C_i$,可以直接加成濃度
    \begin{equation}
      x_A \left(\vec{\bm N}_A + \vec{\bm N}_B\right) = x_A\left(
        C_A \vec{\bm u}_A + C_B \vec{\bm u}_B
      \right) = x_A C \vec{\bm u}^\ast = C_A \vec{\bm u}^\ast
    \end{equation}
    代回(\ref{eq:ficks_law_general})
    \begin{equation}
     \boxed{ \vec{\bm N}_A = C_A \vec{\bm u}^\ast - \underbrace{C D_{AB} \nabla x_A}_{\vec{\bm J}_A^\ast}}
    \end{equation}
    而回到最一開始的Governing Equation(\ref{eq:molar_flux_decomposition_volume_average})\\
    第一項就是對流項,而第二項就是分子擴散項
    \item 如果用小寫的來表示
    \begin{equation}
      \vec{\bm  n}_A = \rho_A \vec{\bm u}^\ast + \vec{\bm j}_A
    \end{equation}
    我們知道$C \vec{\bm u}^\ast = \sum_i C_i \vec{\bm u}_i$\\
    乘上$M_A$,並利用(\ref{eq:species_molar_concentration}),可以得到
    \begin{equation}
      \rho \vec{\bm u}^\ast = \sum_i \rho_i \vec{\bm u}_i
    \end{equation}
    兩者個連接差在一個是以莫耳數表示,一個是以質量表示
    \begin{equation}
      \vec{\bm  N}_A  = \frac{\vec{\bm n}_A}{M_A} = \frac{1}{M_A}\left(
        \rho_A \vec{\bm u} + \vec{\bm j}_A
      \right) =  C_A \vec{\bm u}^\ast - \vec{\bm J}_A^\ast
    \end{equation}
    而$j_A=\rho_A(\vec{\bm u}_A -\vec{\bm_u})$,是相對於體積平均速度的分子擴散\\
    將體積平均速度展開:
    \begin{equation}
      \vec{\bm u} = \frac{\left(\rho_A \vec{\bm u}_A + \rho_B \vec{\bm u}_B\right)}{\rho_A + \rho_B}
    \end{equation}
    定義$\omega_i = \frac{\rho_i}{\sum_j \rho_j}$為物質i的質量分率\\
    則
    \begin{equation}
      \vec{\bm u} = \omega_A \vec{\bm u}_A + \omega_B \vec{\bm u}_B
    \end{equation}
    代入$j_A$中
    \begin{align}
      \vec{\bm j}_A &= \rho_A \left(\vec{\bm u}_A - \vec{\bm u}\right) \nonumber\\
      &= \rho_A \left(\vec{\bm u}_A - \omega_A \vec{\bm u}_A - \omega_B \vec{\bm u}_B\right) \nonumber\\
      &= \rho_A \left(
        (1-\omega_A) \vec{\bm u}_A - \omega_B \vec{\bm u}_B
      \right) \nonumber\\
      &= \rho_A \left(
        \omega_B \vec{\bm u}_A - \omega_B \vec{\bm u}_B
      \right) \nonumber\\
      &= \rho_A \omega_B \left(
        \vec{\bm u}_A - \vec{\bm u}_B
      \right)
    \end{align}
    而相似的,大寫的分子擴散
    \begin{align}
      \vec{\bm J}_A^\ast &= C_A \left(\vec{\bm u}_A - \vec{\bm u}^\ast\right) \nonumber\\
      &= C_A \left(\vec{\bm u}_A - x_A \vec{\bm u}_A - x_B \vec{\bm u}_B\right) \nonumber\\
      &= C_A \left(
        (1-x_A) \vec{\bm u}_A - x_B \vec{\bm u}_B
      \right) \nonumber\\
      &= C_A \left(
        x_B \vec{\bm u}_A - x_B \vec{\bm u}_B
      \right) \nonumber\\
      &= C_A x_B \left(
        \vec{\bm u}_A - \vec{\bm u}_B
      \right)
    \end{align}
    因此兩者之間具有以下比例關係
    \begin{equation}
      \boxed{\frac{\vec{\bm j}_A}{\vec{\bm J}_A^\ast} = \frac{\rho_A \omega_B}{C_A x_B} = \frac{M_A \omega_B}{x_B}
      } \label{eq:molar_to_mass_diffusion_flux}
    \end{equation}
    另外$x$和$\omega$之間的關係為
    \begin{equation}
      x_A = \frac{\frac{\omega_A}{M_A}}{\frac{\omega_A}{M_A} + \frac{\omega_B}{M_B}}
    \end{equation}
    而因為$\omega_B = 1 - \omega_A$,因此
    \begin{equation}
      x_A = \frac{\frac{\omega_A}{M_A}}{\frac{\omega_A}{M_A} + \frac{1-\omega_A}{M_B}}
    \end{equation}
    微分:
    \begin{equation}
      \frac{d}{d\omega_A} x_A =\frac{
        \frac{1}{M_A} \left(
          \frac{\omega_A}{M_A} + \frac{1-\omega_A}{M_B}
        \right) - \frac{\omega_A}{M_A} \left(
          \frac{1}{M_A} - \frac{1}{M_B}
        \right)
      }{\left(
        \frac{\omega_A}{M_A} + \frac{1-\omega_A}{M_B}
      \right)^2}
    \end{equation}
  令平均分子量的倒數
  \begin{equation}
    \frac{1}{\overline{M}} = \frac{\omega_A}{M_A} + \frac{1-\omega_A}{M_B}
  \end{equation}
  則
  \begin{equation}
    \frac{d}{d\omega_A} x_A = \frac{1}{\overline{M}^2} \cdot \frac{1}{M_A M_B}
  \end{equation}
  因此
  \begin{equation}
    \frac{d x_A}{d \omega_A} = \frac{\overline{M}^2}{M_A M_B} \label{eq:dxA_domegaA}
  \end{equation}
  故
  \begin{equation}
    \nabla x_A = \frac{\overline{M}^2}{M_A M_B} \nabla \omega_A \label{eq:grad_xA_to_grad_omegaA}
  \end{equation}
  於是Fick's Law可以寫成以質量分率表示
  \begin{align}
    \vec{\bm J}_A^\ast &= - C D_{AB} \nabla x_A \nonumber\\
    &= - C D_{AB} \cdot \frac{\overline{M}^2}{M_A M_B} \nabla \omega_A 
  \end{align}
  由(\ref{eq:molar_to_mass_diffusion_flux}),可得
  \begin{equation}
    \vec{\bm j}_A = \frac{M_A \omega_B}{x_B} \vec{\bm J}_A^\ast
  \end{equation}
  代入:
  \begin{align}
    \vec{\bm j}_A &= -CD_{AB}\cdot\left(
      \frac{M_A \omega_B}{x_B}
    \right)\cdot \frac{\overline{M}^2}{M_A M_B} \nabla \omega_A \nonumber\\
    &= -CD_{AB} \cdot \frac{\omega_B}{x_B} \cdot \frac{\overline{M}^2}{M_B} \nabla \omega_A
  \end{align}
  而$C\cdot \overline{M} = \rho$,$x_B = \omega_B\frac{\overline{M}}{M_B}$\\
  故
  \begin{align}
    \vec{\bm j}_A &= - CD_{AB} \cdot \frac{\omega_B}{x_B} \cdot \frac{\overline{M}^2}{M_B} \nabla \omega_A \nonumber\\
    &= - \underbrace{C\overline{M}}_{\rho} D_{AB}\cdot \left(
      \underbrace{\omega_B\frac{\overline{M}}{M_B}}_{x_B}\cdot \frac{1}{x_B}
    \right) \nabla \omega_A \nonumber\\
    &= - \rho D_{AB} \nabla \omega_A
  \end{align}
  因此以質量分率表示的Fick's Law為
  \begin{equation}
    \boxed{\vec{\bm j}_A = - \rho D_{AB} \nabla \omega_A} \label{eq:ficks_law_mass_fraction}
  \end{equation}
  或者:
  \begin{equation}
    \vec{\bm n}_A = \rho_A \vec{\bm u}^\ast - \rho D_{AB} \nabla \omega_A
  \end{equation}
  注意這只有二元系統可以這樣換
  \item 簡略形式,二元系統,密度固定,Fickian擴散係數固定,能量平衡式
  \begin{align}
    \frac{\partial \omega_A}{\partial t} + \nabla \cdot \vec{\pm n}_A -r_A &= 0 \nonumber\\
    \frac{\partial \omega_A}{\partial t} + \nabla \cdot \left(
      \rho_A \vec{\bm u} + \vec{\bm j}_A
    \right) - r_A &= 0 \nonumber\\
    \frac{\partial \omega_A}{\partial t} + \nabla \cdot \left(
      \rho_A \vec{\bm u} - \rho D_{AB} \nabla \omega_A
    \right) - r_A &= 0 \nonumber\\
    \frac{\partial \omega_A}{\partial t} + \nabla \cdot \left(
      \rho_A \vec{\bm u} -  D_{AB} \nabla \rho_A
    \right) - r_A &= 0 \nonumber\\
    \frac{\partial \omega_A}{\partial t} + \rho_A \cancelto{0}{\nabla \cdot \vec{\bm u}} 
    + \vec{\bm u} \cdot \nabla \rho_A -D_{AB} \nabla^2 \rho_A - r_A &=0 \nonumber\\
    (\div M_A)\implies \quad \boxed{\frac{\partial C_A}{\partial t} 
    + \vec{\bm u} \cdot \nabla C_A - D_{AB} \nabla^2 C_A - R_A} &=0\label{eq:mass_transport_simplified }
  \end{align}
  \item 更簡略形式,二元系統,密度固定,擴散係數固定,一物濃度極低\\
  令$C_i\ll C_s$,$C_i$是溶質,$C_s$是溶劑\\
  因此這時$C=C_s + C_i \approx C_s$為常數\\
  平均速度也只會是溶劑的,$\vec{\bm u} \approx \vec{\bm u}_s$\\
  化學能變化:
  \begin{align}
    \mu_i &= \mu_i^\circ + RT \ln (x_i P) \nonumber\\
    \nabla \mu_i &= RT \nabla (\ln x_i )
  \end{align}
  從(\ref{eq:maxwell_stefan_diffusion})開始
  \begin{align}
    C_i \nabla \mu_i &= RT\frac{C_i C_s}{C \mathbcal{D}_{is}} \left(
       \vec{\bm u}_s - \vec{\bm u}_i
    \right) \nonumber\\
    &\approx RT \frac{C_i C_s}{C_s \mathbcal{D}_{is}} \left(
       \vec{\bm u} - \vec{\bm u}_i
    \right) \nonumber\\
    C_i RT\nabla (\ln C_i) &= RT \frac{C_i}{\mathbcal{D}_{is}} \left(
       \vec{\bm u} - \vec{\bm u}_i
    \right) \nonumber\\
    \nabla (\ln C_i) &= \frac{1}{\mathbcal{D}_{is}} \left(
       \vec{\bm u} - \vec{\bm u}_i
    \right) \nonumber\\
    \frac{1}{C_i}\nabla C_i &= \frac{1}{\mathbcal{D}_{is}} \left(
       \vec{\bm u} - \vec{\bm u}_i
    \right) \nonumber\\
    -\mathbcal{D}_{is} \nabla C_i &= C_i\vec{\bm u} - \underbrace{C_i \vec{\bm u}_i}_{\vec{\bm N}_i} \nonumber\\
    \vec{\bm N}_i &= C_i \vec{\bm u} - \underbrace{\mathbcal{D}_{is} \nabla C_i}_{\vec{\bm J}_i} \nonumber\\
    \implies \quad &\boxed{\vec{\bm J}_i = -\mathbcal{D}_{is} \nabla C_i} \label{eq:ficks_law_dilute_solution}
  \end{align}
  能量平衡式
  \begin{align}
    \frac{\partial C_i}{\partial t} + \nabla \cdot \vec{\bm N}_i - R_i &= 0 \nonumber\\
    \frac{\partial C_i}{\partial t} + \vec{\bm u} \cdot \nabla C_i - \mathbcal{D}_{is} \nabla^2 C_i - R_i &= 0
  \end{align}
  假設穩態,沒有反應
  \begin{equation}
    \boxed{\vec{\bm u} \cdot \nabla C_i = \mathbcal{D}_{is} \nabla^2 C_i} \label{eq:mass_transport_dilute_solution_steady_state}
  \end{equation}
    \item 對於incomprssible flow,$\nabla\cdot \vec {\bm u} = 0$,則
  \begin{equation}
    \frac{\partial C_A}{\partial t} + \vec {\bm u} \cdot \nabla C_A = D_{AB}\nabla^2 C_A +R_A
  \end{equation}
  或寫成:
  \begin{equation}
    \frac{DC_A}{Dt} = D_{AB}\nabla^2 C_A +R_A
  \end{equation}
  \item 對於incomprssible flow,且沒有質量生成$R_A=0$,則
  \begin{equation}
    \frac{\partial C_A}{\partial t} + \vec {\bm u} \cdot \nabla C_A = D_{AB}\nabla^2 C_A
  \end{equation}
  或寫成:
  \begin{equation}
    \frac{DC_A}{Dt} = D_{AB}\nabla^2 C_A
  \end{equation}
  \item 對於無流體流動$\vec {\bm u}=0$,且沒有反應
  \begin{equation}
    \frac{\partial C_A}{\partial t} = D_{AB}\nabla^2 C_A
  \end{equation}
  此為Fick's Second Law
  \item 若為穩態,且無流體流動$\vec {\bm u}=0$,且沒有反應:
  \begin{equation}
    \nabla^2 C_A = 0
  \end{equation}
  此為Laplace Equation
\end{itemize}
\item 無因次群:
\begin{itemize}
  \item Schmidt Number:
  \begin{equation}
    \text{Sc} = \frac{\nu}{D_{AB}} = \frac{\mu}{\rho D_{AB}} = \frac{\text{Momentum Diffusivity}}{\text{Mass Diffusivity}}
  \end{equation}
  這裡也同熱對流有:
  \begin{equation}
    \text{Sc} = \left(\frac{\delta}{\delta_m}\right)^3 = \left(\frac{\text{Mass Boundary Layer}}{\text{Momentum Boundary Layer}}\right)^3
  \end{equation}
  的關係,但注意$\delta_m$是質量邊界層厚度,而不是熱對流的熱邊界層厚度
  \item Lewis Number:
  \begin{equation}
    \text{Le} = \frac{\alpha}{D_{AB}} = \frac{k}{\rho C_p D_{AB}} = \frac{\text{Thermal Diffusivity}}{\text{Mass Diffusivity}}
  \end{equation}
  \item Prandtl Number:
  \begin{equation}
    \text{Pr} = \frac{\nu}{\alpha} = \frac{C_p\mu}{k} = \frac{\text{Momentum Diffusivity}}{\text{Thermal Diffusivity}}
  \end{equation}
  \item Sherwood Number:\\
  對比熱對流,就是那邊的Nusselt Number,徑向的動量傳輸和質量傳輸比
  \begin{equation}
    \text{Sh} = \frac{k_cL}{D_{AB}} = \frac{\text{Mass Convection Rate}}{\text{Molecular Diffusion Rate}}
  \end{equation}
  若系統為圓柱或球:
  \begin{equation}
    \text{Sh} = \frac{k_cD}{D_{AB}}
  \end{equation}
  \item Peclet Number:\\
  在軸向上的質量傳輸比
  \begin{equation}
    \text{Pe}_M =\text{Re}\cdot\text{Sc} = \frac{\rho u D}{\mu} = \frac{\text{Mass Convection Rate}}{\text{Molecular Diffusion Rate}}
  \end{equation}
  質量跟著液體流動方向傳遞\\
  若$\text{Pe}_M$很大,代表軸向上的質量傳導可以被忽略
  \item Stanton Number:
  \begin{equation}
    \text{St}_M = \frac{\text{Sh}}{\text{Re}\cdot\text{Sc}} = \frac{k_c}{\nu}
  \end{equation}
\end{itemize}
\item 一些質量傳輸的經驗式:
\begin{itemize}
  \item 流體流過平板:
  \begin{equation}
    \text{Sh} = 0.664\text{Re}^{\frac{1}{2}}\text{Sc}^{\frac{1}{3}}
  \end{equation}
  \item 層流流過長直圓管,其壁面濃度均一(泡茶包)
  \begin{equation}
    \text{Sh} = 3.66 + f(\text{Re},\text{Sc},\frac{L}{D})
  \end{equation}
  \item 層流流過長直圓管,給予均勻質量通量(過濾器)
  \begin{equation}
    \text{Sh} = \frac{48}{11} + f(\text{Re},\text{Sc},\frac{L}{D})
  \end{equation}
  \item Plug Flow,流過長直圓管,給予均勻質量通量
  \begin{equation}
    \text{Sh} = 8 + f(\text{Re},\text{Sc},\frac{L}{D})
  \end{equation}
  \item 流過球面的流(泡澡球):
  \begin{equation}
    \text{Sh} = 2 + f(\text{Re},\text{Sc})
  \end{equation}
  \item Reynolds Analogy:(適用於$\text{Sc}=1$)
  \begin{equation}
    \frac{k_c}{u_\infty} = \text{St}_M = \frac{C_f}{2} = \frac{\text{Sh}}{\text{Re}\cdot\text{Sc}}
  \end{equation}
  \item Colburn Analogy:(適用於$0.6<\text{Sc}<2500$) (P.S. 熱對流那邊是$0.5<\text{Pr}<50$)
  \begin{equation}
    j_D = \text{St}_M\cdot \text{Sc}^{\frac{2}{3}} = \frac{C_f}{2}
  \end{equation}
  $j_D$是Colburn $j$ factor for mass transfer\\
  所以:
  \begin{equation}
    j_H =  j_D = \frac{C_f}{2},\quad 0.6<\text{Sc}<2500,\quad 0.5<\text{Pr}<50
  \end{equation}
  \item Prandtl Analogy:(適用於忽略Drag)
  \begin{equation}
    \text{St}_M = \frac{\frac{C_f}{2}}{1+5\sqrt{\frac{C_f}{2}}\left(\text{Sc}-1\right)} = \frac{k_c}{u_\infty}
  \end{equation}
  \item Von Karman Analogy:(適用於忽略Drag)
  \begin{equation}
    \text{St}_M = \frac{\frac{C_f}{2}}{1+5\sqrt{\frac{C_f}{2}\left\{
      \text{Sc}-1+\ln\left[1+\frac{5}{6}\left(\text{Sc}-1\right)\right]
    \right\}}} = \frac{k_c}{u_\infty}
  \end{equation}
\end{itemize}
\end{itemize}
\end{CJK*}
\end{document}